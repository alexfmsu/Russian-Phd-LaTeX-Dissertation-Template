%&preformat-disser
\RequirePackage[l2tabu,orthodox]{nag} % Раскомментировав, можно в логе получать рекомендации относительно правильного использования пакетов и предупреждения об устаревших и нерекомендуемых пакетах
% Формат А4, 14pt (ГОСТ Р 7.0.11-2011, 5.3.6)
\documentclass[a4paper,14pt,oneside,openany]{memoir}

\input{common/setup}            % общие настройки шаблона
\input{common/packages}         % Пакеты общие для диссертации и автореферата
\synopsisfalse                      % Этот документ --- не автореферат
\input{Dissertation/dispackages}    % Пакеты для диссертации
\usepackage{fr-longtable}    %ради \endlasthead
\usepackage{relsize}
\usepackage{bm}
\usepackage{nccmath}

% Листинги с исходным кодом программ
\usepackage{fancyvrb}
\usepackage{listings}

\lccode`\~=0\relax %Без этого хака из-за особенностей пакета listings перестают работать конструкции с \MakeLowercase и т. п. в (xe|lua)latex

% Русская традиция начертания греческих букв
\usepackage{upgreek} % прямые греческие ради русской традиции

%%% Микротипографика
%\ifnumequal{\value{draft}}{0}{% Только если у нас режим чистовика
%    \usepackage[final, babel, shrink=45]{microtype}[2016/05/14] % улучшает представление букв и слов в строках, может помочь при наличии отдельно висящих слов
%}{}

% Отметка о версии черновика на каждой странице
% Чтобы работало надо в своей локальной копии по инструкции
% https://www.ctan.org/pkg/gitinfo2 создать небходимые файлы в папке
% ./git/hooks
% If you’re familiar with tweaking git, you can probably work it out for
% yourself. If not, I suggest you follow these steps:
% 1. First, you need a git repository and working tree. For this example,
% let’s suppose that the root of the working tree is in ~/compsci
% 2. Copy the file post-xxx-sample.txt (which is in the same folder of
% your TEX distribution as this pdf) into the git hooks directory in your
% working copy. In our example case, you should end up with a file called
% ~/compsci/.git/hooks/post-checkout
% 3. If you’re using a unix-like system, don’t forget to make the file executable.
% Just how you do this is outside the scope of this manual, but one
% possible way is with commands such as this:
% chmod g+x post-checkout.
% 4. Test your setup with “git checkout master” (or another suitable branch
% name). This should generate copies of gitHeadInfo.gin in the directories
% you intended.
% 5. Now make two more copies of this file in the same directory (hooks),
% calling them post-commit and post-merge, and you’re done. As before,
% users of unix-like systems should ensure these files are marked as
% executable.
\ifnumequal{\value{draft}}{1}{% Черновик
   \IfFileExists{.git/gitHeadInfo.gin}{
      \usepackage[mark,pcount]{gitinfo2}
      \renewcommand{\gitMark}{rev.\gitAbbrevHash\quad\gitCommitterEmail\quad\gitAuthorIsoDate}
      \renewcommand{\gitMarkFormat}{\rmfamily\color{Gray}\small\bfseries}
   }{}
}{}

\newcommand{\astfootnote}[1]{
	\let\oldthefootnote=\thefootnote
	\setcounter{footnote}{0}
	\renewcommand{\thefootnote}{\fnsymbol{footnote}}
	\footnote{#1}
	\let\thefootnote=\oldthefootnote
}

\newcommand\extrafootertext[1]{
    \bgroup
    \renewcommand\thefootnote{\fnsymbol{footnote}}
    \renewcommand\thempfootnote{\fnsymbol{mpfootnote}}
    \footnotetext[0]{#1}
    \egroup
}
   % Пакеты для специфических пользовательских задач

\input{Dissertation/setup}      % Упрощённые настройки шаблона

% Новые переменные, которые могут использоваться во всём проекте
% ГОСТ 7.0.11-2011
% 9.2 Оформление текста автореферата диссертации
% 9.2.1 Общая характеристика работы включает в себя следующие основные структурные
% элементы:
% актуальность темы исследования;
\newcommand{\actualityTXT}{Актуальность темы.}
% степень ее разработанности;
\newcommand{\progressTXT}{Степень разработанности темы.}
% цели и задачи;
\newcommand{\aimTXT}{Целью}
\newcommand{\tasksTXT}{задачи}
% научную новизну;
\newcommand{\noveltyTXT}{Научная новизна:}
% теоретическую и практическую значимость работы;
%\newcommand{\influenceTXT}{Теоретическая и практическая значимость}
% или чаще используют просто
\newcommand{\influenceTXT}{Практическая значимость}
% методологию и методы исследования;
\newcommand{\methodsTXT}{Методология и методы исследования.}
% положения, выносимые на защиту;
\newcommand{\defpositionsTXT}{Основные положения, выносимые на~защиту:}
% степень достоверности и апробацию результатов.
\newcommand{\reliabilityTXT}{Достоверность}
\newcommand{\probationTXT}{Апробация работы.}

\newcommand{\contributionTXT}{Личный вклад.}
\newcommand{\publicationsTXT}{Публикации.}


%%% Заголовки библиографии:

% для автореферата:
\newcommand{\bibtitleauthor}{\begin{center}Публикации автора по теме диссертации\\ в изданиях, индексируемых в базах данных\\Web of Science, Scopus, RSCI и РИНЦ\end{center}}

% для стиля библиографии `\insertbiblioauthorgrouped`
\newcommand{\bibtitleauthorvak}{В изданиях из списка ВАК РФ}
\newcommand{\bibtitleauthorscopus}{В изданиях, входящих в международную базу цитирования Scopus}
\newcommand{\bibtitleauthorwos}{В изданиях, входящих в международную базу цитирования Web of Science}
\newcommand{\bibtitleauthorother}{В прочих изданиях}
\newcommand{\bibtitleauthorconf}{В сборниках трудов конференций}
\newcommand{\bibtitleauthorpatent}{Зарегистрированные патенты}
\newcommand{\bibtitleauthorprogram}{Зарегистрированные программы для ЭВМ}

% для стиля библиографии `\insertbiblioauthorimportant`:
\newcommand{\bibtitleauthorimportant}{Наиболее значимые \protect\MakeLowercase\bibtitleauthor}

% для списка литературы в диссертации и списка чужих работ в автореферате:
\newcommand{\bibtitlefull}{Список литературы} % (ГОСТ Р 7.0.11-2011, 4)
         % Новые переменные, для всего проекта

%%% Основные сведения %%%
\newcommand{\thesisAuthorLastName}{Кулагин}
\newcommand{\thesisAuthorOtherNames}{Алексей Владимирович}
\newcommand{\thesisAuthorInitials}{\fixme{И.\,О.}}
\newcommand{\thesisAuthor}             % Диссертация, ФИО автора
{%
    \texorpdfstring{% \texorpdfstring takes two arguments and uses the first for (La)TeX and the second for pdf
        \thesisAuthorLastName~\thesisAuthorOtherNames% так будет отображаться на титульном листе или в тексте, где будет использоваться переменная
    }{%
        \thesisAuthorLastName, \thesisAuthorOtherNames% эта запись для свойств pdf-файла. В таком виде, если pdf будет обработан программами для сбора библиографических сведений, будет правильно представлена фамилия.
    }
}
\newcommand{\thesisAuthorShort}        % Диссертация, ФИО автора инициалами
{\thesisAuthorInitials~\thesisAuthorLastName}
%\newcommand{\thesisUdk}                % Диссертация, УДК
%{\fixme{xxx.xxx}}
\newcommand{\thesisTitle}              % Диссертация, название
{\fixme{Длинное название диссертационной работы, состоящее из~достаточно большого
количества слов, совсем длинное длинное длинное длинное название, из~которого
простому обывателю знакомы, в~лучшем случае, лишь отдельные слова}}
\newcommand{\thesisSpecialtyNumber}    % Диссертация, специальность, номер
{\fixme{XX.XX.XX}}
\newcommand{\thesisSpecialtyTitle}     % Диссертация, специальность, название (название взято с сайта ВАК для примера)
{\fixme{Технология обработки, хранения и~переработки злаковых, бобовых культур,
крупяных продуктов, плодоовощной продукции и~виноградарства}}
%% \newcommand{\thesisSpecialtyTwoNumber} % Диссертация, вторая специальность, номер
%% {\fixme{XX.XX.XX}}
%% \newcommand{\thesisSpecialtyTwoTitle}  % Диссертация, вторая специальность, название
%% {\fixme{Теория и~методика физического воспитания, спортивной тренировки,
%% оздоровительной и~адаптивной физической культуры}}
\newcommand{\thesisDegree}             % Диссертация, ученая степень
{кандидата физико-математических наук}
\newcommand{\thesisDegreeShort}        % Диссертация, ученая степень, краткая запись
{\fixme{канд. физ.-мат. наук}}
\newcommand{\thesisCity}               % Диссертация, город написания диссертации
{Москва}
\newcommand{\thesisYear}               % Диссертация, год написания диссертации
{\the\year}
\newcommand{\thesisOrganization}       % Диссертация, организация
{\fixme{Федеральное государственное автономное образовательное учреждение высшего
образования <<Длинное название образовательного учреждения <<АББРЕВИАТУРА>>}}
\newcommand{\thesisOrganizationShort}  % Диссертация, краткое название организации для доклада
{\fixme{НазУчДисРаб}}

\newcommand{\thesisInOrganization}     % Диссертация, организация в предложном падеже: Работа выполнена в ...
{\fixme{учреждении с~длинным длинным длинным длинным названием, в~котором
выполнялась данная диссертационная работа}}

%% \newcommand{\supervisorDead}{}           % Рисовать рамку вокруг фамилии
\newcommand{\supervisorFio}              % Научный руководитель, ФИО
{Ожигов Юрий Игоревич}
\newcommand{\supervisorRegalia}          % Научный руководитель, регалии
{доктор физико-математических наук, профессор}
\newcommand{\supervisorFioShort}         % Научный руководитель, ФИО
{\fixme{И.\,О.~Фамилия}}
\newcommand{\supervisorRegaliaShort}     % Научный руководитель, регалии
{\fixme{уч.~ст.,~уч.~зв.}}

%% \newcommand{\supervisorTwoDead}{}        % Рисовать рамку вокруг фамилии
%% \newcommand{\supervisorTwoFio}           % Второй научный руководитель, ФИО
%% {\fixme{Фамилия Имя Отчество}}
%% \newcommand{\supervisorTwoRegalia}       % Второй научный руководитель, регалии
%% {\fixme{уч. степень, уч. звание}}
%% \newcommand{\supervisorTwoFioShort}      % Второй научный руководитель, ФИО
%% {\fixme{И.\,О.~Фамилия}}
%% \newcommand{\supervisorTwoRegaliaShort}  % Второй научный руководитель, регалии
%% {\fixme{уч.~ст.,~уч.~зв.}}

\newcommand{\opponentOneFio}           % Оппонент 1, ФИО
{\fixme{Фамилия Имя Отчество}}
\newcommand{\opponentOneRegalia}       % Оппонент 1, регалии
{\fixme{доктор физико-математических наук, профессор}}
\newcommand{\opponentOneJobPlace}      % Оппонент 1, место работы
{\fixme{Не очень длинное название для места работы}}
\newcommand{\opponentOneJobPost}       % Оппонент 1, должность
{\fixme{старший научный сотрудник}}

\newcommand{\opponentTwoFio}           % Оппонент 2, ФИО
{\fixme{Фамилия Имя Отчество}}
\newcommand{\opponentTwoRegalia}       % Оппонент 2, регалии
{\fixme{кандидат физико-математических наук}}
\newcommand{\opponentTwoJobPlace}      % Оппонент 2, место работы
{\fixme{Основное место работы c длинным длинным длинным длинным названием}}
\newcommand{\opponentTwoJobPost}       % Оппонент 2, должность
{\fixme{старший научный сотрудник}}

%% \newcommand{\opponentThreeFio}         % Оппонент 3, ФИО
%% {\fixme{Фамилия Имя Отчество}}
%% \newcommand{\opponentThreeRegalia}     % Оппонент 3, регалии
%% {\fixme{кандидат физико-математических наук}}
%% \newcommand{\opponentThreeJobPlace}    % Оппонент 3, место работы
%% {\fixme{Основное место работы c длинным длинным длинным длинным названием}}
%% \newcommand{\opponentThreeJobPost}     % Оппонент 3, должность
%% {\fixme{старший научный сотрудник}}

\newcommand{\leadingOrganizationTitle} % Ведущая организация, дополнительные строки. Удалить, чтобы не отображать в автореферате
{\fixme{Федеральное государственное бюджетное образовательное учреждение высшего
профессионального образования с~длинным длинным длинным длинным названием}}

\newcommand{\defenseDate}              % Защита, дата
{\fixme{DD mmmmmmmm YYYY~г.~в~XX часов}}
\newcommand{\defenseCouncilNumber}     % Защита, номер диссертационного совета
{\fixme{Д\,123.456.78}}
\newcommand{\defenseCouncilTitle}      % Защита, учреждение диссертационного совета
{\fixme{Название учреждения}}
\newcommand{\defenseCouncilAddress}    % Защита, адрес учреждение диссертационного совета
{\fixme{Адрес}}
\newcommand{\defenseCouncilPhone}      % Телефон для справок
{\fixme{+7~(0000)~00-00-00}}

\newcommand{\defenseSecretaryFio}      % Секретарь диссертационного совета, ФИО
{\fixme{Фамилия Имя Отчество}}
\newcommand{\defenseSecretaryRegalia}  % Секретарь диссертационного совета, регалии
{\fixme{д-р~физ.-мат. наук}}            % Для сокращений есть ГОСТы, например: ГОСТ Р 7.0.12-2011 + http://base.garant.ru/179724/#block_30000

\newcommand{\synopsisLibrary}          % Автореферат, название библиотеки
{\fixme{Название библиотеки}}
\newcommand{\synopsisDate}             % Автореферат, дата рассылки
{\fixme{DD mmmmmmmm}\the\year~года}

% To avoid conflict with beamer class use \providecommand
\providecommand{\keywords}%            % Ключевые слова для метаданных PDF диссертации и автореферата
{}
             % Основные сведения
\input{common/fonts}            % Определение шрифтов (частичное)
\input{common/styles}           % Стили общие для диссертации и автореферата
\input{Dissertation/disstyles}  % Стили для диссертации
% для вертикального центрирования ячеек в tabulary
\def\zz{\ifx\[$\else\aftergroup\zzz\fi}
%$ \] % <-- чиним подсветку синтаксиса в некоторых редакторах
\def\zzz{\setbox0\lastbox
\dimen0\dimexpr\extrarowheight + \ht0-\dp0\relax
\setbox0\hbox{\raise-.5\dimen0\box0}%
\ht0=\dimexpr\ht0+\extrarowheight\relax
\dp0=\dimexpr\dp0+\extrarowheight\relax
\box0
}

\lstdefinelanguage{Renhanced}%
{keywords={abbreviate,abline,abs,acos,acosh,action,add1,add,%
        aggregate,alias,Alias,alist,all,anova,any,aov,aperm,append,apply,%
        approx,approxfun,apropos,Arg,args,array,arrows,as,asin,asinh,%
        atan,atan2,atanh,attach,attr,attributes,autoload,autoloader,ave,%
        axis,backsolve,barplot,basename,besselI,besselJ,besselK,besselY,%
        beta,binomial,body,box,boxplot,break,browser,bug,builtins,bxp,by,%
        c,C,call,Call,case,cat,category,cbind,ceiling,character,char,%
        charmatch,check,chol,chol2inv,choose,chull,class,close,cm,codes,%
        coef,coefficients,co,col,colnames,colors,colours,commandArgs,%
        comment,complete,complex,conflicts,Conj,contents,contour,%
        contrasts,contr,control,helmert,contrib,convolve,cooks,coords,%
        distance,coplot,cor,cos,cosh,count,fields,cov,covratio,wt,CRAN,%
        create,crossprod,cummax,cummin,cumprod,cumsum,curve,cut,cycle,D,%
        data,dataentry,date,dbeta,dbinom,dcauchy,dchisq,de,debug,%
        debugger,Defunct,default,delay,delete,deltat,demo,de,density,%
        deparse,dependencies,Deprecated,deriv,description,detach,%
        dev2bitmap,dev,cur,deviance,off,prev,,dexp,df,dfbetas,dffits,%
        dgamma,dgeom,dget,dhyper,diag,diff,digamma,dim,dimnames,dir,%
        dirname,dlnorm,dlogis,dnbinom,dnchisq,dnorm,do,dotplot,double,%
        download,dpois,dput,drop,drop1,dsignrank,dt,dummy,dump,dunif,%
        duplicated,dweibull,dwilcox,dyn,edit,eff,effects,eigen,else,%
        emacs,end,environment,env,erase,eval,equal,evalq,example,exists,%
        exit,exp,expand,expression,External,extract,extractAIC,factor,%
        fail,family,fft,file,filled,find,fitted,fivenum,fix,floor,for,%
        For,formals,format,formatC,formula,Fortran,forwardsolve,frame,%
        frequency,ftable,ftable2table,function,gamma,Gamma,gammaCody,%
        gaussian,gc,gcinfo,gctorture,get,getenv,geterrmessage,getOption,%
        getwd,gl,glm,globalenv,gnome,GNOME,graphics,gray,grep,grey,grid,%
        gsub,hasTsp,hat,heat,help,hist,home,hsv,httpclient,I,identify,if,%
        ifelse,Im,image,\%in\%,index,influence,measures,inherits,install,%
        installed,integer,interaction,interactive,Internal,intersect,%
        inverse,invisible,IQR,is,jitter,kappa,kronecker,labels,lapply,%
        layout,lbeta,lchoose,lcm,legend,length,levels,lgamma,library,%
        licence,license,lines,list,lm,load,local,locator,log,log10,log1p,%
        log2,logical,loglin,lower,lowess,ls,lsfit,lsf,ls,machine,Machine,%
        mad,mahalanobis,make,link,margin,match,Math,matlines,mat,matplot,%
        matpoints,matrix,max,mean,median,memory,menu,merge,methods,min,%
        missing,Mod,mode,model,response,mosaicplot,mtext,mvfft,na,nan,%
        names,omit,nargs,nchar,ncol,NCOL,new,next,NextMethod,nextn,%
        nlevels,nlm,noquote,NotYetImplemented,NotYetUsed,nrow,NROW,null,%
        numeric,\%o\%,objects,offset,old,on,Ops,optim,optimise,optimize,%
        options,or,order,ordered,outer,package,packages,page,pairlist,%
        pairs,palette,panel,par,parent,parse,paste,path,pbeta,pbinom,%
        pcauchy,pchisq,pentagamma,persp,pexp,pf,pgamma,pgeom,phyper,pico,%
        pictex,piechart,Platform,plnorm,plogis,plot,pmatch,pmax,pmin,%
        pnbinom,pnchisq,pnorm,points,poisson,poly,polygon,polyroot,pos,%
        postscript,power,ppoints,ppois,predict,preplot,pretty,Primitive,%
        print,prmatrix,proc,prod,profile,proj,prompt,prop,provide,%
        psignrank,ps,pt,ptukey,punif,pweibull,pwilcox,q,qbeta,qbinom,%
        qcauchy,qchisq,qexp,qf,qgamma,qgeom,qhyper,qlnorm,qlogis,qnbinom,%
        qnchisq,qnorm,qpois,qqline,qqnorm,qqplot,qr,Q,qty,qy,qsignrank,%
        qt,qtukey,quantile,quasi,quit,qunif,quote,qweibull,qwilcox,%
        rainbow,range,rank,rbeta,rbind,rbinom,rcauchy,rchisq,Re,read,csv,%
        csv2,fwf,readline,socket,real,Recall,rect,reformulate,regexpr,%
        relevel,remove,rep,repeat,replace,replications,report,require,%
        resid,residuals,restart,return,rev,rexp,rf,rgamma,rgb,rgeom,R,%
        rhyper,rle,rlnorm,rlogis,rm,rnbinom,RNGkind,rnorm,round,row,%
        rownames,rowsum,rpois,rsignrank,rstandard,rstudent,rt,rug,runif,%
        rweibull,rwilcox,sample,sapply,save,scale,scan,scan,screen,sd,se,%
        search,searchpaths,segments,seq,sequence,setdiff,setequal,set,%
        setwd,show,sign,signif,sin,single,sinh,sink,solve,sort,source,%
        spline,splinefun,split,sqrt,stars,start,stat,stem,step,stop,%
        storage,strstrheight,stripplot,strsplit,structure,strwidth,sub,%
        subset,substitute,substr,substring,sum,summary,sunflowerplot,svd,%
        sweep,switch,symbol,symbols,symnum,sys,status,system,t,table,%
        tabulate,tan,tanh,tapply,tempfile,terms,terrain,tetragamma,text,%
        time,title,topo,trace,traceback,transform,tri,trigamma,trunc,try,%
        ts,tsp,typeof,unclass,undebug,undoc,union,unique,uniroot,unix,%
        unlink,unlist,unname,untrace,update,upper,url,UseMethod,var,%
        variable,vector,Version,vi,warning,warnings,weighted,weights,%
        which,while,window,write,\%x\%,x11,X11,xedit,xemacs,xinch,xor,%
        xpdrows,xy,xyinch,yinch,zapsmall,zip},%
    otherkeywords={!,!=,~,$,*,\%,\&,\%/\%,\%*\%,\%\%,<-,<<-},%$
    alsoother={._$},%$
    sensitive,%
    morecomment=[l]\#,%
    morestring=[d]",%
    morestring=[d]'% 2001 Robert Denham
}%

%решаем проблему с кириллицей в комментариях (в pdflatex) https://tex.stackexchange.com/a/103712
\lstset{extendedchars=true,keepspaces=true,literate={Ö}{{\"O}}1
    {Ä}{{\"A}}1
    {Ü}{{\"U}}1
    {ß}{{\ss}}1
    {ü}{{\"u}}1
    {ä}{{\"a}}1
    {ö}{{\"o}}1
    {~}{{\textasciitilde}}1
    {а}{{\selectfont\char224}}1
    {б}{{\selectfont\char225}}1
    {в}{{\selectfont\char226}}1
    {г}{{\selectfont\char227}}1
    {д}{{\selectfont\char228}}1
    {е}{{\selectfont\char229}}1
    {ё}{{\"e}}1
    {ж}{{\selectfont\char230}}1
    {з}{{\selectfont\char231}}1
    {и}{{\selectfont\char232}}1
    {й}{{\selectfont\char233}}1
    {к}{{\selectfont\char234}}1
    {л}{{\selectfont\char235}}1
    {м}{{\selectfont\char236}}1
    {н}{{\selectfont\char237}}1
    {о}{{\selectfont\char238}}1
    {п}{{\selectfont\char239}}1
    {р}{{\selectfont\char240}}1
    {с}{{\selectfont\char241}}1
    {т}{{\selectfont\char242}}1
    {у}{{\selectfont\char243}}1
    {ф}{{\selectfont\char244}}1
    {х}{{\selectfont\char245}}1
    {ц}{{\selectfont\char246}}1
    {ч}{{\selectfont\char247}}1
    {ш}{{\selectfont\char248}}1
    {щ}{{\selectfont\char249}}1
    {ъ}{{\selectfont\char250}}1
    {ы}{{\selectfont\char251}}1
    {ь}{{\selectfont\char252}}1
    {э}{{\selectfont\char253}}1
    {ю}{{\selectfont\char254}}1
    {я}{{\selectfont\char255}}1
    {А}{{\selectfont\char192}}1
    {Б}{{\selectfont\char193}}1
    {В}{{\selectfont\char194}}1
    {Г}{{\selectfont\char195}}1
    {Д}{{\selectfont\char196}}1
    {Е}{{\selectfont\char197}}1
    {Ё}{{\"E}}1
    {Ж}{{\selectfont\char198}}1
    {З}{{\selectfont\char199}}1
    {И}{{\selectfont\char200}}1
    {Й}{{\selectfont\char201}}1
    {К}{{\selectfont\char202}}1
    {Л}{{\selectfont\char203}}1
    {М}{{\selectfont\char204}}1
    {Н}{{\selectfont\char205}}1
    {О}{{\selectfont\char206}}1
    {П}{{\selectfont\char207}}1
    {Р}{{\selectfont\char208}}1
    {С}{{\selectfont\char209}}1
    {Т}{{\selectfont\char210}}1
    {У}{{\selectfont\char211}}1
    {Ф}{{\selectfont\char212}}1
    {Х}{{\selectfont\char213}}1
    {Ц}{{\selectfont\char214}}1
    {Ч}{{\selectfont\char215}}1
    {Ш}{{\selectfont\char216}}1
    {Щ}{{\selectfont\char217}}1
    {Ъ}{{\selectfont\char218}}1
    {Ы}{{\selectfont\char219}}1
    {Ь}{{\selectfont\char220}}1
    {Э}{{\selectfont\char221}}1
    {Ю}{{\selectfont\char222}}1
    {Я}{{\selectfont\char223}}1
    {і}{{\selectfont\char105}}1
    {ї}{{\selectfont\char168}}1
    {є}{{\selectfont\char185}}1
    {ґ}{{\selectfont\char160}}1
    {І}{{\selectfont\char73}}1
    {Ї}{{\selectfont\char136}}1
    {Є}{{\selectfont\char153}}1
    {Ґ}{{\selectfont\char128}}1
}

% Ширина текста минус ширина надписи 999
\newlength{\twless}
\newlength{\lmarg}
\setlength{\lmarg}{\widthof{999}}   % ширина надписи 999
\setlength{\twless}{\textwidth-\lmarg}

\definecolor{lightgray}{rgb}{0.98,0.98,0.98}
\definecolor{gray}{rgb}{0.5,0.5,0.5}
\definecolor{green}{rgb}{0,0.6,0}
\definecolor{violet}{rgb}{0.79216,0.55686,0.77255}

\lstset{
	backgroundcolor=\color{lightgray},
	basicstyle=\fontsize{10pt}{12pt}\selectfont\ttfamily,
	breakatwhitespace=false,
	breaklines=true,
	captionpos=b,
	commentstyle=\color{gray},
	keepspaces=true,
	keywordstyle=\color{violet},
	numbers=left,
	numbersep=5pt,
	numberstyle=\tiny\color{gray},
	showspaces=false,
	showstringspaces=false,
	showtabs=false,
	stringstyle=\color{green},
	tabsize=2,
	xleftmargin={\lmarg}
}

%http://tex.stackexchange.com/questions/26872/smaller-frame-with-listings
% Окружение, чтобы листинг был компактнее обведен рамкой, если она задается, а не на всю ширину текста
\makeatletter
\newenvironment{SmallListing}[1][]
{\lstset{#1}\VerbatimEnvironment\begin{VerbatimOut}{VerbEnv.tmp}}
{\end{VerbatimOut}\settowidth\@tempdima{%
        \lstinputlisting{VerbEnv.tmp}}
    \minipage{\@tempdima}\lstinputlisting{VerbEnv.tmp}\endminipage}
\makeatother

\DefineVerbatimEnvironment% с шрифтом 12 пт
{Verb}{Verbatim}
{fontsize=\fontsize{12pt}{14pt}\selectfont}

\newfloat[chapter]{ListingEnv}{lol}{Листинг}

\renewcommand{\lstlistingname}{Листинг}

%Общие счётчики окружений листингов
%http://tex.stackexchange.com/questions/145546/how-to-make-figure-and-listing-share-their-counter
% Если смешивать плавающие и не плавающие окружения, то могут быть проблемы с нумерацией
\makeatletter
\AfterEndPreamble{% https://tex.stackexchange.com/a/252682
    \let\c@ListingEnv\relax % drop existing counter "ListingEnv"
    \newaliascnt{ListingEnv}{lstlisting} % команда требует пакет aliascnt
    \let\ftype@lstlisting\ftype@ListingEnv % give the floats the same precedence
}
\makeatother

% значок С++ — используйте команду \cpp
\newcommand{\cpp}{%
    C\nolinebreak\hspace{-.05em}%
    \raisebox{.2ex}{+}\nolinebreak\hspace{-.10em}%
    \raisebox{.2ex}{+}%
}

%%%  Чересстрочное форматирование таблиц
%% http://tex.stackexchange.com/questions/278362/apply-italic-formatting-to-every-other-row
\newcounter{rowcnt}
\newcommand\altshape{\ifnumodd{\value{rowcnt}}{\color{red}}{\vspace*{-1ex}\itshape}}
% \AtBeginEnvironment{tabular}{\setcounter{rowcnt}{1}}
% \AtEndEnvironment{tabular}{\setcounter{rowcnt}{0}}

%%% Ради примера во второй главе
\let\originalepsilon\epsilon
\let\originalphi\phi
\let\originalkappa\kappa
\let\originalle\le
\let\originalleq\leq
\let\originalge\ge
\let\originalgeq\geq
\let\originalemptyset\emptyset
\let\originaltan\tan
\let\originalcot\cot
\let\originalcsc\csc

%%% Русская традиция начертания математических знаков
\renewcommand{\le}{\ensuremath{\leqslant}}
\renewcommand{\leq}{\ensuremath{\leqslant}}
\renewcommand{\ge}{\ensuremath{\geqslant}}
\renewcommand{\geq}{\ensuremath{\geqslant}}
\renewcommand{\emptyset}{\varnothing}

%%% Русская традиция начертания математических функций (на случай копирования из зарубежных источников)
\renewcommand{\tan}{\operatorname{tg}}
\renewcommand{\cot}{\operatorname{ctg}}
\renewcommand{\csc}{\operatorname{cosec}}

%%% Русская традиция начертания греческих букв (греческие буквы вертикальные, через пакет upgreek)
\renewcommand{\epsilon}{\ensuremath{\upvarepsilon}}   %  русская традиция записи
\renewcommand{\phi}{\ensuremath{\upvarphi}}
%\renewcommand{\kappa}{\ensuremath{\varkappa}}
%\renewcommand{\alpha}{\upalpha}
%\renewcommand{\beta}{\upbeta}
\renewcommand{\gamma}{\upgamma}
\renewcommand{\delta}{\updelta}
\renewcommand{\varepsilon}{\upvarepsilon}
\renewcommand{\zeta}{\upzeta}
\renewcommand{\eta}{\upeta}
\renewcommand{\theta}{\uptheta}
\renewcommand{\vartheta}{\upvartheta}
\renewcommand{\iota}{\upiota}
\renewcommand{\kappa}{\upkappa}
\renewcommand{\lambda}{\uplambda}
\renewcommand{\mu}{\upmu}
\renewcommand{\nu}{\upnu}
\renewcommand{\xi}{\upxi}
\renewcommand{\pi}{\uppi}
\renewcommand{\varpi}{\upvarpi}
%\renewcommand{\rho}{\uprho}
%\renewcommand{\varrho}{\upvarrho}
%\renewcommand{\sigma}{\upsigma}
%\renewcommand{\varsigma}{\upvarsigma}
\renewcommand{\tau}{\uptau}
\renewcommand{\upsilon}{\upupsilon}
\renewcommand{\varphi}{\upvarphi}
\renewcommand{\chi}{\upchi}
\renewcommand{\psi}{\uppsi}
\renewcommand{\omega}{\upomega}

\newcommand{\s}{\sigma}
\newcommand{\w}{\omega}
\newcommand{\la}{\langle}
\newcommand{\ra}{\rangle}
\newcommand{\Prec}{\mathlarger{\mathlarger{\prec}}}
\newcommand{\Succ}{\mathlarger{\mathlarger{\succ}}}

\newtheorem{hyp}{Гипотеза}

\newcommand\seminormalsize{\fontsize{13.5}{16}\selectfont}
 % Стили для специфических пользовательских задач

%%% Библиография. Выбор движка для реализации %%%
% Здесь только проверка установленного ключа. Сама настройка выбора движка
% размещена в common/setup.tex
\ifnumequal{\value{bibliosel}}{0}{%
    \input{biblio/predefined}   % Встроенная реализация с загрузкой файла через движок bibtex8
}{
    %%% Реализация библиографии пакетами biblatex и biblatex-gost с использованием движка biber %%%

\usepackage{csquotes} % biblatex рекомендует его подключать. Пакет для оформления сложных блоков цитирования.
%%% Загрузка пакета с основными настройками %%%
\makeatletter
\ifnumequal{\value{draft}}{0}{% Чистовик
\usepackage[%
backend=biber,% движок
bibencoding=utf8,% кодировка bib файла
sorting=none,% настройка сортировки списка литературы
style=gost-numeric,% стиль цитирования и библиографии (по ГОСТ)
language=autobib,% получение языка из babel/polyglossia, default: autobib % если ставить autocite или auto, то цитаты в тексте с указанием страницы, получат указание страницы на языке оригинала
autolang=other,% многоязычная библиография
clearlang=true,% внутренний сброс поля language, если он совпадает с языком из babel/polyglossia
defernumbers=true,% нумерация проставляется после двух компиляций, зато позволяет выцеплять библиографию по ключевым словам и нумеровать не из большего списка
sortcites=true,% сортировать номера затекстовых ссылок при цитировании (если в квадратных скобках несколько ссылок, то отображаться будут отсортированно, а не абы как)
doi=false,% Показывать или нет ссылки на DOI
isbn=false,% Показывать или нет ISBN, ISSN, ISRN
]{biblatex}[2016/09/17]
\ltx@iffilelater{biblatex-gost.def}{2017/05/03}%
{\toggletrue{bbx:gostbibliography}%
\renewcommand*{\revsdnamepunct}{\addcomma}}{}
}{%Черновик
\usepackage[%
backend=biber,% движок
bibencoding=utf8,% кодировка bib файла
sorting=none,% настройка сортировки списка литературы
% defernumbers=true, % откомментируйте, если требуется правильная нумерация ссылок на литературу в режиме черновика. Замедляет сборку
]{biblatex}[2016/09/17]%
}
\makeatother

\providebool{blxmc} % biblatex version needs and has MakeCapital workaround
\boolfalse{blxmc} % setting our new boolean flag to default false
\ifxetexorluatex
\else
% Исправление случая неподдержки знака номера в pdflatex
    \DefineBibliographyStrings{russian}{number={\textnumero}}

% Исправление случая отсутствия прописных букв в некоторых случаях
% https://github.com/plk/biblatex/issues/960#issuecomment-596658282
    \ifdefmacro{\ExplSyntaxOn}{}{\usepackage{expl3}}
    \makeatletter
    \ltx@ifpackagelater{biblatex}{2020/02/23}{
    % Assuming this version of biblatex defines MakeCapital correctly
    }{
        \ltx@ifpackagelater{biblatex}{2019/12/01}{
            % Assuming this version of biblatex defines MakeCapital incorrectly
            \usepackage{expl3}[2020/02/25]
            \@ifpackagelater{expl3}{2020/02/25}{
                \booltrue{blxmc} % setting our new boolean flag to true
            }{}
        }{}
    }
    \makeatother
    \ifblxmc
        \typeout{Assuming this version of biblatex defines MakeCapital
        incorrectly}
        \usepackage{xparse}
        \makeatletter
        \ExplSyntaxOn
        \NewDocumentCommand \blx@maketext@lowercase {m}
          {
            \text_lowercase:n {#1}
          }

        \NewDocumentCommand \blx@maketext@uppercase {m}
          {
            \text_uppercase:n {#1}
          }

        \RenewDocumentCommand \MakeCapital {m}
          {
            \text_titlecase_first:n {#1}
          }
        \ExplSyntaxOff

        \protected\def\blx@biblcstring#1#2#3{%
          \blx@begunit
          \blx@hyphenreset
          \blx@bibstringsimple
          \lowercase{\edef\blx@tempa{#3}}%
          \ifcsundef{#2@\blx@tempa}
            {\blx@warn@nostring\blx@tempa
             \blx@endnounit}
            {#1{\blx@maketext@lowercase{\csuse{#2@\blx@tempa}}}%
             \blx@endunit}}

        \protected\def\blx@bibucstring#1#2#3{%
          \blx@begunit
          \blx@hyphenreset
          \blx@bibstringsimple
          \lowercase{\edef\blx@tempa{#3}}%
          \ifcsundef{#2@\blx@tempa}
            {\blx@warn@nostring\blx@tempa
             \blx@endnounit}
            {#1{\blx@maketext@uppercase{\csuse{#2@\blx@tempa}}}%
             \blx@endunit}}
        \makeatother
    \fi
\fi

\ifsynopsis
\ifnumgreater{\value{usefootcite}}{0}{
    \ExecuteBibliographyOptions{autocite=footnote}
    \newbibmacro*{cite:full}{%
        \printtext[bibhypertarget]{%
            \usedriver{%
                \DeclareNameAlias{sortname}{default}%
            }{%
                \thefield{entrytype}%
            }%
        }%
        \usebibmacro{shorthandintro}%
    }
    \DeclareCiteCommand{\smartcite}[\mkbibfootnote]{%
        \usebibmacro{prenote}%
    }{%
        \usebibmacro{citeindex}%
        \usebibmacro{cite:full}%
    }{%
        \multicitedelim%
    }{%
        \usebibmacro{postnote}%
    }
}{}
\fi

%%% Подключение файлов bib %%%
\addbibresource[label=bl-external]{biblio/external.bib}
\addbibresource[label=bl-author]{biblio/author.bib}
%\addbibresource[label=bl-registered]{biblio/registered.bib}

%http://tex.stackexchange.com/a/141831/79756
%There is a way to automatically map the language field to the langid field. The following lines in the preamble should be enough to do that.
%This command will copy the language field into the langid field and will then delete the contents of the language field. The language field will only be deleted if it was successfully copied into the langid field.
\DeclareSourcemap{ %модификация bib файла перед тем, как им займётся biblatex
    \maps{
        \map{% перекидываем значения полей language в поля langid, которыми пользуется biblatex
            \step[fieldsource=language, fieldset=langid, origfieldval, final]
            \step[fieldset=language, null]
        }
        \map{% перекидываем значения полей numpages в поля pagetotal, которыми пользуется biblatex
            \step[fieldsource=numpages, fieldset=pagetotal, origfieldval, final]
            \step[fieldset=numpages, null]
        }
        \map{% перекидываем значения полей pagestotal в поля pagetotal, которыми пользуется biblatex
            \step[fieldsource=pagestotal, fieldset=pagetotal, origfieldval, final]
            \step[fieldset=pagestotal, null]
        }
        \map[overwrite]{% перекидываем значения полей shortjournal, если они есть, в поля journal, которыми пользуется biblatex
            \step[fieldsource=shortjournal, final]
            \step[fieldset=journal, origfieldval]
            \step[fieldset=shortjournal, null]
        }
        \map[overwrite]{% перекидываем значения полей shortbooktitle, если они есть, в поля booktitle, которыми пользуется biblatex
            \step[fieldsource=shortbooktitle, final]
            \step[fieldset=booktitle, origfieldval]
            \step[fieldset=shortbooktitle, null]
        }
        \map{% если в поле medium написано "Электронный ресурс", то устанавливаем поле media, которым пользуется biblatex, в значение eresource.
            \step[fieldsource=medium,
            match=\regexp{Электронный\s+ресурс},
            final]
            \step[fieldset=media, fieldvalue=eresource]
            \step[fieldset=medium, null]
        }
        \map[overwrite]{% стираем значения всех полей issn
            \step[fieldset=issn, null]
        }
        \map[overwrite]{% стираем значения всех полей abstract, поскольку ими не пользуемся, а там бывают "неприятные" латеху символы
            \step[fieldsource=abstract]
            \step[fieldset=abstract,null]
        }
        \map[overwrite]{ % переделка формата записи даты
            \step[fieldsource=urldate,
            match=\regexp{([0-9]{2})\.([0-9]{2})\.([0-9]{4})},
            replace={$3-$2-$1$4}, % $4 вставлен исключительно ради нормальной работы программ подсветки синтаксиса, которые некорректно обрабатывают $ в таких конструкциях
            final]
        }
        \map[overwrite]{ % стираем ключевые слова
            \step[fieldsource=keywords]
            \step[fieldset=keywords,null]
        }
        % реализация foreach различается для biblatex v3.12 и v3.13.
        % Для версии v3.13 эта конструкция заменяет последующие 7 структур map
        % \map[overwrite,foreach={authorvak,authorscopus,authorwos,authorconf,authorother,authorparent,authorprogram}]{ % записываем информацию о типе публикации в ключевые слова
        %     \step[fieldsource=$MAPLOOP,final=true]
        %     \step[fieldset=keywords,fieldvalue={,biblio$MAPLOOP},append=true]
        % }
        \map[overwrite]{ % записываем информацию о типе публикации в ключевые слова
            \step[fieldsource=authorvak,final=true]
            \step[fieldset=keywords,fieldvalue={,biblioauthorvak},append=true]
        }
        \map[overwrite]{ % записываем информацию о типе публикации в ключевые слова
            \step[fieldsource=authorscopus,final=true]
            \step[fieldset=keywords,fieldvalue={,biblioauthorscopus},append=true]
        }
        \map[overwrite]{ % записываем информацию о типе публикации в ключевые слова
            \step[fieldsource=authorwos,final=true]
            \step[fieldset=keywords,fieldvalue={,biblioauthorwos},append=true]
        }
        \map[overwrite]{ % записываем информацию о типе публикации в ключевые слова
            \step[fieldsource=authorconf,final=true]
            \step[fieldset=keywords,fieldvalue={,biblioauthorconf},append=true]
        }
        \map[overwrite]{ % записываем информацию о типе публикации в ключевые слова
            \step[fieldsource=authorother,final=true]
            \step[fieldset=keywords,fieldvalue={,biblioauthorother},append=true]
        }
        \map[overwrite]{ % записываем информацию о типе публикации в ключевые слова
            \step[fieldsource=authorpatent,final=true]
            \step[fieldset=keywords,fieldvalue={,biblioauthorpatent},append=true]
        }
        \map[overwrite]{ % записываем информацию о типе публикации в ключевые слова
            \step[fieldsource=authorprogram,final=true]
            \step[fieldset=keywords,fieldvalue={,biblioauthorprogram},append=true]
        }
        \map[overwrite]{ % добавляем ключевые слова, чтобы различать источники
            \perdatasource{biblio/external.bib}
            \step[fieldset=keywords, fieldvalue={,biblioexternal},append=true]
        }
        \map[overwrite]{ % добавляем ключевые слова, чтобы различать источники
            \perdatasource{biblio/author.bib}
            \step[fieldset=keywords, fieldvalue={,biblioauthor},append=true]
        }
        \map[overwrite]{ % добавляем ключевые слова, чтобы различать источники
            \perdatasource{biblio/registered.bib}
            \step[fieldset=keywords, fieldvalue={,biblioregistered},append=true]
        }
        \map[overwrite]{ % добавляем ключевые слова, чтобы различать источники
            \step[fieldset=keywords, fieldvalue={,bibliofull},append=true]
        }
%        \map[overwrite]{% стираем значения всех полей series
%            \step[fieldset=series, null]
%        }
        \map[overwrite]{% перекидываем значения полей howpublished в поля organization для типа online
            \step[typesource=online, typetarget=online, final]
            \step[fieldsource=howpublished, fieldset=organization, origfieldval]
            \step[fieldset=howpublished, null]
        }
    }
}

\ifnumequal{\value{mediadisplay}}{1}{
    \DeclareSourcemap{
        \maps{%
            \map{% использование media=text по умолчанию
                \step[fieldset=media, fieldvalue=text]
            }
        }
    }
}{}
\ifnumequal{\value{mediadisplay}}{2}{
    \DeclareSourcemap{
        \maps{%
            \map[overwrite]{% удаление всех записей media
                \step[fieldset=media, null]
            }
        }
    }
}{}
\ifnumequal{\value{mediadisplay}}{3}{
    \DeclareSourcemap{
        \maps{
            \map[overwrite]{% стираем значения всех полей media=text
                \step[fieldsource=media,match={text},final]
                \step[fieldset=media, null]
            }
        }
    }
}{}
\ifnumequal{\value{mediadisplay}}{4}{
    \DeclareSourcemap{
        \maps{
            \map[overwrite]{% стираем значения всех полей media=eresource
                \step[fieldsource=media,match={eresource},final]
                \step[fieldset=media, null]
            }
        }
    }
}{}

\ifsynopsis
\else
\DeclareSourcemap{ %модификация bib файла перед тем, как им займётся biblatex
    \maps{
        \map[overwrite]{% стираем значения всех полей addendum
            \perdatasource{biblio/author.bib}
            \step[fieldset=addendum, null] %чтобы избавиться от информации об объёме авторских статей, в отличие от автореферата
        }
    }
}
\fi

\ifpresentation
% удаляем лишние поля в списке литературы презентации
% их названия можно узнать в файле presentation.bbl
\DeclareSourcemap{
    \maps{
    \map[overwrite,foreach={%
        % {{{ Список лишних полей в презентации
        address,%
        chapter,%
        edition,%
        editor,%
        eid,%
        howpublished,%
        institution,%
        key,%
        month,%
        note,%
        number,%
        organization,%
        pages,%
        publisher,%
        school,%
        series,%
        type,%
        media,%
        url,%
        doi,%
        location,%
        volume,%
        % Список лишних полей в презентации }}}
    }]{
        \perdatasource{biblio/author.bib}
        \step[fieldset=$MAPLOOP,null]
    }
    }
}
\fi

\defbibfilter{vakscopuswos}{%
    keyword=biblioauthorvak or keyword=biblioauthorscopus or keyword=biblioauthorwos
}

\defbibfilter{scopuswos}{%
    keyword=biblioauthorscopus or keyword=biblioauthorwos
}

\defbibfilter{papersregistered}{%
    keyword=biblioauthor or keyword=biblioregistered
}

%%% Убираем неразрывные пробелы перед двоеточием и точкой с запятой %%%
%\makeatletter
%\ifnumequal{\value{draft}}{0}{% Чистовик
%    \renewcommand*{\addcolondelim}{%
%      \begingroup%
%      \def\abx@colon{%
%        \ifdim\lastkern>\z@\unkern\fi%
%        \abx@puncthook{:}\space}%
%      \addcolon%
%      \endgroup}
%
%    \renewcommand*{\addsemicolondelim}{%
%      \begingroup%
%      \def\abx@semicolon{%
%        \ifdim\lastkern>\z@\unkern\fi%
%        \abx@puncthook{;}\space}%
%      \addsemicolon%
%      \endgroup}
%}{}
%\makeatother

%%% Правка записей типа thesis, чтобы дважды не писался автор
%\ifnumequal{\value{draft}}{0}{% Чистовик
%\DeclareBibliographyDriver{thesis}{%
%  \usebibmacro{bibindex}%
%  \usebibmacro{begentry}%
%  \usebibmacro{heading}%
%  \newunit
%  \usebibmacro{author}%
%  \setunit*{\labelnamepunct}%
%  \usebibmacro{thesistitle}%
%  \setunit{\respdelim}%
%  %\printnames[last-first:full]{author}%Вот эту строчку нужно убрать, чтобы автор диссертации не дублировался
%  \newunit\newblock
%  \printlist[semicolondelim]{specdata}%
%  \newunit
%  \usebibmacro{institution+location+date}%
%  \newunit\newblock
%  \usebibmacro{chapter+pages}%
%  \newunit
%  \printfield{pagetotal}%
%  \newunit\newblock
%  \usebibmacro{doi+eprint+url+note}%
%  \newunit\newblock
%  \usebibmacro{addendum+pubstate}%
%  \setunit{\bibpagerefpunct}\newblock
%  \usebibmacro{pageref}%
%  \newunit\newblock
%  \usebibmacro{related:init}%
%  \usebibmacro{related}%
%  \usebibmacro{finentry}}
%}{}

%\newbibmacro{string+doi}[1]{% новая макрокоманда на простановку ссылки на doi
%    \iffieldundef{doi}{#1}{\href{http://dx.doi.org/\thefield{doi}}{#1}}}

%\ifnumequal{\value{draft}}{0}{% Чистовик
%\renewcommand*{\mkgostheading}[1]{\usebibmacro{string+doi}{#1}} % ссылка на doi с авторов. стоящих впереди записи
%\renewcommand*{\mkgostheading}[1]{#1} % только лишь убираем курсив с авторов
%}{}
%\DeclareFieldFormat{title}{\usebibmacro{string+doi}{#1}} % ссылка на doi с названия работы
%\DeclareFieldFormat{journaltitle}{\usebibmacro{string+doi}{#1}} % ссылка на doi с названия журнала
%%% Тире как разделитель в библиографии традиционной руской длины:
\renewcommand*{\newblockpunct}{\addperiod\addnbspace\cyrdash\space\bibsentence}
%%% Убрать тире из разделителей элементов в библиографии:
%\renewcommand*{\newblockpunct}{%
%    \addperiod\space\bibsentence}%block punct.,\bibsentence is for vol,etc.
%%% Изменение точки с запятой на запятую в перечислении библиографических
%%% ссылок:
%\renewcommand*{\multicitedelim}{\addcomma\space}

%%% Возвращаем запись «Режим доступа» %%%
%\DefineBibliographyStrings{english}{%
%    urlfrom = {Mode of access}
%}
%\DeclareFieldFormat{url}{\bibstring{urlfrom}\addcolon\space\url{#1}}

%%% В списке литературы обозначение одной буквой диапазона страниц англоязычного источника %%%
\DefineBibliographyStrings{english}{%
    pages = {p\adddot} %заглавность буквы затем по месту определяется работой самого biblatex
}

%%% В ссылке на источник в основном тексте с указанием конкретной страницы обозначение одной большой буквой %%%
%\DefineBibliographyStrings{russian}{%
%    page = {C\adddot}
%}

%%% Исправление длины тире в диапазонах %%%
% \cyrdash --- тире «русской» длины, \textendash --- en-dash
\DefineBibliographyExtras{russian}{%
  \protected\def\bibrangedash{%
    \cyrdash\penalty\value{abbrvpenalty}}% almost unbreakable dash
  \protected\def\bibdaterangesep{\bibrangedash}%тире для дат
}
\DefineBibliographyExtras{english}{%
  \protected\def\bibrangedash{%
    \cyrdash\penalty\value{abbrvpenalty}}% almost unbreakable dash
  \protected\def\bibdaterangesep{\bibrangedash}%тире для дат
}

%Set higher penalty for breaking in number, dates and pages ranges
\setcounter{abbrvpenalty}{10000} % default is \hyphenpenalty which is 12

%Set higher penalty for breaking in names
\setcounter{highnamepenalty}{10000} % If you prefer the traditional BibTeX behavior (no linebreaks at highnamepenalty breakpoints), set it to ‘infinite’ (10 000 or higher).
\setcounter{lownamepenalty}{10000}

%%% Set low penalties for breaks at uppercase letters and lowercase letters
%\setcounter{biburllcpenalty}{500} %управляет разрывами ссылок после маленьких букв RTFM biburllcpenalty
%\setcounter{biburlucpenalty}{3000} %управляет разрывами ссылок после больших букв, RTFM biburlucpenalty

%%% Список литературы с красной строки (без висячего отступа) %%%
%\defbibenvironment{bibliography} % переопределяем окружение библиографии из gost-numeric.bbx пакета biblatex-gost
%  {\list
%     {\printtext[labelnumberwidth]{%
%       \printfield{prefixnumber}%
%       \printfield{labelnumber}}}
%     {%
%      \setlength{\labelwidth}{\labelnumberwidth}%
%      \setlength{\leftmargin}{0pt}% default is \labelwidth
%      \setlength{\labelsep}{\widthof{\ }}% Управляет длиной отступа после точки % default is \biblabelsep
%      \setlength{\itemsep}{\bibitemsep}% Управление дополнительным вертикальным разрывом между записями. \bibitemsep по умолчанию соответствует \itemsep списков в документе.
%      \setlength{\itemindent}{\bibhang}% Пользуемся тем, что \bibhang по умолчанию принимает значение \parindent (абзацного отступа), который переназначен в styles.tex
%      \addtolength{\itemindent}{\labelwidth}% Сдвигаем правее на величину номера с точкой
%      \addtolength{\itemindent}{\labelsep}% Сдвигаем ещё правее на отступ после точки
%      \setlength{\parsep}{\bibparsep}%
%     }%
%      \renewcommand*{\makelabel}[1]{\hss##1}%
%  }
%  {\endlist}
%  {\item}

%%% Макросы автоматического подсчёта количества авторских публикаций.
% Печатают невидимую (пустую) библиографию, считая количество источников.
% http://tex.stackexchange.com/a/66851/79756
%
\makeatletter
    \newtotcounter{citenum}
    \defbibenvironment{counter}
        {\setcounter{citenum}{0}\renewcommand{\blx@driver}[1]{}} % begin code: убирает весь выводимый текст
        {} % end code
        {\stepcounter{citenum}} % item code: cчитает "печатаемые в библиографию" источники

    \newtotcounter{citeauthorvak}
    \defbibenvironment{countauthorvak}
        {\setcounter{citeauthorvak}{0}\renewcommand{\blx@driver}[1]{}}
        {}
        {\stepcounter{citeauthorvak}}

    \newtotcounter{citeauthorscopus}
    \defbibenvironment{countauthorscopus}
        {\setcounter{citeauthorscopus}{0}\renewcommand{\blx@driver}[1]{}}
        {}
        {\stepcounter{citeauthorscopus}}

    \newtotcounter{citeauthorwos}
    \defbibenvironment{countauthorwos}
        {\setcounter{citeauthorwos}{0}\renewcommand{\blx@driver}[1]{}}
        {}
        {\stepcounter{citeauthorwos}}

    \newtotcounter{citeauthorother}
    \defbibenvironment{countauthorother}
        {\setcounter{citeauthorother}{0}\renewcommand{\blx@driver}[1]{}}
        {}
        {\stepcounter{citeauthorother}}

    \newtotcounter{citeauthorconf}
    \defbibenvironment{countauthorconf}
        {\setcounter{citeauthorconf}{0}\renewcommand{\blx@driver}[1]{}}
        {}
        {\stepcounter{citeauthorconf}}

    \newtotcounter{citeauthor}
    \defbibenvironment{countauthor}
        {\setcounter{citeauthor}{0}\renewcommand{\blx@driver}[1]{}}
        {}
        {\stepcounter{citeauthor}}

    \newtotcounter{citeauthorvakscopuswos}
    \defbibenvironment{countauthorvakscopuswos}
        {\setcounter{citeauthorvakscopuswos}{0}\renewcommand{\blx@driver}[1]{}}
        {}
        {\stepcounter{citeauthorvakscopuswos}}

    \newtotcounter{citeauthorscopuswos}
    \defbibenvironment{countauthorscopuswos}
        {\setcounter{citeauthorscopuswos}{0}\renewcommand{\blx@driver}[1]{}}
        {}
        {\stepcounter{citeauthorscopuswos}}

    \newtotcounter{citeregistered}
    \defbibenvironment{countregistered}
        {\setcounter{citeregistered}{0}\renewcommand{\blx@driver}[1]{}}
        {}
        {\stepcounter{citeregistered}}

    \newtotcounter{citeauthorpatent}
    \defbibenvironment{countauthorpatent}
        {\setcounter{citeauthorpatent}{0}\renewcommand{\blx@driver}[1]{}}
        {}
        {\stepcounter{citeauthorpatent}}

    \newtotcounter{citeauthorprogram}
    \defbibenvironment{countauthorprogram}
        {\setcounter{citeauthorprogram}{0}\renewcommand{\blx@driver}[1]{}}
        {}
        {\stepcounter{citeauthorprogram}}

    \newtotcounter{citeexternal}
    \defbibenvironment{countexternal}
        {\setcounter{citeexternal}{0}\renewcommand{\blx@driver}[1]{}}
        {}
        {\stepcounter{citeexternal}}
\makeatother

\defbibheading{nobibheading}{} % пустой заголовок, для подсчёта публикаций с помощью невидимой библиографии
\defbibheading{pubgroup}{\section*{#1}} % обычный стиль, заголовок-секция
\defbibheading{pubsubgroup}{\noindent\textbf{#1}} % для подразделов "по типу источника"

%%%Сортировка списка литературы Русский-Английский (предварительно удалить dissertation.bbl) (начало)
%%%Источник: https://github.com/odomanov/biblatex-gost/wiki/%D0%9A%D0%B0%D0%BA-%D1%81%D0%B4%D0%B5%D0%BB%D0%B0%D1%82%D1%8C,-%D1%87%D1%82%D0%BE%D0%B1%D1%8B-%D1%80%D1%83%D1%81%D1%81%D0%BA%D0%BE%D1%8F%D0%B7%D1%8B%D1%87%D0%BD%D1%8B%D0%B5-%D0%B8%D1%81%D1%82%D0%BE%D1%87%D0%BD%D0%B8%D0%BA%D0%B8-%D0%BF%D1%80%D0%B5%D0%B4%D1%88%D0%B5%D1%81%D1%82%D0%B2%D0%BE%D0%B2%D0%B0%D0%BB%D0%B8-%D0%BE%D1%81%D1%82%D0%B0%D0%BB%D1%8C%D0%BD%D1%8B%D0%BC
%\DeclareSourcemap{
%    \maps[datatype=bibtex]{
%        \map{
%            \step[fieldset=langid, fieldvalue={tempruorder}]
%        }
%        \map[overwrite]{
%            \step[fieldsource=langid, match=russian, final]
%            \step[fieldsource=presort,
%            match=\regexp{(.+)},
%            replace=\regexp{aa$1}]
%        }
%        \map{
%            \step[fieldsource=langid, match=russian, final]
%            \step[fieldset=presort, fieldvalue={az}]
%        }
%        \map[overwrite]{
%            \step[fieldsource=langid, notmatch=russian, final]
%            \step[fieldsource=presort,
%            match=\regexp{(.+)},
%            replace=\regexp{za$1}]
%        }
%        \map{
%            \step[fieldsource=langid, notmatch=russian, final]
%            \step[fieldset=presort, fieldvalue={zz}]
%        }
%        \map{
%            \step[fieldsource=langid, match={tempruorder}, final]
%            \step[fieldset=langid, null]
%        }
%    }
%}
%Сортировка списка литературы (конец)

%%% Создание команд для вывода списка литературы %%%
\newcommand*{\insertbibliofull}{
    \printbibliography[keyword=bibliofull,section=0,title=\bibtitlefull]
    \ifnumequal{\value{draft}}{0}{
      \printbibliography[heading=nobibheading,env=counter,keyword=bibliofull,section=0]
    }{}
}
\newcommand*{\insertbiblioauthor}{
    \printbibliography[heading=pubgroup, section=0, filter=papersregistered, title=\bibtitleauthor]
}
\newcommand*{\insertbiblioauthorimportant}{
    \printbibliography[heading=pubgroup, section=2, filter=papersregistered, title=\bibtitleauthorimportant]
}

% Вариант вывода печатных работ автора, с группировкой по типу источника.
% Порядок команд `\printbibliography` должен соответствовать порядку в файле common/characteristic.tex
\newcommand*{\insertbiblioauthorgrouped}{
    \section*{\bibtitleauthor}
    \ifsynopsis
    \printbibliography[heading=pubsubgroup, section=0, keyword=biblioauthorvak,    title=\bibtitleauthorvak,resetnumbers=true] % Работы автора из списка ВАК (сброс нумерации)
    \else
    \printbibliography[heading=pubsubgroup, section=0, keyword=biblioauthorvak,    title=\bibtitleauthorvak,resetnumbers=false] % Работы автора из списка ВАК (сквозная нумерация)
    \fi
    \printbibliography[heading=pubsubgroup, section=0, keyword=biblioauthorwos,    title=\bibtitleauthorwos,resetnumbers=false]% Работы автора, индексируемые Web of Science
    \printbibliography[heading=pubsubgroup, section=0, keyword=biblioauthorscopus, title=\bibtitleauthorscopus,resetnumbers=false]% Работы автора, индексируемые Scopus
    \printbibliography[heading=pubsubgroup, section=0, keyword=biblioauthorpatent, title=\bibtitleauthorpatent,resetnumbers=false]% Патенты
    \printbibliography[heading=pubsubgroup, section=0, keyword=biblioauthorprogram,title=\bibtitleauthorprogram,resetnumbers=false]% Программы для ЭВМ
    \printbibliography[heading=pubsubgroup, section=0, keyword=biblioauthorconf,   title=\bibtitleauthorconf,resetnumbers=false]% Тезисы конференций
    \printbibliography[heading=pubsubgroup, section=0, keyword=biblioauthorother,  title=\bibtitleauthorother,resetnumbers=false]% Прочие работы автора
}

\newcommand*{\insertbiblioexternal}{
    \printbibliography[heading=pubgroup,    section=0, keyword=biblioexternal,     title=\bibtitlefull]
}
     % Реализация пакетом biblatex через движок biber
}

% Вывести информацию о выбранных опциях в лог сборки
\typeout{Selected options:}
\typeout{Draft mode: \arabic{draft}}
\typeout{Font: \arabic{fontfamily}}
\typeout{AltFont: \arabic{usealtfont}}
\typeout{Bibliography backend: \arabic{bibliosel}}
\typeout{Precompile images: \arabic{imgprecompile}}
% Вывести информацию о версиях используемых библиотек в лог сборки
\listfiles

%%% Управление компиляцией отдельных частей диссертации %%%
% Необходимо сначала иметь полностью скомпилированный документ, чтобы все
% промежуточные файлы были в наличии
% Затем, для вывода отдельных частей можно воспользоваться командой \includeonly
% Ниже примеры использования команды:
%
%\includeonly{Dissertation/part2}
%\includeonly{Dissertation/contents,Dissertation/appendix,Dissertation/conclusion}
%
% Если все команды закомментированы, то документ будет выведен в PDF файл полностью

\begin{document}
%%% Переопределение именований типовых разделов
% https://tex.stackexchange.com/a/156050
\gappto\captionsrussian{\input{common/renames}\unskip} % for polyglossia and babel
\input{common/renames}

%%% Структура диссертации (ГОСТ Р 7.0.11-2011, 4)
\thispagestyle{empty}

\noindent%
\begin{tabularx}{\textwidth}{@{}lXr@{}}%
    & & \large{На правах рукописи}\\
    \IfFileExists{images/logo.pdf}{\includegraphics[height=2.5cm]{logo}}{\rule[0pt]{0pt}{2.5cm}}  & &
    \ifnumequal{\value{showperssign}}{0}{%
        \rule[0pt]{0pt}{1.5cm}
    }{
        \includegraphics[height=1.5cm]{personal-signature.png}
    }\\
\end{tabularx}

\vspace{0pt plus1fill} %число перед fill = кратность относительно некоторого расстояния fill, кусками которого заполнены пустые места
\begin{center}
\textbf {\large \thesisAuthor}
\end{center}

\vspace{0pt plus3fill} %число перед fill = кратность относительно некоторого расстояния fill, кусками которого заполнены пустые места
\begin{center}
\textbf {\large %\MakeUppercase
\thesisTitle}

\vspace{0pt plus3fill} %число перед fill = кратность относительно некоторого расстояния fill, кусками которого заполнены пустые места
{\large Специальность \thesisSpecialtyNumber\ "---\par <<\thesisSpecialtyTitle>>}

\ifdefined\thesisSpecialtyTwoNumber
{\large Специальность \thesisSpecialtyTwoNumber\ "---\par <<\thesisSpecialtyTwoTitle>>}
\fi

\vspace{0pt plus1.5fill} %число перед fill = кратность относительно некоторого расстояния fill, кусками которого заполнены пустые места
\Large{Автореферат}\par
\large{диссертации на соискание ученой степени\par \thesisDegree}
\end{center}

\vspace{0pt plus4fill} %число перед fill = кратность относительно некоторого расстояния fill, кусками которого заполнены пустые места
{\centering\thesisCity~--- \thesisYear\par}

\newpage
% оборотная сторона обложки
\thispagestyle{empty}
\noindent Работа выполнена {\thesisInOrganization}.

\vspace{0.008\paperheight plus1fill}
\noindent%
\begin{tabularx}{\textwidth}{@{}lX@{}}
    \ifdefined\supervisorTwoFio
    Научные руководители:   & \supervisorRegalia\par
                              \ifdefined\supervisorDead
                              \framebox{\textbf{\supervisorFio}}
                              \else
                              \textbf{\supervisorFio}
                              \fi
                              \par
                              \vspace{0.013\paperheight}
                              \supervisorRegalia\par
                              \ifdefined\supervisorTwoDead
                              \framebox{\textbf{\supervisorTwoFio}}
                              \else
                              \textbf{\supervisorTwoFio}
                              \fi
                              \vspace{0.013\paperheight}\\
    \else
    Научный руководитель:   & \supervisorRegalia\par
                              \ifdefined\supervisorDead
                              \framebox{\textbf{\supervisorFio}}
                              \else
                              \textbf{\supervisorFio}
                              \fi
                              \vspace{0.013\paperheight}\\
    \fi
    Официальные оппоненты:  &
    \ifnumequal{\value{showopplead}}{0}{\vspace{13\onelineskip plus1fill}}{%
        \textbf{\opponentOneFio,}\par
        \opponentOneRegalia,\par
        \opponentOneJobPlace,\par
        \opponentOneJobPost\par
        \vspace{0.01\paperheight}
        \textbf{\opponentTwoFio,}\par
        \opponentTwoRegalia,\par
        \opponentTwoJobPlace,\par
        \opponentTwoJobPost
    \ifdefined\opponentThreeFio
        \par
        \vspace{0.01\paperheight}
        \textbf{\opponentThreeFio,}\par
        \opponentThreeRegalia,\par
        \opponentThreeJobPlace,\par
        \opponentThreeJobPost
    \fi
    }%
    \vspace{0.013\paperheight} \\
    \ifdefined\leadingOrganizationTitle
    Ведущая организация:    &
    \ifnumequal{\value{showopplead}}{0}{\vspace{6\onelineskip plus1fill}}{%
        \leadingOrganizationTitle
    }%
    \fi
\end{tabularx}
\vspace{0.008\paperheight plus1fill}

\noindent Защита состоится \defenseDate~на~заседании диссертационного совета \defenseCouncilNumber~при \defenseCouncilTitle~по адресу: \defenseCouncilAddress.

\vspace{0.008\paperheight plus1fill}
\noindent С диссертацией можно ознакомиться в библиотеке \synopsisLibrary.

\vspace{0.008\paperheight plus1fill}
\noindent Отзывы на автореферат в двух экземплярах, заверенные печатью учреждения, просьба направлять по адресу: \defenseCouncilAddress, ученому секретарю диссертационного совета~\defenseCouncilNumber.

\vspace{0.008\paperheight plus1fill}
\noindent{Автореферат разослан \synopsisDate.}

\noindent Телефон для справок: \defenseCouncilPhone.

\vspace{0.008\paperheight plus1fill}
\noindent%
\begin{tabularx}{\textwidth}{@{}%
>{\raggedright\arraybackslash}b{18em}@{}
>{\centering\arraybackslash}X
r
@{}}
    Ученый секретарь\par
    диссертационного совета\par
    \defenseCouncilNumber,\par
    \defenseSecretaryRegalia
    &
    \ifnumequal{\value{showsecrsign}}{0}{}{%
        \includegraphics[width=2cm]{secretary-signature.png}%
    }%
    &
    \defenseSecretaryFio
\end{tabularx}
           % Титульный лист
\include{Dissertation/contents}        % Оглавление
\ifnumequal{\value{contnumfig}}{1}{}{\counterwithout{figure}{chapter}}
\ifnumequal{\value{contnumtab}}{1}{}{\counterwithout{table}{chapter}}
\chapter*{Введение}                         % Заголовок
\addcontentsline{toc}{chapter}{Введение}    % Добавляем его в оглавление

\newcommand{\actuality}{}
\newcommand{\progress}{}
\newcommand{\aim}{{\textbf\aimTXT}}
\newcommand{\tasks}{\textbf{\tasksTXT}}
\newcommand{\novelty}{\textbf{\noveltyTXT}}
\newcommand{\influence}{\textbf{\influenceTXT}}
%\newcommand{\methods}{\textbf{\methodsTXT}}
%\newcommand{\defpositions}{\textbf{\defpositionsTXT}}
%\newcommand{\reliability}{\textbf{\reliabilityTXT}}
%\newcommand{\probation}{\textbf{\probationTXT}}
\newcommand{\contribution}{\textbf{\contributionTXT}}
\newcommand{\publications}{\textbf{\publicationsTXT}}

Диссертационная работа посвящена разработке математических и про­граммных средств компьютерного моделирования сложной динамики атомных ансамблей и поля в оптических полостях в рамках конечномерных моделей квантовой электродинамики. Ключевой задачей диссертации является разра­ботка подходов к изучению и практическому получению темных состояний многоуровневых атомов, что позволяет пролить свет на понимание структуры темного подпространства.
\\[24pt]
\indent\textbf{Актуальность работы}\\
\indent Развитие математических и программных методов моделирования кванто­вой динамики поля и атомов является ключевым этапом в развитии технологий квантовых устройств, в частности, в построении \textbf{квантового компьютера} \cite{feynman,quantum_computing_zalka,valiev_1,valiev_2}. В особенности, это важно для отечественных исследований в этой об­ласти, где первые принципы квантовой теории для сложных систем лучше всего проверять в \textbf{оптических полостях} \cite{cavity_exp_1,cavity_exp_2,cavity_exp_3} --- весьма дорогостоящем оборудовании. Публикации, как правило, ограничиваются общим описанием эксперимента: технические нюансы его проведения и прибористика остают­ся намеренно нераскрытыми, что затрудняет воспроизводимость полученных результатов и ограничивает возможность их использования другими иссле­дователями. Поэтому особую значимость на сегодняшний день приобретает разработка математических и программных методов компьютерного и супер­компьютерного моделирования многочастичных квантовых процессов. Это дает нам возможность не только быть в курсе производимых в мире современных экс­периментальных работ, но и предвидеть новые, практически важные эффекты квантовой природы, которые можно было бы получить на таком оборудова­нии. Экспоненциальный рост вычислительной сложности \cite{feynman} требует создания эффективных компьютерных программ, моделирующих квантовую динамику в конечномерных моделях. Важнейшими здесь являются модели \textbf{Джейнса-Кам­мингса} \cite{jc_comparison}, \textbf{Тависа-Каммингса} \cite{tc_exact_solution,tc_a_study,tc_dicke_2,tc_improvement,tc_experimental} и их многоуровневые модификации \cite{jc_tc_extension_1,jc_tc_extension_2,jc_tc_extension_3}.

Построение \textbf{квантового компьютера} по первоначальной схеме Р. Фейнмана \cite{feynman} не удается из-за проблемы \textbf{декогерентности} \cite{decoherence_fedichkin,shor_error_2}, носящей фундаментальный характер и в настоящее время описываемой лишь в рамках теории открытых квантовых систем \cite{breuer}. Важнейшей задачей здесь является описание так называемых \textbf{темных состояний} \cite{kok} --- ансамблей атомов, не взаимодействующих с полем. Будучи свободными от декогерентности, темные состояния могут быть использованы для достаточно длительного хранения сложных состояний в \textbf{квантовых вычислениях}, к примеру, в задачах организации \textbf{квантовой памяти} \cite{dark_states_quantum_memory}. Поэтому изучение структуры и разра­ботка методов их практического получения чрезвычайно важны для развития нанотехнологий в целом. В настоящей работе предложен достаточно простой в технической реализации метод получения двухуровневых и многоуровневых темных состояний атомов. Описанию данного метода и его компьютерному мо­делированию посвящена глава \ref{ch:ch3}. Отдельное внимание в данной работе также уделено определению размерности и изучению структуры темного подпростран­ства ансамблей трехуровневых атомов (глава \ref{ch:ch4}).

Особую значимость на сегодняшний день также приобретает компьютер­ное моделирование \textbf{квантовых гейтов} \cite{azuma,fault_tolerant_1}, позволяющее, в частности, дать более точную оценку их качества. Это играет большую роль в непосредственной разработке элементов квантового компьютера и, как следствие, помогает выбрать наиболее эффективный и технологичный путь их реализации. Глава \ref{ch:ch5} данной работы посвящена компьютерному моделированию запутывающего гейта coCSign и оценке его эффективности. Основной запутывающий гейт CNOT (управляемый NOT) реализуется на его основе при помощи однокубитных гейтов.

\textbf{Целью работы} является разработка методов моделирования сложных квантовых процессов в рамках конечномерных моделей квантовой электродина­мики, а также разработка комплекса программ, реализующих данные методы.
\\[18pt]
\indent Важнейшей областью их приложения являются расчет квантовых эф­фектов, моделирование квантовых гейтов, а также исследование темных подпространств, свободных от декогерентности --- главного препятствия кван­товых вычислений.
\\[18pt]
\indent Конкретные \textbf{задачи}, к которым применяются предложенные методы, следующие:
\begin{itemize}
\item{исследование квантовой динамики больших многочастичных систем,}
\item{получение темных состояний ансамблей многоуровневых атомов,}
\item{проверка гипотезы об общем явном виде темного подпространства ансамблей трехуровневых атомов как линейных комбинаций антисим­метричных базисных состояний,
}
\item{моделирование запутывающего гейта coCSign в системе оптических по­лостей.}
\end{itemize}
\
\\
\indent\textbf{Основные положения, выносимые на защиту:}
\begin{enumerate}
\item{Методы компьютерного моделирования сложных квантовых процессов в конечномерных моделях квантовой электродинамики.}
\item{Метод получения темных состояний ансамблей многоуровневых атомов при помощи отбора, основанного на томографии состояния поля вне оптической полости.}
\item{Метод определения размерности темного подпространства многоуров­ ных атомных ансамблей, позволяющий явно описать его структуру.}
\item{Компьютерное моделирование запутывающего гейта coCSign и иссле­
дование факторов, влияющих на точность его срабатывания.}
\end{enumerate}
\
\\
\indent\textbf{Теоретическая и практическая значимость} диссертационной работы состоит в создании математических и программных средств компьютерного и суперкомпьтерного моделирования квантовых процессов в конечномерных мо­ делях, которые позволили
\begin{itemize}
\item{произвести детальный анализ квантовых эффектов, которые не уда­ется предсказать при помощи стандартных математических методов (примером может быть найденный пикообразный характер ансамбле­вых осцилляций групп атомов в оптических полостях),}
\item{предложить метод отбора темных состояний, важный для их практи­ческого применения в наноустройствах и задачах квантовой криптогра­фии,}
\item{произвести определение размерности темного подпространства про­странства многоуровных атомных ансамблей ограниченной численно­сти,}
\item{произвести компьютерное моделирование запутывающего гейта coCSign, а также дать оценку точности его срабатывания.}
\end{itemize}

\clearpage
\indent\textbf{Квантовый компьютер.}

Квантовый компьютер, концепция которого впервые была предложена американским физиком-теоретиком Ричардом Фейнманом \cite{feynman} в начале 1980-х годов, представляет собой революционную идею в области вычислительной техники. Он выдвинул предположение о том, что классические компьютеры принципиально не способны моделировать поведение сложных квантовых си­стем из-за экспоненциального роста требуемых для этого ресурсов. В связи с этим им была предложена идея создания нового типа компьютера, который бы использовал принципы квантовой механики, что позволило бы значительно ускорить процесс обработки информации.

Квантовый компьютер представляет собой вычислительное устройство, которое использует квантовомеханические явления, такие как суперпозиция и запутанность, для выполнения операций над данными. В отличие от класси­ческих компьютеров, которые оперируют битами, принимающими значения 0 или 1, квантовые компьютеры работают с квантовыми битами, или \textbf{кубитами}, каждый из которых может одновременно находиться в состояниях $|0\ra$ и $|1\ra$ (суперпозиции состояний) с определенными вероятностями.

Математически состояние кубита может быть представлено в виде линей­ной комбинации двух ортонормированных базисных состояний:
\begin{align*}
	|\Psi\ra = \alpha|0\ra + \beta|1\ra,\qquad|\alpha|^2 + |\beta|^2 = 1,
\end{align*}
где $|0\ra$ и $|1\ra$ -- базисные состояния кубита (аналогичные 0 и 1 для классического бита), $\alpha$ и $\beta$ -- означающие вероятностные амплитуды этих состояний. Согласно правилу Борна, квадрат модуля амплитуды ($|\alpha|^2$ или $|\beta|^2$) определяет вероятность того, что в результате измерения кубита будет получено соответствующее базисное состояние.

Кубиты могут быть физически реализованы с использованием различных технологий \cite{valiev_1}: ионы в ловушках, сверхпроводящие кубиты, кубиты на основе дефектов в кристаллических решетках и многое другое. Выбор системы для реа­лизации кубита зависит от множества факторов, включая легкость управления, устойчивость к декогерентности, возможность масштабирования и способность к интеграции с другими кубитами для выполнения квантовых операций.

Важным принципом Фейнмановского квантового компьютера является возможность проведения вычислений при помощи квантовых вентилей, или \textbf{квантовых гейтов (quantum gates)} \cite{valiev_1,valiev_2}, --- унитарных операторов, ре­ализующих различные квантовые операции, такие как инверсия, вращение, фазовый сдвиг и прочие преобразования состояний одного или нескольких ку­ битов. \textbf{Универсальным} считается набор квантовых гейтов, достаточных для осуществления любого квантового преобразования на произвольном наборе ку­битов. Его можно получить, взяв, к примеру, все однокубитные гейты и любой \textbf{запутывающий гейт} (скажем, \textbf{CNOT} --- управляемый NOT) \cite{quantum_gates_barenco}. Основная трудность физической реализации квантовых гейтов заключается в обеспече­нии точного управления над состояниями кубитов и минимизации связанных с этим ошибок.

\textbf{Гейтовая модель} --- один из наиболее распространенных подходов к квантовым вычислениям. Она представляет собой описание квантового компьютера, который работает путем последовательного применения кван­товых гейтов к кубитам --- квантовым аналогам битов. 

\textbf{Квантовый алгоритм} в гейтовой модели --- это представление кван­товых вычислений в виде схемы, состоящей из квантовых гейтов, которые применяются к кубитам с тем, чтобы преобразовать входное квантовое состоя­ние в желаемое выходное состояние. В отличие от классических битов, которые могут быть в состоянии 0 или 1, кубиты могут находиться в суперпозиции этих состояний, что позволяет решать некоторые задачи более эффективно, чем это возможно сегодня на классических компьютерах.

\begin{figure}[h!]
	\noindent\centering{
		\includegraphics[width=0.35\textwidth]{Dissertation/images/introduction/swap.png}
		\captionsetup{format=hang,width=0.85\textwidth,justification=centering,singlelinecheck=no}
		\\[6pt]
		\caption{Пример: операция SWAP перестановки кубитов\\может быть реализована троекратным применением гейта CNOT}
	}
\end{figure}

Фейнмановская идея квантовых вычислений также предполагает возмож­ность реализации \textbf{быстрых квантовых алгоритмов}, дающих \textbf{экспоненци­альное ускорение}. Такие алгоритмы могут принципиально отличаться от классических и решать определенные задачи за время, недостижимое для со­временных компьютеров.

\clearpage
К числу таких алгоритмов относятся
\begin{itemize}
\item{
	\textbf{алгоритм Шора факторизации целых чисел \cite{shor}}\\
	Опубликованный в 1994 году алгоритм Шора позволяет факторизовать число $n$ за время $O(\log^{3}n)$, используя $O(\log n)$ кубитов. Его работа была продемонстрирована экспериментально в 2001 году группой спе­циалистов из IBM \cite{shor_exp}: разложение числа 15 на простые множители с использованием 7 кубитов. При достаточном числе кубитов алгоритм Шора способен взламывать такие криптографические протоколы, как RSA, за время, едва превосходящее время шифрования.\\[0pt]
}
\item{
	\textbf{алгоритм Гровера поиска элемента в базе данных \cite{grover}}\\
	Алгоритм использует итеративный процесс применения квантового опе­ратора для увеличения амплитуды искомого элемента в суперпозиции всех элементов, что позволяет найти искомый элемент в неупорядочен­ном списке за время $O(\sqrt{N})$, где $N$ -- количество элементов в списке. Данный алгоритм демонстрирует квадратичное ускорение по сравне­нию с классическими алгоритмами поиска.
}
\end{itemize}
\
\\
\indent С тех пор как Фейнман выдвинул свою идею, \textbf{квантовые вычисления} претерпели значительные теоретические и экспериментальные усовершенство­вания \cite{quantum_computing_strengths_weaknesses,kitaev,quantum_computing_zalka,quantum_computing_universal,decoherence_fedichkin,valiev_1,valiev_2,stean,quantum_computing_abrams_lloyd,quantum_computing_ablayev}. Разработка квантовых компьютеров стала возможной благодаря прогрессу в области квантовой оптики, сверхпроводимости, а также в результате открытий, сделанных в области квантовой информатики и теории сложности. В то же время, несмотря на значительные успехи в создании кван­товых устройств, ряд технических проблем, таких как \textbf{декогерентность}, ошибки в квантовых гейтах и сложность масштабирования, остаются нерешенными. Квантовая коррекция ошибок \cite{shor_error_1,shor_error_2,shor_error_3} и разработка надежных квантовых гейтов \cite{fault_tolerant_1,fault_tolerant_2} являются ключевыми направлениями исследований на пути к реализации полноценного квантового компьютера, описанного Фейнманом.

Фейнмановский квантовый компьютер представляет собой воплощение фундаментальных принципов квантовой теории в практических вычислитель­ных устройствах и, несмотря на все трудности, продолжает оставаться одним из наиболее перспективных направлений в развитии современной вычислитель­ной техники.

\clearpage
\indent\textbf{Роль квантовой запутанности в квантовых вычислениях.}

\textbf{Квантовые вычисления} представляют собой область информатики, которая исследует возможности применения квантовых явлений, таких как суперпозиция и квантовая запутанность, для представления и обработки ин­формации на микроскопическом уровне, что теоретически позволяет достичь значительного ускорения в решении определенного класса вычислительных задач. Ключевым элементом квантовых вычислений является \textbf{квантовая за­путанность}, которая играет важнейшую роль в функционировании квантовых гейтов и построении квантовых алгоритмов.

Явление квантовой запутанности возникает, когда пара или группа частиц генерируется или взаимодействует таким образом, что квантовое состояние каж­дой отдельной частицы не может быть описано независимо от состояний других частиц, даже находясь от них на большом физическом расстоянии.

К примеру, множество всех максимально запутанных двухкубитных кван­товых состояний задается состояниями Белла
\begin{align*}
	\frac{1}{\sqrt{2}}(|00\ra + |11\ra),\qquad \frac{1}{\sqrt{2}}(|01\ra + |10\ra),\\
	\frac{1}{\sqrt{2}}(|00\ra - |11\ra),\qquad \frac{1}{\sqrt{2}}(|01\ra - |10\ra).
\end{align*}

Наличие квантовой запутанности позволяет производить операции над несколькими кубитами одновременно, что приводит к экспоненциальному ускорению в сравнении с классическими вычислениями. С другой стороны, мно­гочастичная запутанность на сегодняшний день остается главным препятствием на пути к созданию масштабируемого квантового компьютера. Одной из фун­даментальных проблем здесь является проблема декогерентности --- процесса разрушения квантового состояния под воздействием внешней среды. В этом случае системы теряют свои квантовые свойства, в частности, запутанность, что приводит к "классическому"\ поведению кубитов и, как следствие, наруше­нию работы квантовых алгоритмов.

Внедрение широкого использования темных состояний есть один из под­ходов к минимизации влияния декогерентности на квантовые системы, что критически важно для развития и практической реализации квантовых вычислительных технологий.

\clearpage
\indent\textbf{Конечномерные модели квантовой электродинамики.}

Конечномерные модели квантовой электродинамики (КЭД) являются ключем к компьютерному моделированию взаимодействия света и вещества, что представляет важность для создания квантовых компьтеров. Они позволяют описывать динамику на уровне отдельных атомов и поля, при котором мы можем точно вычислять квантовые эффекты и непосредственно сравнивать их с экспериментом, в отличии, к примеру, от моделей на твердотельных структурах, где участвуют миллионы атомов (что само по себе приводит к невозможности использования точных вычислительных методов). В особенности, это преимущество конечномерных моделей проявляется при рассмотрении запутанных состояний атомов и поля, играющих важнейшую роль при защите квантовых состояний от декогерентности.

В статье 1963 года \cite{jc_comparison} Э. Джейнс и Ф. Каммингс предложили одну из таких моделей --- модель, описывающую поведение атома, взаимодействующего с модой квантового электромангитного поля. Удержание квантов возбуждения поля (фотонов) осуществляется при помощи двух зеркал, расположенных друг напротив друга, образуя тем самым \textbf{полость}, или \textbf{оптический резонатор}.

\begin{figure}[h!]
	\noindent\centering{
		\includegraphics[width=0.75\textwidth]{Dissertation/images/introduction/optical_cavity.png}
		\captionsetup{format=hang,width=0.8\textwidth,justification=centering,singlelinecheck=no}
		\\[6pt]
		\caption{Оптический резонатор}
	}
\end{figure}

Двухуровневый\footnote{обычно используются два энергетических подуровня в атоме} атом, помещенный в такую полость, может взаимодействовать с полем внутри полости: если энергия перехода между уровнями составляет $E = \hbar w_{a}$ и частота атомного перехода $w_{a}$ примерно равна $w_{c}$ (частоте моды поля), атом переходит из основного состояния $|0\ra$ в возбужденное состояние $|1\ra$ и наоборот. В первом случае происходит поглощение фотона, во втором -- его испускание атомом. Процесс поглощения фотона с его последующим испусканием обратно в полость называется \textbf{осцилляцией Раби \cite{rabi_1,rabi_2,rabi_3}}. 

Для обеспечения времени жизни фотонов в полости, достаточного для нескольких десятков рабиевских осцилляций, нужна большая точность изготовления самих полостей и поддержание высокой степени вакуума в них. Кроме того, для удержания фотона в полости необходимо, чтобы создаваемое им электромагнитное поле порождало конструктивную интерференционную картину внутри полости и деструктивную --- за ее пределами: расстояние $L$ между отражающими зеркалами должно быть кратно половине длины волны фотона $\lambda = 2\pi c/w_{c}$. Используемые в экспериментах полости имеют длину, равную $L = \lambda/2$, то есть порядка 7 мм, что соответствует длине полуволны атома Rb85, два подуровня которого чаще всего используют в экспериментах \cite{rempe,rb_1,rb_2,rb_3}).

Появление одноатомных мазеров сделало возможным изучение взаимодействия отдельного атома с модой электромагнитного поля резонатора. Так, с его помощью в 1987 году Г. Ремпе, Г. Вальтеру и Н. Кляйну \cite{rempe} удалось усилить связь атома с выбранной модой поля (с одновременным подавлением остальных мод) и экспериментально воссоздать динамику, описываемую моделью Джейнса-Каммингса.

Модель Джейнса-Каммингса, предложенная для двухуровневого атома в оптической полости, позже была обобщена на ансамбли таких атомов (модель Тависа-Каммингса \cite{tc_exact_solution,tc_a_study}), а также на несколько полостей, связанных между собой оптическим волокном (модели Джейнса-Каммингса-Хаббарда \cite{jch_time_evolution,jch_site_dependent_control,jch_quench_dynamics} и Тависа-Каммингса-Хаббарда \cite{tch_photon_blockade,tch_transfer,tch_quality}).

На сегодняшний день эксперименты с оптическими полостями \cite{cavity_exp_1,cavity_exp_2,cavity_exp_3} относятся к числу дорогостоящих, так как требуют зеркал с очень высокой степенью отражения (до нескольких десятков тысяч на одну рабиевскую осцилляцию), что достигается использованием сверхпроводящих материалов (к примеру, ниобия) и низких (гелиевых) температур.

Численные же эксперименты не ограничены в своем количестве, дают неоспоримые преимущества в вопросах гибкости настроек своих параметров. Поэтому компьютерное и суперкомпьютерное моделирование динамики квантовых состояний в полостях является необходимым и важнейшим звеном в изучении квантовой теории сложных систем.

\clearpage
\indent\textbf{Модель Джейнса-Каммингса.}

%Конечномерная модель квантовой электродинамики Джейнса-Каммингса описывает взаимодействие двухуровневого атома, помещенного в оптический %резонатор, с одномодовым полем частоты, близкой к собственной частоте резонатора.
Конечномерная модель Джейнса-Каммингса квантовой электродинамики описывает взаимодействие двухуровневого атома, помещенного в оптический резонатор, с одномодовым полем, частота которого близка к частоте атомного перехода.

\begin{figure}[h!]
	\noindent\centering{
		\includegraphics[width=0.3\textwidth]{Dissertation/images/introduction/jc.png}
		\captionsetup{format=hang,width=0.7\textwidth,justification=centering,singlelinecheck=no}
		\caption{Модель Джейнса-Каммингса: двухуровневый атом в оптическом резонаторе}
	}
\end{figure}

\noindent Атом взаимодействуют с электромагнитным полем полости, испуская или поглощая фотон. При поглощении фотона атом возбуждается, при испускании --- переходит в основное состояние. 
\\[12pt]
Обозначим:
$|0\ra$ -- основное состояние,

\hspace{45pt}$|1\ra$ -- возбужденное состояние атома.
\\[12pt]
Введем также следующие операторы:

\vspace{6pt}
\hspace{5pt}$a^{+}, a$ -- операторы рождения и уничтожения фотонов резонаторной моды,

\hspace{5pt}$\s^{+}, \s$ -- повышающий и понижающий операторы атома.
\\[1pt]

\hspace{6pt}$a^{+}|n\ra = \sqrt{n+1}\ |n+1\ra$
\hspace{33pt}$\s^{+}|0\ra = |1\ra$
\hspace{23pt}$\s^{+}|1\ra = 0$

\hspace{15pt}$a|n\ra = \sqrt{n}\ |n-1\ra$
\hspace{70pt}$\s|1\ra =\hspace{2pt}|0\ra$
\hspace{32pt}$\s|0\ra\hspace{1pt} = \hspace{1pt}0$
\\[18pt]
Такая система описывается гамильтонианом Джейнса-Каммингса:
\begin{normalsize}
	\begin{equation}
		H_{\text{JC}} = \underbrace{\hbar w_{c}\ a^{+}a}_{\textstyle H_{field}} + \underbrace{\hbar w_{a}\s^{+}\s}_{\textstyle H_{atom}}~+~\underbrace{g(\s^{+}+\s)(a^{+}+a)}_{\textstyle H_{interaction}},
	\end{equation}
\end{normalsize}

\hspace{25pt}$\hbar $ -- постоянная Планка,

\hspace{22pt}$w_{c}$ -- частота фотонов в полости,

\hspace{22pt}$w_{a}$ -- частота атомного перехода,

\hspace{25pt}$g$ -- интенсивность взаимодействия двухуровневого атома с полем:
%\begin{array}
%\hspace{3em}$\displaystyle g = \sqrt{\frac{\hbar w_{c}}{2 \epsilon_{0} V}} \cdot d$, где\\
%\hspace{3em}$V$ -- эффективный объем полости\\
%\hspace{3em}$d$ -- проекция дипольного момента атома на направление поляризации фотона,\\
%\hspace{3em}$\epsilon_{0}$ -- электрическая постоянная
%\end{array}
\begin{gather*}
	g = \sqrt{\frac{\hbar w_{c}}{2 \epsilon_{0} V}} \cdot d,\\
	V\text{ -- эффективный объем полости},\\
	\epsilon_{0}\text{ -- электрическая постоянная},\\
	\hspace{5em}d\text{ -- проекция дипольного момента атома на направление}\\
	\hspace{4em}\text{поляризации фотона},
\end{gather*}

\vspace{-0.5em}
\hspace{25pt}$|w_{c}-w_{a}|\ll w_{c}+w_{a}$: условие применимости модели JС, означающее\\
\indent\hspace{11em}достаточное время жизни фотона внутри полости.
\\[18pt]
\noindent Пренебрегая членами $\displaystyle{\s^{+}a^{+}}$ и $\displaystyle{\s a}$, не сохраняющими энергию, перепишем гамильтониан в следующем виде:
\begin{equation}\label{jc_rwa}
	H_{\text{JC}}~\approx~H_{\text{JC}}^{\text{RWA}}~=~\underbrace{\hbar w_{c}\ a^{+}a}_{\textstyle H_{field}} + \underbrace{\hbar w_{a}\s^{+}\s}_{\textstyle H_{atom}}~+~\underbrace{g(\s^{+}a+\s a^{+})}_{\textstyle H_{interaction}}.
\end{equation}

\noindent Приближение \eqref{jc_rwa} называется \textbf{приближением вращающейся волны (rotating wave approximation), или RWA} \cite{rwa_rabi_1,rwa_rabi_2,ozhigov_qq}, и имеет место в условии слабого взаимодействия
\begin{equation}
	\frac{g}{\hbar w_{c}}\approx\frac{g}{\hbar w_{a}}\ll1,
\end{equation}
при котором слагаемые $\displaystyle{\s^{+}a^{+}}$ и $\displaystyle{\s a}$ быстро осциллируют, что делает их вклад в гамильтониан незначительным.
\\[28pt]
\indent\textbf{Модель Тависа-Каммингса.}

Рассмотрим $N$ двухуровневых атомов, взаимодействующих с модой электромагнитного поля в идеальном резонаторе. 

Такая система описывается гамильтонианом Тависа-Каммингса:
\begin{normalsize}
	\begin{equation}
		H_{\text{TC}} =  \hbar w_{c}\ a^{+}a + \hbar \sum_{i=1}^N{w_{a_{i}}\s_{i}^{+}\s_{i}} + \sum_{i=1}^N{g_{i}(\s_{i}^{+}+\s_{i})(a^{+}+a)},
	\end{equation}
\end{normalsize}

\begin{quote}
	$\hbar $ -- постоянная Планка,
	
	$w_{c}$ -- частота фотонов в полости,
	
	$w_{a_{i}}$ -- частота перехода $i$-го атома,
	
	$g_{i}$ -- интенсивность взаимодействия $i$-го атома с полем,
	
	$\s_{i}^{+}, \s_{i}$ -- повышающий и понижающий операторы $i$-го атома,
	
	$|w_{c}-w_{a_{i}}|\ll w_{c}+w_{a_{i}} \quad\forall i = \overline{1,N}$: условие применимости модели TС, означающее достаточное время жизни фотона внутри полости.
\end{quote}

\indent Аналогичным образом, пренебрегая членами $\displaystyle{\sigma_{i}^{+}a^{+}}$ и $\displaystyle{\sigma_{i} a}$, не сохраняющими энергию, запишем гамильтониан TC в приближении RWA \cite{rwa_rabi_1,rwa_rabi_2,ozhigov_qq}:
\begin{equation}
	H_{\text{TC}}^{\text{RWA}} = \underbrace{\hbar w_{c}\ a^{+}a}_{\textstyle H_{field}} + \hbar\underbrace{\sum_{i=1}^N{w_{a_{i}}\s_{i}^+\s_{i}}}_{\textstyle H_{atoms}} + \underbrace{\sum_{i=1}^N{g_{i}(\s_{i}^{+}a+\s_{i}a^{+})}}_{\textstyle H_{interaction}},
\end{equation}
\begin{equation}
	\frac{g_{i}}{\hbar w_{c}}\approx\frac{g_{i}}{\hbar w_{a_{i}}}\ll1 \qquad \forall i = \overline{1,N}.
\end{equation}
\vspace{-1em}
\
\\[0pt]
\indent Приближение RWA \cite{rwa_rabi_1,rwa_rabi_2,ozhigov_qq} справедливо для слабой связи атомов с полем и позволяет либо решить задачу аналитически (в случае одного атома), либо существенно упростить компьютерное моделирование квантовой динамики --- для случая нескольких атомов. 

Для RWA-приближения \cite{rwa_rabi_1,rwa_rabi_2,ozhigov_qq} пространство квантовых состояний системы распадается на сумму ортогональных подпространств, инвариантных относительно гамильтониана, и обладающих, к тому же, относительно малой размерностью, которая, в случае одинаковой интенсивности взаимодействия атомов с полем, растет линейно с их числом. В случае же различных интенсивностей взаимодействия этот рост становится экспоненциальным, что обуславливает необходимость применения суперкомпьютеров для установления характера квантовой динамики больших ансамблей.

Модели Джейнса-Каммингса и Тависа-Каммингса могут иметь довольно широкое применение в силу их простоты: к примеру, с их помощью можно описывать переход атомных возбуждений по системе резонаторов, соединенных оптическим волокном, проводимость таких систем \cite{skovoroda_conductivity_1,skovoroda_conductivity_2} и связанные с ней эффекты, такие как квантовое бутылочное горлышко \cite{quantum_bottleneck_victorova}, эффект усиления проводимости дефазирующим шумом (dephasing assisted transport, или DAT) \cite{dat_plenio,dat_quantum_dots} и квантовые блуждания на графах \cite{quantum_walks_ambainis,quantum_walks_ambainis_applications,quantum_walks_mixing,quantum_walks_ambainis_speedup}.

\clearpage
\indent\textbf{Mодель Джейнса-Каммингса-Хаббарда}

Модель Джейнса-Каммингса можно обобщить для случая двух и более взаимодействующих полостей: фотоны могут перемещаться между полостями посредством волноводов из оптического волокна.
\begin{figure}[h!]
	\noindent\centering{
		\includegraphics[width=1.0\textwidth]{Dissertation/images/introduction/JCH.png}
		\captionsetup{format=hang,width=0.7\textwidth,justification=centering,singlelinecheck=no}
		\caption{Модель Джейнса-Каммингса-Хаббарда: цепочка взаимодействующих полостей}
	}
\end{figure}

Такая система описывается моделью Джейнса-Каммингса-Хаббарда \cite{jch_time_evolution,jch_site_dependent_control,jch_quench_dynamics} и ее гамильтониан имеет следующий вид:
\[
H_{\text{JCH}}~=~\sum_{j=1}^J \left(\underbrace{\hbar w_{c_{j}}\ a_j^+a_{j}}_{\textstyle H_{field}}~+~\underbrace{\hbar w_{a_{j}}{\sigma_{j}^+\sigma_{j}}}_{\textstyle H_{atom}}~+~\underbrace{g_j{(\sigma_{j}^{+}+\sigma_{j})(a_{j}^{+}+a_{j})}}_{\textstyle H_{interaction}} \right)
\]
\begin{equation}
	+k_{j,j+1}\sum_{j=1}^{J-1}{\left(a_{j+1}^+a_j+a_{j}^+a_{j+1}\right)},
\end{equation}
\begin{quote}
	$J$ -- количество взаимодействующих полостей (члены с индексом $j$ соответствуют $j$-й полости),\\
	$\sigma_{j}^{+},\sigma_{j}$ -- операторы возбуждения/релаксакции атома в $j$-й полости,\\
	$k_{j,j+1}$ -- интенсивность перелета фотона между $j$-й и $(j+1)$-й поло­стями.
\end{quote}

В приближении вращающейся волны (RWA) \cite{rwa_rabi_1,rwa_rabi_2,ozhigov_qq} гамильтониан JCH определяется естественным образом:
\[
H_{\text{JCH}}^{\text{RWA}}~=~\sum_{j=1}^J \left(\underbrace{\hbar w_{c_j}\ a_j^+a_j}_{\textstyle H_{field}}~+~\underbrace{\hbar w_{a_j}{\sigma_{j}^+\sigma_{j}}}_{\textstyle H_{atom}}~+~\underbrace{g_j{(\sigma_{j}^+a_j+\sigma_{j}a_j^+)}}_{\textstyle H_{interaction}} \right)
\]

\begin{equation}
	+k_{j,j+1}\sum_{j=1}^{J-1}{\left(a_{j+1}^+a_j+a_{j}^+a_{j+1}\right)}.
\end{equation}

\clearpage
\indent\textbf{Mодель Тависа-Каммингса-Хаббарда}

Модель Тависа-Каммингса также обобщается на случай нескольких поло­стей, каждая из которых может содержать один или более атомов.
\begin{figure}[h!]
	\noindent\centering{
		\includegraphics[width=1.0\textwidth]{Dissertation/images/introduction/TCH.png}
		\captionsetup{format=hang,width=0.7\textwidth,justification=centering,singlelinecheck=no}
		\caption{Модель Тависа-Каммингса-Хаббарда: цепочка взаимодействующих полостей}
	}
\end{figure}

Такая система описывается моделью Тависа-Каммингса-Хаббарда \cite{tch_photon_blockade,tch_transfer,tch_quality} и ее га­мильтониан имеет следующий вид:
\[
H_{\text{TCH}}~=~\sum_{j=1}^J \left(\underbrace{\hbar w_{c_{j}}\ a_j^+a_{j}}_{\textstyle H_{field}}~+~\underbrace{\hbar w_{a_{j}}\sum_{j_{i}=1}^{N_{j}}{\sigma_{j_i}^+\sigma_{j_i}}}_{\textstyle H_{atoms}}~+~\underbrace{\sum_{j_{i}=1}^{N_{j}}g_{j_i}{(\sigma_{j_i}^{+}+\sigma_{j_i})(a_{j}^{+}+a_{j})}}_{\textstyle H_{interaction}} \right)
\]
\begin{equation}
	+k_{j,j+1}\sum_{j=1}^{J-1}{\left(a_{j+1}^+a_j+a_{j}^+a_{j+1}\right)},
\end{equation}
\begin{quote}
	$J$ -- количество взаимодействующих полостей (члены с индексом $j$ соответствуют $j$-й полости),\\
	$N_{j}$ -- количество атомов в $j$-й полости,\\
	$\sigma_{j_i}^{+},\sigma_{j_i}$ -- операторы возбуждения/релаксакции $i$-го атома в $j$-й полости,\\
	$k_{j,j+1}$ -- интенсивность перелета фотона между $j$-й и $(j+1)$-й поло­стями.
\end{quote}

В приближении вращающейся волны (RWA) \cite{rwa_rabi_1,rwa_rabi_2,ozhigov_qq} гамильтониан TCH определяется естественным образом:
\[
H_{\text{TCH}}^{\text{RWA}}~=~\sum_{j=1}^J \left(\underbrace{\hbar w_{c_j}\ a_j^+a_j}_{\textstyle H_{field}}~+~\underbrace{\hbar w_{a_j}\sum_{j_{i}=1}^{N_{j}}{\sigma_{j_i}^+\sigma_{j_i}}}_{\textstyle H_{atoms}}~+~\underbrace{\sum_{j_{i}=1}^{N_{j}}g_{j_i}{(\sigma_{j_i}^+a_j+\sigma_{j_i}a_j^+)}}_{\textstyle H_{interaction}} \right)
\]

\begin{equation}
	+k_{j,j+1}\sum_{j=1}^{J-1}{\left(a_{j+1}^+a_j+a_{j}^+a_{j+1}\right)}.
\end{equation}

\clearpage
\indent\textbf{Открытая квантовая система. Основное квантовое уравнение.}

\textbf{Чистое квантовое состояние} изолированной системы описывается ком­плекснозначной волновой функцией \textbf{волновой функцией} $|\Psi\ra$. Оно характеризуется заданием полного набора возможных значений динамических переменных, определяющих состояние системы.

Временн$\acute{\text{a}}$я эволюция волновой функции чистого состояния системы описывается уравнением Шредингера:
\begin{equation}\label{schrodinger}
	i\hbar\frac{\partial}{\partial t}\Psi(r,t) = H\Psi(r,t),
\end{equation}
где $\hbar$ --- постоянная Планка, $\Psi(r,t)$ --- волновая функция системы, $H$ --- гамильтониан системы, определяющий ее полную энергию.

В действительности же квантовая система тесно окружена различными частицами и электромагнитным полем разных мод. В частности, отметим, что в рамках описанных выше конечномерных моделей КЭД атом в полости взаимодействует с фотоном ограниченное время, так как время жизни фотона в полости невелико. При контакте с окружающей средой чистое состояние $|\Psi\ra$ квантовой системы превращается в \textbf{смешанное состояние}, которое уже нельзя описать как вектор состояния $ |\Psi\ra$, но можно описать, используя \textbf{матрицу плотности}. Предложенный Л. Ландау, Дж. фон Нейманом и Ф. Блохом формализм \textbf{матрицы плотности} \cite{landau,belousov,messia} позволяет описывать не только чистые состояния (которые можно представить в виде векторов состояний в гильбертовом пространстве), но и смешанные, представляющие собой статистическую смесь различных чистых состояний.

\textbf{Открытые квантовые системы} \cite{breuer} представляют собой системы, ко­ торые взаимодействуют с окружением, что приводит к обмену энергией и информацией между системами и их окружением. В отличие от замкнутых квантовых систем, открытые квантовые системы не могут быть полностью опи­ саны уравнением Шредингера \eqref{schrodinger} по причине взаимодействия последних с внешней средой.

Матрица плотности является мощным инструментом для описания смешанных состояний, которые типичны для открытых квантовых систем. Наиболее подходящим для них описанием является \textbf{основное квантовое уравнение} \cite{breuer}, которое учитывает диссипативные процессы и декогеренцию:
\begin{equation}\label{klgs}
\begin{gathered}
	\frac{d\rho}{dt} = -\frac{i}{\hbar}[H, \rho] + \mathcal{L}(\rho),\\[12pt]
	{\mathcal L}(\rho)=\sum_{k} l_{k}\left(L_{k} \rho L_{k}^{\dagger} - \frac{1}{2} \{L_{k}^{\dagger}L_{k}, \rho \} \right),
\end{gathered}
\end{equation}
где $L_{k}$ --- операторы Линдблада, которые описывают взаимодействие системы с окружающей средой (называемые также \textbf{факторами декогеренции}), $l_{k}$ --- интенсивности соответствующих факторов декогеренции $L_{k}$. Квадратные скобки означают коммутатор, фигурные скобки --- антикоммутатор.

В отсутствие взаимодействия с окружением временн$\acute{\text{a}}$я эволюция матрицы плотности $\rho_{\Psi} = |\Psi\ra\la\Psi|$, связанной с состоянием $|\Psi\ra$, задается уравнением фон Неймана:
\begin{equation}\label{neumann}
	\frac{d\rho}{dt} = -\frac{i}{\hbar}[H, \rho].
\end{equation}
\noindent Это уравнение описывает эволюцию замкнутой квантовой системы, взаимодей­ствие которой с внешним окружением отсутствует, либо пренебрежимо мало. Оно эквивалентно уравнению Шредингера \eqref{schrodinger} и легко выводится из него.

Уравнение \eqref{klgs} называется основным квантовым уравнением Ко­саковского–Линдблада–Глаубера–Сударшана \cite{breuer}. Оно является обобще­нием уравнения \eqref{neumann} и описывает изменение во времени матрицы плотности системы, взаимодействующей со стационарной средой, не имеющей долговре­менной памяти.

Задача компьютерного моделирования — нахождение численного решения $\rho(t)$ уравнения \eqref{klgs}, которое можно произвести с использованием различных численных методов: как простых (наподобие метода Эйлера), так и более точ­ных (наподобие метода Рунге-Кутты).

Анализ открытых квантовых систем играет ключевую роль во многих современных квантовых технологиях, включая квантовые вычисления и квантовую криптографию \cite{akm2017}, где контролируемое взаимодействие с окру­жением используется для защиты квантовой информации от декогеренции. Открытые квантовые системы являются объектом активного исследования в со­временной физике. Их изучение позволяет понять сложные квантовые явления, которые возникают в условиях реального взаимодействия системы с окружаю­щей средой, тем самым помогая нам расширить понимание фундаментальных принципов квантовой механики.

\clearpage
\indent\textbf{Темные состояния.}

Особый интерес представляет описание так называемых \textbf{темных состояний} \cite{kok,ozhigov_dimension} атомных ансамблей модели Тависа-Каммингса.

Атомный ансамбль, находящийся в \textbf{темном состоянии}, не взаимодействует с полем, и потому способен сохранять свою энергию, не испуская ее в виде фотонов. Простейший вид \textbf{темного состояния} --- состояние 
\begin{equation}\label{intro_ds}
	\frac{1}{\sqrt{2}}(|01\ra - |01\ra),
\end{equation}
именуемое также \textbf{синглетом}.

Темные состояния не подвержены декогерентности, поэтому изучение их структуры и практическое получение важны для развития нанотехнологий и квантовых вычислений в целом. В главе \ref{ch:ch3} настоящей работы предложен до­статочно простой в технической реализации метод оптического отбора темных состояний, основанный на томографии электромагнитного поля вне полости. Алгебраическое описание темных состояний ансамблей многоуровневых атомов можно найти в работе \cite{kok}, однако их явный вид был найден только для двух­ уровневых атомов. В работе \cite{ozhigov_dimension} Ю. Ожиговым было доказано, что размерность темного подпространства пространства $n$ двухуровневых атомов равна
\[
dim(D_{n}^2) =	
\begin{cases}
	C_{n}^{k} - C_{n}^{k-1}~\text{при}~n = 2k, \\
	0, \text{в противном случае},
\end{cases}
\]
и все наборы темных состояний (для четного числа атомов в группе) с учетом нормировки имеют вид 
\[
\frac{1}{2^{n/4}}\bigotimes_{j=1}^{n/2}(|01\ra_{j}-|10\ra_{j}),
\]
где индекс $j$ означает номер пары $j = 1,\dots, n/2$ при произвольном разбиении группы из $n$ атомов. Однако уже для трехуровневых атомов имеется лишь ги­потеза о структруре темных состояний, которая была численно подтверждена для весьма ограниченного числа атомов в рамках данной работы (глава \ref{ch:ch4}).

Темные состояния широко используются в квантовой криптографии: использование синглетных состояний дает дополнительные преимущества в обеспечении устойчивости квантовых протоколов. Так, например, ключевое ме­ сто в построении протокола AKM2017 \cite{akm2017} занимает подготовка синглетного состояния вида \eqref{intro_ds}.

\clearpage
\textbf{Новизна работы}\\
\indent Собственные состояния гамильтониана Тависа-Каммингса исследовались в диссертации Тависа \cite{tc_exact_solution} 1968 года и нигде не публиковалось. Данное им опи­сание использует уже устаревшие численные методы, оно весьма громоздко и не дает ответа на вопрос ни о явной структуре, ни о пути практического по­лучения таких состояний.

Для ансамблей двухуровневых атомов структура и размерность темного подпространства была найдена в 2018-2020 годах \cite{ozhigov_dimension}, однако обобщить найден­ный метод уже на трехуровневые атомы до сих пор не удалось.

Новизна данного круга задач состоит в том, что прямые методы линейной алгебры требуют объемов оперативной памяти, растущих экспоненциально с ростом частиц в ансамбле, поэтому такие методы не могут быть применены для практически значимых задач даже при их суперкомпьютерной реализации. Здесь нужны новые математические и программные средства, учитывающие физический смысл задач и идеологию квантовых вычислений.

В данной диссертации был разработан и программно реализован специ­ альный алгоритм для суперкомпьютера, с помощью которого численно была определена размерность темного подпространства ансамблей, состоящих из двух десятков трехуровневых атомов.

Предложенный в работе алгоритм получения темных состояний много­ уровневых атомов методом оптического отбора --- новый. Он существенно превосходит ранее предложенный Ю. И. Ожиговым метод, основанный на эффекте Штарка-Зеемана \cite{stark,zeeman_1,zeeman_2,zeeman_3}. Метод оптического отбора, в силу его тех­нической простоты, можно использовать для получения темных состояний различных видов.

Математические и программные методы, разработанные в диссерта­ции, позволили детально оценить качество запутывающего квантового гейта coCSign, предложенного в 2020 году Ю. И. Ожиговым \cite{quantum_gates_asynchronous}. Данный гейт ис­пользует одну вспомогательную оптическую полость вместо двух, как в гейте Х. Азумы \cite{azuma} (2010 год). В близкой по идеологии работе Азумы \cite{azuma} также нет полного анализа влияния декогерентности, в частности, неточности гейта даже для идеальных полостей, что сделано в данной диссертации.

\clearpage
\textbf{Апробация работы}

Основные результаты диссертационной работы были представлены на следующих конференциях и научных семинарах:
\\[12pt]
R1. Кулагин А. В. Темные состояния и квантовые эффекты в контексте ис­следования конечномерных моделей (по материалам диссертации на соискание степени кандидата физико-математических наук) // Физико-технологический институт К. А. Валиева РАН, 2 апреля 2024 (научный семинар)
\\[12pt]
R2. Кулагин А. В. Компьютерное моделирование квантовых эффектов в конечномерных моделях (по материалам диссертации на соискание степени кандидата физико-математических наук) // Физико-технологический институт К. А. Валиева РАН, 23 ноября 2023 (научный семинар)
\\[12pt]
R3. Кулагин А. В., Афанасьев В. И., Ли В., Кэли Чж., Мяо Х.-Х., Плужников И., Ожигов Ю. И., Викторова Н. Б. Химический квантовый компьютер // Ломоносовские чтения 2021, секция <<Вычислительная математика и киберне­тика>>, Москва, Россия, 20-29 апреля 2021
\\[12pt]
R4. Ozhigov Y., Kulagin A., Afanasiev V., Keli Zh., Li V., Miao H.-H. About chemical modifications of finite dimensional models of QED // Quantum Informatics 2021, Москва, Россия, 30 марта - 4 апреля 2021
\\[12pt]
R5. Dull R., Ozhigov Y., Kulagin A., Li V., Miao H.-H., Keli Zh. Quality of quantum control by Tavis-Cummings-Hubbard model // Quantum Informatics 2021, Москва, Россия, 30 марта - 4 апреля 2021
\\[12pt]
R6. Kulagin A., Ozhigov Y. Realization of algorithm GSA on the asynchronous atomic excitations // Quantum Informatics 2021, Москва, Россия, 30 марта - 4 апреля 2021
\\[12pt]
R7. Ожигов Ю. И., Кулагин А. В., Ли В., Кэли Чж., Мяо Х.-Х., Дюль Р. Управление атомными ансамблями в модели Тависа-Каммингса-Хаббарда // Тихоновские чтения 2020, МГУ имени М. В. Ломоносова, Москва, Россия, 26-31 октября 2020
\\[12pt]
R8. Сковорода Н. А., Ожигов Ю. И., Ладунов В. Ю., Викторова Н. Б., Кулагин А. В. Ансамбли возбужденных атомов в одномодовых резонаторах // Ломо­носовские чтения 2019, секция <<Вычислительная математика и кибернетика>>, МГУ имени М. В. Ломоносова, Москва, Россия, 15-25 апреля 2019
\\[12pt]
R9. Ожигов Ю. И., Сковорода Н. А., Кулагин А. В., Ладунов В. Ю. Компью­терное моделирование системы зарядов и поля в конечных моделях КЭД // Ломоносовские чтения 2018, секция <<Вычислительная математика и киберне­тика>>, МГУ имени М. В. Ломоносова, Москва, Россия, 16-27 апреля 2018.
\
\\[18pt]
\indent\textbf{Публикации.}

Основные положения и выводы диссертационного исследования в полной мере изложены в 8 печатных работах, из них 7 статей в рецензируемых журна­лах [A1-A7] и 1 статья в сборнике трудов конференции [A8].
\
\\[18pt]
\indent\textbf{Личный вклад автора.}

Все результаты работы, включая предложенный метод оптического отбора темных состояний, получены автором полностью самостоятельно, опубликова­ны в рецензируемых журналах и были представлены на научных конференциях. В публикации [A4] автором был представлен алгоритм определения размерно­сти темного подпространства ансамблей многоуровневых атомов, основанный на редукции сверхбольших графов. Построение алгоритмов, разработка ком­плексов программ и все численные расчеты проведены автором также само­стоятельно. Автор благодарит научного руководителя за постановку задач и обсуждение работы на различных этапах.
\
\\[18pt]
\indent\textbf{Объем и структура работы.}

Диссертация состоит из~введения,
\formbytotal{totalchapter}{глав}{ы}{}{}, заключения и приложения.
%\formbytotal{totalappendix}{приложен}{ия}{ий}{}.
%% на случай ошибок оставляю исходный кусок на месте, закомментированным
%Полный объём диссертации составляет  \ref*{TotPages}~страницу
%с~\totalfigures{}~рисунками и~\totaltables{}~таблицами. Список литературы
%содержит \total{citenum}~наименований.
%
Полный объем диссертации составляет
\formbytotal{TotPages}{страниц}{у}{ы}{}, включая
\formbytotal{totalcount@figure}{рисун}{ок}{ка}{ков}.
%и \formbytotal{totalcount@table}{таблиц}{у}{ы}{}.
Список литературы содержит \formbytotal{citenum}{наименован}{ие}{ия}{ий}.

%\indent\textbf{Теоретическая и практическая значимость.}
%\begin{itemize}
%\item[$\bullet$]{математические и программные средства для компьютерного и суперкомпьтерного моделирования квантовых процессов в %конечномерных моделях, разработанные в диссертации, позволяют понять механизм часто контр-интуитивных квантовых эффектов, которые невозможно %предсказать на основе стандартных математических методов квантовой теории, развитых в 20 веке. Примером может быть найденный пикообразный %характер ансамблевых осцилляций групп атомов в разных оптических полостях,}
%\item[$\bullet$]{предложенный в диссертации метод оценки качества квантовых гейтов позволяет дать более точный и реалистичный прогноз их %построения, чем чисто математические приемы, в частности, он позволяет оценить число кубитов, для которых возможна уверенная реализация %ключевого квантового алгоритма Гровера,}
%\item[$\bullet$]{предложенный алгоритм определения структуры темных состояний может использоваться для аналогичных задач в более сложных %системах, например, в конечномерных моделях химии. Предложенный метод получения темных состояний важен для практического их применения в %наноустройствах и квантовой криптографии.}
%\end{itemize}
%\
%\\[0pt]

\nocite{vynberg}

\nocite{kossakowski}
\nocite{lindblad}

\nocite{quantum_computing_about}

\nocite{quantum_simulation_tc}
\nocite{quantum_simulation_homogeneous}

%---------------------------------------------
\nocite{chemistry_backend}
\nocite{chemistry_foundations}
\nocite{chemistry_quantum_simulation}
\nocite{chemistry_simulation_of_chemical_reaction}
\nocite{chemistry_Simulator_for_quantum_computer}
\nocite{chin}
\nocite{dark_state_three_level}
\nocite{highly}
\nocite{miao}
\nocite{mpi_complete_reference}
\nocite{non_markovianity}
\nocite{ozhigov_qubit_model}
\nocite{pkok}
\nocite{quantum_chemistry_age}
\nocite{robey_zamora}
\nocite{shpakovsky}
\nocite{using_mpi}
\nocite{vibrations_quanta_biology}
%\nocite{zeno_1}
%\nocite{zeno_2}
%\nocite{zeno_3}
%\nocite{zeno_4}
    % Введение
\ifnumequal{\value{contnumfig}}{1}{\counterwithout{figure}{chapter}
}{\counterwithin{figure}{chapter}}
\ifnumequal{\value{contnumtab}}{1}{\counterwithout{table}{chapter}
}{\counterwithin{table}{chapter}}
\chapter{Коллективные осцилляции многоатомных ансамблей}\label{ch:ch1}
В данной главе мы рассмотрим важный тип динамики --- осцилляции между возбужденным состоянием атомного ансамбля и поля, при которых вся энергия системы переходит от атомов к полю и обратно с большой амплитудой. В многоатомном ансамбле имеется множество состояний, по которым амплитуда распределяется в ходе процесса так, что ее концентрация на строго неравновесных состояниях не является очевидной и не может быть непосредственно выведена из вида гамильтониана. Само существование таких осцилляций не следует естественным образом из подобных осцилляций Раби для одного атома \cite{rabi_1,rabi_2,rabi_3}. В случае сложных систем такие осцилляции могут быть проанализированы только в результате численного компьютерного и суперкомпьютерного моделирования.

\section{Постановка задачи}\label{sec:ch1/sec1}
\noindent Пусть ансамбль двухуровневых атомов разделен на две равные половины $A_1$ и $A_2$ с номерами $1,2,...,n/2$ и $(n/2+1),\dots,n$ соответственно и первоначально в полости находится $m$ фотонов (<<накачка>>).
\\[12pt]
Рассмотрим два состояния
\begin{equation}\label{main}
	\begin{split}
		|\Psi_0\ra=|m\ra_{\mathrm{ph}}|0\dots0\ra_{\mathrm{A_{1}}} |1\dots1\ra_{\mathrm{A_{2}}},\\
		|\Psi_1\ra=|m\ra_{\mathrm{ph}}|1\dots1\ra_{\mathrm{A_{1}}} |0\dots0\ra_{\mathrm{A_{2}}}.
	\end{split}
\end{equation}
Переходы между этими состояниями в ходе унитарной динамики называются \textbf{ансамблевыми осцилляциями}. Качество таких осцилляций определяется естественным образом с помощью функций 
\begin{equation}
	\begin{split}
		\displaystyle f_0(t)=|\la \Psi_0 |\Psi(t)\ra |^2,\\f_1(t)=|\la \Psi_1 |\Psi(t)\ra |^2,
	\end{split}
\end{equation}
соответствующих вероятностям получения данных состояний при измерении состояния $|\Psi(t)\ra$ системы в момент времени $t$. В случае неунитарной динамики смешанного состояния $\rho(t)$ вместо функций $f_0, \ f_1$ следует использовать \textbf{функции согласованности (fidelity, или точность Ульмана-Йожи --- Uhlmann-Jozsa fidelity \cite{fidelity_1,fidelity_2,fidelity_3,fidelity_4})}
\[
F_0(t)=\biggl(Tr\sqrt{\sqrt{\rho_0}\ \rho(t)\ \sqrt{\rho_0}}\biggr)^{2}
\]
\begin{center}и\end{center}
\[
F_1(t)=\biggl(Tr\sqrt{\sqrt{\rho_1}\ \rho(t)\ \sqrt{\rho_1}}\biggr)^{2}
\]
соответственно, где $\rho_0=|\Psi_0\ra\la\Psi_0|,\ \rho_1=|\Psi_1\ra\la\Psi_1|$. 

Периодический повтор во времени состояний с близкими к единице показателями качества является критерием наличия ансамблевых осцилляций. Такие точки мы называем \textbf{пиковыми}. Важна также резкость осцилляций --- скорость изменения функции качества в окрестности пиковых точек. Из вида гамильтониана TC вытекает, что вектор чистого состояния при унитарной динамике меняется в пределах подпространства, порожденного базисными состояниями вида $|m\ra_{\mathrm{ph}}|n_1\ra_{\mathrm{A_{1}}}|n_2\ra_{\mathrm{A_{2}}}$, такими что $m+ n_{1}+n_{2}=p$. Его размерность растет экспоненциально от $p$, и потому само существование ансамблевых осцилляций, в отличие от осцилляций Раби \cite{rabi_1,rabi_2,rabi_3}, является совершенно не тривиальным фактом. 

Мы исследуем специфику таких осцилляций, а также промоделируем эф­фект их возрождения в системе двух взаимодействующих полостей.

\section{Выбор базиса, построение гамильтониана}\label{sec:ch1/sec2}
Пусть $\mathrm{A} = \{1, 2, \dots, n\}$ --- множество идентичных двухуровневых атомов, $J(\mathrm{A})$ --- множество их классических базисных состояний. Определим равномерное состояние атомов из $\mathrm{A}$ как
\begin{equation}
	\label{1.1}
	\{k_{\mathrm{A}}\Succ\footnote{для обозначений $\{\mathlarger{\Succ}$, $\mathlarger{\Prec}\}$ будем использовать кет-бра формализм} = \frac{1}{\sqrt{C_{n}^{k}}}\sum_{j \in J(\mathrm{A}):\ w(j) = k}|j\ra,
\end{equation}
где $w(j)$ --- вес Хэмминга двоичного кортежа $j$, равный числу единиц в нем, что соответствует количеству возбужденных атомов в состоянии $|j\ra$. 
Отметим также, что введенные таким образом многоатомные состояния взаимно ортогональны в силу того, что не имеют общих базисных компонент.

Пусть $\mathcal{L}_{\mathrm{A}}$ --- линейная оболочка состояний вида $|\mathrm{ph}\ra\{k_{\mathrm{A}}\Succ\ \ \forall k = \overline{0,|\mathrm{A}|},\  \mathrm{ph} = 0, 1, ...$, где $|\mathrm{ph}\ra$ --- фоковское состояние поля \cite{landau,belousov,messia} (число фотонов) в полости оптического резонатора. Рассмотрим ограничение $\tilde{H}_{\mathrm{TC}}^{\mathrm{A}}$ гамильтониана Тависа-Каммингса $H_{\mathrm{TC}}$ на подпространство $\mathcal{L}_{\mathrm{A}}$. Этот гамильтониан может быть представлен в новом базисе, состоящем из равномерных состояний. Его матрица будет иметь вид
\begin{equation}
	\label{1.2}
	\tilde{H}_{i_{\mathrm{ph}}i_{\mathrm{A}}j_{\mathrm{ph}}j_{\mathrm{A}}} = \Prec i_{\mathrm{A}}\}\la i_{\mathrm{ph}}| H_{\mathrm{TC}} | j_{\mathrm{ph}} \ra \{j_{\mathrm{A}}\Succ,
\end{equation}
где $i_{\mathrm{ph}}, i_{\mathrm{A}}$ и $j_{\mathrm{ph}}, j_{\mathrm{A}}$ --- начальное и конечное число фотонов и атомных возбуждений соответственно. Отметим также, что ненулевые элементы этой матрицы должны удовлетворять условию
\begin{equation}
	\label{1.3}
	i_{\mathrm{ph}} - j_{\mathrm{ph}} = i_{\mathrm{A}} - j_{\mathrm{A}} = \pm 1,
\end{equation}
поскольку взаимодействие атомов с модой электромагнитного поля резонатора осуществляется без потерь.

Выразим эти элементы через числа
\begin{equation}
	\label{1.4}
	p = \mathrm{min}(i_{\mathrm{ph}}, j_{\mathrm{ph}}),\quad a = \mathrm{min}(i_{\mathrm{A}}, j_{\mathrm{A}}).
\end{equation}

Тогда имеем
\begin{equation}
	\label{1.5}
	\tilde{H}_{i_{\mathrm{ph}}i_{\mathrm{A}}j_{\mathrm{ph}}j_{\mathrm{A}}} = 
	\begin{cases}
		\hbar w(p+a), & \text{если $i_{\mathrm{A}} = j_{\mathrm{A}}, i_{\mathrm{ph}} = j_{\mathrm{ph}}$,} \\
		g(n-a)C_{n}^{a}\sqrt{\cfrac{p+1}{C_{n}^{a}C_{n}^{a+1}}}, & \text{в противном случае.}
	\end{cases}
\end{equation}

Значение $\hbar w(p+a)$ соответствует диагональным элементам и равно энергии данного состояния (<<поле>>+<<атомы>>).

Для второго равенства:
\begin{itemize}
	\item[$\diamond$]{
		коэффициент $\sqrt{p+1}$ следует из вида операторов рождения/уничтожения фотонов: 
		\begin{itemize}
			\item{$p = \mathrm{min}(i_{\mathrm{ph}}, j_{\mathrm{ph}})$,}
			\item{$a^{+}|\mathrm{ph}\ra = \sqrt{\mathrm{ph} + 1}|\mathrm{ph} + 1\ra$,}
			\item{$a|\mathrm{ph}\ra = \sqrt{\mathrm{ph}}|\mathrm{ph-1}\ra$,}
		\end{itemize}
	}
	\item[$\diamond$]{ 
		произведение биномиальных коэффициентов $C_{n}^{a}C_{n}^{a+1}$ берется из нормировочных констант в определении 	равномерных состояний (\ref{1.1})},
	\item[$\diamond$]{
		коэффициент $(n-a)$ есть число слагаемых в сумме, равное числу возможных атомов, релаксирующих в данном процессе.}
\end{itemize}

Так, для системы из $n$ идентичных двухуровневых атомов можно ввести состояние атомного ансамбля --- состояние $|k_{1}k_{2}\dots k_{n}\ra$ с уровнями $\{0\Succ, \{1\Succ, \dots, \{n\Succ$. Уровни будут соответствовать суммарному числу возбуждений в ансамбле и будут отстоять друг от друга на величину $E_{k} - E_{k-1} = \hbar w,\ k = \overline{1,n}$.

В последующих главах мы будем рассматривать квантовую динамику в приближении RWA \cite{rwa_rabi_1,rwa_rabi_2,ozhigov_qq} состояний вида $|\Psi_0\ra=|m\ra_{\mathrm{ph}}|0\dots0\ra_{\mathrm{A_{1}}} |1\dots1\ra_{\mathrm{A_{2}}}$ в базисе $|m\ra_{\mathrm{ph}}\{i,j\Succ_{\mathrm{at}}$, где
\\
\null\qquad$|m\ra_{\mathrm{ph}}$ --- начальное фоковское (либо вакуумное) состояние поля \cite{landau,belousov,messia},
\\[6pt]
\null\qquad$\{i,j\Succ_{\mathrm{at}}\!=\!\{i\Succ_{\mathrm{A_{1}}}\{j\Succ_{\mathrm{A_{2}}}$ --- состояние (энергия возбуждения) атомных групп.
\
\\[0pt]
\indent При начальном состоянии системы $|\Psi_0\ra=|m\ra_{\mathrm{ph}}|0\dots0\ra_{\mathrm{A_{1}}} |1\dots1\ra_{\mathrm{A_{2}}}$ через определенный промежуток времени вся амплитуда возбуждений второй группы атомов сконцентрируется на возбуждении первой группы атомов --- состоянии $|\Psi_1\ra=|m\ra_{\mathrm{ph}}|1\dots1\ra_{\mathrm{A_{1}}} |0\dots0\ra_{\mathrm{A_{2}}}$, и данный процесс будет повторяться. Таким образом, мы получим картину коллективных атомных осцилляций ({\color{red}рис. 2.1, 2.2, 2.3, 2.5}), аналогичных осцилляциям Раби для одного атома \cite{rabi_1,rabi_2,rabi_3}. Причина такого поведения кроется в интерференции амплитуд: для состояний вида $|m\ra_{\mathrm{ph}}\{i,j\Succ_{\mathrm{at}}$, отличных от $|\Psi_{0}\ra$ и $|\Psi_{1}\ra$, их амплитуда будет распределяться по всем этим состояниям, снижая тем самым вероятность нахождения системы в каждом из них. Данное свойство позволяет в перспективе создать процедуру переноса многокубитного квантового состояния на некоторое удаленное расстояние весьма простым способом.

Отметим также, что с увеличением фотонной накачки пики амплитуд состояний $|\Psi_{0}\ra$ и $|\Psi_{1}\ra$ будут становиться все более резкими ({\color{red}рис. 2.1, 2.2, 2.3}): первоначальные, или свободные, фотоны в полости будут способствовать процессу осцилляций атомных групп.

В заключение главы мы также рассмотрим данную динамику в условиях, когда атомные группы разделены между полостями, взаимодействующими между собой посредством оптического волновода. Обнаруженное явление перехода состояния атомного ансамбля из одной полости в другую и его возврат в первоначальную полость ({\color{red}рис. 2.13, 2.14}) носит вполне отчетливый характер.

\clearpage
\subsection{Осцилляции с накачкой фотонов}
\vspace{-1em}
\begin{figure}[h!]
	\noindent\centering{
		\includegraphics[width=0.65\textwidth]{Dissertation/images/section_1/20_10/1g_0.05mks_1000nt/oscillations.png}\astfootnote\\
		\includegraphics[width=0.6\textwidth]{Dissertation/images/section_1/20_10/1g_0.05mks_1000nt/fid.png}
		\includegraphics[width=0.6\textwidth]{Dissertation/images/section_1/20_10/1g_5mks_1000nt/fid.png}
		\captionsetup{format=hang,width=0.8\textwidth,justification=centering,singlelinecheck=no}
		\\[6pt]
		\caption{
			Коллективные осцилляции для системы 10x10\\
			$|\Psi_0\ra = |10\ra_{\mathrm{ph}}\{0,10\Succ_{\mathrm{at}} = |10\ra_{\mathrm{ph}}|\protect\underbrace{0 \dots 0}_{10}\ra_{\mathrm{A}_1}|\protect\underbrace{1 \dots 1}_{10}\ra_{\mathrm{A}_2}$\\
			$|\Psi_1\ra = |10\ra_{\mathrm{ph}}\{10,0\Succ_{\mathrm{at}} = |10\ra_{\mathrm{ph}}|\protect\underbrace{1 \dots 1}_{10}\ra_{\mathrm{A}_1}|\protect\underbrace{0 \dots 0}_{10}\ra_{\mathrm{A}_2}$
		}
	}
\end{figure}
\extrafootertext{\hspace{-2em}{\color{red}*}на вертикальной оси --- вероятность $p(t)$ получения состояния $\{i,j\Succ_{\mathrm{at}}$ в момент времени $t$,\\\indent\hspace{-0.5em}на горизонтальных --- рассматриваемый временной интервал и состояния системы $|\Psi_0\ra$, $|\Psi_1\ra$}

\clearpage
\begin{figure}[h!]
	\noindent\centering{
		\includegraphics[width=0.7\textwidth]{Dissertation/images/section_1/50_25/1g_0.05mks_1000nt/oscillations.png}\astfootnote\\
		\includegraphics[width=0.65\textwidth]{Dissertation/images/section_1/50_25/1g_0.05mks_1000nt/fid.png}
		\includegraphics[width=0.65\textwidth]{Dissertation/images/section_1/50_25/1g_5mks_1000nt/fid.png}
		\captionsetup{format=hang,width=0.8\textwidth,justification=centering,singlelinecheck=no}
		\\[6pt]
		\caption{
			Коллективные осцилляции для системы 25x25\\
			$|\Psi_0\ra = |25\ra_{\mathrm{ph}}\{0,25\Succ_{\mathrm{at}} = |25\ra_{\mathrm{ph}}|\protect\underbrace{0 \dots 0}_{25}\ra_{\mathrm{A}_1}|\protect\underbrace{1 \dots 1}_{25}\ra_{\mathrm{A}_2}$\\
			$|\Psi_1\ra = |25\ra_{\mathrm{ph}}\{25,0\Succ_{\mathrm{at}} = |25\ra_{\mathrm{ph}}|\protect\underbrace{1 \dots 1}_{25}\ra_{\mathrm{A}_1}|\protect\underbrace{0 \dots 0}_{25}\ra_{\mathrm{A}_2}$
	}}
\end{figure}
\extrafootertext{\hspace{-2em}{\color{red}*}на вертикальной оси --- вероятность $p(t)$ получения состояния $\{i,j\Succ_{\mathrm{at}}$ в момент времени $t$,\\\indent\hspace{-0.5em}на горизонтальных --- рассматриваемый временной интервал и состояния системы $|\Psi_0\ra$, $|\Psi_1\ra$}

\clearpage
\subsection{Осцилляции произвольных атомных ансамблей}
Произвольные начальные состояния вида 
\begin{equation}\label{good_ensemble}
	\alpha|0\dots0\ra_{\mathrm{A}_1}|1\dots1\ra_{\mathrm{A}_2} + \beta|1\dots1\ra_{\mathrm{A}_1}|0\dots0\ra_{\mathrm{A}_2}
\end{equation}
дают осцилляции хорошего качества в условиях фотонной накачки.
\begin{figure}[h!]
	\noindent\centering{
		\includegraphics[width=0.6\textwidth]{Dissertation/images/section_1/20_10/random/1g_0.05mks_1000nt/oscillations.png}\astfootnote\\
		\includegraphics[width=0.5\textwidth]{Dissertation/images/section_1/20_10/random/1g_0.05mks_1000nt/fidelity.png}
		\captionsetup{format=hang,width=0.75\textwidth,justification=centering,singlelinecheck=no}
		\\[18pt]
		\caption{
			Коллективные осцилляции атомных ансамблей с произвольно выбранными амплитудами\\
			$|\Psi_0\ra=|10\ra_{\mathrm{ph}}\otimes(\alpha\{0\Succ_{\mathrm{A}_1}\{10\Succ_{\mathrm{A}_2}+\beta\{10\Succ_{\mathrm{A}_1}\{0\Succ_{\mathrm{A}_2})$\\
			$|\Psi_1\ra=|10\ra_{\mathrm{ph}}\otimes(\alpha\{10\Succ_{\mathrm{A}_1}\{0\Succ_{\mathrm{A}_2}+\beta\{0\Succ_{\mathrm{A}_1}\{10\Succ_{\mathrm{A}_2})$\\
			$\alpha=0.293+0.245i,\quad\beta=0.204+0.901i$
	}}
\end{figure}
\extrafootertext{\hspace{-2em}{\color{red}*}на вертикальной оси --- вероятность $p(t)$ получения состояния $\{i,j\Succ_{\mathrm{at}}$ в момент времени $t$,\\\indent\hspace{-0.5em}на горизонтальных --- рассматриваемый временной интервал и состояния системы $|\Psi_0\ra$, $|\Psi_1\ra$}

\clearpage
Произвольные начальные состояния вида
\[
|\Psi_{0}\ra = |m\ra_{\text{ph}}\sum_{\substack{\forall\lambda_{k}:~\sum_{k} \lambda_{k} = 1,\\ i,j:~i+j = 2n}}\lambda_{k}\{i\Succ_{\mathrm{A_{1}}}\{j\Succ_{\mathrm{A_{2}}},
\]
отличные от состояний \eqref{good_ensemble}, не осциллируют.

\begin{figure}[ht!]
	\noindent\centering{
		\includegraphics[width=0.5\textwidth]{Dissertation/images/section_1/bad_random_5.png}\astfootnote\\
		\includegraphics[width=0.5\textwidth]{Dissertation/images/section_1/bad_random_10.png}\astfootnote\\
		\captionsetup{format=hang,width=0.8\textwidth,justification=centering,singlelinecheck=no}
		\\[18pt]
		\caption{
			Отсутствие осцилляций групп\\с произвольно выбранными в них изначальными\\количествами атомных возбуждений
	}}
\end{figure}
\extrafootertext{\hspace{-2em}{\color{red}*}на вертикальной оси --- вероятность $p(t)$ получения состояния $\{i,j\Succ_{\mathrm{at}}$ в момент времени $t$,\\\indent\hspace{-0.5em}на горизонтальных --- рассматриваемый временной интервал и состояния системы $|\Psi_0\ra$, $|\Psi_1\ra$}

\clearpage
\subsection{Осцилляции без накачки фотонов: коллапсы и возрождения ансамблевых состояний}
\vspace{-2em}
\noindent Унитарная динамика базисных состояний 
\[
\begin{split}
	|\Psi_0\ra = |0\ra_{\mathrm{ph}}\{0,n\Succ_{\mathrm{at}} = |0\ra_{\mathrm{ph}}\{0\Succ_{\mathrm{A}_1}\{n\Succ_{\mathrm{A}_2} = |0\ra_{\mathrm{ph}}|\protect\underbrace{0 \dots 0}_{n}\ra_{\mathrm{A}_1}|\protect\underbrace{1 \dots 1}_{n}\ra_{\mathrm{A}_2},\\
	|\Psi_1\ra = |0\ra_{\mathrm{ph}}\{n,0\Succ_{\mathrm{at}} = |0\ra_{\mathrm{ph}}\{n\Succ_{\mathrm{A}_1}\{0\Succ_{\mathrm{A}_2} = |0\ra_{\mathrm{ph}}|\protect\underbrace{1 \dots 1}_{n}\ra_{\mathrm{A}_1}|\protect\underbrace{0 \dots 0}_{n}\ra_{\mathrm{A}_2}\ 
\end{split}
\] в отсутствие свободных фотонов в полости будет сопровождаться периодическими коллапсами и возрождениями состояний: пиковые значения меры близости $f(t_{rev})=|\la \Psi_0 |\Psi(t_{rev})\ra|^2$ начального $|\Psi_0\ra$ и <<возрожденного>> начального состояния $|\Psi(t_{rev})\ra$ в момент времени $t_{rev}$ близки к 1.
\begin{figure}[h!]
	\noindent\centering{
		\includegraphics[width=0.54\textwidth]{Dissertation/images/section_1/5_5/1g_0.25mks_1000nt/oscillations.png}\astfootnote\\
		\includegraphics[width=0.49\textwidth]{Dissertation/images/section_1/5_5/1g_0.25mks_1000nt/fid.png}
		\includegraphics[width=0.49\textwidth]{Dissertation/images/section_1/5_5/1g_5mks_1000nt/fid.png}
		\captionsetup{format=hang,width=0.85\textwidth,justification=centering,singlelinecheck=no}
		\\[6pt]
		\caption{
			Коллапсы и возрождения ансамблевых состояний\\
			$|\Psi_0\ra = |0\ra_{\mathrm{ph}}\{0,5\Succ_{\mathrm{at}} = |0\ra_{\mathrm{ph}}|\protect\underbrace{0 \dots 0}_{5}\ra_{\mathrm{A}_1}|\protect\underbrace{1 \dots 1}_{5}\ra_{\mathrm{A}_2}$\\
			$|\Psi_1\ra = |0\ra_{\mathrm{ph}}\{5,0\Succ_{\mathrm{at}} = |0\ra_{\mathrm{ph}}|\protect\underbrace{1 \dots 1}_{5}\ra_{\mathrm{A}_1}|\protect\underbrace{0 \dots 0}_{5}\ra_{\mathrm{A}_2}$
	}}
\end{figure}
\extrafootertext{\hspace{-2em}{\color{red}*}на вертикальной оси --- вероятность $p(t)$ получения состояния $\{i,j\Succ_{\mathrm{at}}$ в момент времени $t$,\\\indent\hspace{-0.5em}на горизонтальных --- рассматриваемый временной интервал и состояния системы $|\Psi_0\ra$, $|\Psi_1\ra$}

\clearpage
\subsection{Зависимость качества осцилляций от числа атомов в группе и силы взаимодействия атомов с полем}
Следующие графики демонстрируют зависимость качества осцилляций от различных параметров квантовой системы.
\\[12pt]
\begin{figure}[h!]
	\noindent\centering{
		\includegraphics[width=0.95\textwidth]{Dissertation/images/section_1/bp_n/ph/1_5_15.png}
		\includegraphics[width=0.95\textwidth]{Dissertation/images/section_1/bp_n/ph/1_15.png}
		\captionsetup{format=hang,width=0.85\textwidth,justification=centering,singlelinecheck=no}
		\caption{
			Снижение пиковых амплитуд и нарушение\\периодичности осцилляций при увеличении числа атомов в группе\\
			$|\Psi_0\ra=|\mathrm{0}\ra_{\mathrm{ph}}\{0,n\Succ_{\mathrm{at}} = |0\ra_{\mathrm{ph}}|\protect\underbrace{0 \dots 0}_{n}\ra_{\mathrm{A}_1}|\protect\underbrace{1 \dots 1}_{n}\ra_{\mathrm{A}_2}$\\
			$|\Psi_1\ra=|\mathrm{0}\ra_{\mathrm{ph}}\{n,0\Succ_{\mathrm{at}} = |0\ra_{\mathrm{ph}}|\protect\underbrace{1 \dots 1}_{n}\ra_{\mathrm{A}_1}|\protect\underbrace{0 \dots 0}_{n}\ra_{\mathrm{A}_2}$
	}}
\end{figure}

\clearpage
\begin{figure}[h!]
	\noindent\centering{
		\includegraphics[width=0.45\textwidth]{Dissertation/images/section_1/Tk/Tgk_10_015_10000.png}
		\includegraphics[width=0.45\textwidth]{Dissertation/images/section_1/Tk/Tgk_20_015_10000.png}\\
		\hspace{24pt}$n = 10$\hspace{190pt}$n = 20$\\[12pt]
		\includegraphics[width=0.45\textwidth]{Dissertation/images/section_1/Tk/Tgk_30_015_10000.png}\\
		\hspace{24pt}$n = 30$\\[12pt]
		\captionsetup{format=hang,width=0.9\textwidth,justification=centering,singlelinecheck=no}
		\caption{
			Зависимость периода осцилляций $T$ от силы взаимодействия $g$ и первоначального числа $m$ фотонов в полости\\
			$|\Psi_0\ra=|m\ra_{\mathrm{ph}}\{0,n\Succ_{\mathrm{at}} = |m\ra_{\mathrm{ph}}|\protect\underbrace{0 \dots 0}_{n}\ra_{\mathrm{A}_1}|\protect\underbrace{1 \dots 1}_{n}\ra_{\mathrm{A}_2}$\\
			$|\Psi_1\ra=|m\ra_{\mathrm{ph}}\{n,0\Succ_{\mathrm{at}} = |m\ra_{\mathrm{ph}}|\protect\underbrace{1 \dots 1}_{n}\ra_{\mathrm{A}_1}|\protect\underbrace{0 \dots 0}_{n}\ra_{\mathrm{A}_2}$
		}
	}
\end{figure}

Увеличение силы взаимодействия $g$ атомов с полем, равно как и усиление фотонной накачки $m$, приводит к существенному уменьшению периода осцилляций. При стремлении $g$ к нулю период осцилляций стремится к бесконечности (значение $g=0$ означает отсутствие взаимодействия атомов с полем).

\clearpage
\begin{figure}[h!]
	\noindent\centering{
		\includegraphics[width=0.55\textwidth]{Dissertation/images/section_1/Tk/02_10000_3.png}\\
		\hspace{-5em}\small{период осцилляций $\mathrm{T=3~ns = const}$}
		\\[36pt]
		\includegraphics[width=0.55\textwidth]{Dissertation/images/section_1/Tk/015_10000_10.png}\\
		\hspace{-5em}\small{период осцилляций $\mathrm{T=10~ns = const}$}
		\\[18pt]
		\captionsetup{format=hang,width=0.9\textwidth,justification=centering,singlelinecheck=no}
		\caption{Удлинение периода осцилляций при уменьшении силы взаимодействия $g$ атомов с полем может быть скомпенсировано увеличением числа $n$ атомов в группе и усилением фотонной накачки
		}
	}
\end{figure}

Представленные изочастотные графики, соответствующие периодам осцилляций $\mathrm{T=3~ns}$ и $\mathrm{T=10~ns}$ для $n=10,~20,~30$ атомов в группе демонстрируют возможность наращивания числа атомов, а также возможность уменьшения силы взаимодействия атомов с полем с одновременным сохранением периода осцилляций.

\clearpage
\begin{figure}[h!]
	\noindent\centering{
		\includegraphics[width=0.8\textwidth]{Dissertation/images/section_1/Tk/g_10_015_10000_edit2.png}
		\captionsetup{format=hang,width=0.9\textwidth,justification=centering,singlelinecheck=no}
		\caption{Зависимость периода осцилляций от\\первоначального числа $m$ фотонов в полости\\для 10 атомов в группе при различных значениях $g$\\[12pt]
			Начальное состояние: $|\Psi_0\ra=|m\ra_{\mathrm{ph}}\{0,10\Succ_{\mathrm{at}} = |m\ra_{\mathrm{ph}}|\protect\underbrace{0 \dots 0}_{10}\ra_{\mathrm{A}_1}|\protect\underbrace{1 \dots 1}_{10}\ra_{\mathrm{A}_2}$
	}}
\end{figure}

\begin{figure}[h!]
	\noindent\centering{
		\includegraphics[width=0.8\textwidth]{Dissertation/images/section_1/Tk/g_20_015_10000_edit2.png}
		\captionsetup{format=hang,width=0.9\textwidth,justification=centering,singlelinecheck=no}
		\caption{Зависимость периода осцилляций от\\первоначального числа $m$ фотонов в полости\\для 20 атомов в группе при различных значениях $g$\\[12pt]
			Начальное состояние: $|\Psi_0\ra=|m\ra_{\mathrm{ph}}\{0,20\Succ_{\mathrm{at}} = |m\ra_{\mathrm{ph}}|\protect\underbrace{0 \dots 0}_{20}\ra_{\mathrm{A}_1}|\protect\underbrace{1 \dots 1}_{20}\ra_{\mathrm{A}_2}$
	}}
\end{figure}

Увеличение фотонной накачки в полости резонатора сопровождается асимптотическим стремлением к нулю периода осцилляций при различных значениях силы взаимодействия атомов с полем в границах приближения RWA.

\clearpage
\begin{figure}[h!]
	\noindent\centering{
		\includegraphics[width=0.73\textwidth]{Dissertation/images/section_1/bp_n/ph/w.png}
		\includegraphics[width=0.73\textwidth]{Dissertation/images/section_1/bp_n/ph/h.png}
		\includegraphics[width=0.73\textwidth]{Dissertation/images/section_1/bp_n/ph/w_h.png}
		\captionsetup{format=hang,width=1.0\textwidth,justification=centering,singlelinecheck=no}
		\caption{Зависимость средних значений ширины основания\\и высоты пика квадрата амплитуды <<возрожденного>>\\начального состояния от числа $n$ атомов в группе\\[12pt]Начальное состояние: $|\Psi_0\ra=|\mathrm{0}\ra_{\mathrm{ph}}\{0,n\Succ_{\mathrm{at}} = |0\ra_{\mathrm{ph}}|\protect\underbrace{0 \dots 0}_{n}\ra_{\mathrm{A}_1}|\protect\underbrace{1 \dots 1}_{n}\ra_{\mathrm{A}_2}$
	}}
\end{figure}

Наблюдается уменьшение средних значений ширины основания и высоты пика осцилляций при увеличении числа атомов в группе. Их отношение близко к константе.

\clearpage
\subsection{Модель Тависа-Каммингса: взаимодействие с внешним окружением}

Рассмотрим модель Тависа-Каммингса в приближении RWA \cite{rwa_rabi_1,rwa_rabi_2,ozhigov_qq} для $n$ двухуровневых атомов, имеющих разные координаты внутри полости. Пусть энергии возбуждения атомов равны $\hbar w_{a_{i}}$ и отличаются от энергии фотонов $\hbar w_{c}$ малой расстройкой $d_i=|\hbar w_{c}-\hbar w_{a_{i}}|$. Гамильтониан такой системы имеет вид
\begin{equation}
	\label{TC}
	H_{\text{TC}}^{\text{RWA}}=\hbar w_{c} a^+a+\hbar \sum\limits_{i=1}^{n}w_{a_{i}}\s^{+}_i\s_i+a\bar\s^{+}+a^{+}\bar\s,
\end{equation}
где $\bar\s=\sum\limits_{i=1}^{n}g_{i}\s_{i}$, $\bar\s^{+}=\sum\limits_{i=1}^ng_{i}\s^{+}_{i}$ --- соответствующие операторы коллективной релаксации и возбуждения группы атомов, силы взаимодействия которых с полем $g_{1},g_{2},\dots,g_{n}$, вообще говоря, различны. 

Реальный резонатор находится в контакте с внешней средой, которая в простейшем случае имеет вид одномодового фотонного резервуара с фиксированной температурой на моде полости. Контакт предполагает возможность обмена фотонами частоты $w_{c}$ между внешней средой и полостью. 

Рассмотрим простейший случай нулевой температуры на моде полости, при котором во внешнем резервуаре нет фотонов и такой контакт ведет к постоянной утечке фотонов из полости. Если $\rho(t)$ --- матрица плотности системы <<поле + атомы>>, то ее динамика в указанном случае описывается основным квантовым уравнением \cite{breuer}:
\begin{equation}\label{lindblad}
	\begin{gathered}
		i\hbar\dot{\rho}={\mathcal{L}}(\rho),\\
		{\mathcal{L}}(\rho)=-\frac{i}{\hbar}[H,\rho]+\frac{1}{\hbar}L(\rho),\\
		L(\rho)=\gamma\biggl(a\rho a^+-\frac{1}{2}\{a^+a,\rho\}\biggr),
	\end{gathered}
\end{equation}
где  $H=H_{\text{TC}}^{\text{RWA}}$, $\gamma$ --- параметр, отвечающий за интенсивность линдбладовского процесса, $L(\rho)=\gamma\bigl(a\rho a^{+}-\frac{1}{2}\{a^{+}a,\rho\}\bigr)$ --- оператор Линдблада, соответствующий утечке фотона, которая выражается оператором уничтожения фотона $a$ \cite{breuer,photon_emission}. 

Мы будем исследовать динамику состояния атомов и поля, идущую задолго до наступления термической стабилизации, рассматривая обмен фотонами с внешним резурвуаром как фактор декогерентности.

\clearpage
\subsection{Осцилляции c утечкой фотонов из полости}
Характерная картина ансамблевых осцилляций сохраняется при условии, когда резонатор находится в контакте с внешней средой, приводящем к постоянной утечке фотонов из полости.
\begin{figure}[h!]
	\noindent\centering{
		\includegraphics[width=0.7\textwidth]{Dissertation/images/section_1/bpl/5_5_1000ns.png}\astfootnote\\
		\includegraphics[width=0.65\textwidth]{Dissertation/images/section_1/bpl/5_5_5mks_fid.png}
		\captionsetup{format=hang,width=0.75\textwidth,justification=centering,singlelinecheck=no}
		\\[12pt]
		\caption{
			Осцилляции c утечкой фотонов из полости\\
			$|\Psi_0\ra=|0\ra_{\mathrm{ph}}\{0,5\Succ_{\mathrm{at}} = |0\ra_{\mathrm{ph}}|\protect\underbrace{0 \dots 0}_{5}\ra_{\mathrm{A}_1}|\protect\underbrace{1 \dots 1}_{5}\ra_{\mathrm{A}_2}$\\
			$|\Psi_1\ra=|0\ra_{\mathrm{ph}}\{5,0\Succ_{\mathrm{at}} = |0\ra_{\mathrm{ph}}|\protect\underbrace{1 \dots 1}_{5}\ra_{\mathrm{A}_1}|\protect\underbrace{0 \dots 0}_{5}\ra_{\mathrm{A}_2}$
	}}
\end{figure}
\extrafootertext{\hspace{-2em}{\color{red}*}на вертикальной оси --- вероятность $p(t)$ получения состояния $\{i,j\Succ_{\mathrm{at}}$ в момент времени $t$,\\\indent\hspace{-0.5em}на горизонтальных --- рассматриваемый временной интервал и состояния системы $|\Psi_0\ra$, $|\Psi_1\ra$}

\subsection{Возрождение состояний атомных ансамблей в системе оптических полостей}
Группы атомов $\mathrm{A}_1$ и $\mathrm{A}_2$, распределенные между полостями, дают хорошие осцилляции при достаточно большой амплитуде $\mu$ перехода возбуждений между полостями ($\mu>g$). 
\\[12pt]
Моделирование следующей динамики проводилось для 10 атомов в каждой полости при $\mu \approx 3.5g$.
\\[12pt]
Начальное состояние $|\Psi_0\ra=|\mathrm{vac}\ra_{\mathrm{ph}_\mathrm{I}}|\mathrm{vac}\ra_{\mathrm{ph}_{\mathrm{II}}}\{0, n\Succ_{\mathrm{at}}$, где
\\
\indent$|\mathrm{vac}\ra_{\mathrm{ph}_\mathrm{I}}$ --- начальное вакуумное состояние поля I полости,
\\[6pt]
\indent$|\mathrm{vac}\ra_{\mathrm{ph}_{\mathrm{II}}}$ --- начальное вакуумное состояние поля II полости,
\\[6pt]
\indent$\{i,j\Succ_{\mathrm{at}} = \{i\Succ_{\mathrm{I}}\{j\Succ_{\mathrm{II}}$ --- состояние атомных групп $\mathrm{A}_1, \mathrm{A}_2$.
\begin{figure}[h!]
	\noindent\centering{
		\includegraphics[width=0.55\textwidth]{Dissertation/images/section_1/bp2/10_10_250ns.png}
		\includegraphics[width=0.5\textwidth]{Dissertation/images/section_1/bp2/10_10_250ns_fid.png}
		\captionsetup{format=hang,width=0.8\textwidth,justification=centering,singlelinecheck=no}
		\\[6pt]
		\caption{$|\Psi_0\ra=|\mathrm{vac}\ra_{\mathrm{ph}_\mathrm{I}}|\mathrm{vac}\ra_{\mathrm{ph}_{\mathrm{II}}}\{0\Succ_{\mathrm{I}}\{10\Succ_{\mathrm{II}}$\\
			коллапсы и возрождения ансамблевых\\состояний между полостями
	}}
\end{figure}

\clearpage
Для 20 атомов в группе была численно найдена граница возникновения осцилляций: амплитуда перехода возбуждений между полостями, равная $\mu \approx 5g$.
\begin{figure}[h!]
	\noindent\centering{
		\includegraphics[width=0.7\textwidth]{Dissertation/images/section_1/bp2/20_20_250ns.png}
		\includegraphics[width=0.65\textwidth]{Dissertation/images/section_1/bp2/20_20_250ns_fid.png}
		\captionsetup{format=hang,width=0.8\textwidth,justification=centering,singlelinecheck=no}
		\\[12pt]
		\caption{$|\Psi_0\ra=|\mathrm{vac}\ra_{\mathrm{ph}_\mathrm{I}}|\mathrm{vac}\ra_{\mathrm{ph}_{\mathrm{II}}}\{0\Succ_{\mathrm{I}}\{20\Succ_{\mathrm{II}}$\\
			высокое качество осцилляций, близкое к 0.8,\\
			коллапсы и возрождения ансамблевых состояний}}
\end{figure}

Фактически такое возрождение квантовых состояний является прямым аналогом эффекта, обнаруженного и продемонстрированного экспериментально группой С. Моисеева \cite{moiseev_1,moiseev_2,moiseev_3,moiseev_4}, --- процесса поглощения пучка фотонов средой с последующим возвратом средой энергии в виде фотонного эха. Такого рода эхо представляет собой важный феномен, поскольку его можно использовать в качестве механизма организации временной квантовой памяти.

\clearpage
\subsection{Выводы главы}
В настоящей главе была рассмотрена динамика квантовых состояний ансамблей двухуровневых атомов и одномодового поля в резонаторе в рамках модели Тависа-Каммингса, а также модели Тависа-Каммингса-Хаббарда для случая двух взаимодействующих полостей в приближении RWA.

По результатам компьютерного моделирования:
\begin{itemize}
	\item{установлен резкий характер коллективных осцилляций между двумя группами атомов равной численности и равной силы взаимодействия атомов с полем,}
	\item{установлено, что резкость осцилляций в ансамбле с четным числом атомов предсказуемо растет с увеличением фотонной накачки в полости и намного превосходит резкость осцилляций Раби для одного атома},
	\item{численно найдена зависимость качества осцилляций от силы взаимодействия атомов с полем. Показано, что удлинение периода осцилляций при уменьшении силы взаимодействия может быть скомпенсировано увеличением числа атомов в группе и усилением фотонной накачки,}
	\item{обнаружено хорошо регистрируемое <<квантовое эхо>>: переход состояния атомного ансамбля из одной полости в другую и возврат этого
состояния в первоначальную полость. Установлено высокое качество такого эха. Была установлена граница его возникновения, выражающаяся через соотношение интенсивности перехода фотонов между полостями и силы взаимодействия атомов с полем, равная $\mu \approx 3.5g$ для 10 атомов и $\mu \approx 5g$ для 20 атомов в каждой полости.}
\end{itemize}

\indent Результаты, полученные для ансамблей, состоящих из 20 и более атомов в группе, необходимых для уверенной фиксации выявленного характера осцил­ляций, потребовали моделирования динамики как чистых, так и смешанных квантовых состояний и не могут быть получены на персональном компью­тере. Моделирование квантовой динамики производилось на графических процессорах (GPU) суперкомпьютера Ломоносов с применением технологии па­раллельных вычислений CUDA, а также пакета матричной алгебры MAGMA (Matrix Algebra on GPU and Multicore Architectures) \cite{magma}.
\\[12pt]
\noindent Листинг программной реализации представлен в \hyperref[appendix]{приложении}.
           % Глава 1
\chapter{Квантовое бутылочное горлышко в атомных превращениях}\label{ch:ch2}

В данной главе будет рассмотрен парадоксальный эффект квантового бутылочного горлышка для процесса интенсивного охлаждения атома, кото­рый переходит в необратимое состояние, находясь в возбужденном состоянии. Превышение некоторого порога интенсивности охлаждения ведет к росту веро­ятности такого перехода, что невозможно при классическом описании процесса.

\section{Постановка задачи}\label{sec:ch2/sec1}
Модель Тависа-Каммингса описывает динамику состояний группы двухуровневых атомов, взаимодействующей с одномодовым полем внутри оптической полости, удерживающей фотоны с энергией атомного возбуждения. Непосредственная экспериментальная реализация такой модели требует оптической полости, зеркала которой сделаны из сверхпроводящего материала (например, из ниобия), функционирующего при температуре жидкого гелия, и атомов, удерживаемых в определенной области внутри полости с помощью оптических пинцетов \cite{cavity_exp_1,cavity_exp_2,cavity_exp_3}. Так можно обеспечить длительность жизни фотона в полости, измеряемой несколькими десятками рабиевских осцилляций \cite{rabi_1,rabi_2,rabi_3}, что составляет не более $10^{-4}$ секунды.

Такие системы могут дать представление о механизме процессов, выходящих за рамки квантовой электродинамики и включающих как перемещение атомов, так и их превращения: в ходе химической реакции или при радиоактивном распаде. В данной главе мы применим модель ТС для анализа вероятности перехода атома в необратимое состояние в результате спонтанной реакции в зависимости от времени при условии интенсивной утечки фотонов. Природа ре­акции, вызывающей такой переход, или \textbf{превращение}, здесь предполагается химической, однако это может быть и радиоактивный распад, индуцированный фотонами (возможность фотоядерных реакций --- экспериментальный факт, описанный, к примеру, в работе \cite{photonuclear_reactions}). Важно, что в результате реакции атом прекращает взаимодействовать с полем в рамках модели TC. Мы предполагаем, что такая реакция происходит только с атомом, находящимся в возбужденном состоянии, в противоположность основному состоянию, в котором атом не может подвергнуться превращению. Соответственно, рост числа фотонов в нашей полости будет эмуляцией вероятности возбуждения атомов, а уменьшение --- эмуляцией вероятности их релаксации.

Гамильтониан модели TC для системы $n$ атомов в приближении вращающейся волны \cite{ozhigov_qq, rwa_1, rwa_2} имеет вид
\[
H_{\text{\text{TC}}}^{\text{\text{RWA}}} = \hbar wa^{+}a + \hbar w\sum_{i=1}^{n}\s_{i}^{+}\s_{i} + a\bar\s^{+}+a^{+}\bar\s,
\]
где $a$, $a^{+}$ --- полевые, а $\s_{i}$, $\s_{i}^{+}$ --- атомные операторы, $\bar\s=\sum\limits_{i=1}^{n}g_{i}\s_{i}$, $\bar\s^{+}=\sum\limits_{i=1}^ng_{i}\s^{+}_{i}$ --- соответствующие операторы коллективной релаксации и возбуждения группы атомов, $w = w_{c} = w_{a}$ --- частота фотонов моды резонатора, равная частоте атомного перехода, $g_{i}$ --- энергия взаимодействия $i$-го атома с полем, символ <<+>> означает сопряжение оператора. Этот гамильтониан может быть дополнен прямым диполь-дипольным взаимодействием между атомами, минующем поле полости.

Контринтуитивный эффект квантового бутылочного горлышка заключается в парадоксальном увеличении времени жизни возбужденного состояния атома в оптической полости при усилении оттока фотонов из нее. Этот эффект имеет чисто квантовую природу.

Мы выясним влияние этого эффекта на описанный выше процесс превращения атома, природа которого для нас, в сущности, не важна. Мы также предполагаем, что вероятность превращения атома, находящегося в основном состоянии, пренебрежимо мала, а для возбужденного состояния вероятность превращения к моменту времени $t$ подчиняется статистике Пуассона $p_{t} = 1-e^{-\gamma t}$.

\section{Фотон-индуцированное превращение атомов}\label{sec:ch2/sec2}

В дополнение к стандартным состояниям атома (основному $|0\ra$ и возбужденному $|1\ra$) мы введем особое состояние $|2\ra$, которое будем называть \textbf{превращенным}. Тогда действие атомных операторов $\s$, $\s^{+}$ на атомный кубит определяется формально как обычно: $\s|0\ra = \s^{+}|1\ra = 0, \s|1\ra = |0\ra, \s^{+}|0\ra = |1\ra$. Оператор же атомного превращения $L_{2}$ определяется так: $L_{2}|1\ra = |2\ra, L_{2}|0\ra = L_{2}|2\ra = 0$.

В превращенном состоянии атом не может взаимодействовать с полем. Его физический смысл состоит в том, что атом либо вступает в химическую реакцию, либо участвует в ядерном превращении. В обоих случаях взаимодействие такого атома с полем в рамках модели Тависа-Каммингса становится невозможным. Ни продукты распада (в случае ядерного превращения), ни энергия того, во что превратится атом, в данном случае интереса не представляют, так что мы будем учитывать только динамику модели TC с дополнительными операторами, выражающими само превращение атома, ведущее к его выходу из данной модели.

Таким образом, искомая математическая модель будет описана основным квантовым уравнением \cite{breuer}
\begin{equation}\label{sec2_master_eq}
	i\hbar\dot\rho = [H_{\text{TC}}^{\text{RWA}}, \rho] + iL(\rho), L(\rho) = \sum_{i=1}^{2}\gamma_{i}(L_{i}\rho L_{i}^{+} - \frac{1}{2}\{L_{i}^{+}L_{i}, \rho\})
\end{equation}
с операторами Линдблада двух типов: $L_{1} = a$ --- вылет фотона из резонатора \cite{breuer,photon_emission} и $L_{2}$ --- оператор атомного превращения. Интенсивности данных операторов обозначим через $\gamma_{1} = \gamma_{\text{out}}$ и $\gamma_{2} = \gamma_{\text{ex}}$ соответственно.

Нас интересует зависимость вероятности превращения атома от интенсивности $\gamma_{1} = \gamma_{\text{out}}$ вылета фотона из резонатора. В данном случае механизм квантового бутылочного горлышка будет заключаться в том, что при большой величине $\gamma_{1} = \gamma_{\text{out}}$ время жизни возбужденного состояния атома удлиняется, что приводит к росту вероятности его превращения. Увеличение времени жизни имеет чисто квантовую природу и подробно описано в статьях \cite{victorova} и \cite{kulagin_homogeneous}.
\\[18pt]
\noindent Выберем базис: $|i\ra_{\text{ph}}|j\ra_{\text{at}}$.

\noindent Здесь\\
\indent\qquad $|i\ra_{\text{ph}}$ -- фоковское состояние поля (число фотонов в полости, $i=\overline{\mbox{0,1}}$),\\
\indent\qquad $|j\ra_{\text{at}}$ -- состояние атома ($j=\overline{\mbox{0,2}}$).
\\[18pt]
\noindent Данная система при наличии одного фотона и одного атома имеет 4 базисных состояния:\\
\indent\qquad $|0\ra_{\text{\text{ph}}}|2\ra_{\text{\text{at}}}$ (превращенное состояние атома),\\
\indent\qquad $|0\ra_{\text{\text{ph}}}|0\ra_{\text{\text{at}}}$ (фотон вне полости),\\
\indent\qquad $|0\ra_{\text{\text{ph}}}|1\ra_{\text{\text{at}}}$ (атом возбужден),\\
\indent\qquad $|1\ra_{\text{\text{ph}}}|0\ra_{\text{\text{at}}}$ (фотон в полости).
\\[18pt]

\noindent Ее гамильтониан:\\
{\normalsize
	\[
	H = \bordermatrix
	{
		&                |0\ra_{\text{\text{ph}}}|2\ra_{\text{\text{at}}} & |0\ra_{\text{\text{ph}}}|0\ra_{\text{\text{at}}} & |0\ra_{\text{\text{ph}}}|1\ra_{\text{\text{at}}} & |1\ra_{\text{\text{ph}}}|0\ra_{\text{\text{at}}} \cr
		|0\ra_{\text{\text{ph}}}|2\ra_{\text{\text{at}}} &      0 &       	   0 &      0 & 0 \cr
		|0\ra_{\text{\text{ph}}}|0\ra_{\text{\text{at}}} &      0 &       	   0 &      0 & 0\cr
		|0\ra_{\text{\text{ph}}}|1\ra_{\text{\text{at}}} &      0 &  		   0 &      \hbar w_{a} & g\cr
		|1\ra_{\text{\text{ph}}}|0\ra_{\text{\text{at}}} &      0 &      	   0 &      g & \hbar w_{c}\cr
	}.
	\]
}
\
\\[18pt]
%\noindent Алгоритм построения гамильтониана систем Джейнса-Каммингса и Тависа-Каммингса с учетом стока, а также компьютерное моделирование их квантовой динамики детально описаны в приложении \hyperref[app:A]{А}.

\noindent Оператор уничтожения фотона (улет фотона из полости):\\
{\normalsize
	\[
	L_{1} = \bordermatrix
	{
		&                |0\ra_{\text{\text{ph}}}|2\ra_{\text{\text{at}}} & |0\ra_{\text{\text{ph}}}|0\ra_{\text{\text{at}}} & |0\ra_{\text{\text{ph}}}|1\ra_{\text{\text{at}}} & |1\ra_{\text{\text{ph}}}|0\ra_{\text{\text{at}}} \cr
		|0\ra_{\text{\text{ph}}}|2\ra_{\text{\text{at}}} &      0 &      0 &      0 & 0 \cr
		|0\ra_{\text{\text{ph}}}|0\ra_{\text{\text{at}}} &      0 &      0 &      0 & 1\cr
		|0\ra_{\text{\text{ph}}}|1\ra_{\text{\text{at}}} &      0 &      0 &      0 & 0\cr
		|1\ra_{\text{\text{ph}}}|0\ra_{\text{\text{at}}} &      0 &      0 &      0 & 0\cr
	}.
	\]
}
\\

\noindent Оператор атомного превращения:\\
{\normalsize
	\[
	L_{2} = \bordermatrix
	{
		&                |0\ra_{\text{\text{ph}}}|2\ra_{\text{\text{at}}} & |0\ra_{\text{\text{ph}}}|0\ra_{\text{\text{at}}} & |0\ra_{\text{\text{ph}}}|1\ra_{\text{\text{at}}} & |1\ra_{\text{\text{ph}}}|0\ra_{\text{\text{at}}} \cr
		|0\ra_{\text{\text{ph}}}|2\ra_{\text{\text{at}}} &      0 &      0 &      1 & 0 \cr
		|0\ra_{\text{\text{ph}}}|0\ra_{\text{\text{at}}} &      0 &      0 &      0 & 0\cr
		|0\ra_{\text{\text{ph}}}|1\ra_{\text{\text{at}}} &      0 &      0 &      0 & 0\cr
		|1\ra_{\text{\text{ph}}}|0\ra_{\text{\text{at}}} &      0 &      0 &      0 & 0\cr
	}.
	\]
}
\\

\noindent Вышеперечисленные матрицы, а также матрица плотности, были реализованы как разреженные, поскольку при численном моделировании динамики для обнаружения эффекта квантового бутылочного горлышка потребовался достаточно малый шаг $dt$ по времени (в пределах 0.1-1 ns) на промежутке 1 mks. Решение основного квантового уравнения \eqref{sec2_master_eq} производилось при помощи метода Эйлера. 

\noindent Каждый шаг по времени состоит из двух этапов:
\[
\tilde{\rho}(t+dt)\ =\ \rho(t)+dt\cdot \frac{i}{\hbar}\cdot L(\rho(t)),
\]
\begin{center}\text(неунитарная динамика: улет фотонов, атомное превращение)\end{center}
\
\\[0pt]
\[
\rho(t+dt)=U_{dt}\cdot \tilde{\rho}(t+dt)\cdot U_{dt}^{*}.
\]
\begin{center}\text(унитарная динамика)\end{center}
\
\\
\indent Наиболее сложной задачей здесь было выявление (за разумное время) тех диапазонов интенсивностей $\gamma_{1} = \gamma_{\text{out}}$ и
$\gamma_{2} = \gamma_{\text{ex}}$, а также их соотношения с интенсивностью фотонно-атомного взаимодействия $g$, при которых эффект квантового бутылочного горлышка был бы обнаружен, и притом, качественно.

Представленные ниже результаты компьютерного моделирования были получены на суперкомпьютере Ломоносов-2 \cite{voevodin_supercomputer}. Вычисления производились параллельно на 100 процессорных узлах: каждый узел моделировал квантовую динамику для конкретного значения $\gamma_{1} = \gamma_{\text{out}}$ и всего выбранного диапазона $\gamma_{2} = \gamma_{\text{ex}}$. Также в ходе численного эксперимента анализировалась квантовая картина для диапазона $g \in [0.001 \cdot w,~0.1 \cdot w]$ --- как для условий ультраслабого взаимодействия, так и для условий границы применимости приближения RWA \cite{ozhigov_qq, rwa_1, rwa_2} в модели ТС. 

\section{Результаты компьютерного моделирования}\label{sec:ch2/sec3}
Здесь будут приведены результаты компьютерного моделирования влияния эффекта квантового бутылочного горлышка на вероятность атомного превращения при различных параметрах квантовой системы.

\clearpage
\begin{figure}[h!]
	\noindent\centering{
		\includegraphics[width=0.65\textwidth]{Dissertation/images/section_2/01.png}	\captionsetup{format=hang,width=0.85\textwidth,justification=centering,singlelinecheck=no}
		\caption{
			Зависимость вероятности превращения атома\\от интенсивности $\gamma_{\text{out}}$ вылета фотона и времени $t$\\
			$\gamma_{\text{out\_min}} = g \qquad \gamma_{\text{out\_max}} = 10g \qquad \gamma_{\text{ex}} = g$
		}
		\
		\\[28pt]
		\includegraphics[width=0.65\textwidth]{Dissertation/images/section_2/02.png}	\captionsetup{format=hang,width=0.85\textwidth,justification=centering,singlelinecheck=no}
		\caption{
			Зависимость вероятности превращения атома\\от интенсивности $\gamma_{\text{out}}$ вылета фотона и времени $t$\\
			$\gamma_{\text{out\_min}} = 0.01g \qquad \gamma_{\text{out\_max}} = 10g \qquad \gamma_{\text{ex}} = g$
		}
	}
\end{figure}

\clearpage
\begin{figure}[h!]
	\noindent\centering{
		\includegraphics[width=0.65\textwidth]{Dissertation/images/section_2/03.png}	\captionsetup{format=hang,width=0.85\textwidth,justification=centering,singlelinecheck=no}
		\caption{
			Зависимость вероятности превращения атома\\от интенсивности $\gamma_{\text{out}}$ вылета фотона и времени $t$\\
			$\gamma_{\text{out\_min}} = 0.01g \qquad \gamma_{\text{out\_max}} = 50g \qquad \gamma_{\text{ex}} = g$
		}
		\
		\\[40pt]
		\includegraphics[width=0.65\textwidth]{Dissertation/images/section_2/04.png}
		\captionsetup{format=hang,width=0.92\textwidth,justification=centering,singlelinecheck=no}
		\caption{
			Зависимость времени насыщения пороговой\\вероятности вылета фотона от интенсивности $\gamma_{\text{out}}$ утечки\\и интенсивности $\gamma_{\text{ex}}$ атомного превращения
			\quad\quad$\gamma_{\text{out\_min}} = 10^{-3}g \qquad \gamma_{\text{out\_max}} = 10^{-2}g$\\
			$\gamma_{\text{ex\_min}} = 0.1g \quad\qquad~ \gamma_{\text{ex\_max}} = g$\\
			\qquad\quad порог вероятности вылета фотона из полости: 0.9
		}
	}
\end{figure}

Шаг моделирования по времени $dt = 0.1~\text{ns}$, $g/\hbar w = 0.01$.

\clearpage
\section{Выводы главы}\label{sec:ch2/sec4}

В результате численного моделирования было обнаружено, что эффект квантового бутылочного горлышка способен существенно повлиять на вероят­ность атомных превращений. Эта возможность реализуется при малой энергии атомного возбуждения относительно энергии атомного превращения. Возбуж­дение атома может затрагивать либо его электронную оболочку (что влияет на химические превращения атома), либо ядро (что способно повлиять на ядерные реакции). При большом усилении интенсивности вылета фотонов возбужда­ющей моды за пределы оптической полости вероятность превращения атома парадоксальным образом повышается. Этот эффект может быть использован в том числе и для понижения вероятности атомного превращения: для этого необходимо увеличить время жизни фотона в пределах оптической полости. Возбуждающая атом фотонная мода служит своеобразным триггером, пере­ключающим режимы динамики самого атома, притом что энергия этой моды на много порядков меньше энергии самого атомного превращения.

Данный эффект, в силу его контринтуитивной, чисто квантовой природы, не учитывается в имеющихся моделях атомных превращений. Его включение в такие модели увеличит область их применения в практических целях --- как в химии, так и в приложениях физики атомного ядра.

\nocite{skovoroda}
\nocite{skovoroda_conductivity}
           % Глава 2
\chapter{Оптический отбор темных состояний ансамблей многоуровневых атомов}\label{ch:ch3}

\section{Темные состояния}\label{subsec:ch3/sect1}

Описание собственных состояний в модели Тависа-Каммингса является весьма непростой задачей. Ее решение для двухуровневых атомов приведено в диссертации Михаэля Тависа \cite{tc_a_study}, и оно нетривиально. Но особый интерес представляет класс собственных состояний, атомная компонента которых состоит из \textbf{темных состояний} \cite{dark_states_dimension}. Так называют атомные состояния, которые не могут ни испустить, ни поглотить фотон в силу того, что интерференция атомов блокирует их взаимодействие с полем. Тем самым атомы мешают друг другу испустить или поглотить фотон.

К примеру, для состояния
\begin{equation}\label{singlet}
	|s\ra = \frac{|01\ra - |10\ra}{\sqrt{2}}
\end{equation}
испускание вторым атомом фотона переведет состояние  $|0\ra_{\text{ph}}|s\ra$ в состояние $|1\ra_{\text{ph}}|00\ra$, а испускание фотона первым атомом переведет состояние  $|0\ra_{\text{ph}}|s\ra$ в состояние $-|1\ra_{\text{ph}}|00\ra$. В итоге попытка испускания фотона состоянием \eqref{singlet} не возымеет никакого эффекта.

Состояние \eqref{singlet} является \textbf{темным} и называется \textbf{синглетом}.

Алгебраическое описание темных состояний ансамблей многоуровневых атомов можно найти в работе \cite{dark_states_kok}, однако их явный вид был найден только для ансамблей двухуровневых атомов. В работе \cite{dark_states_dimension} доказано утверждение о том, что размерность темного подпространства пространства $n$ двухуровневых атомов равна
\begin{equation}
\mathrm{dim}(D_{n}^2) =
\begin{cases}
	C_{n}^{k} - C_{n}^{k-1}~\text{при}~n = 2k, \\
	0, \text{в противном случае},
\end{cases}
\end{equation}
и все наборы темных состояний (для четного числа атомов в группе) с учетом нормировки имеют вид 
\begin{equation}
\frac{1}{2^{n/4}}\bigotimes_{j=1}^{n/2}(|01\ra_{j}-|10\ra_{j}),
\end{equation}
где индекс $j = 1,\dots, n/2$ означает номер пары в произвольном разбиении группы из $n$ атомов.

Для трехуровневых атомов имеется лишь гипотеза о структуре темных состоя­ний, которая была подтверждена в настоящей работе для весьма ограниченного числа атомов в ансамбле (не более 9). Размерность же темного подпространства была численно определена для двух десятков атомов в ансамбле с использованием суперкомпьютера (глава \ref{ch:ch4}).

\section{Практическая значимость темных состояний}\label{subsec:ch3/sect2}
Темные состояния имеют множество применений. В частности, их роль в организации межатомного взаимодействия рассмотрена в работе \cite{dark_states_properties_andre}, для контроля твердотельных спинов --- в работе \cite{dark_states_properties_hansom}, для управления макроскопическими квантовыми системами --- в работе \cite{dark_states_properties_bose_einstein}. Роль участия темного состояния в процессе фотосинтеза описана в работе \cite{dark_states_properties_photosynthesis}. Некоторые методы получения темных состояний в квантовых точках представлены в работах \cite{dark_states_properties_poltl} и \cite{dark_states_properties_tanamoto}. Разрушение темных состояний магнитным полем или модулированной лазерной поляризацией рассматривается в работе \cite{dark_states_properties_destabilization}.

Темные состояния обладают ненулевой энергией и по этой причине могут служить энергетическим микроаккумулятором для различных наноустройств \cite{dark_states_quantum_memory}. Кроме того, будучи свободными от декогерентности (поскольку не взаимодействуют со светом), они могут быть использованы для достаточно длительного хранения сложных состояний в квантовых вычислениях.

Темные состояния широко используются в квантовой криптографии: использование синглетных состояний дает дополнительные преимущества в обеспечении устойчивости квантовых протоколов. Так, например, ключевое ме­сто в построении протокола AKM2017 \cite{akm2017} занимает подготовка синглетного состояния вида \eqref{singlet}.

Получение темных состояний двухуровневых атомов, описанное в работе \cite{dark_states_properties_tanamoto}, опирается на эффект Штарка-Зеемана. В данной главе мы опишем технически более простой метод оптического отбора, основанный на томографии электромагнитного поля вне полости, который, в частности, позволяет получать многоуровневые темные состояния, и притом строго определенного вида.

\clearpage
\section{Многомерные синглеты}\label{sec:ch3/sect2}
Рассмотрим обобщение модели Тависа-Каммингса на $d$-уровневые атомы. Гамильтониан ТС для $n$ атомов с энергиями $g^{j}_{i}$ взаимодействия с полем выделенной моды $j$ имеет вид
\begin{gather}
	H^j_{\text{TC}} = \hbar w a_{j}^{+}a_j + \hbar w \sum\limits_{i=1}^{n}\s_{ji}^{+}\s_{ji} + (a_{j}^{+} + a_{j})\sum\limits_{i=1}^{n}g^{j}_{i}(\s_{ji}^{+} + \s_{ji}),\notag\\
	H^{j,\ \text{RWA}}_{\text{TC}} = \hbar w a_{j}^{+}a_{j} + \hbar w \sum\limits_{i=1}^{n}\s_{ji}^{+}\s_{ji} + a_{j}^{+}\bar\s_{j} + a_{j}\bar\s_{j}^{+},\\
	\bar\s_{j} = \sum\limits_{i=1}^ng^{j}_{i}\s_{ji},\notag
\end{gather}
где RWA-приближение \cite{ozhigov_qq,rwa_rabi_1,rwa_rabi_2} справедливо при $g^{j}_{i}/\hbar w\ll 1$, верхним символом <<+>> обозначено сопряжение операторов. Здесь $a_{j},a_{j}^{+}$ --- полевые операторы уничтожения и рождения фотона моды $j$, $\s_{ji},\s_{ji}^{+}$ --- атомные операторы релаксации и возбуждения атома $i$, соответствующие моде $j$.

Пусть для ансамбля $\text{A}$, состоящего из $n$ одинаковых $d$-уровневых атомов, различающихся только энергиями взаимодействия с модами поля, определен граф $G$ возможных разрешенных переходов между уровнями для каждого $j$-го атома, $j=1,2,\dots,n$. Вершины $G$ соответствуют уровням энергии, ребра --- разрешенным переходам между ними. Тогда $G$ задает набор всевозможных мод $J_{G}$, с которыми может взаимодействовать каждый атом. Многомодовый гамильтониан, соответствующий графу $G$, имеет вид
\begin{gather}
	\label{many}
	H^G_{\text{TC}} = \sum\limits_{j\in J_{G}}H^{j}_{\text{TC}},\notag\\
	H^{G,\ \text{RWA}}_{\text{TC}}=\sum\limits_{j\in J_{G}}H^{j,\ \text{RWA}}_{\text{TC}}\notag.
\end{gather}
Введем обозначение $\bar\s_{G} = \sum\limits_{j\in J_{G}}\bar\s_{j}$. 
Тогда подпространство темных атомных состояний для ансамбля с возможными переходами $G$ есть $Ker(\bar\s_{G}^{+} + \bar\s_{G})$ и $Ker(\bar\s_{G})$ в точной модели и в RWA-приближении \cite{ozhigov_qq,rwa_rabi_1,rwa_rabi_2} соответственно. Заметим, что некоторые моды могут допускать RWA-приближение \cite{ozhigov_qq,rwa_rabi_1,rwa_rabi_2}, тогда как другие --- нет. К разным атомам применимость RWA \cite{ozhigov_qq,rwa_rabi_1,rwa_rabi_2} также может быть различной. Мы будем рассматривать только случай применимости этого приближения ко всем модам и атомам одновременно.

Через $g^{j}(r)$ обозначим амплитуду перехода по ребру $r$, соединяющему пару состояний в графе $G$ для атома $j$.

Сделаем граф $G$ ориентированным, задав ориентацию любого ребра по направлению к уменьшению энергии атомного состояния. Зафиксировав номер атома $j\in\{ 1,2,\dots,n\}$, пометим в графе $G$ ребра $r$ числами $g^{j}(r)$. Получится $n$ графов $G^{j}$, изоморфных $G$, для каждого атома --- свой. Предположим, что каждой паре <<атом $j$, состояние $i$>> можно приписать положительный вес $w(j,~i)$ так, что для любой пары $j,~j'$ атомов отношение $g^{j}(r)/g^{j'}(r)=w(j',~i_{in})/w(j,~i_{in}) = w(j',~i_{fin})/w(j,~i_{fin})$ для любого ребра $r$ с началом $i_{in}$ и концом $i_{fin}$. 

Рассмотрим состояние атомов
\begin{equation}\label{msinglet}
	|D_{G,A}\ra = \sum\limits_{\pi\in S_{d}}(-1)^{\s(\pi)}w(1,\pi(1))\dots w(d,\pi(d))|\pi(1),\dots,\pi(d)\ra,
\end{equation}
где $\pi$ пробегает все перестановки на множестве атомов $1,2,\dots,d$, а $\s(\pi)$ обозначает четность перестановки $\pi$. Состояние $|D_{G,A}\ra$  называется $G,A$-мультисинглетом. Мультисинглет называется равновесным, если все веса $w(j,~i)$ равны единице. Мультисинглет всегда является темным в RWA \cite{ozhigov_qq,rwa_rabi_1,rwa_rabi_2}, а равновесный мультисинглет --- темным для точного гамильтониана. Чтобы показать это, рассмотрим следующий пример.

\textbf{Пример.} Для $d=2$ состояние \eqref{msinglet} с точностью до нормировки примет вид 
\begin{equation}
	\label{g_singlet}
	g^{1}|01\ra-g^{2}|10\ra,
\end{equation}
и оно является темным в RWA-приближении \cite{ozhigov_qq,rwa_rabi_1,rwa_rabi_2}. Из определения весов $w(j,~i)$ следует, что сумма двух слагаемых из \eqref{msinglet}, отличающихся только перестановкой состояний одной пары атомов, будут с точностью до коэффициента иметь вид \eqref{g_singlet}. С другой стороны, состояние \eqref{g_singlet} будет темным для точного гамильтониана тогда и только тогда, когда $g^{1}=g^{2}$. 

Подграф $G'\subseteq G$ называется полным, если вместе с любой своей вершиной он содержит все вершины, соединенные с ней нисходящим ребром, вместе с этим ребром. Набор графов $G_{1},G_{2},\dots,G_{r}$ графа $G$ назовем накрытием, если он состоит из полных подграфов и их объединение дает $G$. Накрытие точное, если любой $G_{i},~i=1,2,\dots,r$ является компонентой связности графа $G$. 

\noindent Для ансамбля, состоящего из $n$ $d$-уровневых атомов, в свете работы \cite{quantum_simulation_homogeneous} правдоподобной является следующая гипотеза о явном виде темных состояний.

\begin{hyp}
	\
	\\
	\indent 1) Любое темное состояние в гамильтониане $H^{G,~\text{RWA}}_{\text{TC}}$ есть линейная комбинация тензорных произведений $G_{i},~\text{A}_{i}$ --- мультисинглетов для некоторых накрытий $\{ G_{i}\}$ графа $G$ и разбиений множества всех атомов $\text{A}$ на подмножества $\text{A}_{i}$. 
	
	2) Темные состояния для точного гамильтониана $H^G_{\text{TC}}$ являются в точности линейными комбинациями равновесных $G_{i},~\text{A}_{i}$ --- мультисинглетов для точных накрытий $\{ G_{i}\}$ графа $G$ и соответствующих разбиений $\text{A}$ на подмножества $\text{A}_{i}$. 
\end{hyp}

В частности, из этого следует, что при связном графе $G$ темные состояния в точной модели бывают лишь для ансамблей с числом атомов, кратным $d$. Данная гипотеза строго доказана только для $d=2$ в работе \cite{dark_states_dimension}.

В ансамбле разнородных атомов, как правило, нет совпадающих частот переходов. Однако в квантовых точках, где <<атомы>> можно фактически формировать искусственно, можно добиться и совпадения частот некоторых переходов в спектрах неодинаковых структур. В этом случае можно исследовать получающиеся темные состояния.

Например, для трехатомного ансамбля трехуровневых атомов с энергиями переходов $\hbar w$ между уровнями $0 \leftrightarrow 1$, $0 \leftrightarrow 2$, $1 \leftrightarrow 2$ единственным темным состоянием в RWA \cite{ozhigov_qq,rwa_rabi_1,rwa_rabi_2} без учета нормировки будет \textbf{мультисинглет}
\begin{equation}
	D_{3} = |012\ra + |120\ra + |201\ra - |021\ra - |102\ra - |210\ra.
\end{equation}

\section{Метод оптического отбора темных состояний}\label{sec:ch3/sect3}
Мы объясним метод оптического отбора на примере ансамбля, состоящего из двух двухуровневых атомов. Будем обозначать базисные состояния системы атомов и поля через $|n\ra_{\text{ph}}|m_1m_2\ra_{\text{at}}$, где $n$ --- число фотонов в резонаторе, $m_1,~m_2$ --- числа возбуждения первого и второго атомов: $|0\ra$ --- основное состояние, $|1\ra$ --- возбужденное. Схема отбора состоит из последовательных шагов отбора, которая начинается с заранее приготовленного состояния поля и атомов $|\Psi(0)\ra|0\ra_{\text{ph}}|\Phi_0\ra_{\text{at}}$, где $|\Phi_{0}\ra_{\text{at}}=\alpha |00\ra+\beta |s\ra$, $|s\ra=\frac{1}{\sqrt{2}}(|01\ra - |10\ra)$ --- двухатомный синглет, $\alpha |00\ra+\beta |s\ra$ --- произвольное состояние двухатомной системы, которое можно получить, выждав необходимое время для испускания фотона двухатомной системой. Например, состояние атомного ансамбля $|01\ra$ можно представить как $|01\ra=\frac{1}{\sqrt 2}(|t\ra+|s\ra)$, где $|t\ra=\frac{1}{\sqrt{2}}(|01\ra+|10\ra)$ --- триплетное состояние, остальные два триплета имеют вид $|00\ra$ и $|11\ra$. 

Шаг процесса с номером $i$ состоит в следующем. В момент времени $\tau_{i}$ мы имеем состояние системы <<атомы + поле>> $\rho_{i}$, при этом вероятность присутствия фотонов в полости исчезающе мала. Мы запускаем в резонатор один фотон, после чего включаем ячейку Поккельса, расположенную внутри резонатора и отражающую фотон в направлении детектора (см. рисунок \ref{fig:2}) и фиксируем время срабатывания детектора. После этого шага делаем следующий точно так же и т.д., набирая статистику времен срабатывания детектора. 

Мы предполагаем, что время запуска фотона в полость мало по сравнению как с временем  рабиевской осцилляции между состояниями $|1\ra_{\text{ph}}|00\ra_{\text{at}}$ и $|0\ra_{\text{ph}}\frac{1}{\sqrt 2}(|01\ra_{\text{at}}+|10\ra_{\text{at}})$, так и с ожидаемым временем вылета фотона из полости, и им можно пренебречь, считая запуск практически мгновенным.

Пусть $\rho'_{i}$ --- априорное состояние системы в полости в момент $i$-го включения ячейки Поккельса. Поскольку фотон появляется в полости очень быстро, можно считать, что это состояние получается из $\rho_i$ добавлением фотона в полость: $\rho'_i=a^+\rho_ia$. После этого мы ждем время $\tau_{\text{click}\_i}$, когда в детектор попадет фотон, вылетевший из полости. 

\begin{figure}[bt]
	\noindent\centering\includegraphics[width=0.75\textwidth]{Dissertation/images/section_3/detector.png}
	\captionsetup{format=hang,width=0.85\textwidth,justification=centering,singlelinecheck=no}
	\caption{Оптический отбор.\\Линдбладовский оператор $L_{1} = a^{+}a$ реализует улет фотона и возврат его обратно в полость после прохождения через детектор.\\ Детектор щелкает всякий раз, когда в него попадает фотон.}
	\label{fig:2}
\end{figure}

Время срабатывания детектора $\tau_{\text{click}\_i}$ на шаге $i$ является случайной величиной, зависящей также от шага $i$, так что решая основное уравнение, мы лишь найдем для нее верхнюю границу $t_i$. Функция распределения $\tau_{\text{click}\_i}$ меняется с каждым шагом и является априорной функцией распределения, которую мы находим по уравнению \eqref{lind}, не прибегая ни к каким экспериментам. Это вычисление нужно лишь для того, чтобы найти верхнюю границу $t_i$ ожидания щелчка детектора на шаге $i$. Матрицу плотности $\rho_{i+1}$ можно найти как решение задачи Коши для основного квантового уравнения \eqref{lind}, соответствующего вылету фотона из полости, для начального состояния $\rho'_i(0)=\rho'_i$, с тем условием, что для момента $t_{i}$ это решение $\rho'_i(t_{i})$ не содержит фотонов в полости с исчезающе малой вероятностью ошибки  (ошибка может произойти только когда мы прекратили ждать срабатывания детектора, а фотон все-таки остался в полости или может быть испущен позже).

Итак, верхнюю границу $t_i$ для $\tau_{\text{click}\_i}$ на шаге $i$ мы ищем численнно, решая основное квантовое уравнение. Полагаем $\rho_{i+1}=\rho'_i(t_{i})$. После вылета фотона из полости состояние атомов внутри полости не меняется, поэтому мы можем произвольно увеличить время ожидания полного вылета до значения, большего найденного $t_i$ для уменьшения вероятности ошибки.

Мы будем делать так определенные последовательные шаги, каждый раз фиксируя время срабатывания детектора на вылетающий из полости фотон. Если момент $\tau_i-\tau_{i-1}$ щелчка детектора на шаге $i$ рассматривается как случайная величина, то функция распределения этой величины находится как $P(t)=\la 0_{\text{ph}}0_10_2|\rho(t)|0_{\text{ph}}0_10_2\ra+\la 0_{\text{ph}}s|\rho(t)|0_{\text{ph}}s\ra$, то есть как вероятность того, что фотон вылетел из полости за время $t$, считая нулевой отметкой начало шага $i$. Плотность распределения времени срабатывания детектора есть $dP(t)/dt$. После достаточно большого числа последовательных шагов мы считаем среднее время $d\tau$ по всем значениям $\tau_{\text{click}\_i}$ срабатывания детектора в наших экспериментах. Далее мы установим факт достаточно быстрого подавления внедиагональных элементов матрицы плотности $\rho_i$ с ростом $i$, так что распределение величины $\tau_{\text{click}\_i}$ для разных $i$ будет практически одинаковым для больших $i$ и сойдется к распределению, характерному  либо для триплета $|00\ra$,  либо для синглета $|s\ra$. Таким образом, величины времен ожидания щелчка детектора $t_i$, начиная с момента исчезновения недиагональных элементов, будут одинаковы. Обозначим их через $dT$. 

Если среднее время $d\tau$ вылета фотона меньше некоторого порога $d\tau_{cr}$, в полости находится темный синглет $|s\ra$. В противном случае мы имеем триплет $|00\ra$, состояние бракуется, и вся серия экспериментов начинается заново --- с выбора случайного начального состояния атомов.

Срабатывание на шаге $i$ детектора, в который попадает фотон, отраженный ячейкой Поккельса, происходит с замедлением, которое меняется между нулем и $\tau_{\text{click}\_i}=\tau_{i}-\tau_{i-1}$. Оно складывается из двух факторов: а) время срабатывания самой ячейки (она может не перекрывать всю полость и потому, даже если в полости есть фотон, он не отразится сразу при прохождении вдоль полости) и б) возможность поглощения фотона компонентой $|00\ra$ атомного состояния  полости.  

Если первоначальное состояние атомов $\rho_0=|s\ra_{\text{at}}\la_{at} s|$, то мы имеем синглет и среднее время вылета $a_s$ фотона из полости будет коротким. Если же $\beta=0$, то мы имеем триплет $\rho_0=|00\ra_{at}\la_{at} 00|$ и среднее время вылета фотона $a_t$ будет длиннее, так как за время бездействия ячейки Поккельса фотон может с ненулевой вероятностью поглотиться ансамблем атомов. Таким образом, достаточно взять статистический барьер для принятия решения $d\tau_{cr}=(a_s+a_t)/2$. 

Считая применимым RWA-приближение \cite{ozhigov_qq,rwa_rabi_1,rwa_rabi_2}, рассмотрим в качестве математической модели шага нашего процесса основное квантовое уравнение\cite{breuer} с оператором Линдблада $A_1=a$ --- удаление фотона из полости:
\begin{gather}
	i\hbar\dot{\rho}=[H,~\rho]+i{\mathcal L}(\rho),\notag\\
	{\mathcal L}(\rho)=\gamma(a\rho a^+-\frac{1}{2}(\{ a^+a,~\rho\}),\label{lind}\\
	H=H_{\text{TC}}\notag.
\end{gather}
Его решение $\rho(t)$ можно приближенно найти, представив в виде последовательности двух шагов, из которых на первом делается один шаг в решении унитарной части \eqref{lind}: $\tilde\rho(t+dt)=e^{-iHdt /\hbar}\rho(t)e^{iHdt /\hbar}$, а на втором --- шаг в решении уравнения \eqref{lind} с удаленным коммутатором:
\begin{equation}
\rho(t+dt)=\tilde\rho(t+dt)+\frac{\gamma}{\hbar}(a\tilde\rho a^+-\frac{1}{2}(\{ a^{+}a,~\tilde\rho\})dt.
\end{equation}

Грубо оценить параметр $\gamma$ можно так. Поскольку изменение матрицы плотности на втором шаге, отнесенное к времени $dt$, за которое свет преодолевает длину полости, составляет величину эффективности ячейки Поккельса $e_p:\ 0<e_p\leq 1$, мы имеем $\frac{\gamma dt}{\hbar}=e_p$, откуда $\gamma=e_p\hbar/dt$. Для атома Rb85 длина полости, равная половине длины волны фотона, составляет $0.7\ cm$, мы имеем $\gamma\approx 10^{-17}e_p$ эрг. 

Предположим, что мы имеем один из вариантов: а) $|\Phi_0\ra_{\text{at}}=|00\ra$ или б) $|\Phi_0\ra_{\text{at}}=|s\ra$. В первом случае время срабатывания детектора, усредненное по большому числу испытаний, будет в силу центральной предельной теоремы очень близко к $t_s$, во втором --- к $t_t$. Поскольку $t_t-t_s$ --- достаточно большая величина, мы сможем статистически достоверно различить эти два случая. Варианты а) или б) имеют место, например, если начальное состояние пары атомов имеет вид $|01\ra$, так как в этом случае щелчок детектора при приготовлении исходного состояния для первого шага уже означает, что мы имеем состояние $|00\ra$, а отсутствие щелчка в течение достаточно длительно времени --- что мы имеем синглет $|s\ra$. 

Теперь пусть оба числа $\alpha$ и $\beta$ ненулевые. Тогда в матрице плотности состояния атомов в базисе $\{|00\ra,~|s\ra\}$, получаемая в результате описанной последовательности шагов внедиагональные члены будут подавляться с числом шагов, так что в пределе матрица плотности полностью распадется на $|00\ra\la 00|$ с вероятностью $|\alpha|^2$ и $|s\ra\la s|$ с вероятностью $|\beta|^2$, и мы придем к уже разобранному случаю двух несовместных альтернатив. Подавление внедиагональных элементов матрицы плотности установлено численным моделированием. Время полного подавления внедиагональных элементов матрицы плотности $T_{\text{non}}$ получается суммированием всех временных отрезков ожидания полного вылета фотона из полости: $T_{\text{non}}=\sum\limits_{i=0}^Lt_i$ где $L$ --- минимальное значение шага, на котором внедиагональные элементы матрицы $\rho_L$ становятся пренебрежимо малыми. График времени полного подавления внедиагональных элементов матрицы плотности $T_{\text{non}}$ в зависимости от энергии $g$ взаимодействия атомов и поля приведен на рисунке \ref{fig:decoh}. 

Моменты щелчков детектора $d\tau_1,d\tau_2,\dots,d\tau_{L_{s,t}}$ в последовательных экспериментах соответствуют независимой выборке из значений данных величин. Значения $L_s,~L_t$ для двух конкурирующих гипотез будут различаться ненамного. По центральной предельной теореме среднее арифметическое $\xi=\sum\limits_{i=1}^Ld\tau_{i}/L$ будет иметь при больших $L$ нормальное распределение с центрами $a_s$ и $a_t$ соответственно, которые представляют собой средние времена вылета фотона для двух альтернативных гипотез: $a_{s}<a_{t}$. 

\clearpage
\begin{figure}[h!]
	\noindent\centering\includegraphics[width=0.9\textwidth]{Dissertation/images/section_3/opt1.png}
	\captionsetup{format=hang,width=1.0\textwidth,justification=centering,singlelinecheck=no}
	
	\caption{
		{\small
			Время $t=T_{\text{non}}$ полного затухания недиагональных элементов\\($\mathrm{abs} < 10^{-3}$)\\[12pt]
			$g/\hbar w$ в интервале $[0.001; 0.01]$ с шагом $0.001$\\
			Интенсивность фотонной утечки: $\gamma / \hbar w = 0.01$\\
			Шаг по времени: $dt = 0.001 /\gamma$
		}
	}
	\label{fig:decoh}
\end{figure}

\begin{figure}[h!]
	\noindent\centering\includegraphics[width=0.95\textwidth]{Dissertation/images/section_3/l_001g.png}
	\caption{
		{\small
			Функция распределения времени жизни фотона в полости\\(слева при $\gamma=0.01g$, справа при $\gamma=g$)
		}
	}
	\label{fig:0.01}
\end{figure}
\
\\
\noindent Пороговая вероятность вылета, при наступлении которой происходит запуск очередного фотона: $0.95$.

На рисунке \ref{fig:0.01} изображены графики функций распределения времени вылета фотона из полости для разных значений $\gamma$. Соответственно, плотности распределения будут производными от этих функций: для $\gamma=g$ график плотности показан на рисунке \ref{fig:dP1}. 

\begin{figure}[ht!]
	\noindent\centering\includegraphics[width=0.9\textwidth]{Dissertation/images/section_3/l_1g.png}
	\captionsetup{format=hang,width=0.95\textwidth,justification=centering,singlelinecheck=no}
	\caption{
		{\small
			Плотность распределения времени срабатывания детектора\\
			$\gamma = g$
		}
	}
	\label{fig:dP1}
\end{figure}

Мы провели прямое моделирование оптического отбора с помощью датчика случайных чисел, последовательностью испытаний. 
В каждом испытании с интервалом $t_{\mathop{\text{click}}}$ моделируется измерение стока, то есть статистическое испытание факта вылета фотона из полости, исходя из рассчитанной по уравнению \eqref{lind} вероятности. При этом уравнение \eqref{lind} решается методом Эйлера с шагом по времени $dt$, причем временные интервалы в вычислительной модели выбирались так, чтобы для любого шага по времени выполнялись бы неравенства $dt<dt_{\mathop{\text{click}}}\ll \tau_{\text{click}\_i}\leq dT$. Если фотон вылетел, испытание считается завершенным, и мы снова запускаем его в полость, изменяя начальное условие для \eqref{lind}, и переходим к следующему испытанию. Число всех испытаний обозначается через $N$, максимальное время одного испытания: $dT=\text{max}(t_i)$.

Ниже приведены результаты численного моделирования для следующих значений параметров: 
\begin{itemize}
	\item{шаг по времени решения уравнения \eqref{lind}:\ $dt = 10 \mathop{\text{ns}}$,}
	\item{интенсивность вылета фотона в сток:\ $\gamma = 0.01g$ }
	\item{период проверки срабатывания детектора $dt_{\mathop{\text{click}}}=50~\text{ns}$, }
	\item{число испытаний (каждое испытание проводится до первого срабатывания детектора) $N = 1000$,}
	\item{$a_t = 1.551\ \mathop{\text{mks}}$, $a_s = 1.125\ \mathop{\text{mks}}$.}
\end{itemize}

Практически можно взять $n_{\text{bor}} = 2T_{\text{gen}}/(a_{s} + a_{t})$ как среднее число щелчков детектора за общее время $T_{\text{gen}}$ наблюдения, $T_{\text{gen}}=NdT$. Применим наш статистический критерий так: при числе щелчков $n_{\text{click}}>n_{\text{bor}}$ мы имеем синглетное состояние $|s\ra$, в противном случае --- триплет $|00\ra$. 

Тогда ошибка первого и второго рода оценится сверху как квантиль $\int\limits_{n_{\text{bor}}}^\infty N_{0,\s}(x)\ dx$ нормального распределения с математическим ожиданием $0$ и дисперсией $1/\text{min}\{L_s,L_t\}$ и может быть сделана сколь угодно малой с увеличением $T_{\text{gen}}$.

\begin{figure}[ht!]
	\noindent\centering\includegraphics[width=1.0\textwidth]{Dissertation/images/section_3/pt_1ns.png}
	\caption{
		Плотность распределения среднего\\времени жизни фотона в полости\\
		(точность моделирования: $\mathop{dt = 1~\text{ns}}$)
	}
	\label{fig:csignga}
\end{figure}

\clearpage
\section{Программная реализация}
\vspace{-3em}
\begin{figure}[h!]
	\noindent\centering{
		\includegraphics[width=0.9\textwidth]{Dissertation/images/section_3/scheme.jpg}
		\captionsetup{format=hang,width=1.0\textwidth,justification=centering,singlelinecheck=no}
		\\[6pt]
		\caption{
			{\small Блок-схема: оптический отбор темных квантовых состояний}
		}
	}
\end{figure}

\clearpage
\section{Оптический отбор многоуровневых темных состояний}\label{sec:ch3/sect4}
\vspace{-3em}
Описанный оптический отбор темных состояний применим и к ансамблям многоуровневых атомов. Здесь надо рассмотреть состояния многоуровневого синглета $|S_{D}\ra$ вида \eqref{msinglet}
и дополнить его до ортонормированного базиса <<светлыми>> состояниями. При этом отбор должен производиться по всем модам, которых в случае $d$ уровней будет не больше $C^{2}_{n}$ (мы учитываем переходы всех порядков). Для трехуровнего случая обозначим мультисинглет через $|D_3\ra$. Мы рассмотрели несколько примеров для ансамблей трех трехуровневых атомов, проведя численное моделирование для параметров $dt = 1 \mathop{\text{ns}}$, $\gamma = g$, $dt_{\mathop{\text{click}}}=100\ \text{ns}$, $N = 1000$. 
Графики функции распределения времени срабатывания детектора даны на рисунке \ref{fig:csignga}, графики плотности распределения среднего времени щелчка детектора для разных состояний --- на рисунке \ref{fig:csignga1}.

Плотность распределения среднего значения времени детектирования фотонов считалась для значений $dt_{\mathop{\text{click}}} = 100~\text{ns}$.
\begin{flushleft}
	\qquad\qquad~~$a_{|10\ra_{\text{ph}}|D_3\ra} = 16.596\ \mathop{\text{mks}}$ \qquad\qquad\qquad~~ $a_{|10\ra_{\text{ph}}|0_{1}0_{2}0_{3}\ra_{\text{at}}} = 22.243\ \mathop{\text{mks}}$\\
	$a_{|10\ra_{\text{ph}}(|0_{1}1_{2}\ra-|1_{1}0_{2}\ra)|0_{3}\ra} = 22.423\ \mathop{\text{mks}}$ \qquad\qquad $a_{|10\ra_{\text{ph}}|0\ra_{2}(|0_{1}1_{3}\ra-|1_{1}0_{3}\ra)} = 22.423\ \mathop{\text{mks}}$\\ 
	$a_{|10\ra_{\text{ph}}|0_{1}\ra(|0_{2}1_{3}\ra-|1_{2}0_{3}\ra)} = 22.423\ \mathop{\text{mks}}$
\end{flushleft}
\begin{figure}[h!]
	\includegraphics[width=0.95\textwidth]{Dissertation/images/section_3/ph1_l1g.png}
	\captionsetup{format=hang,width=0.9\textwidth,justification=centering,singlelinecheck=no}
	\caption{
		Функция распределения среднего\\ \hspace{7em}времени жизни фотона в полости\\ 
		$\gamma = g$
	}
	\label{fig:csignga}
\end{figure}

\clearpage
\begin{figure}[h!]
	\noindent\centering\includegraphics[width=1.0\textwidth]{Dissertation/images/section_3/gauss_l1g.png}
	\captionsetup{format=hang,width=0.9\textwidth,justification=centering,singlelinecheck=no}
	\caption{
		Плотность распределения среднего\\ \hspace{6em}времени жизни фотона в полости\\
		\hspace{6em}$dt_{\text{click}}= 100~\text{ns}$, $\gamma = g$
	}
	\label{fig:csignga1}
\end{figure}
\vspace{-3em}
\section{Выводы главы}\label{sec:ch3/sect5}
\vspace{-3em}
Мы предложили метод генерации темных состояний двух- и трехуровневых атомов, основанный на оптическом отборе. Этот способ очень прост, но, в отличие от предложенного ранее в работе \cite{dark_states_properties_tanamoto}, не требует применения штарковского сдвига уровней. Здесь используется только многократное измерение времени задержки детектирования фотонов, покидающих оптическую полость. Данный способ позволяет получить темное состояние в виде тензорного произведения синглетов в течение не более нескольких десятков микросекунд для спектра атомов Rb85. Он также практически не зависит от выбора начального состояния атомов, помещенных в полость.

Оптический отбор в равной степени применим и к многоуровневым атомам, причем он может быть настроен на получение строго определенного вида темного состояния. Описанный метод, в силу его простоты и скорости, можно использовать для генерации темных состояний ансамблей из нескольких десятков атомов в одной полости, что может быть полезным для производства защищенных от декогерентности квантовых вычислений в оптических полостях.
           % Глава 3
\chapter{Определение размерности темного подпространства пространства многоуровневых атомных ансамблей}\label{ch:ch4}

\section{Введение}\label{sec:ch4/sect1}

Темные состояния атомных ансамблей не взаимодействуют со светом: не могут излучать и
поглощать фотоны по причине возникающей деструктивной интерференции. Таким образом, будучи свободными от декогеренции, они могут быть широко использованы в квантовых вычислениях (в частности, как механизм для создания квантовой памяти). На сегодняшний день структура темных состояний двухуровневых атомов достаточно хорошо изучена. Для ансамблей многоуровневых атомов вопрос об их структуре по-прежнему остается открытым.

В работе \cite{dark_states_dimension} было установлено, что размерность темного подпространства пространства ансамблей двухуровневых атомов соответствует числам Каталана. Обобщение данного утверждения на случай трехуровневых (и тем более, многоуровневых) атомных ансамблей и его строгое доказательство представляется весьма трудоемким и к настоящему времени не проведено.

В данной главе будет представлен суперкомпьютерный алгоритм численного подтверждения данной гипотезы для ансамблей, состоящих из ограниченного числа трехуровневых
атомов.

Рассмотрим модель Тависа-Каммингса, описывающую взаимодействие ансамблей идентичных атомов с фотонами в оптическом резонаторе. Ее гамильтониан в случае слабого
взаимодействия $g \ll \hbar w$ (приближение RWA \cite{ozhigov_qq}) имеет следующий вид:
\[
H_{\text{TC}} = \hbar w_{c}a^{+}a + \hbar w_{a} \sum_{i=1}^{n}\s_{i}^{+}\s_{i} + \sum_{i=1}^{n}g_{i}(a^{+}\s_{i} + a\s^{+}_{i}),
\]
где
\begin{itemize}
	\item[$\bullet$]{$\hbar$ -- постоянная Планка,}
	\item[$\bullet$]{$w_{c}$ -- частота фотонов моды резонатора,}
	\item[$\bullet$]{$w_{a}$ -- частота атомного перехода,}
	\item[$\bullet$]{$g_{i}$ -- сила взаимодействия $i$-го атома с полем,}
	\item[$\bullet$]{$n$ --  число атомов в полости,}
	\item[$\bullet$]{$a^{+}, a$ -- операторы рождения и уничтожения фотона в полости \cite{messia}:\\
		\begin{equation}
			a^{+}|m\ra = \sqrt{m+1}|m+1\ra,\qquad\qquad a|m\ra = \sqrt{m}|m-1\ra,
		\end{equation}
		\begin{center}($m$ -- количество фотонов в полости)\end{center}
		\
		
		\begin{equation}
			a =
			\bordermatrix{
				& |0\rangle & |1\rangle & |2\rangle & \cdots & |m-1\rangle & |m\rangle \cr
				|0\rangle & 0 & 1 & 0 & \cdots & \cdots & 0 \cr
				|1\rangle & \vdots & 0 & \sqrt{2} & \ddots &  & \vdots \cr
				|2\rangle & \vdots &  & \ddots & \ddots & \ddots & \vdots \cr
				\cdots & \vdots &  &  & \ddots & \ddots & 0 \cr
				|m-1\rangle & 0 & \cdots & \cdots & \cdots & 0 & \sqrt{m} \cr
				|m\rangle & 0 & \cdots & \cdots & \cdots & \cdots & 0\cr
			},
		\end{equation}
	},\\[12pt]
	\begin{equation}
		a^{+} =
		\bordermatrix{
			& |0\rangle & |1\rangle & |2\rangle & \cdots & |m-1\rangle & |m\rangle \cr
			|0\rangle & 0 & 0 & \cdots & \cdots & 0 & 0 \cr
			|1\rangle & 1 & 0 &  &  & \vdots & \vdots \cr
			|2\rangle & 0 & \sqrt{2} & \ddots & & \vdots & \vdots \cr
			\cdots & \vdots & \ddots & \ddots & \ddots & \vdots & \vdots \cr
			|m-1\rangle & \vdots &  & \ddots & \ddots & 0 & \vdots \cr
			|m\rangle & 0 & \cdots & \cdots & 0 & \sqrt{m} & 0\cr
		},
	\end{equation}
	,\\
	\item[$\bullet$]{$\s^{+}_{i}, \s_{i}$ -- повышающий и понижающий операторы $i$-го атома, действующие на основное $|0\ra$ и возбужденное $|1\ra$ состояния соответственно:\\
		\begin{equation}
			\begin{split}
				\s_{i}|0\ra_{i} = 0,\qquad\qquad\qquad\qquad\s^{+}_{i}|0\ra_{i} = |1\ra_{i},\\
				\noindent\s_{i}|1\ra_{i} = |0\ra_{i},\qquad\qquad\qquad\qquad\s^{+}_{i}|1\ra_{i} = 0.
			\end{split}
		\end{equation}
		
	}
\end{itemize}
\
\\[12pt]
Для двухуровневых атомов операторы $\s^{+}_{i}$ и $\s_{i}$ имеют следующий вид:
\begin{equation}
	\sigma = \bordermatrix{
		& |0\rangle &
		|1\rangle \cr 
		|0\rangle & 0 & 1 \cr 
		|1\rangle & 0 & 0 \cr },
	\qquad\qquad
	\sigma^{+} = \bordermatrix{ 
		& |0\rangle & |1\rangle \cr
		|0\rangle &
		0 & 0 \cr
		|1\rangle & 1 & 0 \cr 
	}.
\end{equation}

Для простоты будем считать, что частота фотона $w_{c}$ отличается от частоты атомного
перехода $w_{a}$ на величину небольшой расстройки $|w_{c} - w_{a}| \ll w_{c}$ и, кроме того, сила взаимодействия атома с полем одинакова для всех атомов:
\[
g_{i} = g\qquad\forall i = \overline{1,~n}.
\]

Также обозначим через $\overline{\s}$ и $\overline{\s}^{+}$ операторы, действующие на атомный ансамбль:
\[
\overline{\s} = \sum_{i=1}^{n}{\s_{i}} =
\s_{1} \otimes I_{2}  \otimes \dots \otimes I_{n} + I_{1}
\otimes \s_{2} \otimes I_{3} \otimes \dots \otimes I_{n} + \dots
+ I_{1} \otimes \dots \otimes I_{n-1} \otimes \s_{n},
\]
$\displaystyle\overline{\s}^{+} = \sum_{i=1}^{n}{\s^{+}_{i}}$ определяется аналогичным образом.\\[12pt]
Здесь наличие оператора $\s_{j}/\s^{+}_{j}$ означает релаксацию/возбуждение $j$-го атома, наличие оператора $I_{j}$ означает отсутствие воздействия на состояние $j$-го атома.\\[12pt]
Их действие позволит нам определить возможность испускания/поглощения одиночного
фотона хотя бы одним атомом ансамбля.\\[12pt]
Только для темных состояний одновременное действие обоих операторов будет давать нулевой эффект:
\begin{equation}\label{critetion}
	\begin{cases}
		\bm{
			\overline{\s}^{+}|\Psi\ra_{at} = 0 \qquad \textbf{(атомы не могут поглотить фотон)}},\\
		\bm{
			\overline{\s}|\Psi\ra_{at} = 0 ~~\qquad \textbf{(атомы не могут испустить фотон).}}
	\end{cases}
\end{equation}
Здесь $\displaystyle|\Psi\ra_{\text{at}} = \sum_{i=1}^{n}\lambda_{i}|i\ra_{\text{at}}$ --- произвольное состояние атомного ансамбля\\($\lambda_{i}$ соответствует амплитуде состояния, $|i\ra_{\text{at}}$ --- уровню возбуждения $i$-го атома).\\[12pt]
Таким образом, условие \eqref{critetion} является критерием темноты атомного ансамбля, что непосредственно следует из определения темного состояния.

В работе \cite{dark_states_dimension} было сформулировано и строго доказано утверждение о том, что размерность
темного подпространства пространства $n$ двухуровневых атомов равна
\begin{equation}\label{dim_d_2_n}
	\text{dim}(D_{n}^2) =	
	\begin{cases}
		C_{n}^{k} - C_{n}^{k-1}\quad~~\text{при}~n = 2k, \\
		0 \qquad\qquad\qquad\text{в противном случае},
	\end{cases}
\end{equation}
и все наборы темных состояний (для четного числа атомов в группе) с учетом нормировки имеют вид 
\begin{equation}\label{d_2_n}
	\frac{1}{2^{n/4}}\bigotimes_{j=1}^{n/2}(|01\ra_{j}-|10\ra_{j}),
\end{equation}
где индекс $j$ означает номер пары $j = 1,\dots, n/2$ при произвольном разбиении группы из $n$ атомов.
\\[24pt]
\textbf{Приведем несколько примеров:}\\

\noindent
$n = 2$:
\begin{itemize}
	\item[$\triangledown$]{темные состояния: $|0\rangle_{1}|1\rangle_{2} - |1\rangle_{1}|0\rangle_{2}$}
	\item[$\triangledown$]{размерность темного подпространства: $C_{2}^{1} - C_{2}^{0} = 1$\\}
\end{itemize}
$n = 3$:
\begin{itemize}
	\item[$\triangledown$]{нет темных состояний\\}
\end{itemize}
$n = 4$:
\begin{itemize}
	\item[$\triangledown$]{темные состояния:\\$(|0\rangle_{1}|1\rangle_{2} - |1\rangle_{1}|0\rangle_{2})\otimes(|0\rangle_{3}|1\rangle_{4} - |1\rangle_{3}|0\rangle_{4})$\\
		$(|0\rangle_{1}|1\rangle_{3} - |1\rangle_{1}|0\rangle_{3})\otimes(|0\rangle_{2}|1\rangle_{4} - |1\rangle_{2}|0\rangle_{4})$}
	\item[$\triangledown$]{размерность темного подпространства: $C_{4}^{2} - C_{4}^{1} = 2$\\}
\end{itemize}
и так далее.

\section{Постановка задачи}\label{sec:ch4/sect2}
Результат \eqref{dim_d_2_n}, \eqref{d_2_n} справедлив и доказан только для случая двухуровневых атомных ансамблей. Аналогичное утверждение для ансамблей трехуровневых
атомов в качестве гипотезы формулируется следующим образом:
\begin{hyp}
	\label{Th:20}Темное подпространство пространства n трехуровневых атомов есть линейная оболочка состояний $\displaystyle \bigotimes_{j=1}^{n/3}{\widehat{D}_{3}^{(j)}}$,
	где $\widehat{D}_{3}^{(j)}$ --- \textbf{трехатомное состояние, имеющее вид}
	\begin{equation}
		\sum_{\pi \in S_{3}}|\pi(1)\rangle|\pi(2)\rangle|\pi(3)\rangle(-1)^{\sigma(\pi)}
	\end{equation}
	(разбиение $n$ атомов на тройки произвольно),
	и его размерность равна:
	\begin{equation}
		\dim(D^{(3)}_{n}) =
		\begin{cases}
			C_{n}^{k} - C_{n}^{k-1} \qquad\textbf{при n = 3k},\\
			0, \qquad~~\quad\qquad\textbf{в противном случае}.
		\end{cases}\label{eq:dim3}
	\end{equation}
\end{hyp}
\
\\
\indent Примером темного состояния ансамбля трехуровневых атомов является состояние
$
|\Psi\rangle = |012\rangle + |120\rangle + |201\rangle - |021\rangle - |102\rangle - |210\rangle$ (оно же единственное).

Возвращаясь к критерию темноты \eqref{critetion} многоатомного квантового состояния, отметим, что множество решений системы
\begin{equation}\label{slae}
	\begin{cases}
		\overline{\sigma}^{+}|\Psi\rangle_{\text{at}} = 0,\\
		\overline{\sigma}|\Psi\rangle_{\text{at}} = 0,
	\end{cases} \Leftrightarrow
	Ax =
	\begin{pmatrix}
		\overline{\sigma}^{+}\\
		\overline{\sigma}
	\end{pmatrix}
	\begin{pmatrix}
		\lambda_1\\
		\dots\\
		\lambda_N\\
	\end{pmatrix} = 0
\end{equation}
однородных уравнений с соответствующей матрицей системы
$
A=\begin{pmatrix}
	\overline{\sigma}^{+}\\
	\overline{\sigma}
\end{pmatrix}
$ размерности $M \times N$ есть линейное подпространство размерности $N -
rank(A)$. Размерность матрицы системы для трехуровневых атомов равна $(M, N) = (6 \cdot 3^{n}, 3^{n})$. Таким образом, определение размерности темного подпространства сводится к задаче определения ранга матрицы $A$ системы для различных значений $n$.

Предложенный далее алгоритм численно установит, что
\begin{equation}\label{eq:dim3}
	\dim(D^{(3)}_{n}) = 3^{n} - rank(A) =
	\begin{cases}
		C_{n}^{k} - C_{n}^{k-1}\qquad\text{при}~n=3k,\\
		0\quad\qquad\qquad\quad~\text{при}~n \ne 3k.
	\end{cases}
\end{equation}
\
\\[12pt]
\indent Перейдем к перечислению трудностей, возникающих при решении поставленной задачи, а именно точного вычисления ранга сверхбольшой матрицы.

\clearpage
\noindent \textbf{Рассмотрим несколько примеров}\\[12pt]Двухуровневые ансамбли, $\mathbf{p = 2}$:\\
\medskip\hrule\medskip
\noindent$\mathbf{p = 2, n = 2:}$
\begin{equation}\label{eq:matrix2}
	{\footnotesize
		A =
		\bordermatrix{
			& |00\rangle & |01\rangle & |10\rangle & |11\rangle \cr
			& 0 & 1 & 1 & 0 \cr
			& 0 & 0 & 0 & 1 \cr
			& 0 & 0 & 0 & 1 \cr
			& 0 & 0 & 0 & 0 \cr
			& 0 & 0 & 0 & 0 \cr
			& 1 & 0 & 0 & 0 \cr
			& 1 & 0 & 0 & 0 \cr
			& 0 & 1 & 1 & 0 \cr
		}\rightarrow
		\bordermatrix{
			& |00\rangle & |01\rangle & |10\rangle & |11\rangle \cr
			& 1 & 0 & 0 & 0 \cr
			& 0 & 1 & 1 & 0 \cr
			& 0 & 0 & 0 & 1 \cr
		}
	}
\end{equation}
\quad\quad~~Решая систему \eqref{slae}, находим
{\footnotesize
	$\lambda =
	\begin{pmatrix}
		0\\
		\xi\\
		-\xi\\
		0
	\end{pmatrix}
	$
},\
\quad$D^{(2)}_p = \L(\{|01\rangle - |10\rangle\})$.\\[12pt]

\noindent\quad\quad~~~$\dim(D^{(2)}_p) = 2^{p} - rank(A) = 2^{2} - 3 = 1$\\

\medskip\hrule\medskip

\noindent$\mathbf{p = 2, n = 3:}$
{\footnotesize
	\[
	A =
	\begin{pmatrix}
		\overline{\sigma}^{+}\\
		\overline{\sigma}
	\end{pmatrix}=
	\begin{pmatrix}
		0 & 1 & 1 & 0 & 1 & 0 & 0 & 0\\
		0 & 0 & 0 & 1 & 0 & 1 & 0 & 0\\
		0 & 0 & 0 & 1 & 0 & 0 & 1 & 0\\
		0 & 0 & 0 & 0 & 0 & 0 & 0 & 1\\
		0 & 0 & 0 & 0 & 0 & 1 & 1 & 0\\
		0 & 0 & 0 & 0 & 0 & 0 & 0 & 1\\
		0 & 0 & 0 & 0 & 0 & 0 & 0 & 1\\
		0 & 0 & 0 & 0 & 0 & 0 & 0 & 0\\
		0 & 0 & 0 & 0 & 0 & 0 & 0 & 0\\
		1 & 0 & 0 & 0 & 0 & 0 & 0 & 0\\
		1 & 0 & 0 & 0 & 0 & 0 & 0 & 0\\
		0 & 1 & 1 & 0 & 0 & 0 & 0 & 0\\
		1 & 0 & 0 & 0 & 0 & 0 & 0 & 0\\
		0 & 1 & 0 & 0 & 1 & 0 & 0 & 0\\
		0 & 0 & 1 & 0 & 1 & 0 & 0 & 0\\
		0 & 0 & 0 & 1 & 0 & 1 & 1 & 0\\
	\end{pmatrix}\rightarrow
	\begin{pmatrix}
		1 & 0 & 0 & 0 & 0 & 0 & 0 & 0\\
		0 & 1 & 0 & 0 & 0 & 0 & 0 & 0\\
		0 & 0 & 1 & 0 & 0 & 0 & 0 & 0\\
		0 & 0 & 0 & 1 & 0 & 0 & 0 & 0\\
		0 & 0 & 0 & 0 & 1 & 0 & 0 & 0\\
		0 & 0 & 0 & 0 & 0 & 1 & 0 & 0\\
		0 & 0 & 0 & 0 & 0 & 0 & 1 & 0\\
		0 & 0 & 0 & 0 & 0 & 0 & 0 & 1\\
	\end{pmatrix}
	\]
}
\noindent\quad\quad~~~$\dim(D^{(2)}_p) = 2^{p} - rank(A) = 2^{3} - 8 = 0$\\

\clearpage
\indent Исходя из вышеуказанных примеров, можно отметить следующее::
\begin{itemize}
	\item[$\bullet$]{изначально (до процедуры приведения к ступенчатому виду) матрица состоит из нулей и единиц,}
	\item[$\bullet$]{матрица является сильно разреженной с множеством нулевых строк,}
	\item[$\bullet$]{подавляющее большинство строк матрицы нетривиальны: не соответствуют строкам единичной матрицы, в значительной части из них количество единиц сильно превосходит количество нулей,}
	\item[$\bullet$]{размерность матрицы в случае трехуровневой системы равна $6 \cdot 3^{n} \times 3^{n}$, т.к. присутствуют возбуждения и релаксации атомов между уровнями: имеются переходы трех типов --- $\sigma^{0,1}_i$, $\sigma^{1,2}_i$,
		$\sigma^{0,2}_i$.}
\end{itemize}
\
\\[0pt]
\indent Зависимость размерности матрицы системы от числа $n$ атомов в ансамбле носит экспоненциальный характер:

\noindent\begin{tabular}[t]{|p{4em}|p{5em}|p{4em}|p{9em}|p{9em}|}
	\hline
	$n$ & 3 &  & 18 & 21 \\
	\hline
	$M$ x $N$ & $162 \times 27$ & $\quad~\cdots$ & $2.3 \cdot 10^9 \times 387 \cdot 10^6$ & $62 \cdot 10^9 \times 10.4 \cdot 10^9$ \\
	\hline
\end{tabular}
\
\\[12pt]

В связи с этим, перечислим некоторые вычислительные трудности:
\begin{itemize}
	\item[$\bullet$]{вещественные плотные матрицы размера $(6 \cdot 3^{n}) \times 3^{n}$, начиная уже с малых значений $n$, не умещаются целиком в оперативную память,}
	\item[$\bullet$]{алгоритм должен точно вычислять ранг: ошибки округления при работе с действительными числами могут повлиять на вычисления.\\
		Такие инциденты, как
		\[
		\begin{pmatrix}
			0 & \dots & 0.333333 & \dots \\
			0 & \dots & 0.3333334 & \dots \\
		\end{pmatrix}\qquad\text{дают неверный ответ}~rank(A) = 2.
		\]}
	
	Кроме того, было обнаружено, что многие существующие алгоритмы вычисления
	ранга разреженной матрицы дают неверный результат, начиная с $n = 9$ (в частности, метод $\mathsf{linalg.interpolative.estimate\_rank}$ библиотеки $\mathbf{scipy}$). Данный факт означает \textbf{необходимость вычисления ранга в целых числах}.
	\item[$\bullet$]{применение алгоритма Гаусса приведения матрицы к ступенчатому виду занимает неприемлемо большое вычислительное время и не может быть использовано для матриц такого размера}.
\end{itemize}

\section{Описание алгоритма}\label{sec:ch4/sect3}

Перейдем непосредственно к описанию предложенного алгоритма.
Он будет состоять из трех частей:
\begin{enumerate}
	\item{построение разреженной матрицы системы $A$};
	\item{целочисленное приведение разреженной матрицы к ступенчатой форме путем редуцирования соответствующего ей графа};
	\item{окончательное целочисленное приведение матрицы к ступенчатой форме с помощью
		алгоритма Гаусса}.
\end{enumerate}

\subsection{Построение разреженной матрицы системы}\label{subsec:ch4/subsect1}
Генерация разреженных матриц для различных $n = 3\dots21$ выполняется стандартным
образом с использованием научного пакета Python SciPy для разреженных матриц.

Каждая строка содержит номера столбцов ненулевых элементов (нумерация столбцов
начинается с нуля). Полностью нулевые строки плотной матрицы пропускаются.

К примеру, для системы \eqref{eq:matrix2} разреженная матрица записывается в следующем виде:
\begin{flushleft}
	\noindent \qquad 1,2\\
	\noindent\qquad 3,\\
	\noindent\qquad 3,\\
	\noindent\qquad 0,\\
	\noindent\qquad 0,\\
	\noindent\qquad 1,2\\
\end{flushleft}\label{eq:sparse_matrix2}
\
\\[12pt]
\noindent Для трехуровневой системы построение выполняется аналогично.

\clearpage
\noindent Выпишем характерные метрики после первого этапа алгоритма:

\noindent
{\footnotesize
	\begin{tabular}[t]{|p{5em}|p{3em}|p{4em}|p{5em}|p{5em}|p{5em}|p{5em}|p{5em}|}
		\hline
		$n$ & 3 & 6 & 9 & 12 & 15 & 18 & 21 \\
		\hline
		$\mathrm{M}$ & 162 & 4374 & $118 \cdot 10^{3}$ & $3.2 \cdot 10^{6}$ & $86.1 \cdot 10^{6}$ & $2.3 \cdot 10^{9}$ & $62.7 \cdot 10^{9}$ \\
		\hline
		$\mathrm{N}$ & 27 & 729 & $19.6 \cdot 10^{3}$ & $531.4 \cdot 10^{3}$ & $14.3 \cdot 10^{6}$ & $387.4 \cdot 10^{6}$ & $10.4 \cdot 10^{9}$ \\
		\hline
		$\mathbf{nrows}$ & 114 & 3990 & $\approx M$ & $\cdots$ & $\cdots$ & $\cdots$ & $\approx M$ \\
		\hline
		$\mathbf{nonzeros}$ & 162 & 8748 & $354.3 \cdot 10^{3}$ & $12.7 \cdot 10^{6}$ & $430.4 \cdot 10^{6}$ & $13.9 \cdot 10^{9}$ & $439.3 \cdot 10^{9}$ \\
		\hline
	\end{tabular}
}

\begin{itemize}
	\item[$\bullet$]{$n$  --- кол-во атомов},
	\item[$\bullet$]{$\mathrm{M}$ --- кол-во строк в плотной\footnote[1]{плотные матрицы не создаются} матрице,}
	\item[$\bullet$]{$\mathrm{N}$ --- кол-во столбцов в плотной матрице,}
	\item[$\bullet$]{$\mathbf{nrows}$  --- кол-во строк в разреженной матрице,}
	\item[$\bullet$]{$\mathbf{nonzeros}$ --- кол-во ненулевых элементов (единиц) в разреженной матрице.}
\end{itemize}
\
\\[0pt]
\noindent Таким образом, при $n = 21$ мы имеем граф, состоящий из нескольких десятков миллиардов вершин. Обработка такого количества вершин --- стандартная задача для суперкомпьютера с распределенной памятью и тысячами процессорных ядер (в данном случае: Ломоносов-2).

\subsection{Целочисленное приведение матрицы к ступенчатой форме при помощи
	параллельного графового алгоритма}\label{subsec:ch4/subsect2}

Для разреженной матрицы системы построим граф по следующей схеме:
\begin{itemize}
	\item[$\bullet$]{значение вершины --- номер строки в разреженной матрице,}
	\item[$\bullet$]{метки ребра, входящего в вершину, совпадает с позициями единиц в соответствующей
		строке,}
	\item[$\bullet$]{вершины, в которые входят ребра с одинаковыми метками, соединяются ребром с этими метками}.
\end{itemize}
\
\\[0pt]
\indent Для системы \eqref{eq:matrix2} и соответствующей ее разреженной матрицы граф выглядит следующим образом:\\[12pt]

\hspace{1em}\underline{\hspace{4em} Graph \hspace{4em} | \hspace{1em} Sparse matrix \hspace{2em} | \hspace{3em} Dense matrix \hspace{3em}}\\

\[
\hspace{3.5em}\parbox[b][5cm][t]{50mm}{
	\includegraphics[width=36mm]{Dissertation/images/section_4/graph1.eps}
}
\hfill
\parbox[b][5cm][t]{35mm}{
	\begin{flushleft}
		Line 0: \quad 1,2\\
		Line 1: \quad 3,\\
		Line 2: \quad 3,\\
		Line 3: \quad 0,\\
		Line 4: \quad 0,\\
		Line 5: \quad 1,2
	\end{flushleft}
}
\hfill
\parbox[b][5cm][t]{50mm}{
	\bordermatrix{
		& |00\rangle & |01\rangle & |10\rangle & |11\rangle \cr
		& 0 & 1 & 1 & 0 \cr
		& 0 & 0 & 0 & 1 \cr
		& 0 & 0 & 0 & 1 \cr
		& 0 & 0 & 0 & 0 \cr
		& 0 & 0 & 0 & 0 \cr
		& 1 & 0 & 0 & 0 \cr
		& 1 & 0 & 0 & 0 \cr
		& 0 & 1 & 1 & 0 \cr
	}
}\label{eq:graph}
\]
\
\\

Данный граф позволяет производить вычитание строк, и таким образом, чтобы отрицательные числа не появлялись в соответствующей плотной матрице системы. Удаление дубликатов и вычитание строк друг из друга выполняются путем удаления соответствующих ребер графа. Вершины, не имеющие входных и выходных ребер, удаляются (что соответствует удалению нулевых строк в исходной матрице после элементарных преобразований).

Удаление ребер в данном примере происходит следующим образом:
\begin{figure}[h]
	\noindent\centering{
		\includegraphics[width=35mm]{Dissertation/images/section_4/graph2.eps}
	}
\end{figure}

Рассмотрим более сложный пример:
$
A =
\begin{pmatrix}
	\cdots & \cdots & \cdots & \cdots & \cdots\\
	1 & 0 & 0 & 0 & 1 \cr
	0 & 1 & 0 & 0 & 0 \cr
	1 & 0 & 0 & 1 & 1 \cr
	0 & 0 & 1 & 0 & 0 \cr
	\cdots & \cdots & \cdots & \cdots & \cdots
\end{pmatrix}.
$

\clearpage
Соответствующие преобразования графа:
\begin{figure}[h]
	\noindent\centering{
		\includegraphics[width=120mm]{Dissertation/images/section_4/graph7.eps}
	}
	\label{figCurves}
\end{figure}

Последний пример демонстрирует возможность удаления дубликатов и вычитания строк,
находящихся на большом расстоянии. Для этого достаточно определить пересечения меток ребер, входящих и выходящих в связанные вершины. Удаление дубликатов может
осуществляться по цепочке (снизу вверх или наоборот). В результате мы избегаем сравнения строк друг с другом, поиска совпадающих пар и использования операции деления
в ходе элементарных преобразований. Более сложным образом с помощью графа можно
суммировать строки для последующих вычитаний или удаления дубликатов. К примеру, производить следующие элементарные преобразования:
\[
\begin{pmatrix}
	1 & 0 & 0 & 1\\
	0 & 1 & 1 & 0\\
	1 & 0 & 1 & 0\\
	0 & 1 & 0 & 1\\
\end{pmatrix}\rightarrow
\begin{pmatrix}
	1 & 0 & 0 & 1\\
	0 & 1 & 1 & 0\\
	1 & 0 & 1 & 0\\
	1 & 1 & 1 & 1\\
\end{pmatrix}
\rightarrow
\begin{pmatrix}
	1 & 0 & 0 & 1\\
	0 & 1 & 1 & 0\\
	0 & 0 & 1 & 0\\
	1 & 0 & 0 & 1\\
\end{pmatrix}
\rightarrow
\begin{pmatrix}
	1 & 0 & 0 & 1\\
	0 & 1 & 0 & 0\\
	1 & 0 & 1 & 0\\
	0 & 1 & 0 & 1\\
\end{pmatrix}\rightarrow\cdots.
\]

Предложенный алгоритм является параллельным и производит редукцию графа, распределенного по сетке процессоров: разреженные матрицы считываются параллельно ленточным образом (каждый процесс считывает свою часть матрицы и формирует свою часть
общего графа). Если связанные вершины принадлежат разным процессам, информация о
них пересылается в неблокирующем режиме (MPI\_Isend, MPI\_Irecv). При необходимости удаления вершин и/или ребер информация пересылается между соответствующими
ядрами.

\subsection{Окончательное целочисленное приведение матрицы к ступенчатой форме с помощью алгоритма Гаусса}\label{subsec:ch4/subsect3}
Результат второго этапа --- получение матрицы, частично приведенной к ступенчатому виду. Ее достаточно просто восстановить по редуцированному графу. Количество строк в
такой матрице на несколько порядков отличается от числа строк в исходной плотной матрице системы. Окончательное приведение матрицы к ступенчатому виду производится при
помощи алгоритма Гаусса без делений (за ислючением целочисленного деления строки).

\section{Выводы главы}\label{sec:ch4/sect4}
Размерность темного подпространства трехуровневых ансамблей атомов была численно
установлена для следующих значений $n = 3k$:

\noindent\begin{tabular}[t]{|p{5em}|p{3em}|p{3em}|p{3em}|p{3em}|p{4em}|p{4em}|p{4em}|}
	\hline
	$n$ & 3 & 6 & 9 & 12 & 15 & 18 & 21 \\
	\hline
	$\dim(D^{(3)}_n)$ & 1 & 5 & 28 & 165 & 1001 & 6188 & 38 760 \\
	\hline
\end{tabular}\
\\[12pt]

\noindent В остальных случаях для $1 \le n \le 21,~n \ne 3k$ работа алгоритма завершилась приведением матрицы системы \eqref{slae} к единичной матрице c установлением $\dim(D^{(3)}_{n}) = 0$.
\\[12pt]
\noindentТаким образом, для ансамблей, состоящих из $1 \le n \le 21$ трехуровневых атомов, численно была подтверждена гипотеза о соответствии размерности темного подпространства числам Каталана.
           % Глава 4
\chapter*{Заключение}                       % Заголовок
\addcontentsline{toc}{chapter}{Заключение}  % Добавляем его в оглавление

\noindent Приведем в заключении основные результаты работы.
\\[30pt]
\textbf{I. Коллективные осцилляции многоатомных ансамблей.}
\\[18pt]
Была проанализирована динамика квантовых состояний ансамблей двухуров­невых атомов и одномодового поля резонатора в рамках модели Тависа­-Каммингса, а также модели Тависа-Каммингса-Хаббарда для случая двух взаимодействующих полостей в приближении RWA \cite{ozhigov_qq, rwa_1, rwa_2}.
\\[18pt]
\noindent По результатам компьютерного моделирования:
\begin{itemize}
	\item[$\bullet$]{установлен резкий характер осцилляций между двумя группами атомов равной численности и равной силы взаимодействия атомов с полем,}
	\item[$\bullet$]{установлено, что резкость осцилляций в ансамбле с четным числом ато­мов предсказуемо растет с увеличением фотонной накачки в полости и намного превосходит резкость осцилляций Раби \cite{rabi_1,rabi_2,rabi_3,rabi_4} для одного атома,}
	\item[$\bullet$]{численно найдена зависимость качества осцилляций от силы взаимодей­ствия атомов с полем; показано, что удлинение периода осцилляций при уменьшении силы взаимодействия может быть скомпенсировано увели­чением числа атомов в группе и усилением фотонной накачки,}
	\item[$\bullet$]{обнаружено хорошо регистрируемое <<квантовое эхо>>: переход состоя­ния атомного ансамбля из одной полости в другую и возврат этого состояния в первоначальную полость. Установлено высокое качество та­кого эха. По итогам многократных численных экспериментов для ряда квантовых систем была установлена граница его возникновения, выра­жающаяся через соотношение интенсивности перехода фотонов между полостями и силы взаимодействия атомов с полем.
		\\[18pt]
		Такого рода эхо представляет собой важный феномен, поскольку оно может быть использовано для переноса многочастичного состояния в процессе квантовой динамики, а также может послужить механизмом организации временной квантовой памяти.
	}
\end{itemize} 

\clearpage
\noindent\textbf{II. Квантовое бутылочное горлышко в атомных превращениях.}
\\[18pt]
Установлен парадоксальный эффект квантового бутылочного горлышка для процесса интенсивного охлаждения атома, который переходит в необратимое состояние, находясь в возбужденном состоянии.
\\[18pt]
В результате численного моделирования было обнаружено, что превышение некоторого порога интенсивности охлаждения ведет к росту вероятности та­кого перехода, что невозможно при классическом описании процесса.
\\[24pt]
\noindent\textbf{III. Оптический отбор темных состояний ансамблей многоуровневых атомов.}
\\[18pt]
Предложен метод оптического отбора темных состо­яний атомов, основанный на томографии состояния поля вне оптической полости. Данный метод не требует применения штарковского сдвига уровней и практически не зависит от выбора начального состояния ато­мов, помещенных в оптический резонатор.
\\[18pt]
\noindent Процесс оптического отбора был численно промоделирован для ансамблей двух­уровневых и трехуровневых атомов спектра Rb85. На основе собранной статистики времён срабатывания детектора, регистрирующего вылет фотона, было установлено его среднее время жизни  в полости, характерное для темных атомных состояний различного вида. Для трехуровневых атомов это, в частности, позволило сепарировать полностью тем­ное трехатомное состояние (мультисинглет) от состояний, содержащих темную двухатомную компоненту.
\\[18pt]
Описанный метод, в силу технической простоты его реализации, может быть использован для генерации темных состояний ансамблей, состоящих из нескольких десятков атомов, что может быть полезным для организации квантовых вычислений, защищенных от декогерентности.
\\[24pt]
\noindent\textbf{IV. Определение размерности темного подпространства пространства многоуровневых атомных ансамблей.}
\\[18pt]
\noindent Предложен алгоритм определения размерности темного подпространства пространства многоуровных атомных ансамблей, основанный на целочислен­ном вычислении ранга сверхбольшой двоичной матрицы путем параллельной редукции соответствующего ей графа.
\\[18pt]
Данный алгоритм позволил установить размерность темного подпространства для ансамблей, содержащих до 21 трехуровневого атома включительно. Для некратного трем количества атомов в группе работа алгоритма завершилась установлением размерности темного подпространства, равной нулю, означаю­щему отсутствие темных состояний в этом пространстве.
\\[18pt]
Гипотеза о структуре и явном виде темных состояний была подтверждена для 1, 3, 6 и 9 трехуровневых атомов в группе. Было показано, что все темные со­стояния в этом случае есть линейные комбинации мультисинглетов.
\\[24pt]
\noindent\textbf{V. Моделирование запутывающего гейта сoCSign на асинхронных атомных возбуждениях.}
\\[18pt]
\noindent Проведены оценки качества управления гейтом coCSign. Установлено, что ос­новным фактором его снижения в рамках модели JCH является увеличение ширины спектральных линий рабочей моды резонатора: данный фактор имеет фундаментальную природу и проистекает из соотношения неопределенностей <<время-энергия>>. По результатам компьютерного моделирования гейта coCSign в отсутствие данного фактора, а также иных ограничений, чисто технического характера, точность его срабатывания составила порядка 95\%.
\\[24pt]
\hyperref[appendix]{Приложение} содержит листинг программного комплекса первой главы.
      % Заключение
\include{Dissertation/acronyms}        % Список сокращений и условных обозначений
\chapter*{Список сокращений и условных обозначений} % Заголовок
\addcontentsline{toc}{chapter}{Список сокращений и условных обозначений} % Добавляем его в оглавление

\noindent\textbf{JC (Jaynes-Cummings model)} --- модель Джейнса-Каммингса.\\
\noindent\textbf{TC (Tavis-Cummings model)} --- модель Тависа-Каммингса.\\
\noindent\textbf{JCH (Jaynes-Cummings-Hubbard model)} --- модель Джейнса-Каммингса-Хаббарда.\\
\noindent\textbf{TCH (Tavis-Cummings-Hubbard model)} --- модель Тависа-Каммингса-Хаббарда.\\
\noindent\textbf{RWA (rotating-wave approximation)} --- приближение вращающейся волны, при котором слагаемые
$a^{+}\s^{+}$, $a\s$ гамильтониана моделей JC/JCH/TC/TCH, не сохраняющие энергию, можно игнорировать. В картине взаимодействия они быстро осциллируют, что делает их вклад незначительным.\\
\noindent\textbf{GPU (graphics processing unit)} --- графический процессор, который предназначен для ускорения рендеринга графики и параллельных вычислений. GPU содержат от нескольких сотен до нескольких тысяч вычислительных ядер. Благодаря особенностям своей архитектуры и возможностям параллельной обработки данных GPU широко применяются в системах искусственного интеллекта, машинного обучения и высокопроизводительных вычислений.\\
\noindent\textbf{SMP (symmetric multiprocessing}, или \textbf{shared-memory multiprocessing)} --- архитектура многопроцессорных компьютеров, в которой два или более одинаковых процессора сравнимой производительности подключаются единообразно к общей памяти и имеют равный доступ ко всем ресурсам вычислительной системы (в силу чего, система и называется симметричной).\\
\noindent\textbf{MPP (massive parallel processing, массивно-параллельная архитектура)} --- класс архитектур параллельных вычислительных систем. Особенность архитектуры состоит в том, что оперативная память физически разделена между процессорами. Главным преимуществом систем с разделенной памятью является их высокая масштабируемость.\\
\noindent\textbf{MPI (message passing interface, интерфейс передачи сообщений)} --- программный интерфейс передачи информации для обмена сообщениями между процессами, синхронизации выполнения задач, а также управления потоками данных в параллельных вычислениях. Является наиболее распространенным стандартом для построения высокопроизводительных программ для кластеров и суперкомпьютеров.
      % Словарь терминов
\include{Dissertation/references}      % Список литературы
\clearpage
\ifdefmacro{\microtypesetup}{\microtypesetup{protrusion=false}}{} % не рекомендуется применять пакет микротипографики к автоматически генерируемым спискам
\listoffigures  % Список изображений
           % Списки таблиц и изображений (иллюстративный материал)

\setcounter{totalchapter}{\value{chapter}} % Подсчёт количества глав

%%% Настройки для приложений
\appendix
% Оформление заголовков приложений ближе к ГОСТ:
\setlength{\midchapskip}{20pt}
\renewcommand*{\afterchapternum}{\par\nobreak\vskip \midchapskip}
\renewcommand\thechapter{\Asbuk{chapter}} % Чтобы приложения русскими буквами нумеровались

\chapter*{Приложение.\\Коллективные осцилляции многоатомных ансамблей\\(программный комплекс)}\label{appendix}
\addcontentsline{toc}{chapter}{Приложение. Коллективные осцилляции многоатомных\newline\indent\hspace{35pt}ансамблей (программный комплекс)}

\lstinputlisting[mathescape, language=Python, caption=main.py, captionpos=t]{Dissertation/Code/section_1/main.py}
\
\\[0pt]
\lstinputlisting[mathescape, language=Python, caption=config.py, captionpos=t]{Dissertation/Code/section_1/config.py}

\clearpage
\lstinputlisting[mathescape, language=Python, caption=Bipartite/Cavity.py, captionpos=t]{Dissertation/Code/section_1/Bipartite/Cavity.py}
\
\\[0pt]
\lstinputlisting[mathescape, language=Python, caption=Bipartite/Evolution.py, captionpos=t]{Dissertation/Code/section_1/Bipartite/Evolution.py}
\
\\[0pt]
\lstinputlisting[mathescape, language=Python, caption=Bipartite/Hamiltonian.py, captionpos=t]{Dissertation/Code/section_1/Bipartite/Hamiltonian.py}
\
\\[0pt]
\lstinputlisting[mathescape, language=Python, caption=Bipartite/Unitary.py, captionpos=t]{Dissertation/Code/section_1/Bipartite/Unitary.py}
\
\\[0pt]
\lstinputlisting[mathescape, language=Python, caption=Bipartite/WaveFunction.py, captionpos=t]{Dissertation/Code/section_1/Bipartite/WaveFunction.py}

\clearpage
\lstinputlisting[mathescape, language=Python, caption=lib/PyPlot.py, captionpos=t]{Dissertation/Code/section_1/lib/PyPlot.py}

\nocite{chm_1,chm_2,chm_3,chm_4}
\nocite{parallel_voevodin,parallel_high_performance}
\nocite{mpi_complete_reference,mpi_shpakovsky,mpi_using}
\nocite{cpp_effective,cpp_stl,cpp_stroustrup,cpp_optimized,cpp_mastering,cpp_high_performance}
\nocite{python_1,python_2,python_3,python_4}
        % Приложения

\setcounter{totalappendix}{\value{chapter}} % Подсчёт количества приложений

\end{document}
