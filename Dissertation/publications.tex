\clearpage

\ifdefmacro{\microtypesetup}{\microtypesetup{protrusion=false}}{}
\chapter*{
	Публикации автора по теме диссертации в изданиях, индексируемых в базах данных Web of Science, Scopus и RSCI
}
\addcontentsline{toc}{chapter}{Публикации автора по теме диссертации}
\vspace{-1em}
\hypersetup{urlcolor=blue}
{\seminormalsize
\begin{itemize}[leftmargin=24pt]
	\item[A1.]{\textit{Kulagin, A. V.} Realization of Grover Search Algorithm on the Opti­cal Cavities / A. V. Kulagin, Y. I. Ozhigov // Lobachevskii Journal of Mathematics. — 2022. — Vol. 43, no. 4. — P. 864—872. — DOI: \href{https://doi.org/10.1134/S1995080222070162}{10.1134/S1995080222070162}. — [Web of Science, SCOPUS, Impact Factor: 0.7].}
	\item[A2.]{About Chemical Modifications of Finite Dimensional QED Models / Y. I. Ozhigov [et al.] // Nonlinear Phenomena in Complex Systems. — 2021. — Vol. 24, no. 3. — P. 230—241. — DOI: \href{https://doi.org/10.33581/1561-4085-2021-24-3-230-241}{10.33581/1561-4085-2021-24-3-230-241}. — [SCOPUS, Impact Factor: 0.468].}
	\item[A3.]{Quality of Control in the Tavis-Cummings-Hubbard Model / R. Düll [et al.] // Computational Mathematics and Modeling. — 2021. — Vol. 32, no. 1. — P. 75—85. — DOI: \href{https://doi.org/10.1007/s10598-021-09517-y}{10.1007/s10598-021-09517-y}. — [SCOPUS, Impact Factor: 0.602].}
	\item[A4.]{\textit{Kulagin, A. V.} Supercomputer Algorithm for Determining the Dimension of Dark Subspace / A. V. Kulagin // Lobachevskii Journal of Mathemat­ics. — 2021. — Vol. 42, no. 7. — P. 1521—1531. — DOI: \href{https://doi.org/10.1134/s1995080221070143}{10.1134/s1995080221070143}. — [Web of Science, SCOPUS, Impact Factor: 0.7].}
	\item[A5.]{\textit{Kulagin, A. V.} Optical Selection of Dark States of Multilevel Atomic Ensembles / A. V. Kulagin, Y. I. Ozhigov // Computational Mathematics and Modeling. — 2020. — Vol. 31, no. 4. — P. 431—441. — DOI: \href{https://doi.org/10.1007/s10598-021-09504-3}{10.1007/s10598-021-09504-3}. — [SCOPUS, Impact Factor: 0.602].}
	\item[A6.]{\textit{Kulagin, A. V.} Quasi-Classical Description of the ``Quantum Bottleneck'' Effect for Thermal Relaxation of an Atom in a Resonator / A. V. Kulagin, Y. I. Ozhigov, N. B. Victorova // Computational Mathematics and Mod­eling. — 2020. — Vol. 31, no. 1. — P. 1—7. — DOI: \href{https://doi.org/10.1007/s10598-020-09470-2}{10.1007/s10598-020-09470-2}. — [SCOPUS, Impact Factor: 0.602].}
	\item[A7.]{\textit{Викторова, Н. Б.} Квазиклассическое описание эффекта <<Квантовое бутылочное горлышко>> для термической релаксации атома в резона­торе / Н. Б. Викторова, А. В. Кулагин, Ю. И. Ожигов // Прикладная математика и информатика: Труды факультета ВМК МГУ имени М. В. Ломоносова. — 2019. — Т. 62. — С. 5—12. — [RINC].}
	\item[A8.]{Homogeneous atomic ensembles and single-mode field: review of simulation results / A. V. Kulagin [et al.] // Proceedings of SPIE, International Conference on Micro- and Nano-Electronics 2018, The International Society for Optical Engineering (Bellingham, WA, United States). — 2019. — Issue 11022C. — P. 110222C-1-110222C-12. — DOI: \href{http://dx.doi.org/10.1117/12.2521763}{10.1117/12.2521763}. — [SCOPUS].}
\end{itemize}
}
