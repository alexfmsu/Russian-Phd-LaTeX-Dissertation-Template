\chapter{Оценка качества запутывающего гейта coCSign на асинхронных атомных возбуждениях}\label{ch:ch5}

Конечномерные модели квантовой электродинамики Джейнса-Каммингса-Хаббарда (JCH) и Тависа-Каммингса-Хаббарда (TCH) представляют собой наиболее удобное средство компьютерного моделирования квантовой динамики атомных ансамблей, взаимодействующих со светом. Важность такой задачи обусловлена необходимостью отыскания оптимальной технологии реализации гейтов для квантовых вычислений.

Квантовые гейты представляют собой базовые элементы квантового компьютера, обеспечивающие возможность выполнения различных операций над кубитами. Их математическое описание и физическая реализация имеют решающее значение в вопросах эффективной работы квантовых систем. Современные исследования в области квантовых технологий посвящены разработке новых методов реализации квантовых гейтов, их оптимизации и улучшению качества управления кубитами.

Основой квантовых вычислений (\cite{feynman}, \cite{grover}, \cite{shor}) является \textbf{запутывающий гейт}. Как показано в работе \cite{barenco}, такого гейта вместе с однокубитными гейтами достаточно для реализации любых квантовых вычислений: \textbf{универсальным} считается набор квантовых гейтов, достаточный для осуществления любого квантового вычисления на произвольном наборе кубитов. Его можно получить, взяв, к примеру, все однокубитные гейты и любой запутывающий гейт (скажем, гейт CNOT --- управляемый NOT). Основная трудность в физической реализации квантовых гейтов заключается в обеспечении точного управления состояниями кубитов и минимизации ошибок, связанных с этим. Немаловажную роль в возникновение последних вносит декогерентность.

В данной главе мы представим результаты компьютерного моделирования запутывающего гейта coCSign, а также произведем оценку его эффективности. Будут указаны технические ограничения его физической реализации на основе системы оптических полостей. Мы также определим главный фактор, влияющий на снижение качества его работы: он проистекает непосредственно из теоретических границ применимости самой модели Джейнса-Каммингса-Хаббарда.

\clearpage
\section{Запутывающие гейты на оптических полостях}\label{sec:ch5/sect1}

Основным запутывающим гейтом в квантовых вычислениях является гейт CNOT, действующий на базисное состояние кубитов $|x, y\ra$ по следующему правилу:
\[
\text{CNOT:}~|x, y\ra \rightarrow |x, x \oplus y\ra.
\]
В работе \cite{monroe} представлена его реализация на основе колебательных степеней свободы атома. Гейт $\text{CNOT}$ эквивалентен другому запутывающему гейту $\text{CSign}$, действующему по правилу
\[
\text{CSign:}~|x, y\ra \rightarrow (-1)^{xy}|x, y\ra,
\]
поскольку один из них получается из другого при помощи гейта Адамара:
\[
\text{CNOT} = w_{y}~\cdot~\text{CSign}~\cdot~w_{y}.
\]
Здесь гейт Адамара 
\[
\displaystyle w_{y} = \frac{\s_{x} + \s_{y}}{\sqrt{2}}
\]
применяется к кубиту $|y\ra$ базисного состояния $|x, y\ra$.

Гейт $\text{CSign}$ инвертирует фазу состояния $|11\rangle$. Eсли ввести оператор \text{coCSign}, действующий по следующему правилу
\[
\text{coCSign}: |x, y\ra \rightarrow (-1)^{(x \oplus 1)y}|x, y\ra,
\]
он также будет сводиться к гейту $\text{CSign}$ при помощи однокубитных гейтов (в данном случае, гейтов $\s_{x}$, или $\text{NOT}$):
\[
\text{CSign} = \s_{x}(x)~\cdot~\text{coCSign}~\cdot~\s_{x}(x).
\]
\indent Данная схема частично использует идею работы Х. Азумы \cite{azuma}, однако ее реализация требует лишь одну дополнительную полость вместо двух.

Мы займемся компьютерным моделированием гейта coCSign и оценкой его эффективности. Поскольку $\text{CSign} = \s_{x}(x)~\cdot~\text{coCSign}~\cdot~\s_{x}(x)$, а однокубитные гейты реализуются линейными оптическими устройствами (чей вклад в погрешность пренебрежимо мал), эта оценка будет справедливой и для гейта $\text{CNOT}$.

\clearpage
\section{Общая схема гейта coCSign}\label{sec:ch5/sect2}

Кратко опишем общую схему гейта coCSign на асинхронных атомных возбуждениях. \textbf{Данная схема была предложена Ю. Ожиговым и детально описана в работе \cite{ozhigov_quantum_gates}}.

Рассмотрим оптический резонатор и один двухуровневый атом внутри него. Интенсивность взаимодействия $g$ между атомом и полем будем считать малой ($g/\hbar w \ll 1$) --- квантовая динамика в приближении вращающейся волны \cite{ozhigov_qq,rwa_1,rwa_2}, при котором гамильтониан Джейнса-Каммингса системы <<атом+поле>> имеет вид:
\begin{equation}
	\begin{split}
		H = H_{JC} = H_{0} + H_{int},\\
		H_{0} = \hbar w(a^{+}a + \s^{+}\s),\\
		H_{int} = g(a^{+}\s + a\s^{+}),
	\end{split}
\end{equation}
где $a^{+}$, $a$  --- операторы рождения и уничтожения фотона, $\s^{+}$, $\s$ --- операторы возбуждения и релаксации атома соответственно. Запишем базисные состояния поля и атома в виде $|m\ra_{ph}|n\ra_{at}$, где $m = 0, 1, 2$ --- число фотонов в полости, $n = 0, 1$ --- количество атомных возбуждений. В ходе реализации гейта мы будем менять гамильтониан, добавляя к нему слагаемое вида $H_{jump} = \nu(a_{i}a_{j}^{+} + a_{j}a_{i}^{+})$, отвечающее за переход возбуждения из полости $i$ в полость $j$ и наоборот. Значения $\nu_{x}$ и $\nu_y$ соответствуют интенсивностям перехода возбуждения из полостей $x$ и $y$ во вспомогательную полость.

\begin{figure}[h!]
	\noindent\centering{
		\includegraphics[width=0.5\textwidth]{Dissertation/images/section_5/3_cavities.png}
		\captionsetup{format=hang,width=0.8\textwidth,justification=centering,singlelinecheck=no}
		\caption{
			Три полости: $x$, $y$ и вспомогательная при реализации гейта coCSign
		}
		\label{fig:3_cavities}
	}
\end{figure}

Перенос фотона из полости $j$ в полость $i$ и наоборот осуществляется одновременным
включением ячеек Поккельса, что означает добавление $H_{jump}$ к взаимодействию $H_{int}$. При отсутствии атомов это даст ту же динамику, что и рабиевские осцилляции, но с периодом $\tau_{jump} = \pi\hbar/\nu$. Мы будем считать, что $\nu \gg g$, означающее, что перемещение фотона между полостями происходит настолько быстро, что атом не влияет на данный процесс.

В силу несоизмеримости периодов рабиевских осцилляций $\tau_{1}$ и $\tau_{2}$ можно выбрать такие натуральные числа $n_{1}$ и $n_{2}$, что с высокой точностью будет выполняться приближенное равенство
\begin{equation}\label{nt}
2n_{2}\tau_{2} \approx 2n_{1}\tau_{1} + \frac{\tau_{1}}{2},
\end{equation}
которое позволит произвести нелинейный фазовый сдвиг, необходимый для реализации гейта $\text{coCSign}$.

\begin{figure}[h!]
	\noindent\centering{
		\includegraphics[width=0.65\textwidth]{Dissertation/images/section_5/steps.png}
		\captionsetup{format=hang,width=0.8\textwidth,justification=raggedright,singlelinecheck=no}
		\\[6pt]
		\caption{
			Последовательность операций при реализации гейта $\text{coCSign}$ на асинхронных атомных возбуждениях в оптических
			полостях, разбитая на 7 участков по времени, время перехода фотона $\delta\tau = \tau_{jump}/2 \approx \tau_{1(2)}$. 
			После изображенной схемы необходимо подождать время $\tau_{1}/2$.
		}
		\label{fig:steps}
	}
\end{figure}

Состояние кубита $|0\ra$ реализуется в данной модели как состояние оптической полости вида $|0\ra_{ph}|1\ra_{at}$, а состояние кубита $|1\ra$ --- как $|1\ra_{ph}|0\ra_{at}$. Таким образом, состояние $|01\ra$, которому требуется инвертировать фазу, имеет вид $|01\ra_{ph}|10\ra_{at}$, где первый фотонный кубит относится к полости $x$, а второй --- к полости $y$. Заметим, что через время $\tau_{1}/2$ ноль и единица меняются местами с набегом фазы $\pi/2$.

Сначала организуется короткий по длительности обмен фотонами вспомогательной полости и полости $x$, затем, с задержкой $\tau_{1}/2$ --- аналогичный обмен с полостью $y$, затем, через время $2n_{2}\tau_{2}$, снова организуется короткий обмен фотонами вспомогательной полости с полостью $x$, затем, через время $\tau_{1}/2$ --- аналогичный обмен с полостью $y$. Из выбора времен перемещений фотонов вытекает, что в данные моменты в участвующих полостях будет либо один фотон, либо ни одного, поэтому включение ячеек Поккельса на малом временном отрезке $\delta\tau = \pi\hbar/2\nu \ll \tau_{1}$ даст именно перемещение фотонов.

\section{Численное моделирование гейта coCSign}

В построении гейта coCSign участвуют три полости: две полости ($x$ и $y$), реализующие логические кубиты квантовой системы, и одна вспомогательная полость ($aux$). Базисные состояния каждой полости (в рамках модели JC) можно записать в виде $|m\ra_{ph}|n\ra_{at}$, где $m$ --- количество фотонов в полости $(m = 0, 1, 2)$, $n$ --- уровень возбуждения атома $(n = 0, 1)$. Выпишем все базисные состояния, относящиеся к полостям $x$, $y$:
\[
v = \begin{pmatrix} 
|00\ra \\ 
|10\ra \\ 
|01\ra \\ 
|20\ra \\ 
|11\ra
\end{pmatrix}.
\]

Унитарная динамика в каждой полости определяется гамильтонианом Джейнса-Каммингса в приближении RWA \cite{ozhigov_qq, rwa_1, rwa_2}:
\[
H_{\text{JC}}^{\text{RWA}} = \begin{pmatrix} 
0 & 0 & 0 & 0 & 0 \\ 
0 & \hbar w & g & 0 & 0 \\ 
0 & g & \hbar w & 0 & 0 \\ 
0 & 0 & 0 & 2\hbar w & \sqrt{2}g \\ 
0 & 0 & 0 & \sqrt{2}g & 2\hbar w
\end{pmatrix}.
\]

Базисные состояния всей системы (состоящей из трех полостей), можно записать в виде $|m_{x}m_{y}m_{aux}\ra_{ph}|n_{x}n_{y}n_{aux}\ra_{at}$. Всего будет 18 таких состояний:
\[
\begin{pmatrix} 
|000\ra_{ph}|110\ra_{at} \\ 
|010\ra_{ph}|100\ra_{at} \\ 
|100\ra_{ph}|010\ra_{at} \\ 
|110\ra_{ph}|000\ra_{at} \\ 
|000\ra_{ph}|101\ra_{at} \\ 
|100\ra_{ph}|001\ra_{at} \\ 
|001\ra_{ph}|001\ra_{at} \\ 
|101\ra_{ph}|000\ra_{at} \\ 
|000\ra_{ph}|011\ra_{at} \\ 
|001\ra_{ph}|010\ra_{at} \\ 
|010\ra_{ph}|001\ra_{at} \\ 
|011\ra_{ph}|000\ra_{at} \\ 
|002\ra_{ph}|000\ra_{at} \\ 
|001\ra_{ph}|001\ra_{at} \\ 
|020\ra_{ph}|000\ra_{at} \\ 
|010\ra_{ph}|010\ra_{at} \\ 
|200\ra_{ph}|000\ra_{at} \\ 
|100\ra_{ph}|100\ra_{at}
\end{pmatrix}.
\]

Построенный гамильтониан системы, реализующей гейт coCSign, в его грубом представлении изображен на \ref{fig:hcocsign}. Перелет фотона возможен только между вспомогательной полостью и одной из полостей кубита. Наличие $\nu_{x}$ в гамильтониане означает переход
$[x \Leftrightarrow aux]$, значение $\nu_{y}$ означает переход $[y \Leftrightarrow aux]$.
\begin{figure}[h!]
	\noindent\centering{
		\includegraphics[width=1\textwidth]{Dissertation/images/section_5/H.png}
		\captionsetup{format=hang,width=0.8\textwidth,justification=centering,singlelinecheck=no}
		\caption{
			Гамильтониан системы, используемой при реализации гейта coCSign
		}
		\label{fig:hcocsign}
	}
\end{figure}


Ниже приведено соответствие между логическими кубитами (слева), над которыми оперирует гейт, и базисными векторами всей системы
\begin{align*}
|00\ra_{q} \Longleftrightarrow |000\ra_{ph}|110\ra_{at},\\
|01\ra_{q} \Longleftrightarrow |010\ra_{ph}|100\ra_{at},\\
|10\ra_{q} \Longleftrightarrow |100\ra_{ph}|010\ra_{at},\\
|11\ra_{q} \Longleftrightarrow |110\ra_{ph}|000\ra_{at}.
\end{align*}

Эволюция вектора состояния в рамках унитарной динамики описывается уравнением Шредингера
\begin{equation}\label{schrodinger2}
	i\hbar\frac{\partial}{\partial t}\Psi(r,t) = H\Psi(r,t).
\end{equation}

В результате численного моделирования состояние кубита $|10\ra$ претерпело сдвиг фазы $-\pi$, состояния кубитов $|00\ra$, $|01\ra$, $|11\ra$ не изменились. Инвертирование фазы для состояния $|01\ra$ вместо состояния $|10\ra$ получается путем ожидания в течение времени $\tau/2 = \pi\hbar/2g$, за которое осуществляется инверсия кубитов $x$, $y$.

\section{Оценка точности гейта coCSign}

Компьютерное моделирование показало, что уже при значениях $n_{1}$, $n_{2}$ в несколько десятков точность гейта coCSign превышает 90\% (fidelity > 0.9).

Значения $n_{1} = 4$ и $n_{2} = 6$ дают наилучший результат, точность срабатывания гейта coCSign при этом составляет 95.8\% (fidelity $\approx$ 0.958).

Физическое ограничение на качество гейта $\text{coCSign}$ вытекает из соотношения неопределенностей <<энергия-время>> для фотонов. Интенсивность перехода фотонов $\nu$ не может быть слишком большой, так как по мере ее увеличения сокращается время перехода фотона из полости в полость. А это, в свою очередь, приводит ко всё большей неопределенности его энергии и, как следствие, к деструктивной интерференции внутри полости (расстояние $L$ между отражающими зеркалами должно быть кратно половине длины волны фотона $\lambda = 2\pi c/w_{c}$).

Частота фотона в экспериментах с атомом Rb85 составляет приблизительно $10^{10} c^{-1}$. Если мы примем за границу неопределенности частоты значение $10^{9} c^{-1}$, то в силу соотношения неопределенностей $\delta w \cdot \delta t \approx 1$, получим нижнюю оценку $\delta\tau \approx 10^{-9} c^{-1}$ для допустимого интервала времени, в течение которого фотон может перейти из полости в полость. Учитывая время рабиевской осцилляции, равное примерно $\tau \approx 10^{-6} c^{-1}$, получим ограничение $10^{-9} \le \tau \le 10^{-6}$, означающее, что срабатывание гейта coCSign происходит с ошибкой, существенно большей $10^{-3}$.

На точность гейта влияют следующие факторы: неточность определения времени перехода фотона в случае идеальной модели JCH (ошибка равенства \eqref{nt}), технические ограничения ячейки Поккельса, реализующей перелет фотона, конечное время жизни фотона в неидеальном резонаторе и уширение спектральных линий рабочей моды резонатора. Второй и третий факторы носят чисто технический характер и выходят за пределы данной работы. Время жизни фотона в полости можно рассчитать, введя в уравнение \eqref{klgs} факторы декогерентности --- операторы Линдблада $L = a$, реализующие утечку фотонов из полости. Их допустимые интенсивности $l_{i}$ можно рассчитать на основе времени жизни фотона в эксперименте, что составляет несколько миллисекунд (\cite{rempe}). Однако основным фактором декогеренции является последний, он имеет фундаментальную природу и проистекает из соотношения неопределенностей <<время-энергия>>. Чем быстрее из полости в полость перелетает фотон, тем более неопределенной становится его частота, а значит, тем меньшим становится его время жизни в резонаторе.

\section{Выводы главы}\label{sec:ch5/sect3}

По результатам компьютерного моделирования запутывающего гейта $\text{coCSign}$ в оптических полостях точность его срабатывания составила порядка 95\%. Схема данного гейта, предложенная Ю. Ожиговым в работе \cite{ozhigov_quantum_gates}, описывается моделью Джейнса-Каммингса-Хаббарда и заметно проще известных гейтов такого типа. Для ее реализации требуется дополнительная оптическая полость, организация же перелета фотонов из полости в полость осуществляется во временном окне, на которое налагаются ограничения двух типов. Первый связан с технической скоростью срабатывания ячейки Поккельса, второй --- с фундаментальным соотношением неопределенностей <<энергия-время>>. Подобные ограничения присутствуют и в других схемах фотонных компьютеров.

Простота предложенной схемы по сравнению с известными аналогами (\cite{azuma}) делают ее вероятным кандидатом на экспериментальное решение. Главное достоинство предложенной схемы реализации гейтов --- в ее простоте и возможности точного следования теоретической модели JCH, что, несмотря на упомянутые трудности, внушает оптимизм в плане масштабируемости и сравнения теории квантового компьютера с экспериментами на большом числе кубитов.
