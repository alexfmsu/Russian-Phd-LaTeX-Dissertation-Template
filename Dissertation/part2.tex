\chapter{Квантовое бутылочное горлышко в атомных превращениях}\label{ch:ch2}

В данной главе будет рассмотрен парадоксальный эффект квантового бутылочного горлышка для процесса интенсивного охлаждения атома, кото­рый переходит в необратимое состояние, находясь в возбужденном состоянии. Превышение некоторого порога интенсивности охлаждения ведет к росту веро­ятности такого перехода, что невозможно при классическом описании процесса.

\section{Постановка задачи}\label{sec:ch2/sec1}
Модель Тависа-Каммингса \cite{tc_exact_solution,tc_a_study,tc_dicke_2,tc_improvement,tc_experimental} описывает динамику состояний группы двухуровневых атомов, взаимодействующей с одномодовым полем внутри резонатора, удерживающего фотоны с энергией атомного возбуждения. Непосредственная экспериментальная реализация такой модели требует оптической полости, зеркала которой сделаны из сверхпроводящего материала (например, из ниобия), функционирующего при температуре жидкого гелия, и атомов, удерживаемых в определенной области внутри полости с помощью оптических пинцетов \cite{cavity_exp_1,cavity_exp_2,cavity_exp_3}. Так можно обеспечить длительность жизни фотона в полости, измеряемой несколькими десятками рабиевских осцилляций \cite{rabi_1,rabi_2,rabi_3}, что составляет не более $10^{-4}$ секунды.

Такие системы могут дать представление о механизме процессов, выходящих за рамки квантовой электродинамики и включающих в себя как перемещение атомов, так и их превращения: в ходе химической реакции или при радиоактивном распаде. В данной главе мы применим модель ТС для анализа вероятности перехода атома в необратимое состояние в результате спонтанной реакции в зависимости от времени при условии интенсивной утечки фотонов из полости. Природа ре­акции, вызывающей такой переход, или \textbf{превращение}, здесь предполагается химической, однако это может быть и радиоактивный распад, индуцированный фотонами (возможность фотоядерных реакций --- экспериментальный факт, описанный, к примеру, в работе \cite{photonuclear_reactions}). Важно, что в результате реакции атом прекращает взаимодействовать с полем в рамках модели TC. Мы предполагаем, что такая реакция происходит только с атомом, находящимся в возбужденном состоянии  $|1\ra$, в противоположность основному состоянию $|0\ra$, в котором атом не может подвергнуться превращению. Соответственно, рост числа фотонов в полости будет эмуляцией вероятности возбуждения атомов, а уменьшение --- эмуляцией вероятности их релаксации.

Гамильтониан модели TC для системы $n$ атомов в приближении вращающейся волны \cite{ozhigov_qq,rwa_rabi_1,rwa_rabi_2} имеет вид
\begin{equation}
H_{\text{\text{TC}}}^{\text{\text{RWA}}} = \hbar wa^{+}a + \hbar w\sum_{i=1}^{n}\s_{i}^{+}\s_{i} + a\bar\s^{+}+a^{+}\bar\s,
\end{equation}
где $a$, $a^{+}$ --- полевые, а $\s_{i}$, $\s_{i}^{+}$ --- атомные операторы, $\bar\s=\sum\limits_{i=1}^{n}g_{i}\s_{i}$, $\bar\s^{+}=\sum\limits_{i=1}^ng_{i}\s^{+}_{i}$ --- соответствующие операторы коллективной релаксации и возбуждения группы атомов, $w = w_{c} = w_{a}$ --- частота фотонов моды резонатора, равная частоте атомного перехода, $g_{i}$ --- интенсивность взаимодействия $i$-го атома с полем, символ <<+>> означает сопряжение оператора. Этот гамильтониан также может быть дополнен прямым диполь-дипольным взаимодействием между атомами, минующим поле полости.

Контринтуитивный эффект квантового бутылочного горлышка заключается в парадоксальном увеличении времени жизни возбужденного состояния атома в оптической полости при усилении оттока фотонов из нее. Этот эффект имеет чисто квантовую природу.

Мы выясним влияние этого эффекта на описанный выше процесс превращения атома, природа которого для нас, в сущности, не важна. Мы также предполагаем, что вероятность превращения атома, находящегося в основном состоянии, пренебрежимо мала, а для возбужденного состояния вероятность превращения к моменту времени $t$ подчиняется статистике Пуассона $p_{t} = 1-e^{-\gamma t}$.

\section{Фотон-индуцированное превращение атомов}\label{sec:ch2/sec2}

В дополнение к стандартным состояниям атома (основному $|0\ra$ и возбужденному $|1\ra$) мы введем особое состояние $|2\ra$, которое будем называть \textbf{превращенным}. Тогда действие атомных операторов $\s$ и $\s^{+}$ на атомный кубит определяется формально как обычно: 

\hspace{90pt}$\s^{+}|0\ra = |1\ra$
\hspace{90pt}$\s^{+}|1\ra = 0$

\hspace{98pt}$\s|1\ra\hspace{2pt}=\hspace{2pt}|0\ra$
\hspace{98pt}$\s|0\ra\hspace{2pt} = \hspace{1pt}0$\\[12pt]
\noindent Оператор же атомного превращения $L_{2}$ определяется так:\\
\indent\hspace{94pt}$L_{2}|1\ra = |2\ra$
\hspace{95pt}$L_{2}|0\ra = L_{2}|2\ra = 0.$\\[12pt]
\indent В превращенном состоянии атом не может взаимодействовать с полем. Его физический смысл состоит в том, что атом либо вступает в химическую реакцию, либо участвует в ядерном превращении. В обоих случаях взаимодействие такого атома с полем в рамках модели Тависа-Каммингса становится невозможным. Ни продукты распада (в случае ядерного превращения), ни энергия того, во что превратится атом, в данном случае интереса не представляют, так что мы будем учитывать только динамику модели TC с дополнительными операторами, выражающими само превращение атома, ведущее к его выходу из данной модели.

Таким образом, искомая математическая модель будет описываться основным квантовым уравнением \cite{breuer}
\begin{equation}\label{sec2_master_eq}
\begin{gathered}
	i\hbar\dot\rho = [H_{\text{TC}}^{\text{RWA}}, \rho] + iL(\rho),\\
	L(\rho) = \sum_{i=1}^{2}\gamma_{i}(L_{i}\rho L_{i}^{+} - \frac{1}{2}\{L_{i}^{+}L_{i}, \rho\})
\end{gathered}
\end{equation}
с операторами Линдблада двух типов:\\
\indent\qquad$L_{1} = a$ --- вылет фотона из резонатора \cite{breuer,photon_emission},\\
\indent\qquad$L_{2}$ --- оператор атомного превращения.

\noindent Интенсивности данных процессов обозначим через $\gamma_{1} = \gamma_{\text{out}}$ и $\gamma_{2} = \gamma_{\text{ex}}$ соответственно.

Нас интересует зависимость вероятности превращения атома от интенсивности $\gamma_{1} = \gamma_{\text{out}}$ вылета фотона из резонатора. Механизм квантового бутылочного горлышка будет заключаться в том, что при большой величине $\gamma_{1} = \gamma_{\text{out}}$ время жизни возбужденного состояния атома удлиняется, что приводит к росту вероятности его превращения. Увеличение времени жизни имеет чисто квантовую природу и подробно описано в статьях \cite{quantum_bottleneck_victorova} и \cite{quantum_simulation_homogeneous}.
\\[18pt]
\noindent Выберем базис: $|i\ra_{\text{ph}}|j\ra_{\text{at}}$.

\noindent Здесь\\
\indent\qquad $|i\ra_{\text{ph}}$ --- фоковское состояние поля (число фотонов в полости, $i=\overline{\mbox{0,1}}$),\\
\indent\qquad $|j\ra_{\text{at}}$ --- состояние атома ($j=\overline{\mbox{0,2}}$).
\\[18pt]
\noindent Данная система при наличии одного фотона и одного атома имеет 4 базисных состояния:\\
\indent\qquad $|0\ra_{\text{\text{ph}}}|2\ra_{\text{\text{at}}}$ (превращенное состояние атома),\\
\indent\qquad $|0\ra_{\text{\text{ph}}}|0\ra_{\text{\text{at}}}$ (фотон вне полости),\\
\indent\qquad $|0\ra_{\text{\text{ph}}}|1\ra_{\text{\text{at}}}$ (атом возбужден),\\
\indent\qquad $|1\ra_{\text{\text{ph}}}|0\ra_{\text{\text{at}}}$ (фотон в полости).
\\[12pt]

\noindent Ее гамильтониан:\\
{\normalsize
\begin{equation}\label{ch:H}
	H = \bordermatrix
	{
		&                |0\ra_{\text{\text{ph}}}|2\ra_{\text{\text{at}}} & |0\ra_{\text{\text{ph}}}|0\ra_{\text{\text{at}}} & |0\ra_{\text{\text{ph}}}|1\ra_{\text{\text{at}}} & |1\ra_{\text{\text{ph}}}|0\ra_{\text{\text{at}}} \cr
		|0\ra_{\text{\text{ph}}}|2\ra_{\text{\text{at}}} &      0 &       	   0 &      0 & 0 \cr
		|0\ra_{\text{\text{ph}}}|0\ra_{\text{\text{at}}} &      0 &       	   0 &      0 & 0\cr
		|0\ra_{\text{\text{ph}}}|1\ra_{\text{\text{at}}} &      0 &  		   0 &      \hbar w_{a} & g\cr
		|1\ra_{\text{\text{ph}}}|0\ra_{\text{\text{at}}} &      0 &      	   0 &      g & \hbar w_{c}\cr
	}.
\end{equation}
}
\
\\[18pt]
%\noindent Алгоритм построения гамильтониана систем Джейнса-Каммингса и Тависа-Каммингса с учетом стока, а также компьютерное моделирование их квантовой динамики детально описаны в приложении \hyperref[app:A]{А}.

\noindent Оператор уничтожения фотона (улет фотона из полости):\\
{\normalsize
\begin{equation}\label{ch:L1}
	L_{1} = \bordermatrix
	{
		&                |0\ra_{\text{\text{ph}}}|2\ra_{\text{\text{at}}} & |0\ra_{\text{\text{ph}}}|0\ra_{\text{\text{at}}} & |0\ra_{\text{\text{ph}}}|1\ra_{\text{\text{at}}} & |1\ra_{\text{\text{ph}}}|0\ra_{\text{\text{at}}} \cr
		|0\ra_{\text{\text{ph}}}|2\ra_{\text{\text{at}}} &      0 &      0 &      0 & 0 \cr
		|0\ra_{\text{\text{ph}}}|0\ra_{\text{\text{at}}} &      0 &      0 &      0 & 1\cr
		|0\ra_{\text{\text{ph}}}|1\ra_{\text{\text{at}}} &      0 &      0 &      0 & 0\cr
		|1\ra_{\text{\text{ph}}}|0\ra_{\text{\text{at}}} &      0 &      0 &      0 & 0\cr
	}.
\end{equation}
}
\\

\noindent Оператор атомного превращения:\\
{\normalsize
\begin{equation}\label{ch:L2}
	L_{2} = \bordermatrix
	{
		&                |0\ra_{\text{\text{ph}}}|2\ra_{\text{\text{at}}} & |0\ra_{\text{\text{ph}}}|0\ra_{\text{\text{at}}} & |0\ra_{\text{\text{ph}}}|1\ra_{\text{\text{at}}} & |1\ra_{\text{\text{ph}}}|0\ra_{\text{\text{at}}} \cr
		|0\ra_{\text{\text{ph}}}|2\ra_{\text{\text{at}}} &      0 &      0 &      1 & 0 \cr
		|0\ra_{\text{\text{ph}}}|0\ra_{\text{\text{at}}} &      0 &      0 &      0 & 0\cr
		|0\ra_{\text{\text{ph}}}|1\ra_{\text{\text{at}}} &      0 &      0 &      0 & 0\cr
		|1\ra_{\text{\text{ph}}}|0\ra_{\text{\text{at}}} &      0 &      0 &      0 & 0\cr
	}.
\end{equation}
}

\clearpage
\section{Программная реализация}

Алгоритм компьютерного моделирования квантового бутылочного горлышка состоит из следующих этапов:
\begin{enumerate}
\item{
	Составление матрицы $H$ гамильтониана системы \eqref{ch:H}\\
}
\item{
Составление вектора начального состояния\\
$
	|\Psi_{0}\ra = 
	\begin{pmatrix}
		0 \\ 
		0 \\
		1 \\
		0 
	\end{pmatrix}
	\begin{matrix}
    	\quad|0\ra_{\text{\text{ph}}}|2\ra_{\text{\text{at}}} \cr
		\quad|0\ra_{\text{\text{ph}}}|0\ra_{\text{\text{at}}} \cr
    	\quad|0\ra_{\text{\text{ph}}}|1\ra_{\text{\text{at}}} \cr
    	\quad|1\ra_{\text{\text{ph}}}|0\ra_{\text{\text{at}}}
	\end{matrix}
$
\\[24pt]
Составление матрицы плотности начального состояния\\[24pt]
$
	\rho(0) = |\Psi_{0}\ra\la\Psi_{0}| =
	{\small	
	\bordermatrix
	{
		&                |0\ra_{\text{\text{ph}}}|2\ra_{\text{\text{at}}} & |0\ra_{\text{\text{ph}}}|0\ra_{\text{\text{at}}} & |0\ra_{\text{\text{ph}}}|1\ra_{\text{\text{at}}} & |1\ra_{\text{\text{ph}}}|0\ra_{\text{\text{at}}} \cr
		|0\ra_{\text{\text{ph}}}|2\ra_{\text{\text{at}}} &      0 &      0 &      0 & 0 \cr
		|0\ra_{\text{\text{ph}}}|0\ra_{\text{\text{at}}} &      0 &      0 &      0 & 0\cr
		|0\ra_{\text{\text{ph}}}|1\ra_{\text{\text{at}}} &      0 &      0 &      1 & 0\cr
		|1\ra_{\text{\text{ph}}}|0\ra_{\text{\text{at}}} &      0 &      0 &      0 & 0\cr
	}
	}
$\\[12pt]
и операторов Линдблада \eqref{ch:L1}, \eqref{ch:L2}.\\

Матрицы, соответствующие операторам Линдблада \eqref{ch:L1}, \eqref{ch:L2}, а также матрица плотности состояния системы $\rho(t)$ были реализованы как разреженные, поскольку при численном моделировании динамики для обнаружения эффекта квантового бутылочного горлышка потребовался достаточно малый шаг $dt$ по времени (в пределах 0.1-1 ns) на интервале $T = 1~\text{mks}$.\\
}
\item{Вычисление оператора эволюции $\displaystyle U_{dt} = \mathrm{exp}\biggl(\frac{-iHdt}{\hbar}\biggr)$ \eqref{ch1:U}}\\
\item{Моделирование неунитарной квантовой динамики \eqref{ch1:dynamics}\\

Решение основного квантового уравнения \eqref{sec2_master_eq} производилось при помощи метода Эйлера.

\clearpage
\begin{figure}[h!]
	\noindent\centering{
		\includegraphics[width=1.0\textwidth]{Dissertation/images/section_2/scheme.jpg}
		\captionsetup{format=hang,width=1.0\textwidth,justification=centering,singlelinecheck=no}
		\\[6pt]
		\caption{
			{\small Блок-схема: квантовое бутылочное горлышко в атомных превращениях\\ \hspace{13.5em}(для одной пары значений интенсивностей $\gamma_{\textrm{out}}$, $\gamma_{\textrm{ex}}$)}
		}
	}
\end{figure}

\clearpage
Каждый шаг по времени состоит из двух этапов:
\begin{equation}
\tilde{\rho}(t+dt)\ =\ \rho(t)+dt\cdot \frac{i}{\hbar}\cdot L(\rho(t)),
\end{equation}
\begin{center}
	\text(неунитарная динамика: утечка фотонов, атомное превращение)
\end{center}
\begin{equation}
\rho(t+dt)=U_{dt}\cdot \tilde{\rho}(t+dt)\cdot U_{dt}^{*}.
\end{equation}
\begin{center}\text(унитарная динамика)\end{center}
\
\\
Наиболее сложной задачей здесь было выявление (за разумное время) тех диапазонов интенсивностей $\gamma_{1} = \gamma_{\text{out}}$ и
$\gamma_{2} = \gamma_{\text{ex}}$, а также их соотношения с интенсивностью фотонно-атомного взаимодействия $g$, при которых эффект квантового бутылочного горлышка был бы обнаружен и притом качественно.\\[12pt]

Представленные ниже результаты компьютерного моделирования были получены на суперкомпьютере Ломоносов-2 \cite{lomonosov_2}. Вычисления производились параллельно на 100 процессорных узлах: каждый узел моделировал квантовую динамику для конкретного значения $\gamma_{1} = \gamma_{\text{out}}$ и всего выбранного диапазона $\gamma_{2} = \gamma_{\text{ex}}$.\\[12pt]

Также в ходе численного эксперимента анализировалась квантовая картина для диапазона $g \in [0.001 \cdot w,~0.1 \cdot w]$ --- как для условий ультраслабого взаимодействия, так и для условий границы применимости приближения RWA \cite{ozhigov_qq,rwa_rabi_1,rwa_rabi_2} в модели ТС.
}
\end{enumerate}


\section{Результаты компьютерного моделирования}\label{sec:ch2/sec3}
Здесь будут приведены результаты компьютерного моделирования влияния эффекта квантового бутылочного горлышка на вероятность атомного превращения при различных параметрах квантовой системы.

\clearpage
\begin{figure}[h!]
	\noindent\centering{
		\includegraphics[width=0.75\textwidth]{Dissertation/images/section_2/01.png}	\captionsetup{format=hang,width=0.85\textwidth,justification=centering,singlelinecheck=no}
		\caption{
			Зависимость вероятности превращения атома\\от интенсивности $\gamma_{\text{out}}$ вылета фотона и времени $t$\\
			$g \le \gamma_{\text{out}} \le 10g \qquad\qquad \gamma_{\text{ex}} = g = \text{const}$
		}
		\
		\\[28pt]
		\includegraphics[width=0.75\textwidth]{Dissertation/images/section_2/02.png}	\captionsetup{format=hang,width=0.85\textwidth,justification=centering,singlelinecheck=no}
		\caption{
			Зависимость вероятности превращения атома\\от интенсивности $\gamma_{\text{out}}$ вылета фотона и времени $t$\\
			$0.01g \le \gamma_{\text{out}} \le 10g \qquad\qquad \gamma_{\text{ex}} = g = \text{const}$
		}
	}
\end{figure}

\clearpage
\begin{figure}[h!]
	\noindent\centering{
		\includegraphics[width=0.65\textwidth]{Dissertation/images/section_2/03.png}	\captionsetup{format=hang,width=0.85\textwidth,justification=centering,singlelinecheck=no}
		\caption{
			Зависимость вероятности превращения атома\\от интенсивности $\gamma_{\text{out}}$ вылета фотона и времени $t$\\
			$0.01g \le \gamma_{\text{out}} \le 50g \qquad\qquad \gamma_{\text{ex}} = g = \text{const}$
		}
		\
		\\[28pt]
		\includegraphics[width=0.65\textwidth]{Dissertation/images/section_2/04.png}
		\captionsetup{format=hang,width=0.92\textwidth,justification=centering,singlelinecheck=no}
		\caption{
			Зависимость времени достижения целевой\\вероятности вылета фотона от интенсивности $\gamma_{\text{out}}$ утечки\\и интенсивности $\gamma_{\text{ex}}$ атомного превращения\\
			\hspace{1.4em}$10^{-3}g \le \gamma_{\text{out}} \le 10^{-2}g$\\
			$0.1g \le \gamma_{\text{ex}} \le g$\\
			\qquad\quad целевая вероятность вылета фотона из полости: 0.9{\color{red}*}
		}
	}
\end{figure}

Шаг моделирования по времени $dt = 0.1~\text{ns}$, $g/\hbar w = 0.01$.

\extrafootertext{\hspace{-2em}{\color{red}*}при достижении данного значения численный эксперимент прекращается}

\clearpage
\section{Анализ полученных результатов}

\clearpage
\section{Выводы главы}\label{sec:ch2/sec4}

В результате численного моделирования было обнаружено, что эффект квантового бутылочного горлышка способен существенно повлиять на вероят­ность атомных превращений. Эта возможность реализуется при малой энергии атомного возбуждения относительно энергии атомного превращения. Возбуж­дение атома может затрагивать либо его электронную оболочку (что влияет на химические превращения атома), либо ядро (что способно повлиять на ядерные реакции). При большом усилении интенсивности утечки фотонов возбужда­ющей моды за пределы оптической полости вероятность превращения атома парадоксальным образом повышается. Этот эффект может быть использован в том числе и для понижения вероятности атомного превращения: для этого необходимо увеличить время жизни фотона в пределах оптической полости. Возбуждающая атом фотонная мода служит своеобразным триггером, пере­ключающим режимы динамики самого атома, притом что энергия этой моды на много порядков меньше энергии самого атомного превращения.

Данный эффект, в силу его контринтуитивной, чисто квантовой природы, не учитывается в имеющихся моделях атомных превращений. Его включение в такие модели увеличит область их применения в практических целях --- как в химии, так и в приложениях физики атомного ядра.
