\chapter{Оптический отбор темных состояний ансамблей многоуровневых атомов}\label{ch:ch3}

\section{Темные состояния}\label{subsec:ch3/sect1}

Описание собственных состояний в модели Тависа-Каммингса является весьма непростой задачей. Ее решение для двухуровневых атомов приведено в диссертации Михаэля Тависа \cite{tc_a_study}, и оно нетривиально. Но особый интерес представляет класс собственных состояний, атомная компонента которых состоит из \textbf{темных состояний} \cite{dark_states_dimension}. Так называют атомные состояния, которые не могут ни испустить, ни поглотить фотон в силу того, что интерференция атомов блокирует их взаимодействие с полем. Тем самым атомы мешают друг другу испустить или поглотить фотон.

К примеру, для состояния
\begin{equation}\label{singlet}
	|s\ra = \frac{|01\ra - |10\ra}{\sqrt{2}}
\end{equation}
испускание вторым атомом фотона переведет состояние  $|0\ra_{\text{ph}}|s\ra$ в состояние $|1\ra_{\text{ph}}|00\ra$, а испускание фотона первым атомом переведет состояние  $|0\ra_{\text{ph}}|s\ra$ в состояние $-|1\ra_{\text{ph}}|00\ra$. В итоге попытка испускания фотона состоянием \eqref{singlet} не возымеет никакого эффекта.

Состояние \eqref{singlet} является \textbf{темным} и называется \textbf{синглетом}.

Алгебраическое описание темных состояний ансамблей многоуровневых атомов можно найти в работе \cite{dark_states_kok}, однако их явный вид был найден только для ансамблей двухуровневых атомов. В работе \cite{dark_states_dimension} доказано утверждение о том, что размерность темного подпространства пространства $n$ двухуровневых атомов равна
\begin{equation}
\mathrm{dim}(D_{n}^2) =
\begin{cases}
	C_{n}^{k} - C_{n}^{k-1}~\text{при}~n = 2k, \\
	0, \text{в противном случае},
\end{cases}
\end{equation}
и все наборы темных состояний (для четного числа атомов в группе) с учетом нормировки имеют вид 
\begin{equation}
\frac{1}{2^{n/4}}\bigotimes_{j=1}^{n/2}(|01\ra_{j}-|10\ra_{j}),
\end{equation}
где индекс $j = 1,\dots, n/2$ означает номер пары в произвольном разбиении группы из $n$ атомов.

Для трехуровневых атомов имеется лишь гипотеза о структуре темных состоя­ний, которая была подтверждена в настоящей работе для весьма ограниченного числа атомов в ансамбле (не более 9). Размерность же темного подпространства была численно определена для двух десятков атомов в ансамбле с использованием суперкомпьютера (глава \ref{ch:ch4}).

\section{Практическая значимость темных состояний}\label{subsec:ch3/sect2}
Темные состояния имеют множество применений. В частности, их роль в организации межатомного взаимодействия рассмотрена в работе \cite{dark_states_properties_andre}, для контроля твердотельных спинов --- в работе \cite{dark_states_properties_hansom}, для управления макроскопическими квантовыми системами --- в работе \cite{dark_states_properties_bose_einstein}. Роль участия темного состояния в процессе фотосинтеза описана в работе \cite{dark_states_properties_photosynthesis}. Некоторые методы получения темных состояний в квантовых точках представлены в работах \cite{dark_states_properties_poltl} и \cite{dark_states_properties_tanamoto}. Разрушение темных состояний магнитным полем или модулированной лазерной поляризацией рассматривается в работе \cite{dark_states_properties_destabilization}.

Темные состояния обладают ненулевой энергией и по этой причине могут служить энергетическим микроаккумулятором для различных наноустройств \cite{dark_states_quantum_memory}. Кроме того, будучи свободными от декогерентности (поскольку не взаимодействуют со светом), они могут быть использованы для достаточно длительного хранения сложных состояний в квантовых вычислениях.

Темные состояния широко используются в квантовой криптографии: использование синглетных состояний дает дополнительные преимущества в обеспечении устойчивости квантовых протоколов. Так, например, ключевое ме­сто в построении протокола AKM2017 \cite{akm2017} занимает подготовка синглетного состояния вида \eqref{singlet}.

Получение темных состояний двухуровневых атомов, описанное в работе \cite{dark_states_properties_tanamoto}, опирается на эффект Штарка-Зеемана \cite{zeeman_1,zeeman_2}. В данной главе мы опишем технически более простой метод оптического отбора, основанный на томографии электромагнитного поля вне полости, который, в частности, позволяет получать многоуровневые темные состояния, и притом строго определенного вида.

\clearpage
\section{Многомерные синглеты}\label{sec:ch3/sect2}
Рассмотрим обобщение модели Тависа-Каммингса на $d$-уровневые атомы. Гамильтониан ТС для $n$ атомов с энергиями $g^{j}_{i}$ взаимодействия с полем выделенной моды $j$ имеет вид
\begin{gather}
	H^j_{\text{TC}} = \hbar w a_{j}^{+}a_j + \hbar w \sum\limits_{i=1}^{n}\s_{ji}^{+}\s_{ji} + (a_{j}^{+} + a_{j})\sum\limits_{i=1}^{n}g^{j}_{i}(\s_{ji}^{+} + \s_{ji}),\notag\\
	H^{j,\ \text{RWA}}_{\text{TC}} = \hbar w a_{j}^{+}a_{j} + \hbar w \sum\limits_{i=1}^{n}\s_{ji}^{+}\s_{ji} + a_{j}^{+}\bar\s_{j} + a_{j}\bar\s_{j}^{+},\\
	\bar\s_{j} = \sum\limits_{i=1}^ng^{j}_{i}\s_{ji},\notag
\end{gather}
где RWA-приближение \cite{ozhigov_qq,rwa_rabi_1,rwa_rabi_2} справедливо при $g^{j}_{i}/\hbar w\ll 1$, верхним символом <<+>> обозначено сопряжение операторов. Здесь $a_{j},a_{j}^{+}$ --- полевые операторы уничтожения и рождения фотона моды $j$, $\s_{ji},\s_{ji}^{+}$ --- атомные операторы релаксации и возбуждения атома $i$, соответствующие моде $j$.

Пусть для ансамбля $\text{A}$, состоящего из $n$ одинаковых $d$-уровневых атомов, различающихся только энергиями взаимодействия с модами поля, определен граф $G$ возможных разрешенных переходов между уровнями для каждого $j$-го атома, $j=1,2,\dots,n$. Вершины $G$ соответствуют уровням энергии, ребра --- разрешенным переходам между ними. Тогда $G$ задает набор всевозможных мод $J_{G}$, с которыми может взаимодействовать каждый атом. Многомодовый гамильтониан, соответствующий графу $G$, имеет вид
\begin{gather}
	\label{many}
	H^G_{\text{TC}} = \sum\limits_{j\in J_{G}}H^{j}_{\text{TC}},\notag\\
	H^{G,\ \text{RWA}}_{\text{TC}}=\sum\limits_{j\in J_{G}}H^{j,\ \text{RWA}}_{\text{TC}}\notag.
\end{gather}
Введем обозначение $\bar\s_{G} = \sum\limits_{j\in J_{G}}\bar\s_{j}$. 
Тогда подпространство темных атомных состояний для ансамбля с возможными переходами $G$ есть $Ker(\bar\s_{G}^{+} + \bar\s_{G})$ и $Ker(\bar\s_{G})$ в точной модели и в RWA-приближении \cite{ozhigov_qq,rwa_rabi_1,rwa_rabi_2} соответственно. Заметим, что некоторые моды могут допускать RWA-приближение \cite{ozhigov_qq,rwa_rabi_1,rwa_rabi_2}, тогда как другие --- нет. К разным атомам применимость RWA \cite{ozhigov_qq,rwa_rabi_1,rwa_rabi_2} также может быть различной. Мы будем рассматривать только случай применимости этого приближения ко всем модам и атомам одновременно.

Через $g^{j}(r)$ обозначим амплитуду перехода по ребру $r$, соединяющему пару состояний в графе $G$ для атома $j$.

Сделаем граф $G$ ориентированным, задав ориентацию любого ребра по направлению к уменьшению энергии атомного состояния. Зафиксировав номер атома $j\in\{ 1,2,\dots,n\}$, пометим в графе $G$ ребра $r$ числами $g^{j}(r)$. Получится $n$ графов $G^{j}$, изоморфных $G$, для каждого атома --- свой. Предположим, что каждой паре <<атом $j$, состояние $i$>> можно приписать положительный вес $w(j,~i)$ так, что для любой пары $j,~j'$ атомов отношение $g^{j}(r)/g^{j'}(r)=w(j',~i_{in})/w(j,~i_{in}) = w(j',~i_{fin})/w(j,~i_{fin})$ для любого ребра $r$ с началом $i_{in}$ и концом $i_{fin}$. 

Рассмотрим состояние атомов
\begin{equation}\label{msinglet}
	|D_{G,A}\ra = \sum\limits_{\pi\in S_{d}}(-1)^{\s(\pi)}w(1,\pi(1))\dots w(d,\pi(d))|\pi(1),\dots,\pi(d)\ra,
\end{equation}
где $\pi$ пробегает все перестановки на множестве атомов $1,2,\dots,d$, а $\s(\pi)$ обозначает четность перестановки $\pi$. Состояние $|D_{G,A}\ra$  называется $G,A$-мультисинглетом. Мультисинглет называется равновесным, если все веса $w(j,~i)$ равны единице. Мультисинглет всегда является темным в RWA \cite{ozhigov_qq,rwa_rabi_1,rwa_rabi_2}, а равновесный мультисинглет --- темным для точного гамильтониана. Чтобы показать это, рассмотрим следующий пример.

\textbf{Пример.} Для $d=2$ состояние \eqref{msinglet} с точностью до нормировки примет вид 
\begin{equation}
	\label{g_singlet}
	g^{1}|01\ra-g^{2}|10\ra,
\end{equation}
и оно является темным в RWA-приближении \cite{ozhigov_qq,rwa_rabi_1,rwa_rabi_2}. Из определения весов $w(j,~i)$ следует, что сумма двух слагаемых из \eqref{msinglet}, отличающихся только перестановкой состояний одной пары атомов, будут с точностью до коэффициента иметь вид \eqref{g_singlet}. С другой стороны, состояние \eqref{g_singlet} будет темным для точного гамильтониана тогда и только тогда, когда $g^{1}=g^{2}$. 

Подграф $G'\subseteq G$ называется полным, если вместе с любой своей вершиной он содержит все вершины, соединенные с ней нисходящим ребром, вместе с этим ребром. Набор графов $G_{1},G_{2},\dots,G_{r}$ графа $G$ назовем накрытием, если он состоит из полных подграфов и их объединение дает $G$. Накрытие точное, если любой $G_{i},~i=1,2,\dots,r$ является компонентой связности графа $G$. 

\noindent Для ансамбля, состоящего из $n$ $d$-уровневых атомов, в свете работы \cite{quantum_simulation_homogeneous} правдоподобной является следующая гипотеза о явном виде темных состояний.

\begin{hyp}
	\
	\\
	\indent 1) Любое темное состояние в гамильтониане $H^{G,~\text{RWA}}_{\text{TC}}$ есть линейная комбинация тензорных произведений $G_{i},~\text{A}_{i}$ --- мультисинглетов для некоторых накрытий $\{ G_{i}\}$ графа $G$ и разбиений множества всех атомов $\text{A}$ на подмножества $\text{A}_{i}$. 
	
	2) Темные состояния для точного гамильтониана $H^G_{\text{TC}}$ являются в точности линейными комбинациями равновесных $G_{i},~\text{A}_{i}$ --- мультисинглетов для точных накрытий $\{ G_{i}\}$ графа $G$ и соответствующих разбиений $\text{A}$ на подмножества $\text{A}_{i}$. 
\end{hyp}

В частности, из этого следует, что при связном графе $G$ темные состояния в точной модели бывают лишь для ансамблей с числом атомов, кратным $d$. Данная гипотеза строго доказана только для $d=2$ в работе \cite{dark_states_dimension}.

В ансамбле разнородных атомов, как правило, нет совпадающих частот переходов. Однако в квантовых точках, где <<атомы>> можно фактически формировать искусственно, можно добиться и совпадения частот некоторых переходов в спектрах неодинаковых структур. В этом случае можно исследовать получающиеся темные состояния.

Например, для трехатомного ансамбля трехуровневых атомов с энергиями переходов $\hbar w$ между уровнями $0 \leftrightarrow 1$, $0 \leftrightarrow 2$, $1 \leftrightarrow 2$ единственным темным состоянием в RWA \cite{ozhigov_qq,rwa_rabi_1,rwa_rabi_2} без учета нормировки будет \textbf{мультисинглет}
\begin{equation}
	D_{3} = |012\ra + |120\ra + |201\ra - |021\ra - |102\ra - |210\ra.
\end{equation}

\section{Метод оптического отбора темных состояний}\label{sec:ch3/sect3}
Мы объясним метод оптического отбора на примере ансамбля, состоящего из двух двухуровневых атомов. Будем обозначать базисные состояния системы атомов и поля через $|n\ra_{\text{ph}}|m_1m_2\ra_{\text{at}}$, где $n$ --- число фотонов в резонаторе, $m_1,~m_2$ --- числа возбуждения первого и второго атомов: $|0\ra$ --- основное состояние, $|1\ra$ --- возбужденное. Схема отбора состоит из последовательных шагов отбора, которая начинается с заранее приготовленного состояния поля и атомов $|\Psi(0)\ra|0\ra_{\text{ph}}|\Phi_0\ra_{\text{at}}$, где $|\Phi_{0}\ra_{\text{at}}=\alpha |00\ra+\beta |s\ra$, $|s\ra=\frac{1}{\sqrt{2}}(|01\ra - |10\ra)$ --- двухатомный синглет, $\alpha |00\ra+\beta |s\ra$ --- произвольное состояние двухатомной системы, которое можно получить, выждав необходимое время для испускания фотона двухатомной системой. Например, состояние атомного ансамбля $|01\ra$ можно представить как $|01\ra=\frac{1}{\sqrt 2}(|t\ra+|s\ra)$, где $|t\ra=\frac{1}{\sqrt{2}}(|01\ra+|10\ra)$ --- триплетное состояние, остальные два триплета имеют вид $|00\ra$ и $|11\ra$. 

Шаг процесса с номером $i$ состоит в следующем. В момент времени $\tau_{i}$ мы имеем состояние системы <<атомы + поле>> $\rho_{i}$, при этом вероятность присутствия фотонов в полости исчезающе мала. Мы запускаем в резонатор один фотон, после чего включаем ячейку Поккельса, расположенную внутри резонатора и отражающую фотон в направлении детектора (см. рисунок \ref{fig:2}) и фиксируем время срабатывания детектора. После этого шага делаем следующий точно так же и т.д., набирая статистику времен срабатывания детектора. 

Мы предполагаем, что время запуска фотона в полость мало по сравнению как с временем  рабиевской осцилляции между состояниями $|1\ra_{\text{ph}}|00\ra_{\text{at}}$ и $|0\ra_{\text{ph}}\frac{1}{\sqrt 2}(|01\ra_{\text{at}}+|10\ra_{\text{at}})$, так и с ожидаемым временем вылета фотона из полости, и им можно пренебречь, считая запуск практически мгновенным.

Пусть $\rho'_{i}$ --- априорное состояние системы в полости в момент $i$-го включения ячейки Поккельса. Поскольку фотон появляется в полости очень быстро, можно считать, что это состояние получается из $\rho_i$ добавлением фотона в полость: $\rho'_i=a^+\rho_ia$. После этого мы ждем время $\tau_{\text{click}\_i}$, когда в детектор попадет фотон, вылетевший из полости. 

\begin{figure}[bt]
	\noindent\centering\includegraphics[width=0.75\textwidth]{Dissertation/images/section_3/detector.png}
	\captionsetup{format=hang,width=0.85\textwidth,justification=centering,singlelinecheck=no}
	\caption{Оптический отбор.\\Линдбладовский оператор $L_{1} = a^{+}a$ реализует улет фотона и возврат его обратно в полость после прохождения через детектор.\\ Детектор щелкает всякий раз, когда в него попадает фотон.}
	\label{fig:2}
\end{figure}

Время срабатывания детектора $\tau_{\text{click}\_i}$ на шаге $i$ является случайной величиной, зависящей также от шага $i$, так что решая основное уравнение, мы лишь найдем для нее верхнюю границу $t_i$. Функция распределения $\tau_{\text{click}\_i}$ меняется с каждым шагом и является априорной функцией распределения, которую мы находим по уравнению \eqref{lind}, не прибегая ни к каким экспериментам. Это вычисление нужно лишь для того, чтобы найти верхнюю границу $t_i$ ожидания щелчка детектора на шаге $i$. Матрицу плотности $\rho_{i+1}$ можно найти как решение задачи Коши для основного квантового уравнения \eqref{lind}, соответствующего вылету фотона из полости, для начального состояния $\rho'_i(0)=\rho'_i$, с тем условием, что для момента $t_{i}$ это решение $\rho'_i(t_{i})$ не содержит фотонов в полости с исчезающе малой вероятностью ошибки  (ошибка может произойти только когда мы прекратили ждать срабатывания детектора, а фотон все-таки остался в полости или может быть испущен позже).

Итак, верхнюю границу $t_i$ для $\tau_{\text{click}\_i}$ на шаге $i$ мы ищем численнно, решая основное квантовое уравнение. Полагаем $\rho_{i+1}=\rho'_i(t_{i})$. После вылета фотона из полости состояние атомов внутри полости не меняется, поэтому мы можем произвольно увеличить время ожидания полного вылета до значения, большего найденного $t_i$ для уменьшения вероятности ошибки.

Мы будем делать так определенные последовательные шаги, каждый раз фиксируя время срабатывания детектора на вылетающий из полости фотон. Если момент $\tau_i-\tau_{i-1}$ щелчка детектора на шаге $i$ рассматривается как случайная величина, то функция распределения этой величины находится как $P(t)=\la 0_{\text{ph}}0_10_2|\rho(t)|0_{\text{ph}}0_10_2\ra+\la 0_{\text{ph}}s|\rho(t)|0_{\text{ph}}s\ra$, то есть как вероятность того, что фотон вылетел из полости за время $t$, считая нулевой отметкой начало шага $i$. Плотность распределения времени срабатывания детектора есть $dP(t)/dt$. После достаточно большого числа последовательных шагов мы считаем среднее время $d\tau$ по всем значениям $\tau_{\text{click}\_i}$ срабатывания детектора в наших экспериментах. Далее мы установим факт достаточно быстрого подавления внедиагональных элементов матрицы плотности $\rho_i$ с ростом $i$, так что распределение величины $\tau_{\text{click}\_i}$ для разных $i$ будет практически одинаковым для больших $i$ и сойдется к распределению, характерному  либо для триплета $|00\ra$,  либо для синглета $|s\ra$. Таким образом, величины времен ожидания щелчка детектора $t_i$, начиная с момента исчезновения недиагональных элементов, будут одинаковы. Обозначим их через $dT$. 

Если среднее время $d\tau$ вылета фотона меньше некоторого порога $d\tau_{cr}$, в полости находится темный синглет $|s\ra$. В противном случае мы имеем триплет $|00\ra$, состояние бракуется, и вся серия экспериментов начинается заново --- с выбора случайного начального состояния атомов.

Срабатывание на шаге $i$ детектора, в который попадает фотон, отраженный ячейкой Поккельса, происходит с замедлением, которое меняется между нулем и $\tau_{\text{click}\_i}=\tau_{i}-\tau_{i-1}$. Оно складывается из двух факторов: а) время срабатывания самой ячейки (она может не перекрывать всю полость и потому, даже если в полости есть фотон, он не отразится сразу при прохождении вдоль полости) и б) возможность поглощения фотона компонентой $|00\ra$ атомного состояния  полости.  

Если первоначальное состояние атомов $\rho_0=|s\ra_{\text{at}}\la_{at} s|$, то мы имеем синглет и среднее время вылета $a_s$ фотона из полости будет коротким. Если же $\beta=0$, то мы имеем триплет $\rho_0=|00\ra_{at}\la_{at} 00|$ и среднее время вылета фотона $a_t$ будет длиннее, так как за время бездействия ячейки Поккельса фотон может с ненулевой вероятностью поглотиться ансамблем атомов. Таким образом, достаточно взять статистический барьер для принятия решения $d\tau_{cr}=(a_s+a_t)/2$. 

Считая применимым RWA-приближение \cite{ozhigov_qq,rwa_rabi_1,rwa_rabi_2}, рассмотрим в качестве математической модели шага нашего процесса основное квантовое уравнение\cite{breuer} с оператором Линдблада $A_1=a$ --- удаление фотона из полости:
\begin{gather}
	i\hbar\dot{\rho}=[H,~\rho]+i{\mathcal L}(\rho),\notag\\
	{\mathcal L}(\rho)=\gamma(a\rho a^+-\frac{1}{2}(\{ a^+a,~\rho\}),\label{lind}\\
	H=H_{\text{TC}}\notag.
\end{gather}
Его решение $\rho(t)$ можно приближенно найти, представив в виде последовательности двух шагов, из которых на первом делается один шаг в решении унитарной части \eqref{lind}: $\tilde\rho(t+dt)=e^{-iHdt /\hbar}\rho(t)e^{iHdt /\hbar}$, а на втором --- шаг в решении уравнения \eqref{lind} с удаленным коммутатором:
\begin{equation}
\rho(t+dt)=\tilde\rho(t+dt)+\frac{\gamma}{\hbar}(a\tilde\rho a^+-\frac{1}{2}(\{ a^{+}a,~\tilde\rho\})dt.
\end{equation}

Грубо оценить параметр $\gamma$ можно так. Поскольку изменение матрицы плотности на втором шаге, отнесенное к времени $dt$, за которое свет преодолевает длину полости, составляет величину эффективности ячейки Поккельса $e_p:\ 0<e_p\leq 1$, мы имеем $\frac{\gamma dt}{\hbar}=e_p$, откуда $\gamma=e_p\hbar/dt$. Для атома Rb85 длина полости, равная половине длины волны фотона, составляет $0.7\ cm$, мы имеем $\gamma\approx 10^{-17}e_p$ эрг. 

Предположим, что мы имеем один из вариантов: а) $|\Phi_0\ra_{\text{at}}=|00\ra$ или б) $|\Phi_0\ra_{\text{at}}=|s\ra$. В первом случае время срабатывания детектора, усредненное по большому числу испытаний, будет в силу центральной предельной теоремы очень близко к $t_s$, во втором --- к $t_t$. Поскольку $t_t-t_s$ --- достаточно большая величина, мы сможем статистически достоверно различить эти два случая. Варианты а) или б) имеют место, например, если начальное состояние пары атомов имеет вид $|01\ra$, так как в этом случае щелчок детектора при приготовлении исходного состояния для первого шага уже означает, что мы имеем состояние $|00\ra$, а отсутствие щелчка в течение достаточно длительно времени --- что мы имеем синглет $|s\ra$. 

Теперь пусть оба числа $\alpha$ и $\beta$ ненулевые. Тогда в матрице плотности состояния атомов в базисе $\{|00\ra,~|s\ra\}$, получаемая в результате описанной последовательности шагов внедиагональные члены будут подавляться с числом шагов, так что в пределе матрица плотности полностью распадется на $|00\ra\la 00|$ с вероятностью $|\alpha|^2$ и $|s\ra\la s|$ с вероятностью $|\beta|^2$, и мы придем к уже разобранному случаю двух несовместных альтернатив. Подавление внедиагональных элементов матрицы плотности установлено численным моделированием. Время полного подавления внедиагональных элементов матрицы плотности $T_{\text{non}}$ получается суммированием всех временных отрезков ожидания полного вылета фотона из полости: $T_{\text{non}}=\sum\limits_{i=0}^Lt_i$ где $L$ --- минимальное значение шага, на котором внедиагональные элементы матрицы $\rho_L$ становятся пренебрежимо малыми. График времени полного подавления внедиагональных элементов матрицы плотности $T_{\text{non}}$ в зависимости от энергии $g$ взаимодействия атомов и поля приведен на рисунке \ref{fig:decoh}. 

Моменты щелчков детектора $d\tau_1,d\tau_2,\dots,d\tau_{L_{s,t}}$ в последовательных экспериментах соответствуют независимой выборке из значений данных величин. Значения $L_s,~L_t$ для двух конкурирующих гипотез будут различаться ненамного. По центральной предельной теореме среднее арифметическое $\xi=\sum\limits_{i=1}^Ld\tau_{i}/L$ будет иметь при больших $L$ нормальное распределение с центрами $a_s$ и $a_t$ соответственно, которые представляют собой средние времена вылета фотона для двух альтернативных гипотез: $a_{s}<a_{t}$. 

\clearpage
\begin{figure}[h!]
	\noindent\centering\includegraphics[width=0.9\textwidth]{Dissertation/images/section_3/opt1.png}
	\captionsetup{format=hang,width=1.0\textwidth,justification=centering,singlelinecheck=no}
	
	\caption{
		{\small
			Время $t=T_{\text{non}}$ полного затухания недиагональных элементов\\($\mathrm{abs} < 10^{-3}$)\\[12pt]
			$g/\hbar w$ в интервале $[0.001; 0.01]$ с шагом $0.001$\\
			Интенсивность фотонной утечки: $\gamma / \hbar w = 0.01$\\
			Шаг по времени: $dt = 0.001 /\gamma$
		}
	}
	\label{fig:decoh}
\end{figure}

\begin{figure}[h!]
	\noindent\centering\includegraphics[width=0.95\textwidth]{Dissertation/images/section_3/l_001g.png}
	\caption{
		{\small
			Функция распределения времени жизни фотона в полости\\(слева при $\gamma=0.01g$, справа при $\gamma=g$)
		}
	}
	\label{fig:0.01}
\end{figure}
\
\\
\noindent Пороговая вероятность вылета, при наступлении которой происходит запуск очередного фотона: $0.95$.

На рисунке \ref{fig:0.01} изображены графики функций распределения времени вылета фотона из полости для разных значений $\gamma$. Соответственно, плотности распределения будут производными от этих функций: для $\gamma=g$ график плотности показан на рисунке \ref{fig:dP1}. 

\begin{figure}[ht!]
	\noindent\centering\includegraphics[width=0.9\textwidth]{Dissertation/images/section_3/l_1g.png}
	\captionsetup{format=hang,width=0.95\textwidth,justification=centering,singlelinecheck=no}
	\caption{
		{\small
			Плотность распределения времени срабатывания детектора\\
			$\gamma = g$
		}
	}
	\label{fig:dP1}
\end{figure}

Мы провели прямое моделирование оптического отбора с помощью датчика случайных чисел, последовательностью испытаний. 
В каждом испытании с интервалом $t_{\mathop{\text{click}}}$ моделируется измерение стока, то есть статистическое испытание факта вылета фотона из полости, исходя из рассчитанной по уравнению \eqref{lind} вероятности. При этом уравнение \eqref{lind} решается методом Эйлера с шагом по времени $dt$, причем временные интервалы в вычислительной модели выбирались так, чтобы для любого шага по времени выполнялись бы неравенства $dt<dt_{\mathop{\text{click}}}\ll \tau_{\text{click}\_i}\leq dT$. Если фотон вылетел, испытание считается завершенным, и мы снова запускаем его в полость, изменяя начальное условие для \eqref{lind}, и переходим к следующему испытанию. Число всех испытаний обозначается через $N$, максимальное время одного испытания: $dT=\text{max}(t_i)$.

Ниже приведены результаты численного моделирования для следующих значений параметров: 
\begin{itemize}
	\item{шаг по времени решения уравнения \eqref{lind}:\ $dt = 10 \mathop{\text{ns}}$,}
	\item{интенсивность вылета фотона в сток:\ $\gamma = 0.01g$ }
	\item{период проверки срабатывания детектора $dt_{\mathop{\text{click}}}=50~\text{ns}$, }
	\item{число испытаний (каждое испытание проводится до первого срабатывания детектора) $N = 1000$,}
	\item{$a_t = 1.551\ \mathop{\text{mks}}$, $a_s = 1.125\ \mathop{\text{mks}}$.}
\end{itemize}

Практически можно взять $n_{\text{bor}} = 2T_{\text{gen}}/(a_{s} + a_{t})$ как среднее число щелчков детектора за общее время $T_{\text{gen}}$ наблюдения, $T_{\text{gen}}=NdT$. Применим наш статистический критерий так: при числе щелчков $n_{\text{click}}>n_{\text{bor}}$ мы имеем синглетное состояние $|s\ra$, в противном случае --- триплет $|00\ra$. 

Тогда ошибка первого и второго рода оценится сверху как квантиль $\int\limits_{n_{\text{bor}}}^\infty N_{0,\s}(x)\ dx$ нормального распределения с математическим ожиданием $0$ и дисперсией $1/\text{min}\{L_s,L_t\}$ и может быть сделана сколь угодно малой с увеличением $T_{\text{gen}}$.

\begin{figure}[ht!]
	\noindent\centering\includegraphics[width=1.0\textwidth]{Dissertation/images/section_3/pt_1ns.png}
	\caption{
		Плотность распределения среднего\\времени жизни фотона в полости\\
		(точность моделирования: $\mathop{dt = 1~\text{ns}}$)
	}
	\label{fig:csignga}
\end{figure}

\clearpage
\section{Программная реализация}
\vspace{-3em}
\begin{figure}[h!]
	\noindent\centering{
		\includegraphics[width=0.9\textwidth]{Dissertation/images/section_3/scheme.jpg}
		\captionsetup{format=hang,width=1.0\textwidth,justification=centering,singlelinecheck=no}
		\\[6pt]
		\caption{
			{\small Блок-схема: оптический отбор темных квантовых состояний}
		}
	}
\end{figure}

\clearpage
\section{Оптический отбор многоуровневых темных состояний}\label{sec:ch3/sect4}
\vspace{-3em}
Описанный оптический отбор темных состояний применим и к ансамблям многоуровневых атомов. Здесь надо рассмотреть состояния многоуровневого синглета $|S_{D}\ra$ вида \eqref{msinglet}
и дополнить его до ортонормированного базиса <<светлыми>> состояниями. При этом отбор должен производиться по всем модам, которых в случае $d$ уровней будет не больше $C^{2}_{n}$ (мы учитываем переходы всех порядков). Для трехуровнего случая обозначим мультисинглет через $|D_3\ra$. Мы рассмотрели несколько примеров для ансамблей трех трехуровневых атомов, проведя численное моделирование для параметров $dt = 1 \mathop{\text{ns}}$, $\gamma = g$, $dt_{\mathop{\text{click}}}=100\ \text{ns}$, $N = 1000$. 
Графики функции распределения времени срабатывания детектора даны на рисунке \ref{fig:csignga}, графики плотности распределения среднего времени щелчка детектора для разных состояний --- на рисунке \ref{fig:csignga1}.

Плотность распределения среднего значения времени детектирования фотонов считалась для значений $dt_{\mathop{\text{click}}} = 100~\text{ns}$.
\begin{flushleft}
	\qquad\qquad~~$a_{|10\ra_{\text{ph}}|D_3\ra} = 16.596\ \mathop{\text{mks}}$ \qquad\qquad\qquad~~ $a_{|10\ra_{\text{ph}}|0_{1}0_{2}0_{3}\ra_{\text{at}}} = 22.243\ \mathop{\text{mks}}$\\
	$a_{|10\ra_{\text{ph}}(|0_{1}1_{2}\ra-|1_{1}0_{2}\ra)|0_{3}\ra} = 22.423\ \mathop{\text{mks}}$ \qquad\qquad $a_{|10\ra_{\text{ph}}|0\ra_{2}(|0_{1}1_{3}\ra-|1_{1}0_{3}\ra)} = 22.423\ \mathop{\text{mks}}$\\ 
	$a_{|10\ra_{\text{ph}}|0_{1}\ra(|0_{2}1_{3}\ra-|1_{2}0_{3}\ra)} = 22.423\ \mathop{\text{mks}}$
\end{flushleft}
\begin{figure}[h!]
	\includegraphics[width=0.95\textwidth]{Dissertation/images/section_3/ph1_l1g.png}
	\captionsetup{format=hang,width=0.9\textwidth,justification=centering,singlelinecheck=no}
	\caption{
		Функция распределения среднего\\ \hspace{7em}времени жизни фотона в полости\\ 
		$\gamma = g$
	}
	\label{fig:csignga}
\end{figure}

\clearpage
\begin{figure}[h!]
	\noindent\centering\includegraphics[width=1.0\textwidth]{Dissertation/images/section_3/gauss_l1g.png}
	\captionsetup{format=hang,width=0.9\textwidth,justification=centering,singlelinecheck=no}
	\caption{
		Плотность распределения среднего\\ \hspace{6em}времени жизни фотона в полости\\
		\hspace{6em}$dt_{\text{click}}= 100~\text{ns}$, $\gamma = g$
	}
	\label{fig:csignga1}
\end{figure}
\vspace{-3em}
\section{Выводы главы}\label{sec:ch3/sect5}
\vspace{-3em}
Мы предложили метод генерации темных состояний двух- и трехуровневых атомов, основанный на оптическом отборе. Этот способ очень прост, но, в отличие от предложенного ранее в работе \cite{dark_states_properties_tanamoto}, не требует применения штарковского сдвига уровней. Здесь используется только многократное измерение времени задержки детектирования фотонов, покидающих оптическую полость. Данный способ позволяет получить темное состояние в виде тензорного произведения синглетов в течение не более нескольких десятков микросекунд для спектра атомов Rb85. Он также практически не зависит от выбора начального состояния атомов, помещенных в полость.

Оптический отбор в равной степени применим и к многоуровневым атомам, причем он может быть настроен на получение строго определенного вида темного состояния. Описанный метод, в силу его простоты и скорости, можно использовать для генерации темных состояний ансамблей из нескольких десятков атомов в одной полости, что может быть полезным для производства защищенных от декогерентности квантовых вычислений в оптических полостях.
