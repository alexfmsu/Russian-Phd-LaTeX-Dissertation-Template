\chapter*{Заключение}                       % Заголовок
\addcontentsline{toc}{chapter}{Заключение}  % Добавляем его в оглавление

\noindent Приведем в заключении основные результаты работы.
\\[30pt]
\textbf{I. Коллективные осцилляции многоатомных ансамблей.}
\\[18pt]
Была проанализирована динамика квантовых состояний ансамблей двухуров­невых атомов и одномодового поля резонатора в рамках модели Тависа­-Каммингса, а также модели Тависа-Каммингса-Хаббарда для случая двух взаимодействующих полостей в приближении RWA \cite{ozhigov_qq, rwa_1, rwa_2}.
\\[18pt]
\noindent По результатам компьютерного моделирования:
\begin{itemize}
	\item[$\bullet$]{установлен резкий характер осцилляций между двумя группами атомов равной численности и равной силы взаимодействия атомов с полем,}
	\item[$\bullet$]{установлено, что резкость осцилляций в ансамбле с четным числом ато­мов предсказуемо растет с увеличением фотонной накачки в полости и намного превосходит резкость осцилляций Раби \cite{rabi_1,rabi_2,rabi_3,rabi_4} для одного атома,}
	\item[$\bullet$]{численно найдена зависимость качества осцилляций от силы взаимодей­ствия атомов с полем; показано, что удлинение периода осцилляций при уменьшении силы взаимодействия может быть скомпенсировано увели­чением числа атомов в группе и усилением фотонной накачки,}
	\item[$\bullet$]{обнаружено хорошо регистрируемое <<квантовое эхо>>: переход состоя­ния атомного ансамбля из одной полости в другую и возврат этого состояния в первоначальную полость. Установлено высокое качество та­кого эха. По итогам многократных численных экспериментов для ряда квантовых систем была установлена граница его возникновения, выра­жающаяся через соотношение интенсивности перехода фотонов между полостями и силы взаимодействия атомов с полем.
		\\[18pt]
		Такого рода эхо представляет собой важный феномен, поскольку оно может быть использовано для переноса многочастичного состояния в процессе квантовой динамики, а также может послужить механизмом организации временной квантовой памяти.
	}
\end{itemize} 

\clearpage
\noindent\textbf{II. Квантовое бутылочное горлышко в атомных превращениях.}
\\[18pt]
Установлен парадоксальный эффект квантового бутылочного горлышка для процесса интенсивного охлаждения атома, который переходит в необратимое состояние, находясь в возбужденном состоянии.
\\[18pt]
В результате численного моделирования было обнаружено, что превышение некоторого порога интенсивности охлаждения ведет к росту вероятности та­кого перехода, что невозможно при классическом описании процесса.
\\[24pt]
\noindent\textbf{III. Оптический отбор темных состояний ансамблей многоуровневых атомов.}
\\[18pt]
Предложен метод оптического отбора темных состо­яний атомов, основанный на томографии состояния поля вне оптической полости. Данный метод не требует применения штарковского сдвига уровней и практически не зависит от выбора начального состояния ато­мов, помещенных в оптический резонатор.
\\[18pt]
\noindent Процесс оптического отбора был численно промоделирован для ансамблей двух­уровневых и трехуровневых атомов спектра Rb85. На основе собранной статистики времён срабатывания детектора, регистрирующего вылет фотона, было установлено его среднее время жизни  в полости, характерное для темных атомных состояний различного вида. Для трехуровневых атомов это, в частности, позволило сепарировать полностью тем­ное трехатомное состояние (мультисинглет) от состояний, содержащих темную двухатомную компоненту.
\\[18pt]
Описанный метод, в силу технической простоты его реализации, может быть использован для генерации темных состояний ансамблей, состоящих из нескольких десятков атомов, что может быть полезным для организации квантовых вычислений, защищенных от декогерентности.
\\[24pt]
\noindent\textbf{IV. Определение размерности темного подпространства пространства многоуровневых атомных ансамблей.}
\\[18pt]
\noindent Предложен алгоритм определения размерности темного подпространства пространства многоуровных атомных ансамблей, основанный на целочислен­ном вычислении ранга сверхбольшой двоичной матрицы путем параллельной редукции соответствующего ей графа.
\\[18pt]
Данный алгоритм позволил установить размерность темного подпространства для ансамблей, содержащих до 21 трехуровневого атома включительно. Для некратного трем количества атомов в группе работа алгоритма завершилась установлением размерности темного подпространства, равной нулю, означаю­щему отсутствие темных состояний в этом пространстве.
\\[18pt]
Гипотеза о структуре и явном виде темных состояний была подтверждена для 1, 3, 6 и 9 трехуровневых атомов в группе. Было показано, что все темные со­стояния в этом случае есть линейные комбинации мультисинглетов.
\\[24pt]
\noindent\textbf{V. Моделирование запутывающего гейта сoCSign на асинхронных атомных возбуждениях.}
\\[18pt]
\noindent Проведены оценки качества управления гейтом coCSign. Установлено, что ос­новным фактором его снижения в рамках модели JCH является увеличение ширины спектральных линий рабочей моды резонатора: данный фактор имеет фундаментальную природу и проистекает из соотношения неопределенностей <<время-энергия>>. По результатам компьютерного моделирования гейта coCSign в отсутствие данного фактора, а также иных ограничений, чисто технического характера, точность его срабатывания составила порядка 95\%.
\\[24pt]
\hyperref[appendix]{Приложение} содержит листинг программного комплекса первой главы.
