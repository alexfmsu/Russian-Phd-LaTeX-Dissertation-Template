\chapter*{Введение}                         % Заголовок
\addcontentsline{toc}{chapter}{Введение}    % Добавляем его в оглавление

\newcommand{\actuality}{}
\newcommand{\progress}{}
\newcommand{\aim}{{\textbf\aimTXT}}
\newcommand{\tasks}{\textbf{\tasksTXT}}
\newcommand{\novelty}{\textbf{\noveltyTXT}}
\newcommand{\influence}{\textbf{\influenceTXT}}
%\newcommand{\methods}{\textbf{\methodsTXT}}
%\newcommand{\defpositions}{\textbf{\defpositionsTXT}}
%\newcommand{\reliability}{\textbf{\reliabilityTXT}}
%\newcommand{\probation}{\textbf{\probationTXT}}
\newcommand{\contribution}{\textbf{\contributionTXT}}
\newcommand{\publications}{\textbf{\publicationsTXT}}

Диссертационная работа посвящена разработке математических и про­граммных средств компьютерного моделирования сложной динамики атомных ансамблей и поля в оптических полостях в рамках конечномерных моделей квантовой электродинамики. Ключевой задачей диссертации является разра­ботка подходов к изучению и практическому получению темных состояний многоуровневых атомов, что позволяет пролить свет на понимание структуры темного подпространства.
\\[24pt]
\indent\textbf{Актуальность работы}\\
\indent Развитие математических и программных методов моделирования кванто­вой динамики поля и атомов является ключевым этапом в развитии технологий квантовых устройств, в частности, в построении \textbf{квантового компьютера} \cite{feynman,quantum_computing_zalka,valiev_1,valiev_2}. В особенности, это важно для отечественных исследований в этой об­ласти, где первые принципы квантовой теории для сложных систем лучше всего проверять в \textbf{оптических полостях} \cite{cavity_exp_1,cavity_exp_2,cavity_exp_3} --- весьма дорогостоящем оборудовании. Открытые публикации, как правило, ограничиваются общим описанием эксперимента: технические нюансы его проведения и прибористика остают­ся намеренно нераскрытыми, что затрудняет воспроизводимость полученных результатов и ограничивает возможность их использования другими иссле­дователями. Поэтому особую значимость на сегодняшний день приобретает разработка математических и программных методов компьютерного и супер­компьютерного моделирования многочастичных квантовых процессов. Это дает нам возможность не только быть в курсе производимых в мире современных экс­периментальных работ, но и предвидеть новые, практически важные эффекты квантовой природы, которые можно было бы получить на таком оборудова­нии. Экспоненциальный рост вычислительной сложности \cite{feynman} требует создания эффективных компьютерных программ, моделирующих квантовую динамику в конечномерных моделях. Важнейшими здесь являются модели \textbf{Джейнса-Кам­мингса} \cite{jc_comparison,jc_qudit,jc_ladder,jc_descendants,jc_phase_transition}, \textbf{Тависа-Каммингса} \cite{tc_exact_solution,tc_a_study,tc_dicke_2,tc_improvement,tc_experimental} и их многоуровневые модификации \cite{jc_tc_extension_1,jc_tc_extension_2,jc_tc_extension_3}.

Построение \textbf{квантового компьютера} по первоначальной схеме Р. Фейнмана \cite{feynman} не удается из-за проблемы \textbf{декогерентности} \cite{decoherence_fedichkin,breuer}, носящей фундаментальный характер и в настоящее время описываемой лишь в рамках теории открытых квантовых систем \cite{breuer}. Важнейшей задачей здесь является описание так называемых \textbf{темных состояний} \cite{dark_states_kok} --- ансамблей атомов, не взаимодействующих с полем. Будучи свободными от декогерентности, темные состояния могут быть использованы для достаточно длительного хранения сложных состояний в \textbf{квантовых вычислениях}, к примеру, в задачах организации \textbf{квантовой памяти} \cite{dark_states_quantum_memory}. Поэтому изучение структуры и разра­ботка методов их практического получения чрезвычайно важны для развития нанотехнологий в целом. В настоящей работе предложен достаточно простой в технической реализации метод получения двухуровневых и многоуровневых темных состояний атомов. Описанию данного метода и его компьютерному мо­делированию посвящена глава \ref{ch:ch3}. Отдельное внимание в данной работе также уделено определению размерности и изучению структуры темного подпростран­ства ансамблей трехуровневых атомов (глава \ref{ch:ch4}).

Особую значимость на сегодняшний день также приобретает компьютер­ное моделирование \textbf{квантовых гейтов} \cite{azuma,fault_tolerant_1}, позволяющее, в частности, дать более точную оценку их качества. Это играет большую роль в непосредственной разработке элементов квантового компьютера и, как следствие, помогает выбрать наиболее эффективный и технологичный путь их реализации. Глава \ref{ch:ch5} данной работы посвящена компьютерному моделированию запутывающего гейта coCSign и оценке его эффективности. Основной запутывающий гейт CNOT (управляемый NOT) реализуется на его основе при помощи однокубитных гейтов.

\textbf{Целью работы} является разработка методов моделирования сложных квантовых процессов в рамках конечномерных моделей квантовой электродина­мики, а также разработка комплекса программ, реализующих данные методы.
\\[18pt]
\indent Важнейшей областью их приложения являются расчет квантовых эф­фектов, моделирование квантовых гейтов, а также исследование темных подпространств, свободных от декогерентности --- главного препятствия кван­товых вычислений.
\\[18pt]
\indent Конкретные \textbf{задачи}, к которым применяются предложенные методы, следующие:
\begin{itemize}
\item{исследование квантовой динамики больших многочастичных систем,}
\item{получение темных состояний ансамблей многоуровневых атомов,}
\item{проверка гипотезы об общем явном виде темного подпространства ансамблей трехуровневых атомов как линейных комбинаций антисим­метричных базисных состояний,
}
\item{моделирование запутывающего гейта coCSign в системе оптических по­лостей.}
\end{itemize}
\
\\
\indent\textbf{Основные положения, выносимые на защиту:}
\begin{enumerate}
\item{Методы компьютерного моделирования сложных квантовых процессов в конечномерных моделях квантовой электродинамики.}
\item{Метод получения темных состояний ансамблей многоуровневых атомов при помощи отбора, основанного на томографии состояния поля вне оптической полости.}
\item{Метод определения размерности темного подпространства многоуров­ ных атомных ансамблей, позволяющий явно описать его структуру.}
\item{Компьютерное моделирование запутывающего гейта coCSign и иссле­
дование факторов, влияющих на точность его срабатывания.}
\end{enumerate}
\
\\
\indent\textbf{Теоретическая и практическая значимость} диссертационной работы состоит в создании математических и программных средств компьютерного и суперкомпьтерного моделирования квантовых процессов в конечномерных мо­ делях, которые позволили
\begin{itemize}
\item{произвести детальный анализ квантовых эффектов, которые не уда­ется предсказать при помощи стандартных математических методов (примером может быть найденный пикообразный характер ансамбле­вых осцилляций групп атомов в оптических полостях),}
\item{предложить метод отбора темных состояний, важный для их практи­ческого применения в наноустройствах и задачах квантовой криптогра­фии,}
\item{произвести определение размерности темного подпространства про­странства многоуровных атомных ансамблей ограниченной численно­сти,}
\item{произвести компьютерное моделирование запутывающего гейта coCSign, а также дать оценку точности его срабатывания.}
\end{itemize}

\clearpage
\indent\textbf{Квантовый компьютер}

Квантовый компьютер, концепция которого впервые была предложена американским физиком-теоретиком Ричардом Фейнманом \cite{feynman} в начале 1980-х годов, представляет собой революционную идею в области вычислительной техники. Он выдвинул предположение о том, что классические компьютеры принципиально не способны моделировать поведение сложных квантовых си­стем из-за экспоненциального роста требуемых для этого ресурсов. В связи с этим им была предложена идея создания нового типа компьютера, который бы использовал принципы квантовой механики, что позволило бы значительно ускорить процесс обработки информации.

Квантовый компьютер представляет собой вычислительное устройство, которое использует квантовомеханические явления, такие как суперпозиция и запутанность, для выполнения операций над данными. В отличие от класси­ческих компьютеров, которые оперируют битами, принимающими значения 0 или 1, квантовые компьютеры работают с квантовыми битами, или \textbf{кубитами} \cite{valiev_2}, каждый из которых может одновременно находиться в состояниях $|0\ra$ и $|1\ra$ (суперпозиции состояний) с определенными вероятностями.

Математически состояние кубита может быть представлено в виде линей­ной комбинации двух ортонормированных базисных состояний:
\begin{align*}
	|\Psi\ra = \alpha|0\ra + \beta|1\ra,\qquad|\alpha|^2 + |\beta|^2 = 1,
\end{align*}
где $|0\ra$ и $|1\ra$ --- базисные состояния кубита (аналогичные 0 и 1 для классического бита), $\alpha$ и $\beta$ --- амплитуды вероятностей этих состояний. Согласно правилу Борна [{\color{green} 55}], квадрат модуля амплитуды определяет вероятность того, что в результате измерения кубита будет получено соответствующее базисное состояние.

Кубиты могут быть физически реализованы с использованием различных технологий \cite{valiev_1,nielsen_chuang,stean}: ионы в ловушках, сверхпроводящие кубиты, кубиты на основе дефектов в кристаллических решетках и многое другое. Выбор технологии для реа­лизации кубита зависит от множества факторов, включая легкость управления, устойчивость к декогерентности, возможность масштабирования и способность к интеграции с другими кубитами для выполнения квантовых операций.

Важным принципом Фейнмановского квантового компьютера является возможность проведения вычислений при помощи квантовых вентилей, или \textbf{квантовых гейтов (quantum gates)} \cite{valiev_1,valiev_2}, --- унитарных операторов, ре­ализующих различные квантовые операции, такие как инверсия, вращение, фазовый сдвиг и прочие преобразования состояний одного или нескольких ку­битов. \textbf{Универсальным} считается набор квантовых гейтов, достаточных для осуществления любого квантового преобразования над произвольным числом ку­битов. Его можно получить, взяв, к примеру, все однокубитные гейты и любой \textbf{запутывающий гейт} (скажем, \textbf{CNOT} --- управляемый NOT) \cite{quantum_gates_barenco}. Основная трудность физической реализации квантовых гейтов заключается в обеспече­нии точного управления состояниями кубитов и минимизации ошибок, связанных с этим.

\textbf{Гейтовая модель} --- одна из наиболее распространенных моделей организации квантовых вычислений. В этой модели осуществление квантовых вычислений производится путем последовательного применения кван­товых гейтов к кубитам --- квантовым аналогам битов. 

\textbf{Квантовый алгоритм} в гейтовой модели --- это представление кван­товых вычислений в виде схемы, состоящей из квантовых гейтов, которые применяются к кубитам с тем, чтобы преобразовать входное квантовое состоя­ние в целевое выходное состояние. В отличие от классических битов, которые могут быть только в состоянии 0 или 1, кубиты могут находиться в суперпозиции этих состояний, что позволяет решать некоторые задачи более эффективно, чем это возможно сегодня на классических компьютерах.

\begin{figure}[h!]
	\noindent\centering{
		\includegraphics[width=0.35\textwidth]{Dissertation/images/introduction/swap.png}
		\captionsetup{format=hang,width=0.9\textwidth,justification=centering,singlelinecheck=no}
		\\[6pt]
		\caption{Пример: операция SWAP перестановки кубитов\\ \hspace{10em}может быть реализована троекратным\\ \hspace{4.25em}применением гейта CNOT}
	}
\end{figure}

Фейнмановская идея квантовых вычислений также предполагает возмож­ность реализации \textbf{быстрых квантовых алгоритмов}, дающих \textbf{экспоненци­альное ускорение}. Такие алгоритмы могут принципиально отличаться от классических и решать определенные задачи за время, недостижимое для со­временных компьютеров.

\clearpage
К числу таких алгоритмов относятся
\begin{itemize}
\item{
	\textbf{алгоритм Шора факторизации целых чисел \cite{shor}}\\
	Опубликованный в 1994 году алгоритм Шора позволяет факторизовать число $n$ за время $O(\log^{3}n)$, используя $O(\log n)$ кубитов. Его работа была продемонстрирована экспериментально в 2001 году группой спе­циалистов из IBM \cite{shor_exp}: разложение числа 15 на простые множители с использованием 7 кубитов. При достаточном числе кубитов алгоритм Шора способен взламывать такие криптографические протоколы, как RSA, за время, едва превосходящее время шифрования на классическом компьютере.\\[-12pt]
}
\item{
	\textbf{алгоритм Гровера поиска элемента в базе данных \cite{grover}}\\
	Алгоритм использует итеративный процесс применения квантового опе­ратора для увеличения амплитуды искомого элемента в суперпозиции всех элементов, что позволяет найти искомый элемент в неупорядочен­ном списке за время $O(\sqrt{N})$, где $N$ --- количество элементов в списке. Данный алгоритм демонстрирует квадратичное ускорение по сравне­нию с классическими алгоритмами поиска.
}
\end{itemize}
\
\\[-6pt]
\indent С тех пор как Фейнман выдвинул свою идею, квантовые вычисления претерпели значительные теоретические и экспериментальные усовершенство­вания \cite{quantum_computing_strengths_weaknesses,kitaev,quantum_computing_zalka,quantum_computing_universal,decoherence_fedichkin,valiev_1,valiev_2,stean,quantum_computing_abrams_lloyd,quantum_computing_ablayev}. Разработка квантовых компьютеров стала возможной благодаря прогрессу в области квантовой оптики, сверхпроводимости, а также в результате открытий, сделанных в квантовой информатике и теории сложности. В то же время, несмотря на значительные успехи в создании кван­товых устройств, ряд технических проблем, таких как \textbf{декогерентность} \cite{decoherence_fedichkin,breuer}, ошибки в квантовых гейтах и сложность масштабирования, остаются нерешенными. Квантовая коррекция ошибок \cite{shor_error_1,shor_error_2,shor_error_3} и разработка надежных квантовых гейтов \cite{fault_tolerant_1,fault_tolerant_2} являются ключевыми направлениями исследований на пути к реализации полноценного квантового компьютера, описанного Фейнманом.

Фейнмановский квантовый компьютер представляет собой воплощение фундаментальных принципов квантовой теории в практических вычислитель­ных устройствах и, несмотря на все трудности, продолжает оставаться одним из наиболее перспективных направлений в развитии современной вычислитель­ной техники.

\clearpage
\indent\textbf{Роль квантовой запутанности в квантовых вычислениях}

\textbf{Квантовые вычисления} представляют собой область информатики, которая исследует возможности применения квантовых явлений, таких как суперпозиция и квантовая запутанность, для представления и обработки ин­формации на микроскопическом уровне, что теоретически позволяет достичь значительного ускорения в решении определенного класса вычислительных задач. Ключевым элементом квантовых вычислений является \textbf{квантовая за­путанность} \cite{nielsen_chuang}, которая играет важнейшую роль в функционировании квантовых гейтов и построении квантовых алгоритмов.

Явление квантовой запутанности возникает, когда группа частиц генерируется или взаимодействует таким образом, что квантовое состояние каж­дой из них не может быть описано независимо от состояний других, вне зависимости от расстояния между ними.

К примеру, множество всех максимально запутанных двухкубитных кван­товых состояний задается состояниями Белла \cite{nielsen_chuang}
\begin{align*}
	\frac{1}{\sqrt{2}}(|00\ra + |11\ra),\qquad \frac{1}{\sqrt{2}}(|01\ra + |10\ra),\\
	\frac{1}{\sqrt{2}}(|00\ra - |11\ra),\qquad \frac{1}{\sqrt{2}}(|01\ra - |10\ra).
\end{align*}

Наличие квантовой запутанности позволяет производить операции над несколькими кубитами одновременно, что позволяет квантовым алгоритмам давать экспоненциальное ускорение в сравнении с классическими алгоритмами. С другой стороны, мно­гочастичная запутанность на сегодняшний день остается главным препятствием на пути к созданию масштабируемого квантового компьютера. Одной из фун­даментальных проблем здесь является \textbf{проблема декогерентности} \cite{decoherence_fedichkin,breuer} --- процесса разрушения квантового состояния под воздействием внешнего окружения. В этом случае системы теряют свои квантовые свойства (в частности, запутанность), что приводит к "классическому"\ поведению кубитов и, как следствие, наруше­нию работы квантовых алгоритмов.

Внедрение широкого использования темных состояний есть один из под­ходов к минимизации влияния декогерентности на квантовые системы, что критически важно для развития и практической реализации квантовых вычислительных технологий.

\clearpage
\indent\textbf{Конечномерные модели квантовой электродинамики}

Конечномерные модели квантовой электродинамики (КЭД) являются ключем к компьютерному моделированию взаимодействия света и вещества. Они позволяют описывать динамику на уровне отдельных атомов и поля, при котором мы можем точно вычислять квантовые эффекты и непосредственно сравнивать их с экспериментом, в отличии, к примеру, от моделей на твердотельных структурах, где участвуют миллионы атомов (что само по себе приводит к невозможности использования точных вычислительных методов). В особенности, это преимущество конечномерных моделей проявляется при рассмотрении запутанных состояний, играющих важную роль в защите квантовой информации от декогерентности \cite{shor_error_1}.

В статье 1963 года \cite{jc_comparison} Э. Джейнс и Ф. Каммингс предложили одну из таких моделей --- модель, описывающую поведение атома, взаимодействующего с модой квантового электромангитного поля. Удержание квантов возбуждения поля (фотонов) осуществляется при помощи двух зеркал, расположенных друг напротив друга и образующих тем самым своеобразную \textbf{полость}, или \textbf{оптический резонатор}.
\begin{figure}[h!]
	\noindent\centering{
		\includegraphics[width=0.8\textwidth]{Dissertation/images/introduction/optical_cavity.png}
		\captionsetup{format=hang,width=0.8\textwidth,justification=centering,singlelinecheck=no}
		\\[6pt]
		\caption{Оптический резонатор}
	}
\end{figure}

Двухуровневый\footnote{обычно используются два энергетических подуровня в атоме} атом, помещенный в такую полость, может взаимодействовать с полем внутри нее: если энергия перехода между уровнями составляет $E = \hbar w_{a}$ и частота атомного перехода $w_{a}$ примерно совпадает с частотой моды поля $w_{c}$, атом переходит из основного состояния $|0\ra$ в возбужденное состояние $|1\ra$ и наоборот. В первом случае происходит поглощение фотона, во втором --- его испускание атомом. Процесс поглощения фотона с его последующим испусканием обратно в полость называется \textbf{осцилляцией Раби \cite{rabi_1,rabi_2,rabi_3}}. 

Для обеспечения времени жизни фотонов в полости, достаточного для нескольких десятков рабиевских осцилляций, нужна большая точность изготовления самих полостей и поддержание высокой степени вакуума в них. Кроме того, для удержания фотона в полости необходимо, чтобы создаваемое им электромагнитное поле порождало конструктивную интерференционную картину внутри полости: расстояние $L$ между отражающими зеркалами должно быть кратно половине длины волны фотона $\lambda = 2\pi c/w_{c}$. Используемые в экспериментах полости имеют длину, равную $L = \lambda/2$, то есть порядка 7 мм, что соответствует длине полуволны атома Rb85, два подуровня которого чаще всего используют в экспериментах \cite{rempe,rb_1,rb_2,rb_3}).

Появление одноатомных мазеров сделало возможным изучение взаимодействия отдельного атома с модой электромагнитного поля резонатора. Так, с его помощью в 1987 году Г. Ремпе, Г. Вальтеру и Н. Кляйну \cite{rempe} удалось усилить связь атома с выбранной модой поля (с одновременным подавлением остальных мод) и экспериментально воссоздать динамику, описываемую моделью Джейнса-Каммингса.

Модель Джейнса-Каммингса \cite{jc_comparison,jc_qudit,jc_ladder,jc_descendants,jc_phase_transition}, предложенная для двухуровневого атома в оптической полости, позже была обобщена на ансамбли таких атомов (модель Тависа-Каммингса \cite{tc_exact_solution,tc_a_study,tc_dicke_2,tc_improvement,tc_experimental}), а также на системы, включающие в себя несколько полостей, связанных между собой оптическим волокном (модели Джейнса-Каммингса-Хаббарда \cite{jch_time_evolution,jch_site_dependent_control,jch_quench_dynamics} и Тависа-Каммингса-Хаббарда \cite{tch_photon_blockade,tch_transfer,tch_quality}).

На сегодняшний день эксперименты с оптическими полостями \cite{cavity_exp_1,cavity_exp_2,cavity_exp_3} относятся к числу дорогостоящих, так как требуют зеркал с очень высокой степенью отражения (до нескольких десятков тысяч на одну рабиевскую осцилляцию), что достигается использованием сверхпроводящих материалов (к примеру, ниобия) и низких (гелиевых) температур.

Численные же эксперименты не ограничены в своем количестве, дают неоспоримые преимущества в вопросах гибкости настроек своих параметров. Поэтому компьютерное и суперкомпьютерное моделирование динамики квантовых состояний в оптических полостях является необходимым и важнейшим звеном в изучении теории сложных квантовых систем.

\clearpage
\indent\textbf{Модель Джейнса-Каммингса}

%Конечномерная модель квантовой электродинамики Джейнса-Каммингса описывает взаимодействие двухуровневого атома, помещенного в оптический %резонатор, с одномодовым полем частоты, близкой к собственной частоте резонатора.
Конечномерная модель квантовой электродинамики Джейнса-Каммингса описывает взаимодействие двухуровневого атома, помещенного в оптический резонатор, с одномодовым полем, частота которого близка к частоте атомного перехода.

\begin{figure}[h!]
	\noindent\centering{
		\includegraphics[width=0.3\textwidth]{Dissertation/images/introduction/jc.png}
		\captionsetup{format=hang,width=0.7\textwidth,justification=centering,singlelinecheck=no}
		\caption{Модель Джейнса-Каммингса: двухуровневый атом в оптическом резонаторе}
	}
\end{figure}

\noindent Атом взаимодействуют с электромагнитным полем полости, испуская или поглощая фотон. При поглощении фотона атом возбуждается, при испускании --- переходит в основное состояние. 
\\[12pt]
Обозначим:
$|0\ra$ -- основное состояние,

\hspace{45pt}$|1\ra$ -- возбужденное состояние атома.
\\[12pt]
Введем также следующие операторы:

\vspace{6pt}
\hspace{5pt}$a^{+}, a$ -- операторы рождения и уничтожения фотонов резонаторной моды,

\hspace{5pt}$\s^{+}, \s$ -- атомные операторы возбуждения и релаксации.
\\[1pt]

\hspace{6pt}$a^{+}|n\ra = \sqrt{n+1}\ |n+1\ra$
\hspace{33pt}$\s^{+}|0\ra = |1\ra$
\hspace{23pt}$\s^{+}|1\ra = 0$

\hspace{15pt}$a|n\ra = \sqrt{n}\ |n-1\ra$
\hspace{70pt}$\s|1\ra =\hspace{2pt}|0\ra$
\hspace{32pt}$\s|0\ra\hspace{1pt} = \hspace{1pt}0$
\\[18pt]
Такая система описывается гамильтонианом Джейнса-Каммингса \cite{jc_comparison,jc_qudit,jc_ladder,jc_descendants,jc_phase_transition}:
\begin{normalsize}
	\begin{equation}
		H_{\text{JC}} = \underbrace{\hbar w_{c}\ a^{+}a}_{\textstyle H_{field}} + \underbrace{\hbar w_{a}\s^{+}\s}_{\textstyle H_{atom}}~+~\underbrace{g(\s^{+}+\s)(a^{+}+a)}_{\textstyle H_{interaction}},
	\end{equation}
\end{normalsize}

\hspace{25pt}$\hbar $ --- постоянная Планка,

\hspace{22pt}$w_{c}$ --- частота фотонов в полости,

\hspace{22pt}$w_{a}$ --- частота атомного перехода,

\hspace{25pt}$g$ --- интенсивность взаимодействия двухуровневого атома с полем:
%\begin{array}
%\hspace{3em}$\displaystyle g = \sqrt{\frac{\hbar w_{c}}{2 \epsilon_{0} V}} \cdot d$, где\\
%\hspace{3em}$V$ -- эффективный объем полости\\
%\hspace{3em}$d$ -- проекция дипольного момента атома на направление поляризации фотона,\\
%\hspace{3em}$\epsilon_{0}$ -- электрическая постоянная
%\end{array}
\begin{gather*}
	g = \sqrt{\frac{\hbar w_{c}}{2 \epsilon_{0} V}} \cdot d,\\
	~~~~V\text{ --- эффективный объем полости},\\
	\epsilon_{0}\text{ --- электрическая постоянная},\\
	\hspace{5em}d\text{ --- проекция дипольного момента атома}\\
	\hspace{7em}\text{на направление поляризации фотона},
\end{gather*}

\vspace{-0.5em}
\hspace{25pt}$|w_{c}-w_{a}|\ll w_{c}+w_{a}$ --- условие применимости модели.
\\[18pt]
\noindent Пренебрегая членами $\displaystyle{\s^{+}a^{+}}$ и $\displaystyle{\s a}$, не сохраняющими энергию, перепишем гамильтониан в следующем виде:
\begin{equation}\label{jc_rwa}
	H_{\text{JC}}~\approx~H_{\text{JC}}^{\text{RWA}}~=~\underbrace{\hbar w_{c}\ a^{+}a}_{\textstyle H_{field}} + \underbrace{\hbar w_{a}\s^{+}\s}_{\textstyle H_{atom}}~+~\underbrace{g(\s^{+}a+\s a^{+})}_{\textstyle H_{interaction}}.
\end{equation}

\noindent Приближение \eqref{jc_rwa} называется \textbf{приближением вращающейся волны (rotating wave approximation), или RWA} \cite{ozhigov_qq,rwa_rabi_1,rwa_rabi_2}, и имеет место в условии слабого взаимодействия
\begin{equation}
	\frac{g}{\hbar w_{c}}\approx\frac{g}{\hbar w_{a}}\ll1,
\end{equation}
при котором слагаемые $\displaystyle{\s^{+}a^{+}}$ и $\displaystyle{\s a}$ быстро осциллируют, что делает их вклад в квантовую картину незначительным.
\\[28pt]
\indent\textbf{Модель Тависа-Каммингса}

Рассмотрим $N$ двухуровневых атомов, взаимодействующих с модой электромагнитного поля в полости оптического резонатора. 
\\[12pt]
\indent Такая система описывается гамильтонианом Тависа-Каммингса \cite{tc_exact_solution,tc_a_study,tc_dicke_2,tc_improvement,tc_experimental}:
\begin{normalsize}
	\begin{equation}
		H_{\text{TC}} =  \hbar w_{c}\ a^{+}a + \hbar \sum_{i=1}^N{w_{a_{i}}\s_{i}^{+}\s_{i}} + \sum_{i=1}^N{g_{i}(\s_{i}^{+}+\s_{i})(a^{+}+a)},
	\end{equation}
\end{normalsize}

\begin{quote}
	$\hbar $ --- постоянная Планка,
	
	$w_{c}$ --- частота фотонов в полости,
	
	$w_{a_{i}}$ --- частота перехода $i$-го атома,
	
	$g_{i}$ --- интенсивность взаимодействия $i$-го атома с полем,
	
	$\s_{i}^{+}, \s_{i}$ --- операторы возбуждения и релаксации $i$-го атома,
	
	$|w_{c}-w_{a_{i}}|\ll w_{c}+w_{a_{i}} \quad\forall i = \overline{1,N}$ --- условие применимости модели.
\end{quote}

\indent Аналогичным образом, пренебрегая слагаемыми $\displaystyle{\s_{i}^{+}a^{+}}$ и $\displaystyle{\s_{i} a}$, не сохраняющими энергию, запишем гамильтониан TC в приближении RWA \cite{ozhigov_qq,rwa_rabi_1,rwa_rabi_2}:
\begin{equation}
	H_{\text{TC}}^{\text{RWA}} = \underbrace{\hbar w_{c}\ a^{+}a}_{\textstyle H_{field}} + \hbar\underbrace{\sum_{i=1}^N{w_{a_{i}}\s_{i}^+\s_{i}}}_{\textstyle H_{atoms}} + \underbrace{\sum_{i=1}^N{g_{i}(\s_{i}^{+}a+\s_{i}a^{+})}}_{\textstyle H_{interaction}},
\end{equation}
\begin{equation}
	\frac{g_{i}}{\hbar w_{c}}\approx\frac{g_{i}}{\hbar w_{a_{i}}}\ll1 \qquad \forall i = \overline{1,N}.
\end{equation}
\vspace{-1em}
\
\\[0pt]
\indent Приближение RWA \cite{ozhigov_qq,rwa_rabi_1,rwa_rabi_2} справедливо для слабого взаимодействия атомов с полем и позволяет либо решить задачу аналитически (в случае одного атома), либо существенно упростить компьютерное моделирование квантовой динамики (в случае нескольких атомов). 

Для RWA-приближения \cite{ozhigov_qq,rwa_rabi_1,rwa_rabi_2} пространство квантовых состояний системы распадается на сумму ортогональных подпространств, инвариантных относительно гамильтониана и обладающих, к тому же, относительно малой размерностью, которая, в случае одинаковой интенсивности взаимодействия атомов с полем, растет линейно с их числом. В случае же различных интенсивностей взаимодействия этот рост становится экспоненциальным, что обуславливает необходимость применения суперкомпьютеров для установления характера квантовой динамики больших атомных ансамблей.

Модели Джейнса-Каммингса и Тависа-Каммингса могут иметь довольно широкое применение в силу их простоты. К примеру, с их помощью можно описывать переход атомных возбуждений по системе резонаторов, соединенных оптическим волокном, проводимость таких систем \cite{skovoroda_conductivity_1,skovoroda_conductivity_2} и связанные с ней эффекты, такие как квантовое бутылочное горлышко \cite{quantum_bottleneck_victorova}, эффект усиления проводимости дефазирующим шумом (dephasing assisted transport, или DAT) \cite{dat_plenio,dat_quantum_dots} и квантовые блуждания на графах \cite{quantum_walks_ambainis,quantum_walks_ambainis_applications,quantum_walks_mixing,quantum_walks_ambainis_speedup}.

\clearpage
\indent\textbf{Mодель Джейнса-Каммингса-Хаббарда}

Модель Джейнса-Каммингса можно обобщить для случая двух и более взаимодействующих полостей: фотоны могут перемещаться между полостями посредством волноводов из оптического волокна.
\begin{figure}[h!]
	\noindent\centering{
		\includegraphics[width=1.0\textwidth]{Dissertation/images/introduction/JCH.png}
		\captionsetup{format=hang,width=0.7\textwidth,justification=centering,singlelinecheck=no}
		\caption{Модель Джейнса-Каммингса-Хаббарда: цепочка взаимодействующих полостей}
	}
\end{figure}

Такая система описывается моделью Джейнса-Каммингса-Хаббарда \cite{jch_time_evolution,jch_site_dependent_control,jch_quench_dynamics} и ее гамильтониан имеет следующий вид:
\[
H_{\text{JCH}}~=~\sum_{j=1}^J \left(\underbrace{\hbar w_{c_{j}}\ a_j^+a_{j}}_{\textstyle H_{field}}~+~\underbrace{\hbar w_{a_{j}}{\s_{j}^+\s_{j}}}_{\textstyle H_{atom}}~+~\underbrace{g_j{(\s_{j}^{+}+\s_{j})(a_{j}^{+}+a_{j})}}_{\textstyle H_{interaction}} \right)
\]
\begin{equation}
	+\sum_{j=1}^{J-1}{k_{j,j+1}\left(a_{j+1}^+a_j+a_{j}^+a_{j+1}\right)},
\end{equation}
\begin{quote}
	$J$ -- количество взаимодействующих полостей (члены с индексом $j$ соответствуют $j$-й полости),\\
	$\s_{j}^{+},\s_{j}$ -- операторы возбуждения/релаксакции атома в $j$-й полости,\\
	$k_{j,j+1}$ -- интенсивность перелета фотона между $j$-й и $(j+1)$-й поло­стями.
\end{quote}

В приближении вращающейся волны (RWA) \cite{ozhigov_qq,rwa_rabi_1,rwa_rabi_2} гамильтониан JCH определяется естественным образом:
\[
H_{\text{JCH}}^{\text{RWA}}~=~\sum_{j=1}^J \left(\underbrace{\hbar w_{c_j}\ a_j^+a_j}_{\textstyle H_{field}}~+~\underbrace{\hbar w_{a_j}{\s_{j}^+\s_{j}}}_{\textstyle H_{atom}}~+~\underbrace{g_j{(\s_{j}^+a_j+\s_{j}a_j^+)}}_{\textstyle H_{interaction}} \right)
\]

\begin{equation}
	+\sum_{j=1}^{J-1}{k_{j,j+1}\left(a_{j+1}^+a_j+a_{j}^+a_{j+1}\right)}.
\end{equation}

\clearpage
\indent\textbf{Mодель Тависа-Каммингса-Хаббарда}

Модель Тависа-Каммингса также обобщается на случай нескольких поло­стей, каждая из которых может содержать один или более атомов.
\begin{figure}[h!]
	\noindent\centering{
		\includegraphics[width=1.0\textwidth]{Dissertation/images/introduction/TCH.png}
		\captionsetup{format=hang,width=0.7\textwidth,justification=centering,singlelinecheck=no}
		\caption{Модель Тависа-Каммингса-Хаббарда: цепочка взаимодействующих полостей}
	}
\end{figure}

Такая система описывается моделью Тависа-Каммингса-Хаббарда \cite{tch_photon_blockade,tch_transfer,tch_quality} и ее га­мильтониан имеет следующий вид:
\[
H_{\text{TCH}}~=~\sum_{j=1}^J \left(\underbrace{\hbar w_{c_{j}}\ a_j^+a_{j}}_{\textstyle H_{field}}~+~\underbrace{\hbar\sum_{j_{i}=1}^{N_{j}}{w_{a_{j_i}}\s_{j_i}^+\s_{j_i}}}_{\textstyle H_{atoms}}~+~\underbrace{\sum_{j_{i}=1}^{N_{j}}g_{j_i}{(\s_{j_i}^{+}+\s_{j_i})(a_{j}^{+}+a_{j})}}_{\textstyle H_{interaction}} \right)
\]
\begin{equation}
	+\sum_{j=1}^{J-1}{k_{j,j+1}\left(a_{j+1}^+a_j+a_{j}^+a_{j+1}\right)},
\end{equation}
\begin{quote}
	$J$ -- количество взаимодействующих полостей (члены с индексом $j$ соответствуют $j$-й полости),\\
	$N_{j}$ -- количество атомов в $j$-й полости,\\
	$\s_{j_i}^{+},\s_{j_i}$ -- операторы возбуждения/релаксакции $i$-го атома в $j$-й полости,\\
	$k_{j,j+1}$ -- интенсивность перелета фотона между $j$-й и $(j+1)$-й поло­стями.
\end{quote}

В приближении вращающейся волны (RWA) \cite{ozhigov_qq,rwa_rabi_1,rwa_rabi_2} гамильтониан TCH определяется естественным образом:
\[
H_{\text{TCH}}^{\text{RWA}}~=~\sum_{j=1}^J \left(\underbrace{\hbar w_{c_j}\ a_j^+a_j}_{\textstyle H_{field}}~+~\underbrace{\hbar\sum_{j_{i}=1}^{N_{j}}{w_{a_{j_i}}\s_{j_i}^+\s_{j_i}}}_{\textstyle H_{atoms}}~+~\underbrace{\sum_{j_{i}=1}^{N_{j}}g_{j_i}{(\s_{j_i}^+a_j+\s_{j_i}a_j^+)}}_{\textstyle H_{interaction}} \right)
\]

\begin{equation}
	+\sum_{j=1}^{J-1}{k_{j,j+1}\left(a_{j+1}^+a_j+a_{j}^+a_{j+1}\right)}.
\end{equation}

\clearpage
\indent\textbf{Открытая квантовая система. Основное квантовое уравнение.}

\textbf{Чистое квантовое состояние} изолированной системы описывается ком­плекснозначной \textbf{волновой функцией} $|\Psi\ra$. Оно характеризуется заданием полного набора возможных значений динамических переменных, определяющих состояние системы.

Временн$\acute{\text{a}}$я эволюция волновой функции чистого состояния системы описывается уравнением Шредингера:
\begin{equation}\label{schrodinger}
	i\hbar\frac{\partial}{\partial t}\Psi(r,t) = H\Psi(r,t),
\end{equation}
где\\
\indent $\hbar$ --- постоянная Планка,\\
\indent $\Psi(r,t)$ --- волновая функция,\\
\indent $H$ --- гамильтониан системы, определяющий ее полную энергию.
\\[12pt]
\indent В действительности же реальная квантовая система тесно окружена частицами и электромагнитным полем различных мод. В частности, отметим, что в рамках описанных выше конечномерных моделей КЭД атом в полости взаимодействует с фотоном весьма ограниченное время, поскольку время жизни фотона в полости невелико. Взаимодействие с внешним окружением приводит к смешанному состоянию квантовой системы, которое не описывается вектором состояния $|\Psi\ra$. Для его описания используется \textbf{матрица плотности}, формализм которой был предложен Л. Ландау, Дж. фон Нейманом и Ф. Блохом \cite{landau,belousov,messia}. Этот формализм позволяет описывать не только чистые, но и смешанные состояния системы, представляющие собой статистическую смесь различных чистых состояний.

\textbf{Открытые квантовые системы} \cite{breuer} --- это квантовые системы, которые могут обмениваться энергией и веществом с внешним окружением. Такие системы не могут быть полностью описаны уравнением Шредингера \eqref{schrodinger}: декогеренция и диссипация, вызванные взаимодействием с окружением, приводят к потере системой квантовых свойств и энергии соответственно. Для описания динамики открытых квантовых систем требуется более сложный математический аппарат, такой как \textbf{основное квантовое уравнение} \cite{breuer}, которое учитывает диссипативные процессы и декогеренцию:
\begin{equation}\label{klgs}
\begin{gathered}
	\frac{d\rho}{dt} = -\frac{i}{\hbar}[H, \rho] + \mathcal{L}(\rho),\\[12pt]
	{\mathcal L}(\rho)=\sum_{k} l_{k}\left(L_{k} \rho L_{k}^{\dagger} - \frac{1}{2} \{L_{k}^{\dagger}L_{k}, \rho \} \right),
\end{gathered}
\end{equation}
где $L_{k}$ --- операторы Линдблада, которые описывают взаимодействие системы с окружением (так называемыми \textbf{факторами декогеренции}), $l_{k}$ --- интенсивности соответствующих факторов декогеренции. Квадратные скобки означают коммутатор, фигурные скобки --- антикоммутатор.

В отсутствие взаимодействия с окружением временн$\acute{\text{a}}$я эволюция матрицы плотности $\rho_{\Psi} = |\Psi\ra\la\Psi|$, связанной с чистым состоянием $|\Psi\ra$, задается уравнением фон Неймана:
\begin{equation}\label{neumann}
	\frac{d\rho}{dt} = -\frac{i}{\hbar}[H, \rho].
\end{equation}
\noindent Это уравнение описывает эволюцию замкнутой квантовой системы, взаимодей­ствие которой с внешним окружением отсутствует, либо пренебрежимо мало. Оно эквивалентно уравнению Шредингера \eqref{schrodinger} и легко выводится из него.

Уравнение \eqref{klgs} называется \textbf{марковским основным кинетическим уравнением в форме Ко­ссаковского-Линдблада-Глаубера-Сударшана} \cite{breuer,kossakowski,lindblad}. Оно является обобще­нием уравнения \eqref{neumann} и описывает изменение во времени матрицы плотности системы, взаимодействующей со стационарной средой, не имеющей долговре­менной памяти.

Задача компьютерного моделирования — нахождение численного решения $\rho(t)$ уравнения \eqref{klgs}, которое можно произвести с использованием различных численных методов, включая как простые (например, метод Эйлера), так и более точные (например, метод Рунге-Кутты).

Анализ открытых квантовых систем играет ключевую роль во многих современных квантовых технологиях, включая квантовые вычисления и квантовую криптографию \cite{akm2017}, где контролируемое взаимодействие с окру­жением используется для защиты квантовой информации от декогеренции. Открытые квантовые системы являются объектом активного исследования в со­временной физике. Их изучение позволяет понять сложные квантовые явления, которые возникают в условиях реального взаимодействия системы с окружаю­щей средой, тем самым помогая нам расширить понимание фундаментальных принципов квантовой механики.

\clearpage
\indent\textbf{Темные состояния}

Особый интерес представляет описание так называемых \textbf{темных состояний} \cite{dark_states_kok,dark_states_dimension} атомных ансамблей модели Тависа-Каммингса.

Атомный ансамбль, находящийся в \textbf{темном состоянии}, не взаимодействует с полем, и потому способен сохранять свою энергию, не испуская ее в виде фотонов. Простейший вид \textbf{темного состояния} --- состояние 
\begin{equation}\label{intro_ds}
	\frac{|01\ra - |10\ra}{\sqrt{2}},
\end{equation}
именуемое также \textbf{синглетом}.

Темные состояния не подвержены декогерентности, поэтому изучение их структуры и практическое получение важны для развития нанотехнологий и квантовых вычислений в целом. В главе \ref{ch:ch3} настоящей работы предложен до­статочно простой в технической реализации метод оптического отбора темных состояний, основанный на томографии электромагнитного поля вне полости. Алгебраическое описание темных состояний ансамблей многоуровневых атомов можно найти в работе \cite{dark_states_kok}, однако их явный вид был найден только для двух­уровневых атомов. В работе \cite{dark_states_dimension} Ю. Ожиговым было доказано, что размерность темного подпространства пространства $n$ двухуровневых атомов равна
\begin{equation}
\mathrm{dim}(D_{n}^2) =	
\begin{cases}
	C_{n}^{k} - C_{n}^{k-1}~\text{при}~n = 2k, \\
	0, \text{в противном случае},
\end{cases}
\end{equation}
и все наборы темных состояний (для четного числа атомов в группе) с учетом нормировки имеют вид 
\begin{equation}
\frac{1}{2^{n/4}}\bigotimes_{j=1}^{n/2}(|01\ra_{j}-|10\ra_{j}),
\end{equation}
где индекс $j = 1,\dots, n/2$ означает номер пары в произвольном разбиении группы из $n$ атомов. Однако уже для трехуровневых атомов имеется лишь ги­потеза о структруре темных состояний, которая была численно подтверждена для весьма ограниченного числа атомов в рамках данной работы (глава \ref{ch:ch4}).

Темные состояния широко используются в квантовой криптографии: использование синглетных состояний дает дополнительные преимущества в обеспечении устойчивости квантовых протоколов. Так, например, ключевое ме­сто в построении протокола AKM2017 \cite{akm2017} занимает подготовка синглетного состояния вида \eqref{intro_ds}.

\clearpage
\textbf{Новизна работы}\\
\indent Собственные состояния гамильтониана Тависа-Каммингса исследовались в диссертации Тависа \cite{tc_exact_solution} 1968 года и нигде не публиковалось. Данное им опи­сание использует уже устаревшие численные методы, оно весьма громоздко и не дает ответа на вопрос ни о явной структуре, ни о пути практического по­лучения таких состояний.

Для ансамблей двухуровневых атомов структура и размерность темного подпространства была найдена в 2018-2020 годах \cite{dark_states_dimension}, однако обобщить найден­ный метод уже на трехуровневые атомы до сих пор не удалось.

Новизна данного круга задач состоит в том, что прямые методы линейной алгебры требуют объемов оперативной памяти, растущих экспоненциально с ростом частиц в ансамбле, поэтому такие методы не могут быть применены для практически значимых задач даже при их суперкомпьютерной реализации. Здесь нужны новые математические и программные средства, учитывающие физический смысл задач и идеологию квантовых вычислений.

В данной диссертации был разработан и программно реализован специ­ альный алгоритм для суперкомпьютера, с помощью которого численно была определена размерность темного подпространства ансамблей, состоящих из двух десятков трехуровневых атомов.

Предложенный в работе алгоритм получения темных состояний много­ уровневых атомов методом оптического отбора --- новый. Он существенно превосходит ранее предложенный Ю. И. Ожиговым метод, основанный на эффекте Штарка-Зеемана \cite{zeeman_1,zeeman_2}. Метод оптического отбора, в силу его тех­нической простоты, можно использовать для получения темных состояний различных видов.

Математические и программные методы, разработанные в диссерта­ции, позволили детально оценить качество запутывающего квантового гейта coCSign, предложенного в 2020 году Ю. И. Ожиговым \cite{quantum_gates_asynchronous}. Данный гейт ис­пользует одну вспомогательную оптическую полость вместо двух, как в гейте Х. Азумы \cite{azuma} (2010 год). В близкой по идеологии работе Азумы \cite{azuma} также нет полного анализа влияния декогерентности, в частности, неточности гейта даже для идеальных полостей, что сделано в данной диссертации.

\clearpage
\textbf{Апробация работы}

Основные результаты диссертационной работы были представлены на следующих конференциях и научных семинарах:
\\[12pt]
R1. \textit{Кулагин А. В.} Темные состояния и квантовые эффекты в контексте ис­следования конечномерных моделей (по материалам диссертации на соискание степени кандидата физико-математических наук) // Физико-технологический институт К. А. Валиева РАН, 2 апреля 2024 (научный семинар)
\\[12pt]
R2. \textit{Кулагин А. В.} Компьютерное моделирование квантовых эффектов в конечномерных моделях // Физико-технологический институт К. А. Валиева РАН, 23 ноября 2023 (научный семинар)
\\[12pt]
R3. \textit{Кулагин А. В., Афанасьев В. И., Ли В., Кэли Чж., Мяо Х.-Х., Плужников И., Ожигов Ю. И., Викторова Н. Б.} Химический квантовый компьютер // Ломоносовские чтения 2021, секция <<Вычислительная математика и киберне­тика>>, Москва, Россия, 20-29 апреля 2021
\\[12pt]
R4. \textit{Ozhigov Y., Kulagin A., Afanasiev V., Keli Zh., Li V., Miao H.-h.} About chemical modifications of finite dimensional models of QED // Quantum Informatics 2021, Москва, Россия, 30 марта - 4 апреля 2021
\\[12pt]
R5. \textit{Dull R., Ozhigov Y., Kulagin A., Li V., Miao H.-h., Keli Zh.} Quality of quantum control by Tavis-Cummings-Hubbard model // Quantum Informatics 2021, Москва, Россия, 30 марта - 4 апреля 2021
\\[12pt]
R6. \textit{Kulagin A., Ozhigov Y.} Realization of algorithm GSA on the asynchronous atomic excitations // Quantum Informatics 2021, Москва, Россия, 30 марта - 4 апреля 2021
\\[12pt]
R7. \textit{Ожигов Ю. И., Кулагин А. В., Ли В., Кэли Чж., Мяо Х.-Х., Дюль Р.} Управление атомными ансамблями в модели Тависа-Каммингса-Хаббарда // Тихоновские чтения 2020, МГУ имени М. В. Ломоносова, Москва, Россия, 26-31 октября 2020
\\[12pt]
R8. \textit{Сковорода Н. А., Ожигов Ю. И., Ладунов В. Ю., Викторова Н. Б., Кулагин А. В.} Ансамбли возбужденных атомов в одномодовых резонаторах // Ломо­носовские чтения 2019, секция <<Вычислительная математика и кибернетика>>, МГУ имени М. В. Ломоносова, Москва, Россия, 15-25 апреля 2019
\\[12pt]
R9. \textit{Ожигов Ю. И., Сковорода Н. А., Кулагин А. В., Ладунов В. Ю.} Компью­терное моделирование системы зарядов и поля в конечных моделях КЭД // Ломоносовские чтения 2018, секция <<Вычислительная математика и киберне­тика>>, МГУ имени М. В. Ломоносова, Москва, Россия, 16-27 апреля 2018.
\
\\[18pt]
\indent\textbf{Публикации}

Основные положения и выводы диссертационного исследования в полной мере изложены в 8 печатных работах, из них 7 статей в рецензируемых журна­лах [A1-A7] и 1 статья в сборнике трудов конференции [A8].
\
\\[18pt]
\indent\textbf{Личный вклад автора}

Все результаты работы, включая предложенный метод оптического отбора темных состояний, получены автором полностью самостоятельно, опубликова­ны в рецензируемых журналах и были представлены на научных конференциях. В публикации [A4] автором был представлен алгоритм определения размерно­сти темного подпространства пространства ансамблей многоуровневых атомов, основанный на редукции сверхбольших графов. Построение алгоритмов, разработка ком­плексов программ и все численные расчеты проведены автором также само­стоятельно. Автор благодарит научного руководителя за постановку задач и обсуждение работы на различных этапах.
\
\\[18pt]
\indent\textbf{Объем и структура работы}

Диссертация состоит из~введения,
\formbytotal{totalchapter}{глав}{ы}{}{}, заключения и приложения.
%\formbytotal{totalappendix}{приложен}{ия}{ий}{}.
%% на случай ошибок оставляю исходный кусок на месте, закомментированным
%Полный объём диссертации составляет  \ref*{TotPages}~страницу
%с~\totalfigures{}~рисунками и~\totaltables{}~таблицами. Список литературы
%содержит \total{citenum}~наименований.
%
Полный объем диссертации составляет
\formbytotal{TotPages}{страниц}{у}{ы}{}, включая
\formbytotal{totalcount@figure}{рисун}{ок}{ка}{ков}.
%и \formbytotal{totalcount@table}{таблиц}{у}{ы}{}.
Список литературы содержит \formbytotal{citenum}{наименован}{ие}{ия}{ий}.

%\indent\textbf{Теоретическая и практическая значимость.}
%\begin{itemize}
%\item[$\bullet$]{математические и программные средства для компьютерного и суперкомпьтерного моделирования квантовых процессов в %конечномерных моделях, разработанные в диссертации, позволяют понять механизм часто контр-интуитивных квантовых эффектов, которые невозможно %предсказать на основе стандартных математических методов квантовой теории, развитых в 20 веке. Примером может быть найденный пикообразный %характер ансамблевых осцилляций групп атомов в разных оптических полостях,}
%\item[$\bullet$]{предложенный в диссертации метод оценки качества квантовых гейтов позволяет дать более точный и реалистичный прогноз их %построения, чем чисто математические приемы, в частности, он позволяет оценить число кубитов, для которых возможна уверенная реализация %ключевого квантового алгоритма Гровера,}
%\item[$\bullet$]{предложенный алгоритм определения структуры темных состояний может использоваться для аналогичных задач в более сложных %системах, например, в конечномерных моделях химии. Предложенный метод получения темных состояний важен для практического их применения в %наноустройствах и квантовой криптографии.}
%\end{itemize}
%\
%\\[0pt]

% -------------------------------------------------------------------------------------------------

\nocite{vynberg}

\nocite{kossakowski}
\nocite{lindblad}

\nocite{quantum_computing_about}

\nocite{quantum_simulation_tc}
\nocite{quantum_simulation_homogeneous}

\nocite{chemistry_miao}

% -------------------------------------------------------------------------------------------------

\nocite{chemistry_backend}
\nocite{chemistry_foundations}
\nocite{chemistry_quantum_simulation}
\nocite{chemistry_simulation_of_chemical_reaction}
\nocite{chemistry_Simulator_for_quantum_computer}
\nocite{chin}
\nocite{dark_state_three_level}
\nocite{highly}
\nocite{non_markovianity}
\nocite{ozhigov_qubit_model}
\nocite{pkok}
\nocite{quantum_chemistry_age}
\nocite{vibrations_quanta_biology}
%\nocite{zeno_1}
%\nocite{zeno_2}
%\nocite{zeno_3}
%\nocite{zeno_4}

% -------------------------------------------------------------------------------------------------
