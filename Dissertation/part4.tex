\chapter{Определение размерности темного подпространства пространства многоуровневых атомных ансамблей}\label{ch:ch4}

\section{Введение}\label{sec:ch4/sect1}

Темные состояния атомных ансамблей не взаимодействуют со светом: не могут излучать и
поглощать фотоны по причине возникающей деструктивной интерференции. Таким образом, будучи свободными от декогеренции, они могут быть широко использованы в квантовых вычислениях (в частности, как механизм для создания квантовой памяти). На сегодняшний день структура темных состояний двухуровневых атомов достаточно хорошо изучена. Для ансамблей многоуровневых атомов вопрос об их структуре по-прежнему остается открытым.

В работе \cite{ozhigov_dimension} было установлено, что размерность темного подпространства пространства ансамблей двухуровневых атомов соответствует числам Каталана. Обобщение данного утверждения на случай трехуровневых (и тем более, многоуровневых) атомных ансамблей и его строгое доказательство представляется весьма трудоемким и к настоящему времени не проведено.

В данной главе будет представлен суперкомпьютерный алгоритм численного подтверждения данной гипотезы для ансамблей, состоящих из ограниченного числа трехуровневых
атомов.

Рассмотрим модель Тависа-Каммингса, описывающую взаимодействие ансамблей идентичных атомов с фотонами в оптическом резонаторе. Ее гамильтониан в случае слабого
взаимодействия $g \ll \hbar w$ (приближение RWA \cite{ozhigov_qq}) имеет следующий вид:
\[
H_{\text{TC}} = \hbar w_{c}a^{+}a + \hbar w_{a} \sum_{i=1}^{n}\s_{i}^{+}\s_{i} + \sum_{i=1}^{n}g_{i}(a^{+}\s_{i} + a\s^{+}_{i}),
\]
где
\begin{itemize}
	\item[$\bullet$]{$\hbar$ -- постоянная Планка,}
	\item[$\bullet$]{$w_{c}$ -- частота фотонов моды резонатора,}
	\item[$\bullet$]{$w_{a}$ -- частота атомного перехода,}
	\item[$\bullet$]{$g_{i}$ -- сила взаимодействия $i$-го атома с полем,}
	\item[$\bullet$]{$n$ --  число атомов в полости,}
	\item[$\bullet$]{$a^{+}, a$ -- операторы рождения и уничтожения фотона в полости \cite{messia}:\\
		\begin{equation}
			a^{+}|m\ra = \sqrt{m+1}|m+1\ra,\qquad\qquad a|m\ra = \sqrt{m}|m-1\ra,
		\end{equation}
		\begin{center}($m$ -- количество фотонов в полости)\end{center}
		\
		
		\begin{equation}
			a =
			\bordermatrix{
				& |0\rangle & |1\rangle & |2\rangle & \cdots & |m-1\rangle & |m\rangle \cr
				|0\rangle & 0 & 1 & 0 & \cdots & \cdots & 0 \cr
				|1\rangle & \vdots & 0 & \sqrt{2} & \ddots &  & \vdots \cr
				|2\rangle & \vdots &  & \ddots & \ddots & \ddots & \vdots \cr
				\cdots & \vdots &  &  & \ddots & \ddots & 0 \cr
				|m-1\rangle & 0 & \cdots & \cdots & \cdots & 0 & \sqrt{m} \cr
				|m\rangle & 0 & \cdots & \cdots & \cdots & \cdots & 0\cr
			},
		\end{equation}
	},\\[12pt]
	\begin{equation}
		a^{+} =
		\bordermatrix{
			& |0\rangle & |1\rangle & |2\rangle & \cdots & |m-1\rangle & |m\rangle \cr
			|0\rangle & 0 & 0 & \cdots & \cdots & 0 & 0 \cr
			|1\rangle & 1 & 0 &  &  & \vdots & \vdots \cr
			|2\rangle & 0 & \sqrt{2} & \ddots & & \vdots & \vdots \cr
			\cdots & \vdots & \ddots & \ddots & \ddots & \vdots & \vdots \cr
			|m-1\rangle & \vdots &  & \ddots & \ddots & 0 & \vdots \cr
			|m\rangle & 0 & \cdots & \cdots & 0 & \sqrt{m} & 0\cr
		},
	\end{equation}
	,\\
	\item[$\bullet$]{$\s^{+}_{i}, \s_{i}$ -- повышающий и понижающий операторы $i$-го атома, действующие на основное $|0\ra$ и возбужденное $|1\ra$ состояния соответственно:\\
		\begin{equation}
			\begin{split}
				\s_{i}|0\ra_{i} = 0,\qquad\qquad\qquad\qquad\s^{+}_{i}|0\ra_{i} = |1\ra_{i},\\
				\noindent\s_{i}|1\ra_{i} = |0\ra_{i},\qquad\qquad\qquad\qquad\s^{+}_{i}|1\ra_{i} = 0.
			\end{split}
		\end{equation}
		
	}
\end{itemize}
\
\\[12pt]
Для двухуровневых атомов операторы $\s^{+}_{i}$ и $\s_{i}$ имеют следующий вид:
\begin{equation}
	\sigma = \bordermatrix{
		& |0\rangle &
		|1\rangle \cr 
		|0\rangle & 0 & 1 \cr 
		|1\rangle & 0 & 0 \cr },
	\qquad\qquad
	\sigma^{+} = \bordermatrix{ 
		& |0\rangle & |1\rangle \cr
		|0\rangle &
		0 & 0 \cr
		|1\rangle & 1 & 0 \cr 
	}.
\end{equation}

Для простоты будем считать, что частота фотона $w_{c}$ отличается от частоты атомного
перехода $w_{a}$ на величину небольшой расстройки $|w_{c} - w_{a}| \ll w_{c}$ и, кроме того, сила взаимодействия атома с полем одинакова для всех атомов:
\[
g_{i} = g\qquad\forall i = \overline{1,~n}.
\]

Также обозначим через $\overline{\s}$ и $\overline{\s}^{+}$ операторы, действующие на атомный ансамбль:
\[
\overline{\s} = \sum_{i=1}^{n}{\s_{i}} =
\s_{1} \otimes I_{2}  \otimes \dots \otimes I_{n} + I_{1}
\otimes \s_{2} \otimes I_{3} \otimes \dots \otimes I_{n} + \dots
+ I_{1} \otimes \dots \otimes I_{n-1} \otimes \s_{n},
\]
$\displaystyle\overline{\s}^{+} = \sum_{i=1}^{n}{\s^{+}_{i}}$ определяется аналогичным образом.\\[12pt]
Здесь наличие оператора $\s_{j}/\s^{+}_{j}$ означает релаксацию/возбуждение $j$-го атома, наличие оператора $I_{j}$ означает отсутствие воздействия на состояние $j$-го атома.\\[12pt]
Их действие позволит нам определить возможность испускания/поглощения одиночного
фотона хотя бы одним атомом ансамбля.\\[12pt]
Только для темных состояний одновременное действие обоих операторов будет давать нулевой эффект:
\begin{equation}\label{critetion}
	\begin{cases}
		\bm{
			\overline{\s}^{+}|\Psi\ra_{at} = 0 \qquad \textbf{(атомы не могут поглотить фотон)}},\\
		\bm{
			\overline{\s}|\Psi\ra_{at} = 0 ~~\qquad \textbf{(атомы не могут испустить фотон).}}
	\end{cases}
\end{equation}
Здесь $\displaystyle|\Psi\ra_{\text{at}} = \sum_{i=1}^{n}\lambda_{i}|i\ra_{\text{at}}$ --- произвольное состояние атомного ансамбля\\($\lambda_{i}$ соответствует амплитуде состояния, $|i\ra_{\text{at}}$ --- уровню возбуждения $i$-го атома).\\[12pt]
Таким образом, условие \eqref{critetion} является критерием темноты атомного ансамбля, что непосредственно следует из определения темного состояния.

В работе \cite{ozhigov_dimension} было сформулировано и строго доказано утверждение о том, что размерность
темного подпространства пространства $n$ двухуровневых атомов равна
\begin{equation}\label{dim_d_2_n}
	\text{dim}(D_{n}^2) =	
	\begin{cases}
		C_{n}^{k} - C_{n}^{k-1}\quad~~\text{при}~n = 2k, \\
		0 \qquad\qquad\qquad\text{в противном случае},
	\end{cases}
\end{equation}
и все наборы темных состояний (для четного числа атомов в группе) с учетом нормировки имеют вид 
\begin{equation}\label{d_2_n}
	\frac{1}{2^{n/4}}\bigotimes_{j=1}^{n/2}(|01\ra_{j}-|10\ra_{j}),
\end{equation}
где индекс $j$ означает номер пары $j = 1,\dots, n/2$ при произвольном разбиении группы из $n$ атомов.
\\[24pt]
\textbf{Приведем несколько примеров:}\\

\noindent
$n = 2$:
\begin{itemize}
	\item[$\triangledown$]{темные состояния: $|0\rangle_{1}|1\rangle_{2} - |1\rangle_{1}|0\rangle_{2}$}
	\item[$\triangledown$]{размерность темного подпространства: $C_{2}^{1} - C_{2}^{0} = 1$\\}
\end{itemize}
$n = 3$:
\begin{itemize}
	\item[$\triangledown$]{нет темных состояний\\}
\end{itemize}
$n = 4$:
\begin{itemize}
	\item[$\triangledown$]{темные состояния:\\$(|0\rangle_{1}|1\rangle_{2} - |1\rangle_{1}|0\rangle_{2})\otimes(|0\rangle_{3}|1\rangle_{4} - |1\rangle_{3}|0\rangle_{4})$\\
		$(|0\rangle_{1}|1\rangle_{3} - |1\rangle_{1}|0\rangle_{3})\otimes(|0\rangle_{2}|1\rangle_{4} - |1\rangle_{2}|0\rangle_{4})$}
	\item[$\triangledown$]{размерность темного подпространства: $C_{4}^{2} - C_{4}^{1} = 2$\\}
\end{itemize}
и так далее.

\section{Постановка задачи}\label{sec:ch4/sect2}
Результат \eqref{dim_d_2_n}, \eqref{d_2_n} справедлив и доказан только для случая двухуровневых атомных ансамблей. Аналогичное утверждение для ансамблей трехуровневых
атомов в качестве гипотезы формулируется следующим образом:
\begin{hyp}
	\label{Th:20}Темное подпространство пространства n трехуровневых атомов есть линейная оболочка состояний $\displaystyle \bigotimes_{j=1}^{n/3}{\widehat{D}_{3}^{(j)}}$,
	где $\widehat{D}_{3}^{(j)}$ --- \textbf{трехатомное состояние, имеющее вид}
	\begin{equation}
		\sum_{\pi \in S_{3}}|\pi(1)\rangle|\pi(2)\rangle|\pi(3)\rangle(-1)^{\sigma(\pi)}
	\end{equation}
	(разбиение $n$ атомов на тройки произвольно),
	и его размерность равна:
	\begin{equation}
		\dim(D^{(3)}_{n}) =
		\begin{cases}
			C_{n}^{k} - C_{n}^{k-1} \qquad\textbf{при n = 3k},\\
			0, \qquad~~\quad\qquad\textbf{в противном случае}.
		\end{cases}\label{eq:dim3}
	\end{equation}
\end{hyp}
\
\\
\indent Примером темного состояния ансамбля трехуровневых атомов является состояние
$
|\Psi\rangle = |012\rangle + |120\rangle + |201\rangle - |021\rangle - |102\rangle - |210\rangle$ (оно же единственное).

Возвращаясь к критерию темноты \eqref{critetion} многоатомного квантового состояния, отметим, что множество решений системы
\begin{equation}\label{slae}
	\begin{cases}
		\overline{\sigma}^{+}|\Psi\rangle_{\text{at}} = 0,\\
		\overline{\sigma}|\Psi\rangle_{\text{at}} = 0,
	\end{cases} \Leftrightarrow
	Ax =
	\begin{pmatrix}
		\overline{\sigma}^{+}\\
		\overline{\sigma}
	\end{pmatrix}
	\begin{pmatrix}
		\lambda_1\\
		\dots\\
		\lambda_N\\
	\end{pmatrix} = 0
\end{equation}
однородных уравнений с соответствующей матрицей системы
$
A=\begin{pmatrix}
	\overline{\sigma}^{+}\\
	\overline{\sigma}
\end{pmatrix}
$ размерности $M \times N$ есть линейное подпространство размерности $N -
rank(A)$. Размерность матрицы системы для трехуровневых атомов равна $(M, N) = (6 \cdot 3^{n}, 3^{n})$. Таким образом, определение размерности темного подпространства сводится к задаче определения ранга матрицы $A$ системы для различных значений $n$.

Предложенный далее алгоритм численно установит, что
\begin{equation}\label{eq:dim3}
	\dim(D^{(3)}_{n}) = 3^{n} - rank(A) =
	\begin{cases}
		C_{n}^{k} - C_{n}^{k-1}\qquad\text{при}~n=3k,\\
		0\quad\qquad\qquad\quad~\text{при}~n \ne 3k.
	\end{cases}
\end{equation}
\
\\[12pt]
\indent Перейдем к перечислению трудностей, возникающих при решении поставленной задачи, а именно точного вычисления ранга сверхбольшой матрицы.

\clearpage
\noindent \textbf{Рассмотрим несколько примеров}\\[12pt]Двухуровневые ансамбли, $\mathbf{p = 2}$:\\
\medskip\hrule\medskip
\noindent$\mathbf{p = 2, n = 2:}$
\begin{equation}\label{eq:matrix2}
	{\footnotesize
		A =
		\bordermatrix{
			& |00\rangle & |01\rangle & |10\rangle & |11\rangle \cr
			& 0 & 1 & 1 & 0 \cr
			& 0 & 0 & 0 & 1 \cr
			& 0 & 0 & 0 & 1 \cr
			& 0 & 0 & 0 & 0 \cr
			& 0 & 0 & 0 & 0 \cr
			& 1 & 0 & 0 & 0 \cr
			& 1 & 0 & 0 & 0 \cr
			& 0 & 1 & 1 & 0 \cr
		}\rightarrow
		\bordermatrix{
			& |00\rangle & |01\rangle & |10\rangle & |11\rangle \cr
			& 1 & 0 & 0 & 0 \cr
			& 0 & 1 & 1 & 0 \cr
			& 0 & 0 & 0 & 1 \cr
		}
	}
\end{equation}
\quad\quad~~Решая систему \eqref{slae}, находим
{\footnotesize
	$\lambda =
	\begin{pmatrix}
		0\\
		\xi\\
		-\xi\\
		0
	\end{pmatrix}
	$
},\
\quad$D^{(2)}_p = \L(\{|01\rangle - |10\rangle\})$.\\[12pt]

\noindent\quad\quad~~~$\dim(D^{(2)}_p) = 2^{p} - rank(A) = 2^{2} - 3 = 1$\\

\medskip\hrule\medskip

\noindent$\mathbf{p = 2, n = 3:}$
{\footnotesize
	\[
	A =
	\begin{pmatrix}
		\overline{\sigma}^{+}\\
		\overline{\sigma}
	\end{pmatrix}=
	\begin{pmatrix}
		0 & 1 & 1 & 0 & 1 & 0 & 0 & 0\\
		0 & 0 & 0 & 1 & 0 & 1 & 0 & 0\\
		0 & 0 & 0 & 1 & 0 & 0 & 1 & 0\\
		0 & 0 & 0 & 0 & 0 & 0 & 0 & 1\\
		0 & 0 & 0 & 0 & 0 & 1 & 1 & 0\\
		0 & 0 & 0 & 0 & 0 & 0 & 0 & 1\\
		0 & 0 & 0 & 0 & 0 & 0 & 0 & 1\\
		0 & 0 & 0 & 0 & 0 & 0 & 0 & 0\\
		0 & 0 & 0 & 0 & 0 & 0 & 0 & 0\\
		1 & 0 & 0 & 0 & 0 & 0 & 0 & 0\\
		1 & 0 & 0 & 0 & 0 & 0 & 0 & 0\\
		0 & 1 & 1 & 0 & 0 & 0 & 0 & 0\\
		1 & 0 & 0 & 0 & 0 & 0 & 0 & 0\\
		0 & 1 & 0 & 0 & 1 & 0 & 0 & 0\\
		0 & 0 & 1 & 0 & 1 & 0 & 0 & 0\\
		0 & 0 & 0 & 1 & 0 & 1 & 1 & 0\\
	\end{pmatrix}\rightarrow
	\begin{pmatrix}
		1 & 0 & 0 & 0 & 0 & 0 & 0 & 0\\
		0 & 1 & 0 & 0 & 0 & 0 & 0 & 0\\
		0 & 0 & 1 & 0 & 0 & 0 & 0 & 0\\
		0 & 0 & 0 & 1 & 0 & 0 & 0 & 0\\
		0 & 0 & 0 & 0 & 1 & 0 & 0 & 0\\
		0 & 0 & 0 & 0 & 0 & 1 & 0 & 0\\
		0 & 0 & 0 & 0 & 0 & 0 & 1 & 0\\
		0 & 0 & 0 & 0 & 0 & 0 & 0 & 1\\
	\end{pmatrix}
	\]
}
\noindent\quad\quad~~~$\dim(D^{(2)}_p) = 2^{p} - rank(A) = 2^{3} - 8 = 0$\\

\clearpage
\indent Исходя из вышеуказанных примеров, можно отметить следующее::
\begin{itemize}
	\item[$\bullet$]{изначально (до процедуры приведения к ступенчатому виду) матрица состоит из нулей и единиц,}
	\item[$\bullet$]{матрица является сильно разреженной с множеством нулевых строк,}
	\item[$\bullet$]{подавляющее большинство строк матрицы нетривиальны: не соответствуют строкам единичной матрицы, в значительной части из них количество единиц сильно превосходит количество нулей,}
	\item[$\bullet$]{размерность матрицы в случае трехуровневой системы равна $6 \cdot 3^{n} \times 3^{n}$, т.к. присутствуют возбуждения и релаксации атомов между уровнями: имеются переходы трех типов --- $\sigma^{0,1}_i$, $\sigma^{1,2}_i$,
		$\sigma^{0,2}_i$.}
\end{itemize}
\
\\[0pt]
\indent Зависимость размерности матрицы системы от числа $n$ атомов в ансамбле носит экспоненциальный характер:

\noindent\begin{tabular}[t]{|p{4em}|p{5em}|p{4em}|p{9em}|p{9em}|}
	\hline
	$n$ & 3 &  & 18 & 21 \\
	\hline
	$M$ x $N$ & $162 \times 27$ & $\quad~\cdots$ & $2.3 \cdot 10^9 \times 387 \cdot 10^6$ & $62 \cdot 10^9 \times 10.4 \cdot 10^9$ \\
	\hline
\end{tabular}
\
\\[12pt]

В связи с этим, перечислим некоторые вычислительные трудности:
\begin{itemize}
	\item[$\bullet$]{вещественные плотные матрицы размера $(6 \cdot 3^{n}) \times 3^{n}$, начиная уже с малых значений $n$, не умещаются целиком в оперативную память,}
	\item[$\bullet$]{алгоритм должен точно вычислять ранг: ошибки округления при работе с действительными числами могут повлиять на вычисления.\\
		Такие инциденты, как
		\[
		\begin{pmatrix}
			0 & \dots & 0.333333 & \dots \\
			0 & \dots & 0.3333334 & \dots \\
		\end{pmatrix}\qquad\text{дают неверный ответ}~rank(A) = 2.
		\]}
	
	Кроме того, было обнаружено, что многие существующие алгоритмы вычисления
	ранга разреженной матрицы дают неверный результат, начиная с $n = 9$ (в частности, метод $\mathsf{linalg.interpolative.estimate\_rank}$ библиотеки $\mathbf{scipy}$). Данный факт означает \textbf{необходимость вычисления ранга в целых числах}.
	\item[$\bullet$]{применение алгоритма Гаусса приведения матрицы к ступенчатому виду занимает неприемлемо большое вычислительное время и не может быть использовано для матриц такого размера}.
\end{itemize}

\section{Описание алгоритма}\label{sec:ch4/sect3}

Перейдем непосредственно к описанию предложенного алгоритма.
Он будет состоять из трех частей:
\begin{enumerate}
	\item{построение разреженной матрицы системы $A$};
	\item{целочисленное приведение разреженной матрицы к ступенчатой форме путем редуцирования соответствующего ей графа};
	\item{окончательное целочисленное приведение матрицы к ступенчатой форме с помощью
		алгоритма Гаусса}.
\end{enumerate}

\subsection{Построение разреженной матрицы системы}\label{subsec:ch4/subsect1}
Генерация разреженных матриц для различных $n = 3\dots21$ выполняется стандартным
образом с использованием научного пакета Python SciPy для разреженных матриц.

Каждая строка содержит номера столбцов ненулевых элементов (нумерация столбцов
начинается с нуля). Полностью нулевые строки плотной матрицы пропускаются.

К примеру, для системы \eqref{eq:matrix2} разреженная матрица записывается в следующем виде:
\begin{flushleft}
	\noindent \qquad 1,2\\
	\noindent\qquad 3,\\
	\noindent\qquad 3,\\
	\noindent\qquad 0,\\
	\noindent\qquad 0,\\
	\noindent\qquad 1,2\\
\end{flushleft}\label{eq:sparse_matrix2}
\
\\[12pt]
\noindent Для трехуровневой системы построение выполняется аналогично.

\clearpage
\noindent Выпишем характерные метрики после первого этапа алгоритма:

\noindent
{\footnotesize
	\begin{tabular}[t]{|p{5em}|p{3em}|p{4em}|p{5em}|p{5em}|p{5em}|p{5em}|p{5em}|}
		\hline
		$n$ & 3 & 6 & 9 & 12 & 15 & 18 & 21 \\
		\hline
		$\mathrm{M}$ & 162 & 4374 & $118 \cdot 10^{3}$ & $3.2 \cdot 10^{6}$ & $86.1 \cdot 10^{6}$ & $2.3 \cdot 10^{9}$ & $62.7 \cdot 10^{9}$ \\
		\hline
		$\mathrm{N}$ & 27 & 729 & $19.6 \cdot 10^{3}$ & $531.4 \cdot 10^{3}$ & $14.3 \cdot 10^{6}$ & $387.4 \cdot 10^{6}$ & $10.4 \cdot 10^{9}$ \\
		\hline
		$\mathbf{nrows}$ & 114 & 3990 & $\approx M$ & $\cdots$ & $\cdots$ & $\cdots$ & $\approx M$ \\
		\hline
		$\mathbf{nonzeros}$ & 162 & 8748 & $354.3 \cdot 10^{3}$ & $12.7 \cdot 10^{6}$ & $430.4 \cdot 10^{6}$ & $13.9 \cdot 10^{9}$ & $439.3 \cdot 10^{9}$ \\
		\hline
	\end{tabular}
}

\begin{itemize}
	\item[$\bullet$]{$n$  --- кол-во атомов},
	\item[$\bullet$]{$\mathrm{M}$ --- кол-во строк в плотной\footnote[1]{плотные матрицы не создаются} матрице,}
	\item[$\bullet$]{$\mathrm{N}$ --- кол-во столбцов в плотной матрице,}
	\item[$\bullet$]{$\mathbf{nrows}$  --- кол-во строк в разреженной матрице,}
	\item[$\bullet$]{$\mathbf{nonzeros}$ --- кол-во ненулевых элементов (единиц) в разреженной матрице.}
\end{itemize}
\
\\[0pt]
\noindent Таким образом, при $n = 21$ мы имеем граф, состоящий из нескольких десятков миллиардов вершин. Обработка такого количества вершин --- стандартная задача для суперкомпьютера с распределенной памятью и тысячами процессорных ядер (в данном случае: Ломоносов-2).

\subsection{Целочисленное приведение матрицы к ступенчатой форме при помощи
	параллельного графового алгоритма}\label{subsec:ch4/subsect2}

Для разреженной матрицы системы построим граф по следующей схеме:
\begin{itemize}
	\item[$\bullet$]{значение вершины --- номер строки в разреженной матрице,}
	\item[$\bullet$]{метки ребра, входящего в вершину, совпадает с позициями единиц в соответствующей
		строке,}
	\item[$\bullet$]{вершины, в которые входят ребра с одинаковыми метками, соединяются ребром с этими метками}.
\end{itemize}
\
\\[0pt]
\indent Для системы \eqref{eq:matrix2} и соответствующей ее разреженной матрицы граф выглядит следующим образом:\\[12pt]

\hspace{1em}\underline{\hspace{4em} Graph \hspace{4em} | \hspace{1em} Sparse matrix \hspace{2em} | \hspace{3em} Dense matrix \hspace{3em}}\\

\[
\hspace{3.5em}\parbox[b][5cm][t]{50mm}{
	\includegraphics[width=36mm]{Dissertation/images/section_4/graph1.eps}
}
\hfill
\parbox[b][5cm][t]{35mm}{
	\begin{flushleft}
		Line 0: \quad 1,2\\
		Line 1: \quad 3,\\
		Line 2: \quad 3,\\
		Line 3: \quad 0,\\
		Line 4: \quad 0,\\
		Line 5: \quad 1,2
	\end{flushleft}
}
\hfill
\parbox[b][5cm][t]{50mm}{
	\bordermatrix{
		& |00\rangle & |01\rangle & |10\rangle & |11\rangle \cr
		& 0 & 1 & 1 & 0 \cr
		& 0 & 0 & 0 & 1 \cr
		& 0 & 0 & 0 & 1 \cr
		& 0 & 0 & 0 & 0 \cr
		& 0 & 0 & 0 & 0 \cr
		& 1 & 0 & 0 & 0 \cr
		& 1 & 0 & 0 & 0 \cr
		& 0 & 1 & 1 & 0 \cr
	}
}\label{eq:graph}
\]
\
\\

Данный граф позволяет производить вычитание строк, и таким образом, чтобы отрицательные числа не появлялись в соответствующей плотной матрице системы. Удаление дубликатов и вычитание строк друг из друга выполняются путем удаления соответствующих ребер графа. Вершины, не имеющие входных и выходных ребер, удаляются (что соответствует удалению нулевых строк в исходной матрице после элементарных преобразований).

Удаление ребер в данном примере происходит следующим образом:
\begin{figure}[h]
	\noindent\centering{
		\includegraphics[width=35mm]{Dissertation/images/section_4/graph2.eps}
	}
\end{figure}

Рассмотрим более сложный пример:
$
A =
\begin{pmatrix}
	\cdots & \cdots & \cdots & \cdots & \cdots\\
	1 & 0 & 0 & 0 & 1 \cr
	0 & 1 & 0 & 0 & 0 \cr
	1 & 0 & 0 & 1 & 1 \cr
	0 & 0 & 1 & 0 & 0 \cr
	\cdots & \cdots & \cdots & \cdots & \cdots
\end{pmatrix}.
$

\clearpage
Соответствующие преобразования графа:
\begin{figure}[h]
	\noindent\centering{
		\includegraphics[width=120mm]{Dissertation/images/section_4/graph7.eps}
	}
	\label{figCurves}
\end{figure}

Последний пример демонстрирует возможность удаления дубликатов и вычитания строк,
находящихся на большом расстоянии. Для этого достаточно определить пересечения меток ребер, входящих и выходящих в связанные вершины. Удаление дубликатов может
осуществляться по цепочке (снизу вверх или наоборот). В результате мы избегаем сравнения строк друг с другом, поиска совпадающих пар и использования операции деления
в ходе элементарных преобразований. Более сложным образом с помощью графа можно
суммировать строки для последующих вычитаний или удаления дубликатов. К примеру, производить следующие элементарные преобразования:
\[
\begin{pmatrix}
	1 & 0 & 0 & 1\\
	0 & 1 & 1 & 0\\
	1 & 0 & 1 & 0\\
	0 & 1 & 0 & 1\\
\end{pmatrix}\rightarrow
\begin{pmatrix}
	1 & 0 & 0 & 1\\
	0 & 1 & 1 & 0\\
	1 & 0 & 1 & 0\\
	1 & 1 & 1 & 1\\
\end{pmatrix}
\rightarrow
\begin{pmatrix}
	1 & 0 & 0 & 1\\
	0 & 1 & 1 & 0\\
	0 & 0 & 1 & 0\\
	1 & 0 & 0 & 1\\
\end{pmatrix}
\rightarrow
\begin{pmatrix}
	1 & 0 & 0 & 1\\
	0 & 1 & 0 & 0\\
	1 & 0 & 1 & 0\\
	0 & 1 & 0 & 1\\
\end{pmatrix}\rightarrow\cdots.
\]

Предложенный алгоритм является параллельным и производит редукцию графа, распределенного по сетке процессоров: разреженные матрицы считываются параллельно ленточным образом (каждый процесс считывает свою часть матрицы и формирует свою часть
общего графа). Если связанные вершины принадлежат разным процессам, информация о
них пересылается в неблокирующем режиме (MPI\_Isend, MPI\_Irecv). При необходимости удаления вершин и/или ребер информация пересылается между соответствующими
ядрами.

\subsection{Окончательное целочисленное приведение матрицы к ступенчатой форме с помощью алгоритма Гаусса}\label{subsec:ch4/subsect3}
Результат второго этапа --- получение матрицы, частично приведенной к ступенчатому виду. Ее достаточно просто восстановить по редуцированному графу. Количество строк в
такой матрице на несколько порядков отличается от числа строк в исходной плотной матрице системы. Окончательное приведение матрицы к ступенчатому виду производится при
помощи алгоритма Гаусса без делений (за ислючением целочисленного деления строки).

\section{Выводы главы}\label{sec:ch4/sect4}
Размерность темного подпространства трехуровневых ансамблей атомов была численно
установлена для следующих значений $n = 3k$:

\noindent\begin{tabular}[t]{|p{5em}|p{3em}|p{3em}|p{3em}|p{3em}|p{4em}|p{4em}|p{4em}|}
	\hline
	$n$ & 3 & 6 & 9 & 12 & 15 & 18 & 21 \\
	\hline
	$\dim(D^{(3)}_n)$ & 1 & 5 & 28 & 165 & 1001 & 6188 & 38 760 \\
	\hline
\end{tabular}\
\\[12pt]

\noindent В остальных случаях для $1 \le n \le 21,~n \ne 3k$ работа алгоритма завершилась приведением матрицы системы \eqref{slae} к единичной матрице c установлением $\dim(D^{(3)}_{n}) = 0$.
\\[12pt]
\noindentТаким образом, для ансамблей, состоящих из $1 \le n \le 21$ трехуровневых атомов, численно была подтверждена гипотеза о соответствии размерности темного подпространства числам Каталана.
