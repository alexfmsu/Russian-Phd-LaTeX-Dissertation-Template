\chapter{Определение размерности темного подпространства пространства многоуровневых атомных ансамблей}\label{ch:ch4}

\section{Введение}\label{sec:ch4/sect1}

Темные состояния атомных ансамблей не взаимодействуют со светом: не могут излучать и
поглощать фотоны по причине возникающей деструктивной интерференции. Таким образом, будучи свободными от декогеренции, они могут быть широко использованы в квантовых вычислениях (в частности, как механизм для создания квантовой памяти). На сегодняшний день структура темных состояний двухуровневых атомов достаточно хорошо изучена. Для ансамблей многоуровневых атомов вопрос об их структуре по-прежнему остается открытым.

В работе \cite{dark_states_dimension} было установлено, что размерность темного подпространства пространства ансамблей двухуровневых атомов соответствует числам Каталана. Обобщение данного утверждения на случай трехуровневых (и тем более, многоуровневых) атомных ансамблей и его строгое доказательство представляется весьма трудоемким и к настоящему времени не проведено.

В данной главе будет представлен суперкомпьютерный алгоритм численного подтверждения данной гипотезы для ансамблей, состоящих из ограниченного числа трехуровневых
атомов.

Рассмотрим модель Тависа-Каммингса, описывающую взаимодействие ансамблей идентичных атомов с фотонами в оптическом резонаторе. Ее гамильтониан в случае слабого
взаимодействия $g \ll \hbar w$ (приближение RWA \cite{ozhigov_qq}) имеет следующий вид:
\[
H_{\text{TC}} = \hbar w_{c}a^{+}a + \hbar w_{a} \sum_{i=1}^{n}\s_{i}^{+}\s_{i} + \sum_{i=1}^{n}g_{i}(a^{+}\s_{i} + a\s^{+}_{i}),
\]
где
\begin{itemize}
	\item[$\bullet$]{$\hbar$ -- постоянная Планка,}
	\item[$\bullet$]{$w_{c}$ -- частота фотонов моды резонатора,}
	\item[$\bullet$]{$w_{a}$ -- частота атомного перехода,}
	\item[$\bullet$]{$g_{i}$ -- сила взаимодействия $i$-го атома с полем,}
	\item[$\bullet$]{$n$ --  число атомов в полости,}
	\item[$\bullet$]{$a^{+}, a$ -- операторы рождения и уничтожения фотона в полости \cite{messia}:\\
		\begin{equation}
			a^{+}|m\ra = \sqrt{m+1}|m+1\ra,\qquad\qquad a|m\ra = \sqrt{m}|m-1\ra,
		\end{equation}
		\begin{center}($m$ -- количество фотонов в полости)\end{center}
		\
		
		\begin{equation}
			a =
			\bordermatrix{
				& |0\ra & |1\ra & |2\ra & \cdots & |m-1\ra & |m\ra \cr
				|0\ra & 0 & 1 & 0 & \cdots & \cdots & 0 \cr
				|1\ra & \vdots & 0 & \sqrt{2} & \ddots &  & \vdots \cr
				|2\ra & \vdots &  & \ddots & \ddots & \ddots & \vdots \cr
				\cdots & \vdots &  &  & \ddots & \ddots & 0 \cr
				|m-1\ra & 0 & \cdots & \cdots & \cdots & 0 & \sqrt{m} \cr
				|m\ra & 0 & \cdots & \cdots & \cdots & \cdots & 0\cr
			},
		\end{equation}
	},\\[12pt]
	\begin{equation}
		a^{+} =
		\bordermatrix{
			& |0\ra & |1\ra & |2\ra & \cdots & |m-1\ra & |m\ra \cr
			|0\ra & 0 & 0 & \cdots & \cdots & 0 & 0 \cr
			|1\ra & 1 & 0 &  &  & \vdots & \vdots \cr
			|2\ra & 0 & \sqrt{2} & \ddots & & \vdots & \vdots \cr
			\cdots & \vdots & \ddots & \ddots & \ddots & \vdots & \vdots \cr
			|m-1\ra & \vdots &  & \ddots & \ddots & 0 & \vdots \cr
			|m\ra & 0 & \cdots & \cdots & 0 & \sqrt{m} & 0\cr
		},
	\end{equation}
	,\\
	\item[$\bullet$]{$\s^{+}_{i}, \s_{i}$ -- повышающий и понижающий операторы $i$-го атома, действующие на основное $|0\ra$ и возбужденное $|1\ra$ состояния соответственно:\\
		\begin{equation}
			\begin{split}
				\s_{i}|0\ra_{i} = 0,\qquad\qquad\qquad\qquad\s^{+}_{i}|0\ra_{i} = |1\ra_{i},\\
				\noindent\s_{i}|1\ra_{i} = |0\ra_{i},\qquad\qquad\qquad\qquad\s^{+}_{i}|1\ra_{i} = 0.
			\end{split}
		\end{equation}
		
	}
\end{itemize}
\
\\[12pt]
Для двухуровневых атомов операторы $\s^{+}_{i}$ и $\s_{i}$ имеют следующий вид:
\begin{equation}
	\s_{i} = \bordermatrix{
		& |0\ra_{i} & |1\ra_{i} \cr 
		|0\ra_{i} & 0 & 1 \cr 
		|1\ra_{i} & 0 & 0 \cr
	},
	\qquad\qquad
	\s_{i}^{+} = \bordermatrix{ 
		& |0\ra_{i} & |1\ra_{i} \cr
		|0\ra_{i} & 0 & 0 \cr
		|1\ra_{i} & 1 & 0 \cr 
	}.
\end{equation}

Для простоты будем считать, что частота фотона $w_{c}$ отличается от частоты атомного
перехода $w_{a}$ на величину небольшой расстройки $|w_{c} - w_{a}| \ll w_{c}$ и, кроме того, сила взаимодействия атома с полем одинакова для всех атомов:
\[
g_{i} = g\qquad\forall i = \overline{1,~n}.
\]

Также обозначим через $\overline{\s}$ и $\overline{\s}^{+}$ операторы, действующие на атомный ансамбль:
\[
\overline{\s} = \sum_{i=1}^{n}{\s_{i}} =
\s_{1} \otimes I_{2}  \otimes \dots \otimes I_{n} + I_{1}
\otimes \s_{2} \otimes I_{3} \otimes \dots \otimes I_{n} + \dots
+ I_{1} \otimes \dots \otimes I_{n-1} \otimes \s_{n},
\]
$\displaystyle\overline{\s}^{+} = \sum_{i=1}^{n}{\s^{+}_{i}}$ определяется аналогичным образом.\\[12pt]
Здесь наличие оператора $\s_{j}/\s^{+}_{j}$ означает релаксацию/возбуждение $j$-го атома, наличие оператора $I_{j}$ означает отсутствие воздействия на состояние $j$-го атома.

\noindent Их действие позволит нам определить возможность испускания/поглощения одиночного фотона хотя бы одним атомом ансамбля.

Только для темных состояний одновременное действие обоих операторов будет давать нулевой эффект:
\begin{equation}\label{critetion}
	\begin{cases}
		\bm{
			\overline{\s}^{+}|\Psi\ra_{at} = 0 \qquad \textbf{(атомы не могут поглотить фотон)}},\\
		\bm{
			\overline{\s}|\Psi\ra_{at} = 0 ~~\qquad \textbf{(атомы не могут испустить фотон).}}
	\end{cases}
\end{equation}
\begin{center}
$\displaystyle|\Psi\ra_{\text{at}}$ --- произвольное состояние атомного ансамбля
\end{center}

\noindent Таким образом, условие \eqref{critetion} является \textbf{критерием темноты} атомного ансамбля, что непосредственно следует из определения темного состояния.

В работе \cite{dark_states_dimension} было сформулировано и строго доказано утверждение о том, что размерность темного подпространства пространства $n$ двухуровневых атомов равна
\begin{equation}\label{dim_d_2_n}
	\text{dim}(D_{n}^2) =	
	\begin{cases}
		C_{n}^{k} - C_{n}^{k-1}\quad~~\text{при}~n = 2k, \\
		0 \qquad\qquad\qquad\text{в противном случае},
	\end{cases}
\end{equation}
и все наборы темных состояний (для четного числа атомов в группе) с учетом нормировки имеют вид 
\begin{equation}\label{d_2_n}
	\frac{1}{2^{n/4}}\bigotimes_{j=1}^{n/2}(|01\ra_{j}-|10\ra_{j}),
\end{equation}
где индекс $j$ означает номер пары $j = 1,\dots, n/2$ при произвольном разбиении группы из $n$ атомов.
\\[24pt]
\textbf{Приведем несколько примеров:}\\

\noindent
$n = 2$:
\begin{itemize}
	\item[$\triangledown$]{темные состояния: $|0\ra_{1}|1\ra_{2} - |1\ra_{1}|0\ra_{2}$}
	\item[$\triangledown$]{размерность темного подпространства: $C_{2}^{1} - C_{2}^{0} = 1$\\}
\end{itemize}
$n = 3$:
\begin{itemize}
	\item[$\triangledown$]{нет темных состояний\\}
\end{itemize}
$n = 4$:
\begin{itemize}
	\item[$\triangledown$]{
		темные состояния:\\
		$(|0\ra_{1}|1\ra_{2} - |1\ra_{1}|0\ra_{2})\otimes(|0\ra_{3}|1\ra_{4} - |1\ra_{3}|0\ra_{4})$\\
		$(|0\ra_{1}|1\ra_{3} - |1\ra_{1}|0\ra_{3})\otimes(|0\ra_{2}|1\ra_{4} - |1\ra_{2}|0\ra_{4})$\\
		$(|0\ra_{1}|1\ra_{4} - |1\ra_{1}|0\ra_{4})\otimes(|0\ra_{2}|1\ra_{3} - |1\ra_{2}|0\ra_{3})$
	}
	\item[$\triangledown$]{размерность темного подпространства: $C_{4}^{2} - C_{4}^{1} = 2$.}
\end{itemize}

\section{Постановка задачи}\label{sec:ch4/sect2}
Результат \eqref{d_2_n} установлен и доказан только для случая двухуровневых атомных ансамблей. Аналогичное утверждение для ансамблей трехуровневых
атомов в качестве гипотезы формулируется следующим образом:
\begin{hyp}
	\label{Th:20}Темное подпространство пространства n трехуровневых атомов есть линейная оболочка состояний $\displaystyle \bigotimes_{j=1}^{n/3}{\widehat{D}_{3}^{(j)}}$,
	где $\widehat{D}_{3}^{(j)}$ --- \textbf{трехатомное состояние, имеющее вид}
	\begin{equation}
		\sum_{\pi \in S_{3}}|\pi(1)\ra|\pi(2)\ra|\pi(3)\ra(-1)^{\s(\pi)}
	\end{equation}
	\begin{center}(разбиение $n$ атомов на тройки произвольно).\end{center}
\end{hyp}

\clearpage
\indent Примером темного состояния ансамбля трехуровневых атомов является состояние
$|\Psi\ra = |012\ra + |120\ra + |201\ra - |021\ra - |102\ra - |210\ra$ (оно же единственное).

Возвращаясь к \textbf{критерию темноты} \eqref{critetion} многоатомного квантового состояния, отметим, что множество решений системы
\begin{equation}\label{slae}
	\begin{cases}
		\overline{\s}^{+}|\Psi\ra_{\text{at}} = 0,\\
		\overline{\s}|\Psi\ra_{\text{at}} = 0,
	\end{cases} \Leftrightarrow
	Ax =
	\begin{pmatrix}
		\overline{\s}^{+}\\
		\overline{\s}
	\end{pmatrix}
	\begin{pmatrix}
		\lambda_1\\
		\dots\\
		\lambda_N\\
	\end{pmatrix} = 0
\end{equation}
однородных уравнений с соответствующей матрицей системы
$
A=\begin{pmatrix}
	\overline{\s}^{+}\\
	\overline{\s}
\end{pmatrix}
$ размерности $M \times N$ есть линейное подпространство размерности $N -
rank(A)$. 

Для случая трехуровневых атомов размерность матрицы системы равна $(M, N) = (6 \cdot 3^{n}, 3^{n})$, а размерность темного подпространства определяется как
\begin{equation}\label{dim_3_level}
\mathrm{dim}(D_{n}^{3}) = 3^{n} - rank(A).
\end{equation}

\noindent Таким образом, определение размерности темного подпространства сводится к задаче определения ранга матрицы $A$ системы для различных значений $n$.
\\[24pt]
\noindent \textbf{Рассмотрим несколько примеров}\\[12pt]
Определим размерности темного подпространства для ряда двухуровневых атомных ансамблей. По аналогии с \eqref{dim_3_level} размерность будет равна
\[
\mathrm{dim}(D_{n}^{2}) = 2^{n} - rank(A).
\]
\
\\
\medskip\hrule\medskip
\noindent$\mathbf{n = 2:}$
\begin{equation}\label{eq:matrix2}
	{\footnotesize
		A =
		\bordermatrix{
			& |00\ra & |01\ra & |10\ra & |11\ra \cr
			& 0 & 1 & 1 & 0 \cr
			& 0 & 0 & 0 & 1 \cr
			& 0 & 0 & 0 & 1 \cr
			& 0 & 0 & 0 & 0 \cr
			& 0 & 0 & 0 & 0 \cr
			& 1 & 0 & 0 & 0 \cr
			& 1 & 0 & 0 & 0 \cr
			& 0 & 1 & 1 & 0 \cr
		}\rightarrow
		\bordermatrix{
			& |00\ra & |01\ra & |10\ra & |11\ra \cr
			& 1 & 0 & 0 & 0 \cr
			& 0 & 1 & 1 & 0 \cr
			& 0 & 0 & 0 & 1 \cr
		}
	}
\end{equation}
\quad\quad~~Решая систему \eqref{slae}, находим
{\footnotesize
	$\lambda =
	\begin{pmatrix}
		0\\
		\xi\\
		-\xi\\
		0
	\end{pmatrix}
	$
},\
\quad$D^{(2)}_p = \L(\{|01\ra - |10\ra\})$.\\[12pt]

\noindent\quad\quad~~~$\dim(D^{(2)}_p) = 2^{p} - rank(A) = 2^{2} - 3 = 1$\\

\medskip\hrule\medskip

\noindent$\mathbf{n = 3:}$
{\footnotesize
	\[
	A =
	\begin{pmatrix}
		\overline{\s}^{+}\\
		\overline{\s}
	\end{pmatrix}=
	\begin{pmatrix}
		0 & 1 & 1 & 0 & 1 & 0 & 0 & 0\\
		0 & 0 & 0 & 1 & 0 & 1 & 0 & 0\\
		0 & 0 & 0 & 1 & 0 & 0 & 1 & 0\\
		0 & 0 & 0 & 0 & 0 & 0 & 0 & 1\\
		0 & 0 & 0 & 0 & 0 & 1 & 1 & 0\\
		0 & 0 & 0 & 0 & 0 & 0 & 0 & 1\\
		0 & 0 & 0 & 0 & 0 & 0 & 0 & 1\\
		0 & 0 & 0 & 0 & 0 & 0 & 0 & 0\\
		0 & 0 & 0 & 0 & 0 & 0 & 0 & 0\\
		1 & 0 & 0 & 0 & 0 & 0 & 0 & 0\\
		1 & 0 & 0 & 0 & 0 & 0 & 0 & 0\\
		0 & 1 & 1 & 0 & 0 & 0 & 0 & 0\\
		1 & 0 & 0 & 0 & 0 & 0 & 0 & 0\\
		0 & 1 & 0 & 0 & 1 & 0 & 0 & 0\\
		0 & 0 & 1 & 0 & 1 & 0 & 0 & 0\\
		0 & 0 & 0 & 1 & 0 & 1 & 1 & 0\\
	\end{pmatrix}\rightarrow
	\begin{pmatrix}
		1 & 0 & 0 & 0 & 0 & 0 & 0 & 0\\
		0 & 1 & 0 & 0 & 0 & 0 & 0 & 0\\
		0 & 0 & 1 & 0 & 0 & 0 & 0 & 0\\
		0 & 0 & 0 & 1 & 0 & 0 & 0 & 0\\
		0 & 0 & 0 & 0 & 1 & 0 & 0 & 0\\
		0 & 0 & 0 & 0 & 0 & 1 & 0 & 0\\
		0 & 0 & 0 & 0 & 0 & 0 & 1 & 0\\
		0 & 0 & 0 & 0 & 0 & 0 & 0 & 1\\
	\end{pmatrix}
	\]
}
\
\\[12pt]

\noindent\qquad~~~$\dim(D^{(2)}_p) = 2^{p} - rank(A) = 2^{3} - 8 = 0$\\

\medskip\hrule\medskip
\
\\[12pt]
\noindent Перейдем к перечислению трудностей, возникающих при решении поставленной задачи, а именно задачи точного вычисления ранга сверхбольшой матрицы.\\Исходя из вышеуказанных примеров, можно отметить следующее:
\begin{itemize}
	\item[$\bullet$]{
		изначально (до процедуры приведения к ступенчатому виду) матрица состоит из нулей и единиц,}
	\item[$\bullet$]{
		матрица является сильно разреженной с множеством нулевых строк,}
	\item[$\bullet$]{
		подавляющее большинство строк матрицы нетривиальны: не соответствуют строкам единичной матрицы, в значительной части из них количество единиц сильно превосходит количество нулей,}
	\item[$\bullet$]{
		размерность матрицы в случае трехуровневой системы равна $6 \cdot 3^{n} \times 3^{n}$, поскольку возможны возбуждения и релаксации атомов трех типов --- $\s_{i}^{+\{0,1\}} / \s_{i}^{\{0,1\}}$, $\s_{i}^{+\{1,2\}} / \s_{i}^{\{1,2\}}$, $\s_{i}^{+\{0,2\}} / \s_{i}^{\{0,2\}}$.
	}
\end{itemize}
\
\\[0pt]
\noindent\textbf{Зависимость размерности матрицы системы от числа $n$ атомов в ансамбле носит экспоненциальный характер:}

\noindent\begin{tabular}[t]{|p{4em}|p{5em}|p{4em}|p{9em}|p{9em}|}
	\hline
	$n$ & 3 &  & 18 & 21 \\
	\hline
	$M$ x $N$ & $162 \times 27$ & $\quad~\cdots$ & $2.3 \cdot 10^9 \times 387 \cdot 10^6$ & $62 \cdot 10^9 \times 10.4 \cdot 10^9$ \\
	\hline
\end{tabular}
\
\\[12pt]

\noindent Перечислим также возникающие вычислительные трудности:
\begin{itemize}
	\item[$\bullet$]{вещественные плотные матрицы размера $(6 \cdot 3^{n}) \times 3^{n}$, начиная уже с малых значений $n$, не умещаются целиком в оперативную память,}
	\item[$\bullet$]{алгоритм должен \textbf{точно вычислять ранг}: ошибки округления при работе с вещественными числами могут повлиять на корректность вы­числений.\\
		Такие инциденты, как
		\[
		\begin{pmatrix}
			0 & \dots & 0.3333333 & \dots \\
			0 & \dots & 0.3333334 & \dots \\
		\end{pmatrix}
		\]}приводят к вычислению неверного ранга.\\[0pt]

	Кроме того, было обнаружено, что многие существующие на сего­ дняшний день алгоритмы и пакеты программ для вычисления ран­га разреженных матриц дают неверный ответ (в частности, метод $\mathsf{linalg.interpolative.estimate\_rank}$ библиотеки $\mathbf{scipy}$ начиная с $n = 9$).\\[18pt]
	Данный факт означает \textbf{необходимость вычисления ранга в целых числах}.
	\item[$\bullet$]{применение алгоритма Гаусса приведения матрицы к ступенчатому ви­ду занимает неприемлемо большое вычислительное время и не может быть использовано для матриц подобных размеров.}
\end{itemize}

\section{Описание алгоритма}\label{sec:ch4/sect3}

Перейдем непосредственно к описанию предложенного алгоритма.
Он будет состоять из трех частей:
\begin{enumerate}
	\item{построение разреженной матрицы системы $A$};
	\item{целочисленное приведение разреженной матрицы к ступенчатой форме путем редуцирования соответствующего ей графа};
	\item{окончательное целочисленное приведение матрицы к ступенчатой форме с помощью
		алгоритма Гаусса}.
\end{enumerate}

\subsection{Построение разреженной матрицы системы}\label{subsec:ch4/subsect1}
Генерация разреженных матриц для различных $n = 3\dots21$ выполняется стандартным
образом с использованием научного пакета Python SciPy для разреженных матриц.

Каждая строка содержит номера столбцов ненулевых элементов (нумерация столбцов
начинается с нуля). Полностью нулевые строки плотной матрицы пропускаются.

К примеру, для системы \eqref{eq:matrix2} разреженная матрица записывается в следующем виде:
\begin{flushleft}
	\noindent \qquad 1,2\\
	\noindent\qquad 3,\\
	\noindent\qquad 3,\\
	\noindent\qquad 0,\\
	\noindent\qquad 0,\\
	\noindent\qquad 1,2\\
\end{flushleft}\label{eq:sparse_matrix2}
\
\\[12pt]
\noindent Для трехуровневой системы построение выполняется аналогично.

\clearpage
\noindent Выпишем характерные метрики после первого этапа алгоритма:

\noindent
{\footnotesize
	\begin{tabular}[t]{|p{5em}|p{3em}|p{4em}|p{5em}|p{5em}|p{5em}|p{5em}|p{5em}|}
		\hline
		$n$ & 3 & 6 & 9 & 12 & 15 & 18 & 21 \\
		\hline
		$\mathrm{M}$ & 162 & 4374 & $118 \cdot 10^{3}$ & $3.2 \cdot 10^{6}$ & $86.1 \cdot 10^{6}$ & $2.3 \cdot 10^{9}$ & $62.7 \cdot 10^{9}$ \\
		\hline
		$\mathrm{N}$ & 27 & 729 & $19.6 \cdot 10^{3}$ & $531.4 \cdot 10^{3}$ & $14.3 \cdot 10^{6}$ & $387.4 \cdot 10^{6}$ & $10.4 \cdot 10^{9}$ \\
		\hline
		$\mathbf{nrows}$ & 114 & 3990 & $\approx M$ & $\cdots$ & $\cdots$ & $\cdots$ & $\approx M$ \\
		\hline
		$\mathbf{nonzeros}$ & 162 & 8748 & $354.3 \cdot 10^{3}$ & $12.7 \cdot 10^{6}$ & $430.4 \cdot 10^{6}$ & $13.9 \cdot 10^{9}$ & $439.3 \cdot 10^{9}$ \\
		\hline
	\end{tabular}
}

\begin{itemize}
	\item[$\bullet$]{$n$  --- кол-во атомов},
	\item[$\bullet$]{$\mathrm{M}$ --- кол-во строк в плотной\footnote[1]{плотные матрицы не создаются} матрице,}
	\item[$\bullet$]{$\mathrm{N}$ --- кол-во столбцов в плотной матрице,}
	\item[$\bullet$]{$\mathbf{nrows}$  --- кол-во строк в разреженной матрице,}
	\item[$\bullet$]{$\mathbf{nonzeros}$ --- кол-во ненулевых элементов (единиц) в разреженной матрице.}
\end{itemize}
\
\\[0pt]
\noindent Как мы видим, для $n = 21$ разреженная матрица системы содержит несколько десятков миллиардов строк.

\subsection{Элементарные преобразования над матрицей системы путем редукции соответствующего ей графа}\label{subsec:ch4/subsect2}

Для разреженной матрицы системы построим граф по следующей схеме:
\begin{itemize}
	\item[$\bullet$]{значение вершины --- номер строки в разреженной матрице,}
	\item[$\bullet$]{метка ребра, входящего в вершину, содержит позиции единиц в соот­ветствующей строке разреженной матрицы (что также соответствует номерам столбцов единичных элементов в плотной матрице),\\[12pt]
	нумерация столбцов в плотной матрице производится с нуля
	}
	\item[$\bullet$]{вершины, в которые входят ребра с общими метками, соединяются реб­ром с этими метками.}
\end{itemize}
\
\\[0pt]
\indent Для системы \eqref{eq:matrix2} и соответствующей ее разреженной матрицы граф выглядит следующим образом:

\clearpage
\noindent\textbf{Пример 1.}
\\[18pt]
\underline{\hspace{1.5em} Плотная матрица \hspace{1.5em} | \hspace{1em} Разреженная матрица \hspace{1em} | \hspace{3em} Граф \hspace{3.5em}}\\

\[
\hspace{1em}{\footnotesize
\parbox[b][5cm][t]{50mm}{
	\bordermatrix{
		& |00\ra & |01\ra & |10\ra & |11\ra \cr
		& 0 & 1 & 1 & 0 \cr
		& 0 & 0 & 0 & 1 \cr
		& 0 & 0 & 0 & 1 \cr
		& 0 & 0 & 0 & 0 \cr
		& 0 & 0 & 0 & 0 \cr
		& 1 & 0 & 0 & 0 \cr
		& 1 & 0 & 0 & 0 \cr
		& 0 & 1 & 1 & 0 \cr
	}
}
\hfill
\hspace{9.5em}\parbox[b][5cm][t]{35mm}{
	\begin{flushleft}
		1,2\\
		3\\
		3\\
		0\\
		0\\
		1,2
	\end{flushleft}
}
\hfill
\hspace{1.5em}\parbox[b][5cm][t]{50mm}{
	\includegraphics[width=36mm]{Dissertation/images/section_4/graph1.eps}
}
}\label{eq:graph}
\]

Данный граф позволяет производить вычитание строк, и таким образом, чтобы отрицательные числа не появлялись в соответствующей ему плотной матрице системы. Удаление дубликатов и вычитание строк друг из друга выполняются путем удаления соответствующих ребер графа. 

Вершины, не имеющие входных и выходных ребер, удаляются, что соответствует удалению нулевых строк в исходной матрице в результате элементарных преобразований.

\[
{\small
\begin{pmatrix}
    \cdots & \cdots & \cdots & \cdots & \cdots & \cdots & \cdots & \cdots & \cdots \cr
    0 & \fbox{1} & 0 & \fbox{1} & \fbox{1} & \fbox{1} & \fbox{1} & 0 & \fbox{1} \cr
    \cdots & \cdots & \cdots & \cdots & \cdots & \cdots & \cdots & \cdots & \cdots \cr
    0 & \fbox{1} & 0 & \fbox{1} & 0 & \fbox{1} & \fbox{1} & 0 & \fbox{1} \cr
    \cdots & \cdots & \cdots & \cdots & \cdots & \cdots & \cdots & \cdots & \cdots
\end{pmatrix}
\rightarrow
\begin{pmatrix}
    \cdots & \cdots & \cdots & \cdots & \cdots & \cdots & \cdots & \cdots & \cdots \cr
    0 & 0 & 0 & 0 & 1 & 0 & 0 & 0 & 0 \cr
    \cdots & \cdots & \cdots & \cdots & \cdots & \cdots & \cdots & \cdots & \cdots \cr
    0 & \fbox{1} & 0 & \fbox{1} & 0 & \fbox{1} & \fbox{1} & 0 & \fbox{1} \cr
    \cdots & \cdots & \cdots & \cdots & \cdots & \cdots & \cdots & \cdots & \cdots
\end{pmatrix}
}
\]

\[
{\small
\begin{pmatrix}
    \cdots & \cdots & \cdots & \cdots & \cdots & \cdots & \cdots & \cdots & \cdots \cr
    0 & \fbox{1} & 0 & \fbox{1} & 0 & \fbox{1} & \fbox{1} & 0 & \fbox{1} \cr
    \cdots & \cdots & \cdots & \cdots & \cdots & \cdots & \cdots & \cdots & \cdots \cr
    0 & \fbox{1} & 0 & \fbox{1} & \fbox{1} & \fbox{1} & \fbox{1} & 0 & \fbox{1} \cr
    \cdots & \cdots & \cdots & \cdots & \cdots & \cdots & \cdots & \cdots & \cdots
\end{pmatrix}
\rightarrow
\begin{pmatrix}
    \cdots & \cdots & \cdots & \cdots & \cdots & \cdots & \cdots & \cdots & \cdots \cr
    0 & \fbox{1} & 0 & \fbox{1} & 0 & \fbox{1} & \fbox{1} & 0 & \fbox{1} \cr
    \cdots & \cdots & \cdots & \cdots & \cdots & \cdots & \cdots & \cdots & \cdots \cr
    0 & 0 & 0 & 0 & 1 & 0 & 0 & 0 & 0 \cr
    \cdots & \cdots & \cdots & \cdots & \cdots & \cdots & \cdots & \cdots & \cdots
\end{pmatrix}
}
\]

\[
{\small
\begin{pmatrix}
    \cdots & \cdots & \cdots & \cdots & \cdots & \cdots & \cdots & \cr
    0 & \fbox{1} & \fbox{1} & \fbox{1} & \fbox{1} & 0 & \fbox{1} \cr
    \cdots & \cdots & \cdots & \cdots & \cdots & \cdots & \cdots & \cr
    0 & \fbox{1} & \fbox{1} & \fbox{1} & \fbox{1} & 0 & \fbox{1} \cr
    \cdots & \cdots & \cdots & \cdots & \cdots & \cdots & \cdots & \cr
    0 & \fbox{1} & \fbox{1} & \fbox{1} & \fbox{1} & 0 & \fbox{1} \cr
    \cdots & \cdots & \cdots & \cdots & \cdots & \cdots & \cdots & \cr
    0 & \fbox{1} & \fbox{1} & \fbox{1} & \fbox{1} & 0 & \fbox{1} \cr
    \cdots & \cdots & \cdots & \cdots & \cdots & \cdots & \cdots
\end{pmatrix}
\rightarrow
\begin{pmatrix}
    \cdots & \cdots & \cdots & \cdots & \cdots & \cdots & \cdots & \cr
    0 & \fbox{1} & \fbox{1} & \fbox{1} & \fbox{1} & 0 & \fbox{1} \cr
    \cdots & \cdots & \cdots & \cdots & \cdots & \cdots & \cdots
\end{pmatrix}
}
\]

Другим словами, это вычитания из «старшей» строки $A$ «младшей» стро­ки $B$, при которых строка $A$ содержит все единицы строки $B$ на тех же позициях, но, кроме того, может также содержать единицы на позициях, кото­рые отсутствуют в строке $B$. Соответствующие вершины $A'$ и $B'$ графа будут соединены ребром (с ненулевым множеством меток), а множество меток ребра, входящего в вершину $B'$, будет подмножеством меток ребра, входящего в вер­шину $A'$.
\\[24pt]
\noindent Для матрицы
\[
{\small
\begin{pmatrix}
    \cdots & \cdots & \cdots & \cdots & \cdots & \cdots & \cdots & \cdots & \cdots \cr
    0 & \fbox{1} & 0 & \fbox{1} & \fbox{1} & \fbox{1} & \fbox{1} & 0 & \fbox{1} \cr
    \cdots & \cdots & \cdots & \cdots & \cdots & \cdots & \cdots & \cdots & \cdots \cr
    0 & \fbox{1} & 0 & \fbox{1} & 0 & \fbox{1} & \fbox{1} & 0 & \fbox{1} \cr
    \cdots & \cdots & \cdots & \cdots & \cdots & \cdots & \cdots & \cdots & \cdots
\end{pmatrix}
}
\]
возможны следующие фрагменты соответствующего ей графа (двух типов):

\begin{figure}[h!]
	\noindent\centering{
		\includegraphics[width=0.45\textwidth]{Dissertation/images/section_4/A.png}
	}
\end{figure}

\clearpage
\noindent Аналогично для матрицы
\[
{\small
\begin{pmatrix}
    \cdots & \cdots & \cdots & \cdots & \cdots & \cdots & \cdots & \cdots & \cdots \cr
    0 & \fbox{1} & 0 & \fbox{1} & 0 & \fbox{1} & \fbox{1} & 0 & \fbox{1} \cr
    \cdots & \cdots & \cdots & \cdots & \cdots & \cdots & \cdots & \cdots & \cdots \cr
    0 & \fbox{1} & 0 & \fbox{1} & \fbox{1} & \fbox{1} & \fbox{1} & 0 & \fbox{1} \cr
    \cdots & \cdots & \cdots & \cdots & \cdots & \cdots & \cdots & \cdots & \cdots
\end{pmatrix}
}
\]

\begin{figure}[h!]
	\noindent\centering{
		\includegraphics[width=0.17\textwidth]{Dissertation/images/section_4/B.png}
	}
\end{figure}

\noindent и для случая строк-дубликатов

\[
{\small
\begin{pmatrix}
    \cdots & \cdots & \cdots & \cdots & \cdots & \cdots & \cdots & \cr
    0 & \fbox{1} & \fbox{1} & \fbox{1} & \fbox{1} & 0 & \fbox{1} \cr
    \cdots & \cdots & \cdots & \cdots & \cdots & \cdots & \cdots & \cr
    0 & \fbox{1} & \fbox{1} & \fbox{1} & \fbox{1} & 0 & \fbox{1} \cr
    \cdots & \cdots & \cdots & \cdots & \cdots & \cdots & \cdots & \cr
    0 & \fbox{1} & \fbox{1} & \fbox{1} & \fbox{1} & 0 & \fbox{1} \cr
    \cdots & \cdots & \cdots & \cdots & \cdots & \cdots & \cdots & \cr
    0 & \fbox{1} & \fbox{1} & \fbox{1} & \fbox{1} & 0 & \fbox{1} \cr
    \cdots & \cdots & \cdots & \cdots & \cdots & \cdots & \cdots
\end{pmatrix}
}
\]

\begin{figure}[h!]
	\noindent\centering{
		\includegraphics[width=0.12\textwidth]{Dissertation/images/section_4/C.png}
	}
\end{figure}

\clearpage
Удаление ребер в \textbf{примере 1} происходит следующим образом:

\begin{figure}[h!]
	\noindent\centering{
		\includegraphics[width=0.23	\textwidth]{Dissertation/images/section_4/graph2.eps}
	}
\end{figure}

Плотная матрица, соответствующая редуцированному графу, после эле­ментарных преобразований будет иметь вид:
\[
{\small
	\bordermatrix{
		& |00\ra & |01\ra & |10\ra & |11\ra \cr
		& 1 & 0 & 0 & 0 \cr
		& 0 & 1 & 1 & 0 \cr
		& 0 & 0 & 0 & 1
	}
}
\]
\
\\[12pt]
\noindent Рассмотрим более сложный пример:

\noindent\textbf{Пример 2.}
\[
{\small
A =
\begin{pmatrix}
	\cdots & \cdots & \cdots & \cdots & \cdots\\
	1 & 0 & 0 & 0 & 1 \cr
	0 & 1 & 0 & 0 & 0 \cr
	1 & 0 & 0 & 1 & 1 \cr
	0 & 0 & 1 & 0 & 0 \cr
	\cdots & \cdots & \cdots & \cdots & \cdots
\end{pmatrix}.
}
\]

\noindent Соответствующие преобразования графа:
\begin{figure}[h]
	\noindent\centering{
		\includegraphics[width=140mm]{Dissertation/images/section_4/graph7.eps}
	}
	\label{figCurves}
\end{figure}

\clearpage
Последний пример демонстрирует возможность удаления дубликатов и вычитания строк, находящихся на расстоянии. «Проход» по цепочке связанных вершин может осуществляться как сверху вниз, так и снизу вверх. Для это­го достаточно определять пересечения меток ребер, входящих и выходящих в связанные вершины. В результате мы избегаем сравнения строк между собой, поиска совпадающих пар и использования операции деления с неизбежным воз­никновением вещественных чисел в ходе элементарных преобразований.

Предложенный подход позволяет на несколько порядков сократить экспо­ненциально растущую с ростом $n$ размерность плотной матрицы системы. К примеру, для 21 атома десятки миллиардов строк в исходной матрице системы будут <<коллапсированы>> до не более чем десяти тысяч строк, что позволит в дальнейшем произвести точное вычисление ее ранга.

Представленный алгоритм является параллельным и производит редукцию графа, распределенного по сетке процессоров: разреженные матрицы считываются параллельно <<ленточным>> образом (каждый процесс считывает свою часть матрицы и формирует свою часть общего графа). Если связанные вершины принадлежат разным процессам, информация об этом (номера вер­шин и множества меток, соответствующих входящим и выходящим ребрам) пересылается в неблокирующем режиме (MPI\_Isend, MPI\_Irecv). При необхо­димости удаления вершин и/или ребер соответствующая информация также пересылается между процессами.

Работа алгоритма начинается с удаления дубликатов строк каждым про­цессом, а именно с сокращения длин тех цепочек, в которых множество меток соединяющих их ребер будет общим для всех вершин. В результате данной процедуры каждая строка матрицы будет содержаться в одном экземпляре, а вершины, отвечающие дубликатам строк, будут удалены. После чего произво­дится процесс вычитания таких строк из других в порядке обхода кратности вершин, соответствующих этим строкам. Кратность вершины в данном случае есть число входящих в нее ребер с отличными друг от друга множествами ме­ток. Данная процедура приводит к появлению строк-дубликатов: в этом случае вновь запускается процесс их удаления с возможной рассылкой информации об этом другим процессорам.

Вся процедура повторяется до тех пор, пока в матрице не остается ни одной пары строк, которые можно было бы вычесть друг из друга:
\begin{itemize}
\item{каждый отдельный процессор не может удалить ребро или вершину,}
\item{невозможно удалить ребро, соединяющее вершины, находящиеся на раз­ных процессорах.}
\end{itemize}

Выполнение работы данного алгоритма производилось на суперкомпью­ тере Ломоносов-2 с использованием 256 процессорных узлов, содержащих 2048 процессорных ядер. Максимально доступное число атомов в рамках имеющихся в распоряжении вычислительных ресурсов для поставленной задачи состави­ло 21.

\subsection{Окончательное целочисленное приведение матрицы к ступенчатой форме с помощью алгоритма Гаусса}\label{subsec:ch4/subsect3}
Результат второго этапа --- получение матрицы значительно меньшего размера, не содержащей дубликатов строк, прошедшую процесс элементарных преобразований над строками. Ее достаточно просто восстановить по редуци­рованному графу. Количество строк в такой матрице на несколько порядков отличается от числа строк в исходной плотной матрице системы. Окончатель­ное приведение матрицы к ступенчатому виду производится с применением алгоритма Гаусса в целых числах, допускающего следующие операции:
\begin{itemize}
\item{сложение и вычитание строк,}
\item{перестановка строк,}
\item{целочисленное умножение и деление строки на число.}
\end{itemize}

В результате умножений строк на число и вычитаний в матрице появля­ются отрицательные числа, а также числа, по модулю существенно большие единицы. В процессе вычислений было обнаружено, что рост абсолютных зна­чений не является неограниченным и не приводит к переполнению типа (в данном случае использовался 4-байтный тип $\mathsf{int}$). Это эмпирический факт, кото­рый позволил произвести процесс точного вычисления ранга матрицы до конца. В случае возникшего переполнения типа для элементов матрицы были бы за­ действованы большеразмерные типы ($\mathsf{long}$, $\mathsf{long~long}$), либо был бы пересмотрен подход к процедуре окончательного приведения матрицы к ступенчатому виду в пользу других альтернатив.

\clearpage
Отсутствие переполнения может объясняться рядом особенностей мат­рицы, полученной на этапе 2:
\begin{itemize}
\item{значительная часть строк матрицы соответствует строками единичной матрицы и в ходе процедуры приведения матрицы к ступенчатому виду требует лишь перестановки,}
\item{размерность матрицы на несколько порядков меньше размерности ис­ходной матрицы,}
\item{матрица является значительно более разреженной, чем исходная мат­рица: строки, изначально состоящие практически целиком из единиц стали в еще большей степени разреженными, поскольку из них вычи­тались строки с меньшим количеством единиц, в том числе, и строки единичной матрицы.}
\end{itemize}
\
\\
\indent По окончании данного этапа плотная матрица системы, сокращенная в своей в размерности на этапе 2, приводится к единичному виду и ее ранг соот­ветствует количеству ее строк.
\\[12pt]
\indent По итогам работы алгоритма размерность темного подпространства трех­уровневых ансамблей атомов была численно установлена для следующих значений $n = 3k$:

\noindent\begin{tabular}[t]{|p{5em}|p{3em}|p{3em}|p{3em}|p{3em}|p{4em}|p{4em}|p{4em}|}
	\hline
	$n$ & 3 & 6 & 9 & 12 & 15 & 18 & 21 \\
	\hline
	$\dim(D^{(3)}_n)$ & 1 & 5 & 28 & 165 & 1001 & 6188 & 38 760 \\
	\hline
\end{tabular}
\
\\[24pt]
\indent Для значений $n = \overline{1, 9}$ при $n$ кратном 3 мультисинглеты вида \eqref{dim_3_level} были выписаны в явном виде. Размерность линейной оболочки, натянутой на эти состояния, совпала с размерностью темного подпространства, полученной в результате работы представленного выше алгоритма, подтвердив тем самым гипотезу о структуре темного подпространства. А именно подтверждено, что для ансамблей, состоящих из $n = \overline{1, 9}~(n = 3k)$ атомов, всё множество тем­ных состояний исчерпывается мультисинглетами вида \eqref{dim_3_level} и других темных состояний в ансамблях с указанной численностью атомов нет.

\clearpage
\noindent $\textbf{n = 3:}$
\begin{figure}[h]
	\noindent{
		\includegraphics[width=1.0\textwidth]{Dissertation/images/section_4/n_3.png}
	}
	\label{figCurves}
\end{figure}

\noindent $\textbf{n = 6:}$
\begin{figure}[h]
	\noindent{
		\includegraphics[width=1.0\textwidth]{Dissertation/images/section_4/n_6.png}
	}
	\label{figCurves}
\end{figure}

\clearpage
\noindent $\textbf{n = 9:}$
\begin{figure}[h]
	\noindent{
		\includegraphics[width=0.4\textwidth]{Dissertation/images/section_4/n_9_1.png}
	}
	\label{figCurves}
\end{figure}

\begin{figure}[h]
	\noindent{
		\includegraphics[width=1.0\textwidth]{Dissertation/images/section_4/n_9_2.png}
	}
	\label{figCurves}
\end{figure}

\clearpage
\section{Выводы главы}\label{sec:ch4/sect4}
Был предложен алгоритм определения размерности темного подпростран­ства пространства многоуровных атомных ансамблей, основанный на целочисленном вычислении ранга сверхбольшой двоичной матрицы путем параллельной редукции соответствующего ей графа.

Данный алгоритм позволил установить размерность темного подпростран­ства для ансамблей, содержащих до 21 трехуровневого атома включительно. Для некратного трем количества атомов в группе работа алгоритма завер­шилась установлением размерности темного подпространства, равной нулю, означающему отсутствие темных состояний в этом пространстве.

Гипотеза о структуре и явном виде темных состояний была подтверждена для 1, 3, 6 и 9 трехуровневых атомов в группе. Было показано, что все темные состояния в данном случае есть линейные комбинации мультисинглетов.